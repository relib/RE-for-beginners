\subsection{32-bit ARM (AArch32)}

\subsubsection{General purpose registers}

\begin{itemize}
\myindex{ARM!\Registers!R0}
	\item R0 --- function result is usually returned using R0
	\item R1...R12 --- \ac{GPR}s
	\item R13 --- \ac{AKA} SP (\gls{stack pointer})
\myindex{ARM!\Registers!Link Register}
	\item R14 --- \ac{AKA} LR (\gls{link register})
	\item R15 --- \ac{AKA} PC (program counter)
\end{itemize}

\myindex{ARM!\Registers!scratch registers}
\Reg{0}-\Reg{3} are also called \q{scratch registers}: the function's arguments are usually passed in them,
and the values in them are not required to be restored upon the function's exit.

\subsubsection{Current Program Status Register (CPSR)}

\begin{center}
\begin{tabular}{ | l | l | }
\hline
\headercolor\ Bit &
\headercolor\ Description \\
\hline
0..4           & M --- processor mode \\
\hline
5              & T --- Thumb state \\
\hline
6              & F --- FIQ disable \\
\hline
7              & I --- IRQ disable \\
\hline
8              & A --- imprecise data abort disable \\
\hline
9              & E --- data endianness \\
\hline
10..15, 25, 26 & IT --- if-then state \\
\hline
16..19         & GE --- greater-than-or-equal-to \\
\hline
20..23         & DNM --- do not modify \\
\hline
24             & J --- Java state \\
\hline
27             & Q --- sticky overflow \\
\hline
28             & V --- overflow \\
\hline
29             & C --- carry/borrow/extend \\
\hline
\myindex{ARM!\Registers!Z}
30             & Z --- zero bit \\
\hline
31             & N --- negative/less than \\
\hline
\end{tabular}
\end{center}

% TODO
% \myindex{ARM!\Registers!APSR}
% \subsubsection{Application Program Status Register (APSR)}

% TODO
% \myindex{ARM!\Registers!FPSCR}
% \subsubsection{Floating-Point Status and Control Register (FPPSR)}
% http://infocenter.arm.com/help/index.jsp?topic=/com.arm.doc.ddi0344b/Chdfafia.html

\subsubsection{VFP (floating point) and NEON registers}
\label{ARM_VFP_registers}

% http://infocenter.arm.com/help/index.jsp?topic=/com.arm.doc.dht0002a/ch01s03s02.html

\myindex{ARM!D-\registers{}}
\myindex{ARM!S-\registers{}}
\begin{center}
\begin{tabular}{ | l | l | l | l | }
\hline
0..31\textsuperscript{bits} & 32..64 & 65..96 & 97..127 \\
\hline
\multicolumn{4}{ | c | }{Q0\textsuperscript{128 bits}} \\
\hline
\multicolumn{2}{ | c | }{D0\textsuperscript{64 bits}} & \multicolumn{2}{ c | }{D1} \\
\hline
S0\textsuperscript{32 bits} & S1 & S2 & S3 \\
\hline
\end{tabular}
\end{center}

S-registers are 32-bit, used for the storage of single precision numbers.

D-registers are 64-bit ones, used for the storage of double precision numbers.

D- and S-registers share the same physical space in the CPU---it is possible to access 
a D-register via the S-registers (it is senseless though).

Likewise, the \gls{NEON} Q-registers are 128-bit ones and share the same physical space in the CPU 
with the other floating point registers.

In VFP 32 S-registers are present: S0..S31.

In VFPv2 there 16 D-registers are added, which in fact occupy the same space as S0..S31.

In VFPv3 (\gls{NEON} or \q{Advanced SIMD}) there are 16 more D-registers, D0..D31, but the D16..D31 registers are not sharing space with any other S-registers.

In \gls{NEON} or \q{Advanced SIMD} another 16 128-bit Q-registers were added, which share the same space as D0..D31.

\subsection{64-bit ARM (AArch64)}

\subsubsection{General purpose registers}
\label{ARM64_GPRs}

The number of registers has been doubled since AArch32.

\begin{itemize}
\myindex{ARM!\Registers!X0}
	\item X0 --- function result is usually returned using X0
        \item X0...X7 --- Function arguments are passed here.
	\item X8
	\item X9...X15 --- are temporary registers, the callee function can use and not restore them.
	\item X16
	\item X17
	\item X18
	\item X19...X29 --- callee function can use them, but must restore them upon exit.
	\item X29 --- used as \ac{FP} (at least GCC)
	\item X30 --- \q{Procedure Link Register} \ac{AKA} \ac{LR} (\gls{link register}).
	\item X31---register always contains zero \ac{AKA} XZR or \q{Zero Register}. It's 32-bit part is called WZR.
	\item \ac{SP}, not a general purpose register anymore.
\end{itemize}

See also: \ARMPCS.

The 32-bit part of each X-register is also accessible via W-registers (W0, W1, etc.).

\begin{center}
\begin{tabular}{ | l | l | }
\hline
	High 32-bit part & low 32-bit part \\
\hline
\multicolumn{2}{ | c | }{X0} \\
\hline
\multicolumn{1}{ | c | }{} & \multicolumn{1}{ c | }{W0} \\
\hline
\end{tabular}
\end{center}

