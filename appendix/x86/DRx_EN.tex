\subsection{Debugging registers}

Used for hardware breakpoints control.

\begin{itemize}
	\item DR0 --- address of breakpoint \#1
	\item DR1 --- address of breakpoint \#2
	\item DR2 --- address of breakpoint \#3
	\item DR3 --- address of breakpoint \#4
	\item DR6 --- a cause of break is reflected here
	\item DR7 --- breakpoint types are set here
\end{itemize}

\subsubsection{DR6}
\myindex{x86!\Registers!DR6}

\begin{center}
\begin{tabular}{ | l | l | }
\hline
\headercolor\ Bit (mask) &
\headercolor\ Description \\
\hline
0 (1)       &  B0 --- breakpoint \#1 has been triggered \\
\hline
1 (2)       &  B1 --- breakpoint \#2 has been triggered \\
\hline
2 (4)       &  B2 --- breakpoint \#3 has been triggered \\
\hline
3 (8)       &  B3 --- breakpoint \#4 has been triggered \\
\hline
13 (0x2000) &  BD --- modification attempt of one of the DRx registers. \\
            &  may be raised if GD is enabled \\
\hline
14 (0x4000) &  BS --- single step breakpoint (TF flag has been set in EFLAGS). \\
	    &  Highest priority. Other bits may also be set. \\
\hline
% TODO: describe BT
15 (0x8000) &  BT (task switch flag) \\
\hline
\end{tabular}
\end{center}

N.B. A single step breakpoint is a breakpoint which occurs after each instruction.
It can be enabled by setting TF in EFLAGS (\myref{EFLAGS}).

\subsubsection{DR7}
\myindex{x86!\Registers!DR7}

Breakpoint types are set here.

%\small
\begin{center}
\begin{tabular}{ | l | l | }
\hline
\headercolor\ Bit (mask) &
\headercolor\ Description \\
\hline
0 (1)       &  L0 --- enable breakpoint \#1 for the current task \\
\hline
1 (2)       &  G0 --- enable breakpoint \#1 for all tasks \\
\hline
2 (4)       &  L1 --- enable breakpoint \#2 for the current task \\
\hline
3 (8)       &  G1 --- enable breakpoint \#2 for all tasks \\
\hline
4 (0x10)    &  L2 --- enable breakpoint \#3 for the current task \\
\hline
5 (0x20)    &  G2 --- enable breakpoint \#3 for all tasks \\
\hline
6 (0x40)    &  L3 --- enable breakpoint \#4 for the current task \\
\hline
7 (0x80)    &  G3 --- enable breakpoint \#4 for all tasks \\
\hline
8 (0x100)   &  LE --- not supported since P6 \\
\hline
9 (0x200)   &  GE --- not supported since P6 \\
\hline
13 (0x2000) &  GD --- exception is to be raised if any MOV instruction \\
            & tries to modify one of the DRx registers \\
\hline
16,17 (0x30000)    &  breakpoint \#1: R/W --- type \\
\hline
18,19 (0xC0000)    &  breakpoint \#1: LEN --- length \\
\hline
20,21 (0x300000)   &  breakpoint \#2: R/W --- type \\
\hline
22,23 (0xC00000)   &  breakpoint \#2: LEN --- length \\
\hline
24,25 (0x3000000)  &  breakpoint \#3: R/W --- type \\
\hline
26,27 (0xC000000)  &  breakpoint \#3: LEN --- length \\
\hline
28,29 (0x30000000) &  breakpoint \#4: R/W --- type \\
\hline
30,31 (0xC0000000) &  breakpoint \#4: LEN --- length \\
\hline
\end{tabular}
\end{center}
%\normalsize

The breakpoint type is to be set as follows (R/W):

\begin{itemize}
\item 00 --- instruction execution
\item 01 --- data writes
\item 10 --- I/O reads or writes (not available in user-mode)
\item 11 --- on data reads or writes
\end{itemize}

N.B.: breakpoint type for data reads is absent, indeed. \\
\\
Breakpoint length is to be set as follows (LEN):

\begin{itemize}
\item 00 --- one-byte
\item 01 --- two-byte
\item 10 --- undefined for 32-bit mode, eight-byte in 64-bit mode
\item 11 --- four-byte
\end{itemize}
