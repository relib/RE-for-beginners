\mysection{MIPS}

\subsection{\Registers}
\label{MIPS_registers_ref}

\myindex{MIPS!O32}
( O32 calling convention )

\subsubsection{General purpose registers (\ac{GPR})}

%\small
\begin{center}
\begin{tabular}{ | l | l | l | }
\hline
\HeaderColor Number & 
\HeaderColor Pseudoname & 
\HeaderColor Description \\
\hline
\$0             & \$ZERO          & Always zero. Writing to this register is like \ac{NOP}. \\
\hline
\$1             & \$AT            & Used as a temporary register \\
                &                 & for assembly macros and pseudo instructions. \\
\hline
\$2 \dots \$3   & \$V0 \dots \$V1 & Function result is returned here. \\
\hline
\$4 \dots \$7   & \$A0 \dots \$A3 & Function arguments. \\
\hline
\$8 \dots \$15  & \$T0 \dots \$T7 & Used for temporary data. \\
\hline
\$16 \dots \$23 & \$S0 \dots \$S7 & Used for temporary data\AsteriskOne{}. \\
\hline
\$24 \dots \$25 & \$T8 \dots \$T9 & Used for temporary data. \\
\hline
\$26 \dots \$27 & \$K0 \dots \$K1 & Reserved for \ac{OS} kernel. \\
\hline
\$28            & \$GP            & Global Pointer\AsteriskTwo{}. \\
\hline
\$29            & \$SP            & \ac{SP}\AsteriskOne{}. \\
\hline
\$30            & \$FP            & \ac{FP}\AsteriskOne{}. \\
\hline
\$31            & \$RA            & \ac{RA}. \\
\hline
n/a             & PC              & \ac{PC}. \\
\hline
n/a             & HI              & high 32 bit of multiplication or division remainder\AsteriskThree{}. \\
\hline
n/a             & LO              & low 32 bit of multiplication and division remainder\AsteriskThree{}. \\
\hline
\end{tabular}
\end{center}
%\normalsize

\subsubsection{Floating-point registers}
\label{MIPS_FPU_registers}

\begin{center}
\begin{tabular}{ | l | l | l | }
\hline
\HeaderColor Name & \HeaderColor Description \\
\hline
\$F0..\$F1   & Function result returned here. \\
\hline
\$F2..\$F3   & Not used. \\
\hline
\$F4..\$F11  & Used for temporary data. \\
\hline
\$F12..\$F15 & First two function arguments. \\
\hline
\$F16..\$F19 & Used for temporary data. \\
\hline
\$F20..\$F31 & Used for temporary data\AsteriskOne{}. \\
\hline
%fcr31 & Control/status register. \\
%\hline
\end{tabular}
\end{center}

\AsteriskOne{} --- \Gls{callee} must preserve the value.\\
\AsteriskTwo{} --- \Gls{callee} must preserve the value (except in \ac{PIC} code).\\
\myindex{MIPS!\Instructions!MFLO}
\myindex{MIPS!\Instructions!MFHI}
\AsteriskThree{} --- accessible using the \TT{MFHI} and \TT{MFLO} instructions.\\

\subsection{\Instructions}

There are 3 kinds of instructions:

\begin{itemize}

\item R-type: those which have 3 registers. R-instruction usually have the following form:

\begin{lstlisting}
instruction destination, source1, source2
\end{lstlisting}

One important thing to keep in mind is that when the first and second register are the same, 
IDA may show the instruction in its shorter form:

\begin{lstlisting}
instruction destination/source1, source2
\end{lstlisting}

That somewhat reminds us of the Intel syntax for x86 assembly language.

\item I-type: those which have 2 registers and a 16-bit immediate value.

\item J-type: jump/branch instructions, have 26 bits for encoding the offset.

\end{itemize}

\subsubsection{Jump instructions}

What is the difference between B-instructions (\INS{BEQ}, \INS{B}, etc.) and J- ones (\INS{JAL}, \INS{JALR}, etc.)?

The B-instructions have an I-type, hence, the B-instructions' offset is encoded as a 16-bit immediate.
\INS{JR} and \INS{JALR} are R-type and jump to an absolute address specified in a register.
\INS{J} and \INS{JAL} are J-type, hence the offset is encoded as a 26-bit immediate.

In short, B-instructions can encode a condition 
(\INS{B} is in fact pseudo instruction for \TT{BEQ \$ZERO, \$ZERO, LABEL}), 
while J-instructions can't.

