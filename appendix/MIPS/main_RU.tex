\mysection{MIPS}

\subsection{\Registers}
\label{MIPS_registers_ref}

\myindex{MIPS!O32}
( Соглашение о вызовах O32 )

\subsubsection{Регистры общего пользования (\ac{GPR})}

%\small
\begin{center}
\begin{tabular}{ | l | l | l | }
\hline
\HeaderColor Номер & 
\HeaderColor Псевдоимя & 
\HeaderColor Описание \\
\hline
\$0             & \$ZERO          & Всегда ноль. Запись в этот регистр это как \ac{NOP}. \\
\hline
\$1             & \$AT            & Используется как временный регистр \\
                &                 & для ассемблерных макросов и псевдоинструкций. \\
\hline
\$2 \dots \$3   & \$V0 \dots \$V1 & Здесь возвращается результат функции. \\
\hline
\$4 \dots \$7   & \$A0 \dots \$A3 & Аргументы функции. \\
\hline
\$8 \dots \$15  & \$T0 \dots \$T7 & Используется для временных данных. \\
\hline
\$16 \dots \$23 & \$S0 \dots \$S7 & Используется для временных данных\AsteriskOne{}. \\
\hline
\$24 \dots \$25 & \$T8 \dots \$T9 & Используется для временных данных. \\
\hline
\$26 \dots \$27 & \$K0 \dots \$K1 & Зарезервировано для ядра \ac{OS}. \\
\hline
\$28            & \$GP            & Глобальный указатель\AsteriskTwo{}. \\
\hline
\$29            & \$SP            & \ac{SP}\AsteriskOne{}. \\
\hline
\$30            & \$FP            & \ac{FP}\AsteriskOne{}. \\
\hline
\$31            & \$RA            & \ac{RA}. \\
\hline
n/a             & PC              & \ac{PC}. \\
\hline
n/a             & HI              & старшие 32 бита результата умножения или остаток от деления\AsteriskThree{}. \\
\hline
n/a             & LO              & младшие 32 бита результата умножения или результат деления\AsteriskThree{}. \\
\hline
\end{tabular}
\end{center}
%\normalsize

\subsubsection{Регистры для работы с числами с плавающей точкой}
\label{MIPS_FPU_registers}

\begin{center}
\begin{tabular}{ | l | l | l | }
\hline
\HeaderColor Название & \HeaderColor Описание \\
\hline
\$F0..\$F1   & Здесь возвращается результат функции. \\
\hline
\$F2..\$F3   & Не используется. \\
\hline
\$F4..\$F11  & Используется для временных данных. \\
\hline
\$F12..\$F15 & Первые два аргумента функции. \\
\hline
\$F16..\$F19 & Используется для временных данных. \\
\hline
\$F20..\$F31 & Используется для временных данных\AsteriskOne{}. \\
\hline
%fcr31 & Control/status register. \\
%\hline
\end{tabular}
\end{center}

\AsteriskOne{} --- \Gls{callee} должен сохранять.\\
\AsteriskTwo{} --- \Gls{callee} должен сохранять (кроме \ac{PIC}-кода).\\
\myindex{MIPS!\Instructions!MFLO}
\myindex{MIPS!\Instructions!MFHI}
\AsteriskThree{} --- доступны используя инструкции \TT{MFHI} и \TT{MFLO}.\\

\subsection{\Instructions}

Есть три типа инструкций.

\begin{itemize}

\item Тип R: имеющие 3 регистра. R-инструкции обычно имеют такой вид:

\begin{lstlisting}
instruction destination, source1, source2
\end{lstlisting}

Важно помнить, что если первый и второй регистр один и тот же, IDA может показать инструкцию в сокращенной
форме:

\begin{lstlisting}
instruction destination/source1, source2
\end{lstlisting}

Это немного напоминает Интеловский синтаксис ассемблера x86.

\item Тип I: имеющие 2 регистра и 16-битное \q{immediate}-значение.

\item Тип J: инструкции перехода, имеют 26 бит для кодирования смещения.

\end{itemize}

\subsubsection{Инструкции перехода}

Какая разница между инструкциями начинающихся с B- (\INS{BEQ}, \INS{B}, итд) и с J- (\INS{JAL}, \INS{JALR}, итд)?

B-инструкции имеют тип I, так что, смещение в этих инструкциях кодируется как 16-битное значение.
Инструкции \INS{JR} и \INS{JALR} имеют тип R, и они делают переход по абсолютному адресу указанному в регистре.
\INS{J} и \INS{JAL} имеют тип J, так что смещение кодируется как 26-битное значение.

Коротко говоря, в B-инструкциях можно кодировать условие 
(\INS{B} на самом деле это псевдоинструкция для \TT{BEQ \$ZERO, \$ZERO, LABEL}),
а в J-инструкциях нельзя.

