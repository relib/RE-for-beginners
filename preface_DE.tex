\section*{Vorwort}

\subsection*{Warum zwei Titel?}
\label{TwoTitles}

Dieses Buch hieß von 2014-2018 ``Reverse Engineering for Beginners'', jedoch hatte ich immer die Befürchtung, dass es den Leserkreis zu sehr einengen würde.

Infosec Leute kennen sich mit ``Reverse Engineering'' aus, jedoch hörte ich selten das Wort ``Assembler'' von Ihnen.

Desweiteren ist der Begriff ``Reverse Engineering'' etwas zu kryptisch für den Großteil von Programmierern, diesen ist jedoch ``Assembler'' geläufig.

Im Juli 2018 änderte ich als Experiment den Titel zu ``Assembly Language for Beginners'' und veröffentlichte den Link auf der Hacker News-Website\footnote{\url{https://news.ycombinator.com/item?id=17549050}}. Das Buch kam allgemein gut an.

Aus diesem Grund hat das Buch nun zwei Titel.

Ich habe den zweiten Titel zu ``Understanding Assembly Language'' geändert, da es bereits eine Erscheinung mit dem Titel ``Assembly Language for Beginners'' gab. Einige Leute sind der Meinung, dass ``for Beginners'' etwas sarkastisch klingt, für ein Buch mit \textasciitilde{}1000 Seiten.

Die beiden Bücher unterscheiden sich lediglich im Titel, dem Dateinamen (UAL-XX.pdf beziehungsweise RE4B-XX.pdf), URL und ein paar der einleitenden Seiten.

\subsection*{Über Reverse Engineering}

Es gibt verschiedene verbreitete Interpretationen des Begriffs Reverse Engineering:\\
1) Reverse Engineering von Software: Rückgewinnung des Quellcodes bereits kompilierter Programme;\\
2) Das Erfassen von 3D Strukturen und die digitalen Manipulationen die zur Duplizierung notwendig sind;\\
3) Nachbilden von \ac{DBMS}-Strukturen.\\
Dieses Buch behandelt die erste Interpretation.

\subsection*{Voraussetzungen}

Grundlegende Kenntnisse der Programmiersprache C.
Empfohlene Literatur: \myref{CCppBooks}.

\subsection*{Übungen und Aufgaben}
\dots 
befinden sich nun alle auf der Website: \url{http://challenges.re}.

\subsection*{Über den Autor}
\begin{tabularx}{\textwidth}{ l X }

\raisebox{-\totalheight}{
\includegraphics[scale=0.60]{Dennis_Yurichev.jpg}
}

&
Dennis Yurichev ist ein erfahrener Reverse Engineer und Programmierer.
Er kann per E-Mail kontaktiert werden: \textbf{\EMAIL{}}.

% FIXME: no link. \tablefootnote doesn't work
\end{tabularx}

% subsections:
\input{praise}
% TBT \input{uni_DE}
\ifdefined\RUSSIAN
\newcommand{\PeopleMistakesInaccuraciesRusEng}{Александр Лысенко, Федерико Рамондино, Марк Уилсон, Разихова Мейрамгуль Кайратовна, Анатолий Прокофьев, Костя Бегунец, Валентин ``netch'' Нечаев, Александр Плахов, Артем Метла, Александр Ястребов, Влад Головкин\footnote{goto-vlad@github}, Евгений Прошин, Александр Мясников, Алексей Третьяков}
\else
\newcommand{\PeopleMistakesInaccuraciesRusEng}{Alexander Lysenko, Federico Ramondino, Mark Wilson, Razikhova Meiramgul Kayratovna, Anatoly Prokofiev, Kostya Begunets, Valentin ``netch'' Nechayev, Aleksandr Plakhov, Artem Metla, Alexander Yastrebov, Vlad Golovkin\footnote{goto-vlad@github}, Evgeny Proshin, Alexander Myasnikov, Alexey Tretiakov}
\fi

\newcommand{\PeopleMistakesInaccuracies}{\PeopleMistakesInaccuraciesRusEng{}, Zhu Ruijin, Changmin Heo, Vitor Vidal, Stijn Crevits, Jean-Gregoire Foulon\footnote{\url{https://github.com/pixjuan}}, Ben L., Etienne Khan, Norbert Szetei\footnote{\url{https://github.com/73696e65}}, Marc Remy, Michael Hansen, Derk Barten, The Renaissance\footnote{\url{https://github.com/TheRenaissance}}, Hugo Chan, Emil Mursalimov, Tanner Hoke, Tan90909090@GitHub, Ole Petter Orhagen, Sourav Punoriyar, Vitor Oliveira, Alexis Ehret, Maxim Shlochiski,
Greg Paton, Pierrick Lebourgeois.}

\newcommand{\PeopleItalianTranslators}{Federico Ramondino\footnote{\url{https://github.com/pinkrab}},
Paolo Stivanin\footnote{\url{https://github.com/paolostivanin}}, twyK, Fabrizio Bertone, Matteo Sticco, Marco Negro\footnote{\url{https://github.com/Internaut401}}, bluepulsar}

\newcommand{\PeopleFrenchTranslators}{Florent Besnard\footnote{\url{https://github.com/besnardf}}, Marc Remy\footnote{\url{https://github.com/mremy}}, Baudouin Landais, Téo Dacquet\footnote{\url{https://github.com/T30rix}}, BlueSkeye@GitHub\footnote{\url{https://github.com/BlueSkeye}}}

\newcommand{\PeopleGermanTranslators}{Dennis Siekmeier\footnote{\url{https://github.com/DSiekmeier}},
Julius Angres\footnote{\url{https://github.com/JAngres}}, Dirk Loser\footnote{\url{https://github.com/PolymathMonkey}}, Clemens Tamme, Philipp Schweinzer}

\newcommand{\PeopleSpanishTranslators}{Diego Boy, Luis Alberto Espinosa Calvo, Fernando Guida, Diogo Mussi, Patricio Galdames,
Emiliano Estevarena}

\newcommand{\PeoplePTBRTranslators}{Thales Stevan de A. Gois, Diogo Mussi, Luiz Filipe, Primo David Santini}

\newcommand{\PeoplePolishTranslators}{Kateryna Rozanova, Aleksander Mistewicz, Wiktoria Lewicka, Marcin Sokołowski}

\newcommand{\PeopleJapaneseTranslators}{%
shmz@github\footnote{\url{https://github.com/shmz}},%
4ryuJP@github\footnote{\url{https://github.com/4ryuJP}}}

\EN{\input{thanks_EN}}
\ES{\input{thanks_ES}}
\NL{\input{thanks_NL}}
\RU{\input{thanks_RU}}
\IT{\input{thanks_IT}}
\FR{\input{thanks_FR}}
\DE{\input{thanks_DE}}
%\CN{\input{thanks_CN}}
\JA{\input{thanks_JA}}
\PL{\subsection*{Podziękowania}

Za cierpliwe odpowiadanie na wszystkie moje pytania: SkullC0DEr.

Za wskazanie błędów i nieścisłości: \PeopleMistakesInaccuracies{}

Za inną pomoc:
Andrew Zubinski,
Arnaud Patard (rtp on \#debian-arm IRC),
noshadow on \#gcc IRC,
Aliaksandr Autayeu,
Mohsen Mostafa Jokar,
Peter Sovietov,
Misha ``tiphareth'' Verbitsky.

Za przetłumaczenie tej książki na język chiński uproszczony:
Antiy Labs (\href{http://antiy.cn}{antiy.cn}), Archer.

Za tłumaczenie na język koreański: Byungho Min.

Za tłumaczenie na język holenderski: Cedric Sambre (AKA Midas).

Za tłumaczenie na język hiszpański: \PeopleSpanishTranslators{}.

Za tłumaczenie na język portugalski: \PeoplePTBRTranslators{}.

Za tłumaczenie na język włoski: \PeopleItalianTranslators{}.

Za tłumaczenie na język francuski: \PeopleFrenchTranslators{}.

Za tłumaczenie na język niemiecki: \PeopleGermanTranslators{}.

Za tłumaczenie na język polski: \PeoplePolishTranslators{}.

Za tłumaczenie na język japoński: \PeopleJapaneseTranslators{}.

Za korektę:
Vladimir Botov,
Andrei Brazhuk,
Mark ``Logxen'' Cooper, Yuan Jochen Kang, Mal Malakov, Lewis Porter, Jarle Thorsen, Hong Xie.

Vasil Kolev\footnote{\url{https://vasil.ludost.net/}} wprowadził wiele poprawek i wskazał sporo błędów.

Dziękuję również wszystkim użytkownikom z github.com za ich komentarze i poprawki.

Użyłem wielu pakietów \LaTeX. Chciałbym podziękować również ich autorom.

\subsubsection*{Darczyńcy}

Tym wszystkim, którzy mnie wspierali w czasie pisania tej książki:

\input{donors}

bardzo dziękuję.
}
\CN{\input{thanks_CN}}


\input{FAQ_DE}

\subsection*{Über die koreanische Übersetzung}

Im Januar 2015 hat die Acorn Publishing Company (\href{http://www.acornpub.co.kr}{www.acornpub.co.kr}) in Süd-Korea
viel Aufwand in die Übersetzung und Veröffentlichung meines Buchs ins Koreanische (mit Stand August 2014) investiert.

Es ist jetzt unter dieser \href{http://go.yurichev.com/17343}{Webseite} verfügbar.

\iffalse
\begin{figure}[H]
\centering
\includegraphics[scale=0.3]{acorn_cover.jpg}
\end{figure}
\fi

Der Übersetzer ist Byungho Min (\href{http://go.yurichev.com/17344}{twitter/tais9}).
Die Cover-Gestaltung wurde von meinem künstlerisch begabten Freund Andy Nechaevsky erstellt:
\href{http://go.yurichev.com/17023}{facebook/andydinka}.
Die Acorn Publishing Company besetzt die Urheberrechte an der koreanischen Übersetzung.

Wenn Sie also ein \emph{echtes} Buch in Ihrem Buchregal auf koreanisch haben und 
mich bei meiner Arbeit unterstützen wollen, können Sie das Buch nun kaufen.

\subsection*{Über die persische Übersetzung (Farsi)}

In 2016 wurde das Buch von Mohsen Mostafa Jokar übersetzt, der in der iranischen Community
auch für die Übersetzung des Radare
Handbuchs\footnote{\url{http://rada.re/get/radare2book-persian.pdf}} bekannt ist).
Es ist auf der Homepage des Verlegers\footnote{\url{http://goo.gl/2Tzx0H}} (Pendare Pars)
verfügbar.

Hier ist ein Link zu einem 40-seitigen Auszug: \url{https://beginners.re/farsi.pdf}.

National Library of Iran registration information: \url{http://opac.nlai.ir/opac-prod/bibliographic/4473995}.

\subsection*{Über die chinesische Übersetzung}

Im April 2017 wurde die chinesische Übersetzung von Chinese PTPress fertiggestellt, bei denen
auch das Copyright liegt.

Die chinesische Version kann hier bestellt werden: \url{http://www.epubit.com.cn/book/details/4174}.
Ein Auszug und die Geschichte der Übersetzung kann hier gefunden werden: \url{http://www.cptoday.cn/news/detail/3155}.

Der Hauptübersetzer ist Archer, dem der Autor sehr iel verdankt.
Archer war extrem akribisch (im positiven Sinne) und meldete
den Großteil der bekannten Fehler, was für Literatur wie dieses Buch extrem wichtig ist.
Der Autor empfiehlt diesen Service jedem anderen Autor!

Die Mitarbeiter von \href{http://www.antiy.net/}{Antiy Labs} halfen ebenfalls bei der Übersetzung.
\href{http://www.epubit.com.cn/book/onlinechapter/51413}{Hier ist das Vorwort} von ihnen.
