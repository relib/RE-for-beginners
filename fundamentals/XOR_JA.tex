\mysection{XOR (排他的論理和)}
\label{XOR_property}

\XOR は、特定のビットを反転するだけの場合に広く使用されています。
確かに、1を適用したXOR演算は実質的に1ビットを反転します。

\input{fundamentals/XOR_table}

逆に、0を適用した \XOR 演算は何もしません。つまり、アイドル演算です。
これは \XOR 演算の非常に重要な特性であり、暗記することを強くお勧めします。


\subsection{Logical difference}

\myindex{Cray}
In Cray-1 supercomputer (1976-1977) manual
\footnote{\url{http://www.bitsavers.org/pdf/cray/CRAY-1/HR-0004-CRAY_1_Hardware_Reference_Manual-PRELIMINARY-1975.OCR.pdf}},
you can find XOR instruction was called \emph{logical difference}.

Indeed, XOR(a,b)=1 if a!=b.

\subsection{日常語}

XOR演算は一般的な日常会話に存在します。
誰かが ``りんごやバナナを買ってください``
と尋ねるとき、これは通常 ``最初の物か二番目の物を買うが両方ではない'' という意味です。
論理ORは ``両方の物も良い'' を意味するからです。

排他的論理和ではなく論理和を使用することを強調するために、
日常会話では ``and/or'' を使用することをお勧めします:\url{https://en.wikipedia.org/wiki/And/or}

\subsection{暗号化}

XORは、(少なくとも\emph{Feistelネットワークでは})アマチュア(\ref{simple_XOR_encryption})と\emph{実際の}暗号化の両方で頻繁に使用されています。

XORは、$cipher\_text = plain\_text \oplus key$ であり、
$(plain\_text \oplus key) \oplus key = plain\_text$
であるため、ここでは非常に便利です。

\subsection{\ac{RAID}4}
\index{RAID4}

\ac{RAID}4は、ハードディスクを保護するための非常に簡単な方法を提供します。
たとえば、複数のディスク($D_1$, $D_2$, $D_3$ など)と1つのパリティディスク($P$)があります。
パリティディスクに書き込まれる各ビット/バイトは、計算されてオンザフライで書き込まれます。

\begin{equation} \label{eq:RAID4}
P = D_1 \oplus D_2 \oplus D_3
\end{equation}

$D_2$など、いずれかのディスクに障害が発生した場合は、まったく同じ方法で復元されます。

\begin{equation}
D_2 = D_1 \oplus P \oplus D_3
\end{equation}

パリティディスクが故障した場合は、\ref{eq:RAID4}の方法を使用して復元されます。
いずれかのディスクが2台故障した場合、両方を復元することは不可能です。

\ac{RAID}5はより高度ですが、このXORプロパティはまだそこで利用されています。

そのため、 \ac{RAID} コントローラには、大量の書き込みデータをオンザフライでXOR処理するためのハードウェア 
''XORアクセラレータ'' があります。コンピュータがどんどん速くなると、 \ac{SIMD} を使ってソフトウェアレベルで実行できるようになります。

\subsection{XORスワップアルゴリズム}

信じがたいことですが、このコードは他の追加のレジスタやメモリセルを使わずに \EAX と \EBX の値を入れ替えます。

\begin{lstlisting}[style=customasmx86]
xor eax, ebx
xor ebx, eax
xor eax, ebx
\end{lstlisting}

それがどのように機能するのか調べてみましょう。
まず、x86アセンブリ言語とは別にステップを書き換えます。

\begin{lstlisting}
X = X XOR Y
Y = Y XOR X
X = X XOR Y
\end{lstlisting}

各ステップでXとYは何を持っていますか?
単純な規則を覚えておいてください:XとYの値はどれでも $(X \oplus Y) \oplus Y = X$

みてみましょう。
最初のステップの後の $X$ には $X \oplus Y$ があります。
2ステップ目以降の $Y$ は $Y \oplus (X \oplus Y) = X$ です。
3ステップ後の $X$ は $(X \oplus Y) \oplus X = Y$ です。

誰かがこのトリックを使うべきかどうかを言うのは難しいですが、それはXORプロパティの良いデモンストレーションの例として役立ちます。

ウィキペディアの記事(\url{https://en.wikipedia.org/wiki/XOR_swap_algorithm})にはまた別の説明があります。
XORの代わりに加算および減算演算を使用できます。

\begin{lstlisting}
X = X + Y
Y = X - Y
X = X - Y
\end{lstlisting}

Let's see:
$X$ after 1st step has $X+Y$;
$Y$ after 2nd step has $X+Y-Y=X$;
$X$ after 3rd step has $X+Y-X=Y$.

\subsection{XORリンクリスト}
\index{Doubly linked list}

二重リンクリストは、各要素が前の要素と次の要素へのリンクを持つリストです。
したがって、リストを前後にトラバースするのはとても簡単です。 
C++の\TT{std::list}は、この本でも検討されている二重リンクリストを実装しています:\ref{std_list}

つまり、各要素には2つのポインタがあります。
おそらく小さい\ac{RAM}の環境では可能でしょうか。フットプリント、2つではなく1つのポインタですべての機能を維持するには?
はい、それが $prev \oplus next$ の値であればこのメモリセルに格納されます。これは通常``link''と呼ばれます。

たぶん、前の要素へのアドレスは次の要素のアドレスを使って「暗号化」され、そうでなければ
次の要素アドレスは前の要素アドレスを使って「暗号化」されると言うことができます。

このリストを前方にたどると、常に前の要素のアドレスがわかるので、このフィールドを「復号化」して
次の要素のアドレスを取得できます。
同様に、このリストを逆方向にたどり、次の要素のアドレスを使用してこのフィールドを「復号化」することもできます。

ただし、最初の要素のアドレスがわからないと、特定の要素の前後の要素のアドレスを
見つけることはできません。

この解決策を完成させるためのいくつかのこと:最初の要素はXORを行わずに次の要素のアドレスを持ち、
最後の要素はXORを行わずに前の要素のアドレスを持つようになります。

それでは要約しましょう。これは5要素の二重リンクリストの例です。 
$A_x$は要素のアドレスです。

\begin{center}
\begin{tabular}{ | l | l | }
	\hline
	\HeaderColor アドレス & \HeaderColor \emph{link} フィールドの中身 \\
	\hline
	$A_0$ & $A_1$ \\
	\hline
	$A_1$ & $A_0 \oplus A_2$ \\
	\hline
	$A_2$ & $A_1 \oplus A_3$ \\
	\hline
	$A_3$ & $A_2 \oplus A_4$ \\
	\hline
	$A_4$ & $A_3$ \\
	\hline
\end{tabular}
\end{center}

繰り返しになりますが、このトリッキーなハックを誰かが使用する必要があるかどうかは言い難いですが、これもXORプロパティの優れたデモンストレーションです。
XORスワップアルゴリズムと同様に、それに関するウィキペディアの記事でもXORの代わりに加算または減算を使用する方法を提供しています:
\url{https://en.wikipedia.org/wiki/XOR_linked_list}

% subsection:
\subsection{値の切り替えトリック}

... found in Jorg Arndt --- Matters Computational / Ideas, Algorithms, Source Code
\footnote{https://www.jjj.de/fxt/fxtbook.pdf}.

You want a variable to be switching between 123 and 456.
You may write something like:

\begin{lstlisting}
if (a==123)
    a=456;
else
    a=123;
\end{lstlisting}

But this can be done using a single operation:

\lstinputlisting[style=customc]{fundamentals/XOR_switch.c}

It works because $123 \oplus 123 \oplus 456=0 \oplus 456=456$ and
$456 \oplus 123 \oplus 456=456 \oplus 456 \oplus 123=0 \oplus 123=123$.

One can argue, worth it using or not, especially keeping in mind code readability.
But this is yet another demonstration of XOR properties.


\subsection{Zobrist hashing / tabulation hashing}
\myindex{Chess}
\myindex{Zobrist hashing}
\myindex{Tabulation hashing}

If you work on a chess engine, you traverse a game tree many times per second, and often, you can encounter
the same position, which has already been processed.

So you have to use a method to store already calculated positions somewhere.
But chess position can require a lot of memory, and a hash function would be used instead.

Here is a way to compress a chess position into 64-bit value, called Zobrist hashing:

\begin{lstlisting}[style=customc]
// 8*8ボードと12ピースがあります(白側のために6つ、黒側のために6つ)

uint64_t table[12][8][8]; // 乱数値で満たす

int position[8][8]; // ボード上の各正方形に。 0 - ピースなし 1..12 - ピース

uint64_t hash;

for (int row=0; row<8; row++)
	for (int col=0; col<8; col++)
	{
		int piece=position[row][col];

		if (piece!=0)
			hash=hash^table[piece][row][col];
	};

return hash;
\end{lstlisting}

Now the most interesting part: if the next (modified) chess position differs only by one (moved) piece,
you don't need to recalculate hash for the whole position, all you need is:

\begin{lstlisting}[style=customc]
hash=...; // (すでに計算済み)

// 古いピースに関する情報を差し引く
hash=hash^table[old_piece][old_row][old_col];

// 新しいピースに関する情報を加える
hash=hash^table[new_piece][new_row][new_col];
\end{lstlisting}


\subsection{ところで}

通常の\emph{OR}は、\emph{排他的OR}とは対照的に、\emph{包含的OR}(または\emph{IOR})と呼ばれることもあります。 
1つの場所は、\emph{operator} Pythonライブラリです。ここでは、\emph{operator.ior}と呼ばれています。
