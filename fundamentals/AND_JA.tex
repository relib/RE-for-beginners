\mysection{AND}

\subsection{値が $2^n$ 境界になっているかをチェックする}

値が$2^n$の数で割り切れるかどうか(1024、4096など)を余りなく確認する必要がある場合は、 
\CCpp で\TT{\%}演算子を使用できますが、もっと簡単な方法があります。
4096は0x1000なので、常に$4*3=12$下位ビットがクリアされています。

必要なのは、

\lstinputlisting[style=customc]{fundamentals/AND_4096_JA.c}

つまり、このコードは下位12ビットの間にビットセットがあるかどうかを確認します。
副作用として、下位12ビットは常に4096による値の除算の余りです($2^n$による除算は単なる右シフトであり、
シフトされた(そしてドロップされた)ビットは余りのビットです)。

数が奇数か偶数かをチェックしたい場合も同じ話です。

\begin{lstlisting}[style=customc]
if (value&1)
	// 奇数
else
	// 偶数
\end{lstlisting}

これは2で割って1ビットの剰余を得るのと同じです。

\subsection{KOI-8R キリル文字のエンコーディング}

8ビット\ac{ASCII}テーブルは、電子メールを含む一部のインターネットサービスではサポートされていませんでした。
サポートされているものもあれば、されないものもあります。

ラテン文字以外の文字を書くシステムが、ラテン文字以外の文字に対応するために8ビットのASCIIテーブルの後半部分を使った時もありました。
いくつかの人気のあるキリル文字のエンコーディングがありましたが、KOI-8R(Andrey ``ache'' Chernovによって考案されたもの)
は他と比べるとややユニークです。

% TODO invert arrow 
% TODO text latex form instead of png!
\begin{figure}[H]
\centering
\includegraphics[width=0.5\textwidth]{fundamentals/koi8r.png}
\caption{KOI8-R テーブル}
\end{figure}

キリル文字はラテン文字とほぼ同じ順序で割り当てられることを誰かが気付くかもしれません。
これは1つの重要な特性につながります:KOI-8Rでエンコードされたキリル文字のテキストのすべての8番目のビットがリセットされるなら、
テキストはキリル文字の代わりにラテン文字で音訳されたテキストに変換します。
たとえば、ロシア語の文は、

\begin{framed}
\begin{quotation}
Мой дядя самых честных правил, Когда не в шутку занемог, Он уважать себя заставил, И лучше выдумать не мог.
\end{quotation}
\end{framed}

\dots KOI-8Rでエンコードされてから8番目のビットが取り除かれた場合、次のように変換されます。

\begin{framed}
\begin{quotation}
mOJ DQDQ SAMYH \^{}ESTNYH PRAWIL, kOGDA NE W [UTKU ZANEMOG, oN UWAVATX SEBQ ZASTAWIL, i LU\^{}[E WYDUMATX NE MOG.
\end{quotation}
\end{framed}

\dots おそらくこれは美的には魅力的ではありませんが、このテキストはロシア語を母語とする人間にはまだ読めます。

したがって、古い7ビットサービスを通過したKOI-8Rでエンコードされたキリル文字のテキストは、音訳されていても
読みやすいテキストになります。

8ビット目を削除すると、(任意の)8ビット\ac{ASCII}テーブルの後半から任意の文字が
最初のテーブルの同じ場所に転置されます(テーブルの右にある赤い矢印を見てください)。
文字が既に前半に配置されている(つまり、標準の7ビット\ac{ASCII}テーブルの)場合、転置されません。

おそらく、文字変換されたように見えた文字に8ビット目を追加すれば、
文字変換されたテキストはまだ回復可能です。

欠点は明らかです。KOI-8Rテーブルに割り当てられたキリル文字は、
ロシア語/ブルガリア語/ウクライナ語などと同じ順序ではありません。たとえば、これはソートには適していません。
