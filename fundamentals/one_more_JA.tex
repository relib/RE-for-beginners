\subsection{2の補数の形式に関する加算の組}

\epigraph{Exercise 2-1. Write a program to determine the ranges of \TT{char}, \TT{short}, \TT{int}, and \TT{long}
variables, both \TT{signed} and \TT{unsigned}, by printing appropriate values from standard headers
and by direct computation.}{\KRBook}

\subsubsection{\gls{word}の最大数を取得する}

符号なし数の最大値はすべてのビットがセットされている数:\emph{0xFF....FF} です。
(\gls{word}が符号付き整数として扱われるの場合は-1になります)
\gls{word}を取って、ビットをすべてセットし、値を取得します:

\begin{lstlisting}[style=customc]
#include <stdio.h>

int main()
{
	unsigned int val=~0; // "unsigned char"に変更して、符号なし8ビットバイトの最大値を取得
	// 0-1も動作する、または単に-1
	printf ("%u\n", val); // %uで符号なし
};
\end{lstlisting}

32ビット整数の場合、これは4294967295です。

\subsubsection{ある符号付 \gls{word} の最小値を取得する}

最小の符号付き整数は\emph{0x80....00}とエンコードされます。つまり、最上位ビットが設定され、その他はクリアされます。
最大の符号付き整数も同じ方法でエンコードされますが、ビットはすべて反転されます:\emph{0x7F....FF}

それが消えるまで少しだけ左にシフトしましょう:

\begin{lstlisting}[style=customc]
#include <stdio.h>

int main()
{
	signed int val=1; // "signed char"に変更して、符号ありバイトの値を見つける
	while (val!=0)
	{
		printf ("%d %d\n", val, ~val);
		val=val<<1;
	};
};
\end{lstlisting}

出力は以下のとおりです。

\begin{lstlisting}
...

536870912 -536870913
1073741824 -1073741825
-2147483648 2147483647
\end{lstlisting}

最後の2つの数字は、それぞれ最小および最大の符号付き32ビット \emph{int}です。
