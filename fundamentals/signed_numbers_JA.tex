\mysection{\SignedNumbersSectionName}
\label{sec:signednumbers}
\myindex{Signed numbers}

\newcommand{\URLS}{\href{http://go.yurichev.com/17117}{wikipedia}}

符号付き数を表現するには方法がいくつかあります。
しかし\q{2の補数}はコンピュータでは最もポピュラーです。

これは、いくつかのバイト値の表です。

\begin{center}
\begin{tabular}{ | l | l | l | l | }
\hline
\HeaderColor 2進数 & \HeaderColor 16進数 & \HeaderColor 符号なし & \HeaderColor 符号あり \\
\hline
01111111 & 0x7f & 127 & 127 \\
\hline
01111110 & 0x7e & 126 & 126 \\
\hline
\multicolumn{4}{ |c| }{...} \\
\hline
00000110 & 0x6 & 6 & 6 \\
\hline
00000101 & 0x5 & 5 & 5 \\
\hline
00000100 & 0x4 & 4 & 4 \\
\hline
00000011 & 0x3 & 3 & 3 \\
\hline
00000010 & 0x2 & 2 & 2 \\
\hline
00000001 & 0x1 & 1 & 1 \\
\hline
00000000 & 0x0 & 0 & 0 \\
\hline
11111111 & 0xff & 255 & -1 \\
\hline
11111110 & 0xfe & 254 & -2 \\
\hline
11111101 & 0xfd & 253 & -3 \\
\hline
11111100 & 0xfc & 252 & -4 \\
\hline
11111011 & 0xfb & 251 & -5 \\
\hline
11111010 & 0xfa & 250 & -6 \\
\hline
\multicolumn{4}{ |c| }{...} \\
\hline
10000010 & 0x82 & 130 & -126 \\
\hline
10000001 & 0x81 & 129 & -127 \\
\hline
10000000 & 0x80 & 128 & -128 \\
\hline
\end{tabular}
\end{center}

\myindex{x86!\Instructions!JA}
\myindex{x86!\Instructions!JB}
\myindex{x86!\Instructions!JL}
\myindex{x86!\Instructions!JG}
符号付き数字と符号なし数字の違いは、\TT{0xFFFFFFFE} と \TT{0x00000002}を表す場合です。 
符号なしの場合、最初の数字(4294967294)は2番目の数字(2)よりも大きいです。 
それらを表すなら 両方とも符号付きで、最初のものは-2になり、2番目のものより小さくなります(2)。 
それが、条件付きジャンプ(~(\myref{sec:Jcc}))は、符号付き( \JG, \JL など)と符号なし(\INS{JA}、 \JB)の
両方の操作に存在する理由です。

簡単にするために、これは知っておくべきことです。

\begin{itemize}
\item 数値は符号付きでも符号なしでもかまいません。

\item \CCpp の符号付き型

  \begin{itemize}
    \item \TT{int64\_t} (-9,223,372,036,854,775,808 .. 9,223,372,036,854,775,807)
	  (-~9.2..~9.2 クィンティリオン) または \\
                \TT{0x8000000000000000..0x7FFFFFFFFFFFFFFF}),
    \item \Tint (-2,147,483,648..2,147,483,647 (-~2.15..~2.15Gb) または \\
	    \TT{0x80000000..0x7FFFFFFF}),
    \item \Tchar (-128..127 または \TT{0x80..0x7F}),
    \item \TT{ssize\_t}.
   \end{itemize}

	符号なし
  \begin{itemize}
	  \item \TT{uint64\_t} (0..18,446,744,073,709,551,615 
		  (~18 quintillions) または \TT{0..0xFFFFFFFFFFFFFFFF}),
   \item \TT{unsigned int} (0..4,294,967,295 (~4.3Gb) または \TT{0..0xFFFFFFFF}),
   \item \TT{unsigned char} (0..255 または \TT{0..0xFF}), 
   \item \TT{size\_t}.
  \end{itemize}

\item 符号付きタイプは \ac{MSB}に 符号があります:1は\q{マイナス}、0は\q{プラス}を意味します。

\item より大きなデータ型に拡張するのは簡単です:
\myref{subsec:sign_extending_32_to_64}

\label{sec:signednumbers:negation}
\item 否定は簡単です。すべてのビットを反転して1を加えるだけです。

反対側のゼロから同じ距離のところに多数の逆符号があることを頭に入れておくことができます。
途中にゼロがあるので、1を加える必要があります。

\myindex{x86!\Instructions!IDIV}
\myindex{x86!\Instructions!DIV}
\myindex{x86!\Instructions!IMUL}
\myindex{x86!\Instructions!MUL}
\myindex{x86!\Instructions!CBW}
\myindex{x86!\Instructions!CWD}
\myindex{x86!\Instructions!CWDE}
\myindex{x86!\Instructions!CDQ}
\myindex{x86!\Instructions!CDQE}
\myindex{x86!\Instructions!MOVSX}
\myindex{x86!\Instructions!SAR}
\item 
	加算演算と減算演算は、符号付き値と符号なし値の両方に適しています。
	しかし、チップ分割と除算の操作、x86には異なる指示があります。
	\TT{IDIV}/\TT{IMUL}は符号付き、\TT{DIV}/\TT{MUL}は符号なし用です。
\item
	符号付き番号を使用したその他の指示は次のとおりです。
	\TT{CBW/CWD/CWDE/CDQ/CDQE} (\myref{ins:CBW_CWD_etc})、 \TT{MOVSX} (\myref{MOVSX})、 \TT{SAR} (\myref{ins:SAR})
\end{itemize}

いくつかの負および正の値(\ref{signed_tbl})の表は、摂氏スケールの温度計のように見えます。
これは、加算と減算が符号付きと符号なしの両方の数値で同じように機能する理由です。
最初の加数が温度計のマークとして表され、2番目の加数を追加する必要がある場合は
正です。 
2回目の追加2番目の加数が負の場合は、2番目の加数の絶対値までマークを下げます。

2つの負数の加算は次のように機能します。
たとえば、16ビットレジスタを使用して-2と-3を追加する必要があります。 
-2と-3は、それぞれ0xfffeと0xfffdです。
これらの数字を符号なしとして追加すると、0xfffe+0xfffd=0x1fffbになります。
しかし、16ビットレジスタを扱うので、結果は\emph{切り捨て}られ、最初の1はドロップされ、
0xfffbは残され、これは-5です。
これは、-2(または0xfffe)は、次のように平易な英語で表現できるためです。
``2は、16ビットレジスタの最大値+ 1までの値に欠けています''。
-3は、``\dots…3まではこの値が足りない''と表すことができます。 
16ビットレジスタ+1の最大値は0x10000です。 
2つの数の足し算と$2^{16}$の法による\emph{カットオフ}の間、$2+3=5$は\emph{欠けます}。

% subsections:
\subsection{MULよりもIMULを使用する}
\label{IMUL_over_MUL}

\myindex{x86!\Instructions!MUL}
\myindex{x86!\Instructions!IMUL}
2つの符号なしの値が乗算される\lstref{unsigned_multiply_C}のような例では、 \MUL の代わりに \IMUL が使用される\lstref{unsigned_multiply_lst}にコンパイルされます。

これは \MUL 命令と \IMUL 命令の両方の重要な特性です。
まず、2つの32ビット値を乗算すると64ビット値が生成され、2つの64ビット値を乗算すると128ビット値が生成されます
(32ビット環境で可能な最大\gls{product}は\GTT{0xffffffff*0xffffffff=0xfffffffe00000001})。
しかし、 \CCpp 標準では結果の上位半分にアクセスすることはできず、\gls{product}は常に被乗数と同じサイズになります。% TODO \gls{}?
上位半分が無視された場合、 \MUL 命令と \IMUL 命令の両方が同じように機能します。つまり、両方とも
同じ下位半分を生成します。
これは、符号付き数字を表す\q{2の補数}の方法の重要な特性です。

そのため、\CCpp コンパイラはこれらの命令をどれでも使用できます。

\MUL は \AX/\EAX/\RAX レジスタに格納された被乗数の1つを必要としますが、 \IMUL は \MUL よりも汎用性があります。
それ以上に:MULは32ビット環境では\GTT{EDX:EAX}ペア、64ビット環境では\GTT{RDX:RAX}ペアを格納するので、結果全体を常に計算します。
それどころか、ペアではなくIMULを使用している間に単一のデスティネーションレジスタを設定し、次に\ac{CPU}を設定することは可能です。より低い半分だけを計算するでしょう、それはより速く働きます([Torborn Granlundの\emph{Instruction latencies and throughput for AMD and Intel x86 processors}\footnote{\url{http://yurichev.com/mirrors/x86-timing.pdf}]}を参照)。

以上より、 \CCpp コンパイラは \MUL よりも頻繁に \IMUL 命令を生成する可能性があります。

\myindex{Compiler intrinsic}
それにもかかわらず、コンパイラ組み込み関数を使用して、符号なし乗算を実行して\emph{完全}な結果を得ることは依然として可能です。
これは\emph{拡張乗算}とも呼ばれます。 
MSVCには、これに\emph{\_\_emul}\footnote{\url{https://msdn.microsoft.com/en-us/library/d2s81xt0(v=vs.80).aspx}}と呼ばれる組み込み関数と、もう1つ:
\emph{\_umul128}\footnote{\url{https://msdn.microsoft.com/library/3dayytw9%28v=vs.100%29.aspx}} があります。 
GCCは\emph{\_\_int128}データ型を提供しており、64ビットの被乗数が最初に128ビットの1に昇格されると、
\gls{product}は別の\emph{\_\_int128}値に格納され、結果は64ビット右にシフトされ、
結果の上位半分が得られます\footnote{例: \url{http://stackoverflow.com/a/13187798}}。

\subsubsection{WindowsでのMulDiv()関数}
\myindex{Windows!Win32!MulDiv()}

WindowsにはMulDiv()関数
\footnote{\url{https://msdn.microsoft.com/en-us/library/windows/desktop/aa383718(v=vs.85).aspx}}、
融合乗算/除算関数があり、2つの32ビット整数を中間の64ビット値に乗算し、
それを3番目の32ビット整数で除算します。 
2つのコンパイラ組み込み関数を使用するよりも簡単なので、Microsoftの開発者はそのための特別な関数を作りました。 
そして、これはその使用法から判断すると、忙しい機能です。

\subsection{2の補数の形式に関する加算の組}

\epigraph{Exercise 2-1. Write a program to determine the ranges of \TT{char}, \TT{short}, \TT{int}, and \TT{long}
variables, both \TT{signed} and \TT{unsigned}, by printing appropriate values from standard headers
and by direct computation.}{\KRBook}

\subsubsection{\gls{word}の最大数を取得する}

符号なし数の最大値はすべてのビットがセットされている数:\emph{0xFF....FF} です。
(\gls{word}が符号付き整数として扱われるの場合は-1になります)
\gls{word}を取って、ビットをすべてセットし、値を取得します:

\begin{lstlisting}[style=customc]
#include <stdio.h>

int main()
{
	unsigned int val=~0; // "unsigned char"に変更して、符号なし8ビットバイトの最大値を取得
	// 0-1も動作する、または単に-1
	printf ("%u\n", val); // %uで符号なし
};
\end{lstlisting}

32ビット整数の場合、これは4294967295です。

\subsubsection{ある符号付 \gls{word} の最小値を取得する}

最小の符号付き整数は\emph{0x80....00}とエンコードされます。つまり、最上位ビットが設定され、その他はクリアされます。
最大の符号付き整数も同じ方法でエンコードされますが、ビットはすべて反転されます:\emph{0x7F....FF}

それが消えるまで少しだけ左にシフトしましょう:

\begin{lstlisting}[style=customc]
#include <stdio.h>

int main()
{
	signed int val=1; // "signed char"に変更して、符号ありバイトの値を見つける
	while (val!=0)
	{
		printf ("%d %d\n", val, ~val);
		val=val<<1;
	};
};
\end{lstlisting}

出力は以下のとおりです。

\begin{lstlisting}
...

536870912 -536870913
1073741824 -1073741825
-2147483648 2147483647
\end{lstlisting}

最後の2つの数字は、それぞれ最小および最大の符号付き32ビット \emph{int}です。


