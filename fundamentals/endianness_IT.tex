\mysection{Endianness}
\label{sec:endianness}

Con il termine endianness si intende un modo di rappresentare i dati in memoria.

\subsection{Big-endian}

Nelle architetture big-endian il valore \TT{0x12345678} viene rappresentato in memoria come:

\begin{center}
\begin{tabular}{ | l | l | }
\hline
\HeaderColor Indirizzo in memoria & \HeaderColor Valore in byte \\
\hline
+0 & 0x12 \\
\hline
+1 & 0x34 \\
\hline
+2 & 0x56 \\
\hline
+3 & 0x78 \\
\hline
\end{tabular}
\end{center}

Tra le CPU big-endian ricordiamo Motorola 68k e IBM POWER.

\subsection{Little-endian}

Nelle architetture little-endian il valore \TT{0x12345678} viene rappresentato in memoria come:

\begin{center}
\begin{tabular}{ | l | l | }
\hline
\HeaderColor Indirizzo in memoria & \HeaderColor Valore in byte \\
\hline
+0 & 0x78 \\
\hline
+1 & 0x56 \\
\hline
+2 & 0x34 \\
\hline
+3 & 0x12 \\
\hline
\end{tabular}
\end{center}

Tra le CPU aventi architettura little-endian ricordiamo l'Intel x86.
Un importante esempio di utilizzo di endianness little-endian si trova in questo libro:
\myref{{LargeInteger}}.

\subsection{\Example}

Si consideri un sistema Linux installato su un'architettura MIPS big-endian e disponibile in QEMU
\footnote{Scaricabile qui: \url{http://go.yurichev.com/17008}}.

E si compili il seguente semplice esempio:

\begin{lstlisting}[style=customc]
#include <stdio.h>

int main()
{
	int v;

	v=123;

	printf ("%02X %02X %02X %02X\n", 
		*(char*)&v,
		*(((char*)&v)+1),
		*(((char*)&v)+2),
		*(((char*)&v)+3));
};
\end{lstlisting}

Eseguendolo si ottiene:

\begin{lstlisting}
root@debian-mips:~# ./a.out 
00 00 00 7B
\end{lstlisting}

Il risultato è
0x7B, ovvero 123 in decimale.
Nelle architetture little-endian il valore 7B verrà memorizzato nel primo byte di memoria (è possibile verificarlo su x86 o x86-64), 
mentre in questo caso viene memorizzato nell'ultimo byte, poiché il byte più alto viene memorizzato per primo.

Ecco perché esistono diverse distribuzioni Linux per architetture MIPS
(\q{mips} (big-endian) e \q{mipsel} (little-endian)).
Non è possibile che un file binario compilato per una specifica endianness possa essere eseguito su un \ac{OS} avente diversa endianness.

Si può trovare un altro esempio di architettura MIPS big-endian in questo libro: \myref{MIPS_structure_big_endian}.

\subsection{Bi-endian}

Esistono poi alcune CPU che possono cambiare tra diverse endianness. Tra queste vi sono ARM, PowerPC, SPARC, MIPS, \ac{IA64}, etc.

\subsection{Converting data}

\myindex{x86!\Instructions!BSWAP}
L'istruzione \TT{BSWAP} può essere usata per la conversione di endianness.

\myindex{TCP/IP}
I pacchetti di rete TCP/IP usano la convenzione big-endian quindi un programma che lavora su un'architettura little-endian deve convertire i dati ricevuti.
Per questo motivo le funzioni \TT{htonl()} e \TT{htons()} sono tipicamente usate per questo scopo.

Nel mondo TCP/IP il big-endian viene anche chiamato \q{network byte order}, mentre il byte order sul computer viene chiamato \q{host byte order}.
L'\q{host byte order} è little-endian sulle architetture Intel x86 e altre architetture little-endian,
ma è big-endian su architettura IBM POWER, quindi le funzioni \TT{htonl()} e \TT{htons()} non cambiano l'ordinamento dei byte in quest'ultima.

