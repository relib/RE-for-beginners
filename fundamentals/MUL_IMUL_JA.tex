\subsection{MULよりもIMULを使用する}
\label{IMUL_over_MUL}

\myindex{x86!\Instructions!MUL}
\myindex{x86!\Instructions!IMUL}
2つの符号なしの値が乗算される\lstref{unsigned_multiply_C}のような例では、 \MUL の代わりに \IMUL が使用される\lstref{unsigned_multiply_lst}にコンパイルされます。

これは \MUL 命令と \IMUL 命令の両方の重要な特性です。
まず、2つの32ビット値を乗算すると64ビット値が生成され、2つの64ビット値を乗算すると128ビット値が生成されます
(32ビット環境で可能な最大\gls{product}は\GTT{0xffffffff*0xffffffff=0xfffffffe00000001})。
しかし、 \CCpp 標準では結果の上位半分にアクセスすることはできず、\gls{product}は常に被乗数と同じサイズになります。% TODO \gls{}?
上位半分が無視された場合、 \MUL 命令と \IMUL 命令の両方が同じように機能します。つまり、両方とも
同じ下位半分を生成します。
これは、符号付き数字を表す\q{2の補数}の方法の重要な特性です。

そのため、\CCpp コンパイラはこれらの命令をどれでも使用できます。

\MUL は \AX/\EAX/\RAX レジスタに格納された被乗数の1つを必要としますが、 \IMUL は \MUL よりも汎用性があります。
それ以上に:MULは32ビット環境では\GTT{EDX:EAX}ペア、64ビット環境では\GTT{RDX:RAX}ペアを格納するので、結果全体を常に計算します。
それどころか、ペアではなくIMULを使用している間に単一のデスティネーションレジスタを設定し、次に\ac{CPU}を設定することは可能です。より低い半分だけを計算するでしょう、それはより速く働きます([Torborn Granlundの\emph{Instruction latencies and throughput for AMD and Intel x86 processors}\footnote{\url{http://yurichev.com/mirrors/x86-timing.pdf}]}を参照)。

以上より、 \CCpp コンパイラは \MUL よりも頻繁に \IMUL 命令を生成する可能性があります。

\myindex{Compiler intrinsic}
それにもかかわらず、コンパイラ組み込み関数を使用して、符号なし乗算を実行して\emph{完全}な結果を得ることは依然として可能です。
これは\emph{拡張乗算}とも呼ばれます。 
MSVCには、これに\emph{\_\_emul}\footnote{\url{https://msdn.microsoft.com/en-us/library/d2s81xt0(v=vs.80).aspx}}と呼ばれる組み込み関数と、もう1つ:
\emph{\_umul128}\footnote{\url{https://msdn.microsoft.com/library/3dayytw9%28v=vs.100%29.aspx}} があります。 
GCCは\emph{\_\_int128}データ型を提供しており、64ビットの被乗数が最初に128ビットの1に昇格されると、
\gls{product}は別の\emph{\_\_int128}値に格納され、結果は64ビット右にシフトされ、
結果の上位半分が得られます\footnote{例: \url{http://stackoverflow.com/a/13187798}}。

\subsubsection{WindowsでのMulDiv()関数}
\myindex{Windows!Win32!MulDiv()}

WindowsにはMulDiv()関数
\footnote{\url{https://msdn.microsoft.com/en-us/library/windows/desktop/aa383718(v=vs.85).aspx}}、
融合乗算/除算関数があり、2つの32ビット整数を中間の64ビット値に乗算し、
それを3番目の32ビット整数で除算します。 
2つのコンパイラ組み込み関数を使用するよりも簡単なので、Microsoftの開発者はそのための特別な関数を作りました。 
そして、これはその使用法から判断すると、忙しい機能です。
