% To translators: don't bother to translate this... english-only version.

\begin{center}
\LARGE{} Это моя собственная доска объявления \normalsize{}
\end{center}

\textbf{Обучаю (пока) бесплатно}

В качестве эксперимента, я могу попытаться учить программированию (но не reverse engineering-у и не инфобезу). С сильным geek-овским уклоном в теоретическую computer science и (немного) дискретную математику, ну и всё, что рядом. Вне зависимости от возраста, образования и бэкграунда. Пока что бесплатно (в режиме демо-версии), в моё свободнее время, удаленно.

-> \url{https://yurichev.com/news/20200405_teaching/}

\myhrule{}

\textbf{Эта книга наверняка уже устарела}.
(Если только не была скачана прямо сейчас с \url{https://beginners.re/}.)

Книга \href{\RepoURL/ChangeLog}{меняется очень часто},
контент добавляется, ошибки (будем надеяться) исправляются.
Также, в первую очередь книга пишется на английском, а перевод на русский немного запаздывает.
Последняя версия всегда на \url{https://beginners.re/}.

А PDF-файл, который вы сейчас читаете, был скомпилирован \today{}.

\myhrule{}

Если вы распечатали эту книгу на бумаге, не могли бы вы прислать мне её фотографию, для коллекции?\\
\EMAILS{}.
Сама коллекция, пока что: \url{https://yurichev.com/news/20200222_printed_RE4B/}.

\myhrule{}

Мои дорогие читатели! Время от времени, у меня появляются вопросы, и я не знаю, кого (или где) спросить.
Или я просто ленив...
Поможете мне?

\myhrule{}

Есть винчестер на 2TB, который непрерывно тарахтит.
Даже когда OS не загружена!
Что сводит меня с ума.
Странная вещь -- когда в SMART включаю режим тестирования "long" \\
(\verb|smartctl --test=long /dev/sda|),
его почти не слышно, но на ~5 часов.
Когда тест заканчивается, винт снова очень шумный.
В SMART-инфе вроде всё нормально.
Винт новый.
Может это какой-то maintenance, который надо просто переждать?
В SMART-инфе ничего подозрительного: \url{https://pastebin.com/5eFePXgG}.

\myhrule{}

Есть у меня какие-нибудь знакомые в Украине, покупающие Биткоины в обмен на наличные доллары-евро?

\myhrule{}

Есть большой граф, например, миллион узлов (вершин).
Нужно его визуализировать как-то, чтобы пользователь мог ходить по графу при помощи мыши.
Нажал на линк (ребро), переместился на другой узел (вершину).
Вот примерно как в IDA.
Может быть, при помощи JavaScript.
Есть какие-то опенсорсные готовые решения?

\myhrule{}

Помните ли вы игру ``The Incredible Machine'' под DOS?
Знаете ли о машинах Руби Голдберга?
Что в наше время можно использовать для симуляции оных?
Может быть, какой-нибудь физический движок?

\myhrule{}

Где можно накачать баз и телефонных справочников, которые используются на сайте \url{http://nomerorg.website/}?

\myhrule{}

Кто-нибудь может помочь мне с Low Fragmentation Heap в Windows?

\myhrule{}

Соц.опрос: как вы используете DLL injection кроме как для перехвата вызовов API?

\myhrule{}

Блютузовые наушники ERGO BT-590 имеют сенсорные кнопки, слишком чувствительные, и их легко задеть одеждой.
Как в Андроиде сделать так, чтобы Андроид игнорировал сообщения от наушников о нажатии кнопок?

\myhrule{}

Если знаете что-то, пожалуйста помогите мне: \EMAILS{}.

