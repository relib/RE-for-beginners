\mysection{\oracle: fichiers .MSB-files}
\myindex{\oracle}
\epigraph{When working toward the solution of a problem, it always helps if you know the answer.}{Murphy's Laws, Rule of Accuracy}

Ceci est un fichier binaire qui contient les messages d'erreur avec leur numéro
correspondant.
Essayons de comprendre son format et de trouver un moyen de les extraire.

Il y a des fichiers de messages d'erreur d'\oracle au format texte,
donc nous pouvons comparer le texte et les fichiers binaires paqués\footnote{Les
fichiers texte open-source n'existent pas dans \oracle pour chaque fichier .MSB,
c'est donc pourquoi nous allons travailler sur leur format de fichier}.

Ceci est le début du fichier texte ORAUS.MSG avec des commentaires non pertinents
supprimés:

\begin{lstlisting}[caption=Beginning of ORAUS.MSG file without comments]
00000, 00000, "normal, successful completion"
00001, 00000, "unique constraint (%s.%s) violated"
00017, 00000, "session requested to set trace event"
00018, 00000, "maximum number of sessions exceeded"
00019, 00000, "maximum number of session licenses exceeded"
00020, 00000, "maximum number of processes (%s) exceeded"
00021, 00000, "session attached to some other process; cannot switch session"
00022, 00000, "invalid session ID; access denied"
00023, 00000, "session references process private memory; cannot detach session"
00024, 00000, "logins from more than one process not allowed in single-process mode"
00025, 00000, "failed to allocate %s"
00026, 00000, "missing or invalid session ID"
00027, 00000, "cannot kill current session"
00028, 00000, "your session has been killed"
00029, 00000, "session is not a user session"
00030, 00000, "User session ID does not exist."
00031, 00000, "session marked for kill"
...
\end{lstlisting}

Le premier nombre est le code d'erreur.
Le second contient peut-être des flags supplémentaires.

\clearpage
Maintenant ouvrons le fichier binaire ORAUS.MSB et trouvons ces chaînes de teste.
Et elles y sont:

\begin{figure}[H]
\centering
\myincludegraphics{ff/Oracle_MSB/1.png}
\caption{Hiew: first block}
\label{fig:oracle_MSB_1}
\end{figure}

Nous voyons les chaînes de texte (celles du début du fichier ORAUS.MSG inclues)
imbriquées avec d'autres sortes de valeurs binaire.
Avec un examen rapide, nous pouvons voir que la partie principale du fichier
binaire est divisée en bloc de taille 0x200 (512) octets.

\clearpage
Regardons le contenu du premier bloc:

\begin{figure}[H]
\centering
\myincludegraphics{ff/Oracle_MSB/2.png}
\caption{Hiew: first block}
\label{fig:oracle_MSB_2}
\end{figure}

Ici nous voyons les textes des premiers messages d'erreur.
Ce que nous voyons aussi, c'est qu'il n'y a pas d'octet à zéro entre les messages d'erreur.
Ceci implique que ce ne sont pas des chaînes C terminées par null.
Par conséquent, la longueur de chaque message d'erreur doit être encodée
d'une façon ou d'une autre.
Essayons aussi de trouver le numéro d'erreur.
Le fichier ORAUS.MSG débute par ceci:
0, 1, 17 (0x11), 18 (0x12), 19 (0x13), 20 (0x14), 21 (0x15), 22 (0x16), 23 (0x17), 24 (0x18)...
Nous allons trouver ces nombres au début du bloc et les marquer avec des lignes
rouge.
La période entre les codes d'erreur est de 6 octets.

Ceci implique qu'il y a probablement 6 octets d'information alloués pour chaque
message d'erreur.

La première valeur 16-bit (0xA ici ou 10) indique le nombre de messages dans chaque
bloc: ceci peut être vérifié en examinant d'autres blocs.
En effet: les messages d'erreur ont une taille arbitraire.
Certains sont plus long, d'autres plus court.
Mais la taille du bloc est toujours fixée, ainsi, vous ne savez jamais combien
de messages d'erreur sont stockés dans chaque bloc.

Comme nous l'avons déjà noté, puisqu'il ne s'agit pas de chaîne C terminée par null,
leur taille doit être encodée quelque part.
La taille de la première chaîne \q{normal, successful completion} est de 29
(0x1D) octets.
La taille de la seconde chaîne \q{unique constraint (\%s.\%s) violated} est de 34
(0x22) octets.
Nous ne trouvons pas ces valeurs (0x1D ou/et 0x22) dans le bloc.%

Il y a aussi une autre chose.
\oracle 
doit déterminer la position de la chaîne qu'il y a besoin de charger dans le bloc, exact?
La première chaîne \q{normal, successful completion} débute à la position 0x1444
(si nous comptons depuis le début du fichier) ou en 0x44 (depuis le début du bloc).
La seconde chaîne \q{unique constraint (\%s.\%s) violated} débute à la position
0x1461 (depuis le début du fichier) ou en 0x61 (depuis le début du bloc).
Ces nombres nous sont quelque peu familier!
Nous pouvons clairement les voir au début du bloc.

Donc, chaque bloc de 6 octets est:

\begin{itemize}
\item 16-bit numéro d'erreur;
\item 16-bit à zéro (peut-être des flags additionnels);
\item position du début de la chaîne de texte dans le bloc courant.
\end{itemize}

Nous pouvons rapidement vérifier les autres valeurs et être sûr que notre supposition
est correcte.
Et il y a aussi le dernier bloc \q{factice} de 6 octets avec un numéro d'erreur à
zéro et débutant après le dernier caractère du dernier message d'erreur.
Peut-être est-ce ainsi que la longueur du texte du message est déterminée?
Nous énumérons juste les blocs de 6 octets pour trouver le numéro d'erreur
dont nous avons besoin, puis nous obtenons la position de la chaîne de texte, puis
la longueur de la chaîne de texte en cherchant le bloc de 6 octets suivant!
De cette façon nous déterminons les limites de la chaîne!
Cette méthode nous permet d'économiser un peu d'espace en ne sauvegardant pas
la taille de la chaîne de texte dans le fichier!

Il n'est pas possible de dire si ça sauve beaucoup d'espace, mais c'est un truc
astucieux.

\clearpage
Revenons à l'entête de fichier .MSB:

\begin{figure}[H]
\centering
\myincludegraphics{ff/Oracle_MSB/3.png}
\caption{Hiew: entête de fichier}
\label{fig:oracle_MSB_3}
\end{figure}

Maintenant nous pouvons trouver rapidement le nombre de blocs dans le fichier
(marqué en rouge).
Nous pouvons vérifier d'autres fichiers .MSB et nous voyons que c'est vrai pour
chacun d'entre eux.

Il y a de nombreuses autres valeurs, mais nous ne voulons pas les examiner,
puisque notre job (un utilitaire d'extraction) est fait.

Si nous devions écrire un générateur de fichier .MSB, nous devrions probablement
comprendre la signification des autres valeurs.

\clearpage
Il y a aussi une table qui vient après l'entête, qui contient probablement des
valeurs 16-bit:

\begin{figure}[H]
\centering
\myincludegraphics{ff/Oracle_MSB/4.png}
\caption{Hiew: table last\_errnos}
\label{fig:oracle_MSB_4}
\end{figure}

Leurs tailles peuvent être déterminées visuellement (les lignes rouges sont
dessinées ici).

En regardant ces valeurs, nous avons trouvé que chaque nombre 16-bit est le dernier
code d'erreur pour chaque bloc.

C'est donc ainsi qu'\oracle trouve rapidement le message d'erreur:

\begin{itemize}
\item charger la table que nous appellerons last\_errnos (qui contient le dernier
numéro d'erreur pour chaque bloc);

\item trouver un bloc qui contient le numéro d'erreur que nous cherchons, en assumant
que les codes d'erreur augmentent dans les blocs et aussi dans le fichier;

\item charger le bloc spécifique;

\item énumérer les structures de 6 octets jusqu'à trouver le numéro d'erreur;

\item obtenir la position du premier du bloc de 6 octets courant;

\item obtenir la position du dernier caractère du bloc de 6 octets suivant;

\item charger tous les caractères du message dans cet intervalle.
\end{itemize}

Ceci est un programme C que nous avons écrit qui extrait les fichiers .MSB:
\href{http://go.yurichev.com/17213}{beginners.re}.

Voici aussi les deux fichiers que nous avons utilisé dans l'exemple
(\oracle 11.1.0.6):
\href{http://go.yurichev.com/17214}{beginners.re},
\href{http://go.yurichev.com/17215}{beginners.re}.

\subsection{Résumé}

Cette méthode est probablement désuète pour les ordinateurs moderne.
Je suppose que ce format de fichier a été développé au milieu des années 80 par
quelqu'un qui a aussi codé pour les \emph{big iron}\footnote{NDT: Gros ordinateur
de type mainframe.} en ayant à l'esprit l'économie d'espace et de mémoire.
Néanmoins, ça a été une tâche intéressante mais facile de comprendre
un format de fichier propriétaire sans regarder dans le code d'\oracle.
