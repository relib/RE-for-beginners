\subsection{Правила де Моргана и декомпиляция}
\myindex{Правила де Моргана}

Иногда оптимизатор компилятора может использовать правила де Моргана чтобы сделать код короче/быстрее.

Например, это:

\begin{lstlisting}[style=customc]
void f(int a, int b, int c, int d)
{
	if (a>0 && b>0)
		printf ("§оба a и b больше нуля§\n");
	else if (c>0 && d>0)
		printf ("§оба c и d больше нуля§\n");
	else
		printf ("§что-то еще§\n");
};
\end{lstlisting}

... выглядит очень невинно, когда компилируется GCC 5.4.0 x64:

\begin{lstlisting}[style=customasmx86]
; \verb|int __fastcall f(int a, int b, int c, int d)|
                public f
f               proc near
                test    edi, edi
                jle     short loc_8
                test    esi, esi
                jg      short loc_30

loc_8:
                test    edx, edx
                jle     short loc_20
                test    ecx, ecx
                jle     short loc_20
                mov     edi, offset s   ; "оба c и d больше нуля"
                jmp     puts

loc_20:
                mov     edi, offset aSomethingElse ; "что-то еще"
                jmp     puts

loc_30:
                mov     edi, offset aAAndBPositive ; "оба a и b больше нуля"

loc_35:
                jmp     puts
f               endp
\end{lstlisting}

... тоже выглядит невинно, но Hex-Rays 2.2.0 не может видеть, что на самом деле в исходном коде было использовано две операции И:

\begin{lstlisting}[style=customc]
int __fastcall f(int a, int b, int c, int d)
{
  int result;

  if ( a > 0 && b > 0 )
  {
    result = puts("§оба a и b больше нуля§");
  }
  else if ( c <= 0 || d <= 0 )
  {
    result = puts("§что-то еще§");
  }
  else
  {
    result = puts("§оба c и d больше нуля§");
  }
  return result;
}
\end{lstlisting}

Выражение \verb~c <= 0 || d <= 0~ это обратное от \verb|c>0 && d>0|, т.к.,
$\overline{A \cup B} = \overline{A} \cap \overline{B}$ и
$\overline{A \cap B} = \overline{A} \cup \overline{B}$,
Иными словами,
\verb~!(cond1 || cond2) == !cond1 && !cond2~ и 
\verb~!(cond1 && cond2) == !cond1 || !cond2~.

Эти правила полезно держать в голове, потому что эта оптимизация компилятора используется почти везде.

Иногда полезно инвертировать условие, чтобы понять код лучше.
Это фрагмент реального кода декомпилированный при помощи Hex-Rays:

\begin{lstlisting}[style=customc]
	for (int i=0; i<12; i++)
	{
		if (v1[i-12] != 0.0 || v1[i] != 0.0)
		{
			v108=min(v108, (float)v0[i*24 -2]);
			v113=max(v113, (float)v0[i*24]);
		};
	}
\end{lstlisting}

... его можно переписать так:

\begin{lstlisting}[style=customc]
	for (int i=0; i<12; i++)
	{
		if (v1[i-12] == 0.0 && v1[i] == 0.0)
			continue;

		v108=min(v108, (float)v0[i*24 -2]);
		v113=max(v113, (float)v0[i*24]);
	}
\end{lstlisting}

Что лучше? Пока не знаю, но для лучшего понимания, хорошо бы посмотреть на обе версии.

