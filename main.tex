\documentclass[a4paper,oneside]{book}

% http://www.tex.ac.uk/FAQ-noroom.html
\usepackage{etex}

\usepackage[table,usenames,dvipsnames]{xcolor}

\usepackage{fontspec}
% fonts
%\setmonofont{DroidSansMono}
%\setmainfont[Ligatures=TeX]{PT Sans}
%\setmainfont{DroidSans}
\setmainfont{DejaVu Sans}
\setmonofont{DejaVu Sans Mono}
\usepackage{polyglossia}
\defaultfontfeatures{Scale=MatchLowercase} % ensure all fonts have the same 1ex
\usepackage{ucharclasses}
\usepackage{csquotes}

\ifdefined\ENGLISH
\setmainlanguage{english}
\setotherlanguage{russian}
\fi

\ifdefined\RUSSIAN
\setmainlanguage{russian}
%\newfontfamily\cyrillicfont{LiberationSans}
%\newfontfamily\cyrillicfonttt{LiberationMono}
%\newfontfamily\cyrillicfontsf{lmsans10-regular.otf}
\setotherlanguage{english}
\fi

\ifdefined\GERMAN
%\wlog{main GERMAN defined OK}
\setmainlanguage{german}
\setotherlanguage{english}
\fi

\ifdefined\SPANISH
\setmainlanguage{spanish}
\setotherlanguage{english}
\fi

\ifdefined\ITALIAN
\setmainlanguage{italian}
\setotherlanguage{english}
\fi

\ifdefined\BRAZILIAN
\setmainlanguage{portuges}
\setotherlanguage{english}
\fi

\ifdefined\POLISH
\setmainlanguage{polish}
\setotherlanguage{english}
\fi

\ifdefined\DUTCH
\setmainlanguage{dutch}
\setotherlanguage{english}
\fi

\ifdefined\TURKISH
\setmainlanguage{turkish}
\setotherlanguage{english}
\fi

\ifdefined\THAI
\setmainlanguage{thai}
%\usepackage[thai]{babel}
%\usepackage{fonts-tlwg}
\setmainfont[Script=Thai]{TH SarabunPSK}
\newfontfamily{\thaifont}[Script=Thai]{TH SarabunPSK}
\let\thaifonttt\ttfamily
\setotherlanguage{english}
\fi

\ifdefined\FRENCH
\setmainlanguage{french}
\setotherlanguage{english}
\fi

\ifdefined\JAPANESE
\usepackage{xeCJK}
\xeCJKallowbreakbetweenpuncts
\defaultfontfeatures{Ligatures=TeX,Scale=MatchLowercase}
\setCJKmainfont{IPAGothic}
\setCJKsansfont{IPAGothic}
\setCJKmonofont{IPAGothic}
\DeclareQuoteStyle{japanese}
  {「}
  {」}
  {『}
  {』}
\setquotestyle{japanese}
\setmainlanguage{japanese}
\setotherlanguage{english}
\fi

\usepackage{microtype}
\usepackage{fancyhdr}
\usepackage{listings}
\usepackage{ulem}
\usepackage{url}
\usepackage{graphicx}
\usepackage{makeidx}
\usepackage[cm]{fullpage}
%\usepackage{color}
\usepackage{fancyvrb}
\usepackage{xspace}
\usepackage{tabularx}
\usepackage{framed}
\usepackage{parskip}
\usepackage{epigraph}
\usepackage{ccicons}
\usepackage[nottoc]{tocbibind}
\usepackage{longtable}
\usepackage[footnote,printonlyused,withpage]{acronym}
\usepackage[]{bookmark,hyperref} % must be last
\usepackage[official]{eurosym}
\usepackage[usestackEOL]{stackengine}

% ************** myref
% http://tex.stackexchange.com/questions/228286/how-to-mix-ref-and-pageref#228292
\ifdefined\RUSSIAN
\newcommand{\myref}[1]{%
  \ref{#1}
  (стр.~\pageref{#1})%
  }
% FIXME: I wasn't able to force varioref to output russian text...
\else
\usepackage{varioref}
\newcommand{\myref}[1]{\vref{#1}}
\fi
% ************** myref

\usepackage{glossaries}
\usepackage{tikz}
%\usepackage{fixltx2e}
\usepackage{bytefield}

\usepackage{amsmath}
\usepackage{MnSymbol}
\undef\mathdollar

\usepackage{float}

\usepackage{shorttoc}
\usetikzlibrary{calc,positioning,chains,arrows}
\usepackage[margin=0.5in,headheight=15.5pt]{geometry}

%--------------------
% to prevent clashing of numbers and titles in TOC:
% https://tex.stackexchange.com/a/64124
\usepackage{tocloft}% http://ctan.org/pkg/tocloft
\makeatletter
\renewcommand{\numberline}[1]{%
  \@cftbsnum #1\@cftasnum~\@cftasnumb%
}
\makeatother
%--------------------

\newcommand{\footnoteref}[1]{\textsuperscript{\ref{#1}}}

%\definecolor{lstbgcolor}{rgb}{0.94,0.94,0.94}

% I don't know why this voodoo works, but without all-caps, it can't find LIGHT-GRAY color. WTF?
% see also: https://tex.stackexchange.com/questions/64298/error-with-xcolor-package
\definecolor{light-gray}{gray}{0.87}
\definecolor{LIGHT-GRAY}{gray}{0.87}
\definecolor{RED}{rgb}{1,0,0}
\makeindex

\include{macros}
\include{glossary}

\makeglossaries

\hypersetup{
    colorlinks=true,
    allcolors=blue,
    pdfauthor={\AUTHOR},
    pdftitle={\TitleMain}
    }

%\ifdefined\RUSSIAN
\newcommand{\LstStyle}{\ttfamily\small}
%\else
%\newcommand{\LstStyle}{\ttfamily}
%\fi

% inspired by http://prismjs.com/
\definecolor{digits}{RGB}{0,0,0}
\definecolor{bg}{RGB}{255,252,250}
\definecolor{col1}{RGB}{154,20,150}
\definecolor{col2}{RGB}{112,128,144}
\definecolor{col3}{RGB}{10,120,180}
\definecolor{col4}{RGB}{106,164,108}


\lstset{
    %backgroundcolor=\color{lstbgcolor},
    %backgroundcolor=\color{light-gray},
    backgroundcolor=\color{bg},
    basicstyle=\LstStyle,
    breaklines=true,
    %prebreak=\raisebox{0ex}[0ex][0ex]{->},
    %postbreak=\raisebox{0ex}[0ex][0ex]{->},
    prebreak=\raisebox{0ex}[0ex][0ex]{\ensuremath{\rhookswarrow}},
    postbreak=\raisebox{0ex}[0ex][0ex]{\ensuremath{\rcurvearrowse\space}},
    frame=single,
    columns=fullflexible,keepspaces,
    escapeinside=§§,
    inputencoding=utf8
}

% I'm giving up with syntax highlighting so far.
% one problem is: make hexadecimal numbers and keywords distinct
% anyone can try to solve it
% see also: http://tex.stackexchange.com/questions/347784/package-listing-colorizing-assembly-listings

\lstdefinestyle{customc}{
  %language=C,
  %showstringspaces=false,
  %backgroundcolor=\color{bg},
  %keywordstyle=\color{col3},
  %commentstyle=\color{col2},
  %identifierstyle=\color{col4},
  %stringstyle=\color{col1}
}

\lstdefinestyle{custommath}{
  %language=Mathematica,
  %showstringspaces=false,
  %backgroundcolor=\color{bg},
  %keywordstyle=\color{col3},
  %commentstyle=\color{col2},
  %identifierstyle=\color{col4},
  %stringstyle=\color{col1}
}

\lstdefinestyle{custompy}{
  %language=Python,
  %showstringspaces=false,
  %backgroundcolor=\color{bg},
  %keywordstyle=\color{col3},
  %commentstyle=\color{col2},
  %identifierstyle=\color{col4},
  %stringstyle=\color{col1}
}

\lstdefinestyle{customjava}{
  %language=Java,
  %showstringspaces=false,
  %backgroundcolor=\color{bg},
  %keywordstyle=\color{col3},
  %commentstyle=\color{col2},
  %identifierstyle=\color{col4},
  %stringstyle=\color{col1}
}

\lstdefinestyle{customasmx86}{
  %morekeywords={push,mov,sub,call,and,leave,ret,lea,and,retn,xor,add,db,%
%	  eax,ebx,ecx,edx,esi,edi,esp,ebp,eip%
%	  rax,rbx,rcx,rdx,rsi,rdi,rbp,rsp,rip},%
  %alsoletter=.,alsodigit=?,%
  %alsodigit=abcdefhABCDEFhH,%
  %sensitive=f,%
  %morestring=[b]",%
  %morestring=[b]',%
  %morecomment=[l];%
  %showstringspaces=false,
  %basicstyle=\color{red},
  %backgroundcolor=\color{bg},
  %keywordstyle=\color{red},
  %commentstyle=\color{green},
  %identifierstyle=\color{black},
  %identifierstyle=\color{blue},
  %stringstyle=\color{yellow},
}

\lstdefinestyle{customasmARM}{
  %morekeywords={stmfd,mov,adr,bl,ldmfd,push,movs,pop,b,stp,add,adrp,ldp,ret,str,sub,movt,stmfa,movw,mov.w,movt.w,add.w,stmia.w,str.w,blx,%
%		  sp,pc,r0,r1,r2,r3,r4,r5,r6,r7,r8,r9,r10,r11,r12,lr,%
%		  x0,x1,x2,x3,x4,x5,x6,x7,x8,x9,x10,%
%		  w0,w1,w2,w3,w4,w5,w6,w7,w8,w9,w10},
  %alsoletter=.,alsodigit=?,%
  %sensitive=f,%
  %morestring=[b]",%
  %morestring=[b]',%
  %morecomment=[l];%  
  %showstringspaces=false,
  %backgroundcolor=\color{bg},
  %keywordstyle=\color{col3},
  %commentstyle=\color{col2},
  %identifierstyle=\color{black},
  %identifierstyle=\color{col4},
  %stringstyle=\color{col1},
}

\lstdefinestyle{customasmMIPS}{
  %morekeywords={lui,addiu,sw,lw,li,jalr,move,j,la,jr,or,%
  %		\$0,\$1, \$2, \$3, \$4, \$5, \$6, \$7, \$8, \$9, \$10, \$11, \$12, \$13, \$14, \$15, \$16, \$17, \$18, \$19,%
  %		\$20, \$21, \$22, \$23, \$24, \$25, \$26, \$27, \$28, \$29, \$30,\$31,\$sp,\$gp,\$fp,%
%		\%lo,\%hi,
%		},
  %alsoletter=.,alsodigit=?,%
  %sensitive=f,%
  %morestring=[b]",%
  %morestring=[b]',%
  %morecomment=[l];%  
  %showstringspaces=false,
  %backgroundcolor=\color{bg},
  %keywordstyle=\color{col3},
  %commentstyle=\color{col2},
  %identifierstyle=\color{black}, 
  %stringstyle=\color{col1},
}

\lstdefinestyle{customasmPPC}{
  %showstringspaces=false,
}

\ifdefined\RUSSIAN
\renewcommand\lstlistingname{Листинг}
\renewcommand\lstlistlistingname{Листинг}
\fi

\DeclareMathSizes{12}{30}{16}{12}%

% see also:
% http://tex.stackexchange.com/questions/129225/how-can-i-get-get-makeindex-to-ignore-capital-letters
% http://tex.stackexchange.com/questions/18336/correct-sorting-of-index-entries-containing-macros
\def\myindex#1{\expandafter\index\expandafter{#1}}

\begin{document}

% fancyhdr
\pagestyle{fancy}
\setlength{\headheight}{13pt}
\fancyhead[R]{} % suppress chapter name

\VerbatimFootnotes

\frontmatter

%\RU{\include{1st_page_RU}}
\EN{\include{1st_page_EN}}
%\DE{\include{1st_page_DE}}
%\ES{\include{1st_page_ES}}
%\CN{\include{1st_page_CN}}

\RU{\include{dedication_RU}}
\EN{\include{dedication_EN}}
\FR{\include{dedication_FR}}
\JPN{\include{dedication_JPN}}
\DE{\include{dedication_DE}}

\include{page_after_cover}
\include{call_for_translators}

\shorttoc{%
    \RU{Краткое оглавление}%
    \EN{Abridged contents}%
    \ES{Contenidos abreviados}%
    \PTBRph{}%
    \DE{Inhaltsverzeichnis (gekürzt)}%
    \PLph{}%
    \ITAph{}%
    \THAph{}\NLph{}%
    \FR{Contenus abrégés}%
    \JPN{簡略版}
    \TR{İçindekiler}
}{0}

\tableofcontents
\cleardoublepage

\cleardoublepage
\include{preface}

\mainmatter

\include{parts}

\EN{\part*{\RU{Приложение}\EN{Appendix}\DE{Anhang}\FR{Appendice}\JPN{付録}}
\appendix
\addcontentsline{toc}{part}{\RU{Приложение}\EN{Appendix}\DE{Anhang}\FR{Appendice}\JPN{付録}}

% chapters
\EN{\input{appendix/x86/main_EN}}
\RU{\input{appendix/x86/main_RU}}
\DE{\input{appendix/x86/main_DE}}
\FR{\input{appendix/x86/main_FR}}
\JPN{\mysection{x86}

\subsection{Terminology}

Common for 16-bit (8086/80286), 32-bit (80386, etc.), 64-bit.

\myindex{IEEE 754}
\myindex{MS-DOS}
\begin{description}
	\item[byte] 8-bit.
		The DB assembly directive is used for defining variables and arrays of bytes.
		Bytes are passed in the 8-bit part of registers: \TT{AL/BL/CL/DL/AH/BH/CH/DH/SIL/DIL/R*L}.
	\item[word] 16-bit. 
		DW assembly directive \dittoclosing.
		Words are passed in the 16-bit part of the registers:\\
			\TT{AX/BX/CX/DX/SI/DI/R*W}.
	\item[double word] (\q{dword}) 32-bit.
		DD assembly directive \dittoclosing.
		Double words are passed in registers (x86) or in the 32-bit part of registers (x64). 
		In 16-bit code, double words are passed in 16-bit register pairs.
	\item[quad word] (\q{qword}) 64-bit.
		DQ assembly directive \dittoclosing.
		In 32-bit environment, quad words are passed in 32-bit register pairs.
	\item[tbyte] (10 bytes) 80-bit or 10 bytes (used for IEEE 754 FPU registers).
	\item[paragraph] (16 bytes)---term was popular in MS-DOS environment. % TODO link to a part about 8086 memory model...
\end{description}

\myindex{Windows!API}

Data types of the same width (BYTE, WORD, DWORD) are also the same in Windows \ac{API}.

\input{appendix/x86/registers} % subsection
\subsection{\RU{Инструкции}\EN{Instructions}}
\label{sec:x86_instructions}

\RU{Инструкции, отмеченные как (M) обычно не генерируются компилятором: если вы видите её, очень может быть
это вручную написанный фрагмент кода, либо это т.н. compiler intrinsic}
\EN{Instructions marked as (M) are not usually generated by the compiler: if you see one of them, it is probably
a hand-written piece of assembly code, or a compiler intrinsic} (\myref{sec:compiler_intrinsic}).

% TODO ? обратные инструкции

\RU{Только наиболее используемые инструкции перечислены здесь}
\EN{Only the most frequently used instructions are listed here}.
\EN{You can read \myref{x86_manuals} for a full documentation.}%
\RU{Обращайтесь к \myref{x86_manuals} для полной документации.}

\RU{Нужно ли заучивать опкоды инструкций на память?}\EN{Do you have to know all instruction's opcodes by heart?}
\RU{Нет, только те, которые часто используются для модификации кода}\EN{No, only those
which are used for code patching} (\myref{x86_patching}).
\RU{Остальные запоминать нет смысла.}\EN{All the rest of the opcodes don't need to be memorized.}

\subsubsection{\RU{Префиксы}\EN{Prefixes}}

\myindex{x86!\Prefixes!LOCK}
\myindex{x86!\Prefixes!REP}
\myindex{x86!\Prefixes!REPE/REPNE}
\begin{description}
\label{x86_lock}
\item[LOCK] \RU{используется чтобы предоставить эксклюзивный доступ к памяти в многопроцессорной среде}
\EN{forces CPU to make exclusive access to the RAM in multiprocessor environment}.
\RU{Для упрощения, можно сказать, что когда исполняется инструкция с этим префиксом, остальные процессоры
в системе останавливаются}\EN{For the sake of simplification, it can be said that when an instruction
with this prefix is executed, all other CPUs in a multiprocessor system are stopped}.
\RU{Чаще все это используется для критических секций, семафоров, мьютексов}\EN{Most often
it is used for critical sections, semaphores, mutexes}.
\RU{Обычно используется с}\EN{Commonly used with} ADD, AND, BTR, BTS, CMPXCHG, OR, XADD, XOR.
\RU{Читайте больше о критических секциях}\EN{You can read more about critical sections here} (\myref{critical_sections}).

\item[REP] \RU{используется с инструкциями}\EN{is used with the} MOVSx \AndENRU STOSx\EN{ instructions}:
\RU{инструкция будет исполняться в цикле, счетчик расположен в регистре CX/ECX/RCX}
\EN{execute the instruction in a loop, the counter is located in the CX/ECX/RCX register}.
\RU{Для более детального описания, читайте больше об инструкциях}
\EN{For a detailed description, read more about the} MOVSx (\myref{REP_MOVSx}) 
\AndENRU STOSx (\myref{REP_STOSx})\EN{ instructions}.

\RU{Работа инструкций с префиксом REP зависит от флага DF, он задает направление}
\EN{The instructions prefixed by REP are sensitive to the DF flag, which is used to set the direction}.

\item[REPE/REPNE] (\ac{AKA} REPZ/REPNZ) \RU{используется с инструкциями}\EN{used with} CMPSx \AndENRU
SCASx\EN{ instructions}:
\RU{инструкция будет исполняться в цикле, счетчик расположен в регистре \TT{CX}/\TT{ECX}/\TT{RCX}}
\EN{execute the last instruction in a loop, the count is set in the \TT{CX}/\TT{ECX}/\TT{RCX} register}. 
\RU{Выполнение будет прервано если ZF будет 0 (REPE) либо если ZF будет 1 (REPNE)}
\EN{It terminates prematurely if ZF is 0 (REPE) or if ZF is 1 (REPNE)}.

\RU{Для более детального описания, читайте больше об инструкциях}
\EN{For a detailed description, you can read more about the} CMPSx (\myref{REPE_CMPSx}) 
\AndENRU SCASx (\myref{REPNE_SCASx})\EN{ instructions}.

\RU{Работа инструкций с префиксами REPE/REPNE зависит от флага DF, он задает направление}
\EN{Instructions prefixed by REPE/REPNE are sensitive to the DF flag, which is used to set the direction}.

\end{description}

\subsubsection{\RU{Наиболее часто используемые инструкции}\EN{Most frequently used instructions}}

\RU{Их можно заучить в первую очередь}\EN{These can be memorized in the first place}.

\begin{description}
% in order to keep them easily sorted...
\input{appendix/x86/instructions/ADC}
\input{appendix/x86/instructions/ADD}
\input{appendix/x86/instructions/AND}
\input{appendix/x86/instructions/CALL}
\input{appendix/x86/instructions/CMP}
\input{appendix/x86/instructions/DEC}
\input{appendix/x86/instructions/IMUL}
\input{appendix/x86/instructions/INC}
\input{appendix/x86/instructions/JCXZ}
\input{appendix/x86/instructions/JMP}
\input{appendix/x86/instructions/Jcc}
\input{appendix/x86/instructions/LAHF}
\input{appendix/x86/instructions/LEAVE}
\input{appendix/x86/instructions/LEA}
\input{appendix/x86/instructions/MOVSB_W_D_Q}
\input{appendix/x86/instructions/MOVSX}
\input{appendix/x86/instructions/MOVZX}
\input{appendix/x86/instructions/MOV}
\input{appendix/x86/instructions/MUL}
\input{appendix/x86/instructions/NEG}
\input{appendix/x86/instructions/NOP}
\input{appendix/x86/instructions/NOT}
\input{appendix/x86/instructions/OR}
\input{appendix/x86/instructions/POP}
\input{appendix/x86/instructions/PUSH}
\input{appendix/x86/instructions/RET}
\input{appendix/x86/instructions/SAHF}
\input{appendix/x86/instructions/SBB}
\input{appendix/x86/instructions/SCASB_W_D_Q}
\input{appendix/x86/instructions/SHx}
\input{appendix/x86/instructions/SHRD}
\input{appendix/x86/instructions/STOSB_W_D_Q}
\input{appendix/x86/instructions/SUB}
\input{appendix/x86/instructions/TEST}
\input{appendix/x86/instructions/XOR}
\end{description}

\subsubsection{\RU{Реже используемые инструкции}\EN{Less frequently used instructions}}

\begin{description}
\input{appendix/x86/instructions/BSF}
\input{appendix/x86/instructions/BSR}
\input{appendix/x86/instructions/BSWAP}
\input{appendix/x86/instructions/BTC}
\input{appendix/x86/instructions/BTR}
\input{appendix/x86/instructions/BTS}
\input{appendix/x86/instructions/BT}
\input{appendix/x86/instructions/CBW_CWDE_CDQ}
\input{appendix/x86/instructions/CLD}
\input{appendix/x86/instructions/CLI}
\input{appendix/x86/instructions/CMC}
\input{appendix/x86/instructions/CMOVcc}
\input{appendix/x86/instructions/CMPSB_W_D_Q}
\input{appendix/x86/instructions/CPUID}
\input{appendix/x86/instructions/DIV}
\input{appendix/x86/instructions/IDIV}
\input{appendix/x86/instructions/INT}
\input{appendix/x86/instructions/IN}
\input{appendix/x86/instructions/IRET}
\input{appendix/x86/instructions/LOOP}
\input{appendix/x86/instructions/OUT}
\input{appendix/x86/instructions/POPA}
\input{appendix/x86/instructions/POPCNT}
\input{appendix/x86/instructions/POPF}
\input{appendix/x86/instructions/PUSHA}
\input{appendix/x86/instructions/PUSHF}
\input{appendix/x86/instructions/RCx}
\input{appendix/x86/instructions/ROx}
\input{appendix/x86/instructions/SAL}
\input{appendix/x86/instructions/SAR}
\input{appendix/x86/instructions/SETcc}
\input{appendix/x86/instructions/STC}
\input{appendix/x86/instructions/STD}
\input{appendix/x86/instructions/STI}
\input{appendix/x86/instructions/SYSCALL}
\input{appendix/x86/instructions/SYSENTER}
\input{appendix/x86/instructions/UD2}
\input{appendix/x86/instructions/XCHG}
\end{description}

\subsubsection{\RU{Инструкции FPU}\EN{FPU instructions}}

\RU{Суффикс \TT{-R} в названии инструкции обычно означает, что операнды поменяны местами, суффикс \TT{-P} означает
что один элемент выталкивается из стека после исполнения инструкции, суффикс \TT{-PP} означает, что
выталкиваются два элемента}%
\EN{\TT{-R} suffix in the mnemonic usually implies that the operands are reversed,
\TT{-P} suffix implies that one element is popped
from the stack after the instruction's execution, \TT{-PP} suffix implies that two elements are popped}.

\TT{-P} \RU{инструкции часто бывают полезны, когда нам уже больше не нужно хранить значение в 
FPU-стеке после операции.}%
\EN{instructions are often useful when we do not need the value in the FPU stack to be 
present anymore after the operation.}

\begin{description}
\input{appendix/x86/instructions/FABS}
\input{appendix/x86/instructions/FADD} % + FADDP
\input{appendix/x86/instructions/FCHS}
\input{appendix/x86/instructions/FCOM} % + FCOMP + FCOMPP
\input{appendix/x86/instructions/FDIVR} % + FDIVRP
\input{appendix/x86/instructions/FDIV} % + FDIVP
\input{appendix/x86/instructions/FILD}
\input{appendix/x86/instructions/FIST} % + FISTP
\input{appendix/x86/instructions/FLD1}
\input{appendix/x86/instructions/FLDCW}
\input{appendix/x86/instructions/FLDZ}
\input{appendix/x86/instructions/FLD}
\input{appendix/x86/instructions/FMUL} % + FMULP
\input{appendix/x86/instructions/FSINCOS}
\input{appendix/x86/instructions/FSQRT}
\input{appendix/x86/instructions/FSTCW} % + FNSTCW
\input{appendix/x86/instructions/FSTSW} % + FNSTSW
\input{appendix/x86/instructions/FST}
\input{appendix/x86/instructions/FSUBR} % + FSUBRP
\input{appendix/x86/instructions/FSUB} % + FSUBP
\input{appendix/x86/instructions/FUCOM} % + FUCOMP + FUCOMPP
\input{appendix/x86/instructions/FXCH}
\end{description}

%\subsubsection{\RU{SIMD-инструкции}\EN{SIMD instructions}}

% TODO

%\begin{description}
%\input{appendix/x86/instructions/DIVSD}
%\input{appendix/x86/instructions/MOVDQA}
%\input{appendix/x86/instructions/MOVDQU}
%\input{appendix/x86/instructions/PADDD}
%\input{appendix/x86/instructions/PCMPEQB}
%\input{appendix/x86/instructions/PLMULHW}
%\input{appendix/x86/instructions/PLMULLD}
%\input{appendix/x86/instructions/PMOVMSKB}
%\input{appendix/x86/instructions/PXOR}
%\end{description}

% SHLD !
% SHRD !
% BSWAP !
% CMPXCHG
% XADD !
% CMPXCHG8B
% RDTSC !
% PAUSE!

% xsave
% fnclex, fnsave
% movsxd, movaps, wait, sfence, lfence, pushfq
% prefetchw
% REP RETN
% REP BSF
% movnti, movntdq, rdmsr, wrmsr
% ldmxcsr, stmxcsr, invlpg
% swapgs
% movq, movd
% mulsd
% POR
% IRETQ
% pslldq
% psrldq
% cqo, fxrstor, comisd, xrstor, wbinvd, movntq
% fprem
% addsb, subsd, frndint

% rare:
%\item[ENTER]
%\item[LES]
% LDS
% XLAT

\subsubsection{\RU{Инструкции с печатаемым ASCII-опкодом}\EN{Instructions having printable ASCII opcode}}

(\RU{В 32-битном режиме}\EN{In 32-bit mode}).

\label{printable_x86_opcodes}
\myindex{Shellcode}
\RU{Это может пригодиться для создания шеллкодов}\EN{These can be suitable for shellcode construction}.
\RU{См. также}\EN{See also}: \myref{subsec:EICAR}.

% FIXME: break table
% FIXME: start at 0x20...
\begin{center}
\begin{longtable}{ | l | l | l | }
\hline
\HeaderColor ASCII\RU{-символ}\EN{ character} & 
\HeaderColor \RU{шестнадцатеричный код}\EN{hexadecimal code} & 
\HeaderColor x86\RU{-инструкция}\EN{ instruction} \\
\hline
0	 &30	 &XOR \\
1	 &31	 &XOR \\
2	 &32	 &XOR \\
3	 &33	 &XOR \\
4	 &34	 &XOR \\
5	 &35	 &XOR \\
7	 &37	 &AAA \\
8	 &38	 &CMP \\
9	 &39	 &CMP \\
:	 &3a	 &CMP \\
;	 &3b	 &CMP \\
<	 &3c	 &CMP \\
=	 &3d	 &CMP \\
?	 &3f	 &AAS \\
@	 &40	 &INC \\
A	 &41	 &INC \\
B	 &42	 &INC \\
C	 &43	 &INC \\
D	 &44	 &INC \\
E	 &45	 &INC \\
F	 &46	 &INC \\
G	 &47	 &INC \\
H	 &48	 &DEC \\
I	 &49	 &DEC \\
J	 &4a	 &DEC \\
K	 &4b	 &DEC \\
L	 &4c	 &DEC \\
M	 &4d	 &DEC \\
N	 &4e	 &DEC \\
O	 &4f	 &DEC \\
P	 &50	 &PUSH \\
Q	 &51	 &PUSH \\
R	 &52	 &PUSH \\
S	 &53	 &PUSH \\
T	 &54	 &PUSH \\
U	 &55	 &PUSH \\
V	 &56	 &PUSH \\
W	 &57	 &PUSH \\
X	 &58	 &POP \\
Y	 &59	 &POP \\
Z	 &5a	 &POP \\
\lbrack{}	 &5b	 &POP \\
\textbackslash{}	 &5c	 &POP \\
\rbrack{}	 &5d	 &POP \\
\verb|^|	 &5e	 &POP \\
\_	 &5f	 &POP \\
\verb|`|	 &60	 &PUSHA \\
a	 &61	 &POPA \\
f	 &66	 &\RU{(в 32-битном режиме) переключиться на}\EN{(in 32-bit mode) switch to}\\
   & & \RU{16-битный размер операнда}\EN{16-bit operand size} \\
g	 &67	 &\RU{(в 32-битном режиме) переключиться на}\EN{in 32-bit mode) switch to}\\
   & & \RU{16-битный размер адреса}\EN{16-bit address size} \\
h	 &68	 &PUSH\\
i	 &69	 &IMUL\\
j	 &6a	 &PUSH\\
k	 &6b	 &IMUL\\
p	 &70	 &JO\\
q	 &71	 &JNO\\
r	 &72	 &JB\\
s	 &73	 &JAE\\
t	 &74	 &JE\\
u	 &75	 &JNE\\
v	 &76	 &JBE\\
w	 &77	 &JA\\
x	 &78	 &JS\\
y	 &79	 &JNS\\
z	 &7a	 &JP\\
\hline
\end{longtable}
\end{center}

\myindex{x86!\Instructions!AAA}
\myindex{x86!\Instructions!AAS}
\myindex{x86!\Instructions!CMP}
\myindex{x86!\Instructions!DEC}
\myindex{x86!\Instructions!IMUL}
\myindex{x86!\Instructions!INC}
\myindex{x86!\Instructions!JA}
\myindex{x86!\Instructions!JAE}
\myindex{x86!\Instructions!JB}
\myindex{x86!\Instructions!JBE}
\myindex{x86!\Instructions!JE}
\myindex{x86!\Instructions!JNE}
\myindex{x86!\Instructions!JNO}
\myindex{x86!\Instructions!JNS}
\myindex{x86!\Instructions!JO}
\myindex{x86!\Instructions!JP}
\myindex{x86!\Instructions!JS}
\myindex{x86!\Instructions!POP}
\myindex{x86!\Instructions!POPA}
\myindex{x86!\Instructions!PUSH}
\myindex{x86!\Instructions!PUSHA}
\myindex{x86!\Instructions!XOR}

\RU{В итоге}\EN{In summary}:
AAA, AAS, CMP, DEC, IMUL, INC, JA, JAE, JB, JBE, JE, JNE, JNO, JNS, JO, JP, JS, POP, POPA, PUSH, PUSHA, 
XOR.

 % subsection
\subsection{npad}
\label{sec:npad}

\RU{Это макрос в ассемблере, для выравнивания некоторой метки по некоторой границе.}
\EN{It is an assembly language macro for aligning labels on a specific boundary.}
\DE{Dies ist ein Assembler-Makro um Labels an bestimmten Grenzen auszurichten.}
\FR{C'est une macro du langage d'assemblage pour aligner les labels sur une limite
spécifique.}
\JPN{It is an assembly language macro for aligning labels on a specific boundary.}

\RU{Это нужно для тех \IT{нагруженных} меток, куда чаще всего передается управление, например, 
начало тела цикла. 
Для того чтобы процессор мог эффективнее вытягивать данные или код из памяти, через шину с памятью, 
кэширование, итд.}
\EN{That's often needed for the busy labels to where the control flow is often passed, e.g., loop body starts.
So the CPU can load the data or code from the memory effectively, through the memory bus, cache lines, etc.}
\DE{Dies ist oft nützlich Labels, die oft Ziel einer Kotrollstruktur sind, wie Schleifenköpfe.
Somit kann die CPU Daten oder Code sehr effizient vom Speicher durch den Bus, den Cache, usw. laden.}
\FR{C'est souvent necessaire pour des labels très utilisés, comme par exemple le
début d'un corps de boucle. Ainsi, le CPU peut charger les données ou le code depuis
la mémoire efficacement, à travers le bus mémoire, les caches, etc.}

\RU{Взято из}\EN{Taken from}\DE{Entnommen von}\FR{Pris de} \TT{listing.inc} (MSVC):

\myindex{x86!\Instructions!NOP}
\RU{Это, кстати, любопытный пример различных вариантов \NOP{}-ов. 
Все эти инструкции не дают никакого эффекта, но отличаются разной длиной.}
\EN{By the way, it is a curious example of the different \NOP variations.
All these instructions have no effects whatsoever, but have a different size.}
\DE{Dies ist übrigens ein Beispiel für die unterschiedlichen \NOP-Variationen.
Keine dieser Anweisungen hat eine Auswirkung, aber alle haben eine unterschiedliche Größe.}
\FR{À propos, c'est un exemple curieux des différentes variations de \NOP. Toutes
ces instructions n'ont pas d'effet, mais ont une taille différente.}

\RU{Цель в том, чтобы была только одна инструкция, а не набор NOP-ов, 
считается что так лучше для производительности CPU.}
\EN{Having a single idle instruction instead of couple of NOP-s,
is accepted to be better for CPU performance.}
\DE{Eine einzelne Idle-Anweisung anstatt mehrerer NOPs hat positive Auswirkungen
auf die CPU-Performance.}
\FR{Avoir une seule instruction sans effet au lieu de plusieurs est accepté comme
étant meilleur pour la performance du CPU.}

\begin{lstlisting}[style=customasmx86]
;; LISTING.INC
;;
;; This file contains assembler macros and is included by the files created
;; with the -FA compiler switch to be assembled by MASM (Microsoft Macro
;; Assembler).
;;
;; Copyright (c) 1993-2003, Microsoft Corporation. All rights reserved.

;; non destructive nops
npad macro size
if size eq 1
  nop
else
 if size eq 2
   mov edi, edi
 else
  if size eq 3
    ; lea ecx, [ecx+00]
    DB 8DH, 49H, 00H
  else
   if size eq 4
     ; lea esp, [esp+00]
     DB 8DH, 64H, 24H, 00H
   else
    if size eq 5
      add eax, DWORD PTR 0
    else
     if size eq 6
       ; lea ebx, [ebx+00000000]
       DB 8DH, 9BH, 00H, 00H, 00H, 00H
     else
      if size eq 7
	; lea esp, [esp+00000000]
	DB 8DH, 0A4H, 24H, 00H, 00H, 00H, 00H 
      else
       if size eq 8
        ; jmp .+8; .npad 6
	DB 0EBH, 06H, 8DH, 9BH, 00H, 00H, 00H, 00H
       else
        if size eq 9
         ; jmp .+9; .npad 7
         DB 0EBH, 07H, 8DH, 0A4H, 24H, 00H, 00H, 00H, 00H
        else
         if size eq 10
          ; jmp .+A; .npad 7; .npad 1
          DB 0EBH, 08H, 8DH, 0A4H, 24H, 00H, 00H, 00H, 00H, 90H
         else
          if size eq 11
           ; jmp .+B; .npad 7; .npad 2
           DB 0EBH, 09H, 8DH, 0A4H, 24H, 00H, 00H, 00H, 00H, 8BH, 0FFH
          else
           if size eq 12
            ; jmp .+C; .npad 7; .npad 3
            DB 0EBH, 0AH, 8DH, 0A4H, 24H, 00H, 00H, 00H, 00H, 8DH, 49H, 00H
           else
            if size eq 13
             ; jmp .+D; .npad 7; .npad 4
             DB 0EBH, 0BH, 8DH, 0A4H, 24H, 00H, 00H, 00H, 00H, 8DH, 64H, 24H, 00H
            else
             if size eq 14
              ; jmp .+E; .npad 7; .npad 5
              DB 0EBH, 0CH, 8DH, 0A4H, 24H, 00H, 00H, 00H, 00H, 05H, 00H, 00H, 00H, 00H
             else
              if size eq 15
               ; jmp .+F; .npad 7; .npad 6
               DB 0EBH, 0DH, 8DH, 0A4H, 24H, 00H, 00H, 00H, 00H, 8DH, 9BH, 00H, 00H, 00H, 00H
              else
	       %out error: unsupported npad size
               .err
              endif
             endif
            endif
           endif
          endif
         endif
        endif
       endif
      endif
     endif
    endif
   endif
  endif
 endif
endif
endm
\end{lstlisting}
 % subsection

}
\subsection{UNIX: struct tm}

% subsections here:
\EN{\input{patterns/15_structs/3_tm_linux/linux_EN}}
\RU{\input{patterns/15_structs/3_tm_linux/linux_RU}}
\DE{\input{patterns/15_structs/3_tm_linux/linux_DE}}
\FR{\input{patterns/15_structs/3_tm_linux/linux_FR}}
\JPN{\subsubsection{Linux}

Let's take the \TT{tm} structure from \TT{time.h} in Linux for example:

\lstinputlisting[style=customc]{patterns/15_structs/3_tm_linux/GCC_tm.c}

Let's compile it in GCC 4.4.1:

\lstinputlisting[caption=GCC 4.4.1,style=customasmx86]{patterns/15_structs/3_tm_linux/GCC_tm_EN.asm}

Somehow, \IDA did not write the local variables' names in the local stack.
But since we already are experienced reverse engineers :-) we may do it without this information in 
this simple example.

\myindex{x86!\Instructions!LEA}

Please also pay attention to the \TT{lea edx, [eax+76Ch]}~---this instruction just adds \TT{0x76C} (1900) to value in \EAX,
but doesn't modify any flags. See also the relevant section about \LEA{}~(\myref{sec:LEA}).

\myparagraph{GDB}

Let's try to load the example into GDB
\footnote{The \IT{date} result is slightly corrected for demonstration purposes.
Of course, it's not possible to run GDB that quickly, in the same second.}:

\lstinputlisting[caption=GDB]{patterns/15_structs/3_tm_linux/GCC_tm_GDB.txt}

We can easily find our structure in the stack.
First, let's see how it's defined in \IT{time.h}:

\begin{lstlisting}[caption=time.h, label=struct_tm,style=customc]
struct tm
{
  int	tm_sec;
  int	tm_min;
  int	tm_hour;
  int	tm_mday;
  int	tm_mon;
  int	tm_year;
  int	tm_wday;
  int	tm_yday;
  int	tm_isdst;
};
\end{lstlisting}

Pay attention that
32-bit \Tint is used here instead of WORD in SYSTEMTIME.
So, each field occupies 32-bit.

Here are the fields of our structure in the stack:

\begin{lstlisting}
0xbffff0dc:	0x080484c3	0x080485c0	0x000007de	0x00000000
0xbffff0ec:	0x08048301	0x538c93ed	0x00000025 sec	0x0000000a min
0xbffff0fc:	0x00000012 hour	0x00000002 mday	0x00000005 mon 	0x00000072 year
0xbffff10c:	0x00000001 wday	0x00000098 yday	0x00000001 isdst0x00002a30
0xbffff11c:	0x0804b090	0x08048530	0x00000000	0x00000000
\end{lstlisting}

Or as a table:

\begin{center}
\begin{tabular}{ | l | l | l | }
\hline
\headercolor{} Hexadecimal number & 
\headercolor{} decimal number & 
\headercolor{} field name \\
\hline
0x00000025 & 37 	& tm\_sec \\
\hline
0x0000000a & 10 	& tm\_min \\
\hline
0x00000012 & 18 	& tm\_hour \\	
\hline
0x00000002 & 2 		& tm\_mday \\	
\hline
0x00000005 & 5 		& tm\_mon \\	
\hline
0x00000072 & 114 	& tm\_year \\
\hline
0x00000001 & 1 		& tm\_wday \\	
\hline
0x00000098 & 152 	& tm\_yday \\	
\hline
0x00000001 & 1 		& tm\_isdst \\
\hline
\end{tabular}
\end{center}

Just like SYSTEMTIME (\myref{sec:SYSTEMTIME}), 

there are also other fields available that are not used, like tm\_wday, tm\_yday, tm\_isdst.
}

\EN{\input{patterns/15_structs/3_tm_linux/ARM/main_EN}}
\RU{\input{patterns/15_structs/3_tm_linux/ARM/main_RU}}
\DE{\input{patterns/15_structs/3_tm_linux/ARM/main_DE}}
\FR{\input{patterns/15_structs/3_tm_linux/ARM/main_FR}}
\JPN{\subsubsection{ARM}

\myparagraph{\OptimizingKeilVI (\ThumbMode)}

Same example:

\lstinputlisting[caption=\OptimizingKeilVI (\ThumbMode),style=customasmARM]{patterns/15_structs/3_tm_linux/ARM/tm_ARM_keil_thumb.asm}

\myparagraph{\OptimizingXcodeIV (\ThumbTwoMode)}

\IDA \q{knows} the \TT{tm} structure 
(because \IDA \q{knows} the types of the arguments of library functions like \TT{localtime\_r()}), 

so it shows here structure elements accesses and their names.

\lstinputlisting[caption=\OptimizingXcodeIV (\ThumbTwoMode),style=customasmARM]{patterns/15_structs/3_tm_linux/ARM/tm_ARM_xcode_thumb.asm}
}

\EN{\input{patterns/15_structs/3_tm_linux/MIPS/main_EN}}
\RU{\input{patterns/15_structs/3_tm_linux/MIPS/main_RU}}
\DE{\input{patterns/15_structs/3_tm_linux/MIPS/main_DE}}
\FR{\input{patterns/15_structs/3_tm_linux/MIPS/main_FR}}
\JPN{\subsubsection{MIPS}

\lstinputlisting[caption=\Optimizing GCC 4.4.5 (IDA),numbers=left,style=customasmMIPS]{patterns/15_structs/3_tm_linux/MIPS/MIPS_O3_IDA_EN.lst}

This is an example where the branch delay slots can confuse us.

For example, there is the instruction \INS{addiu \$a1, 1900} at line 35 which adds 1900 to the year number.

It's executed before the corresponding \INS{JALR} at line 34, do not forget about it.

}

% subsection:
\EN{\input{patterns/15_structs/3_tm_linux/as_array/main_EN}}
\RU{\input{patterns/15_structs/3_tm_linux/as_array/main_RU}}
\DE{\input{patterns/15_structs/3_tm_linux/as_array/main_DE}}
\FR{\input{patterns/15_structs/3_tm_linux/as_array/main_FR}}
\JPN{\subsubsection{Structure as a set of values}

In order to illustrate that the structure is just variables laying side-by-side in one place, 
let's rework our example while looking at the \IT{tm} structure definition again: \lstref{struct_tm}.

\lstinputlisting[style=customc]{patterns/15_structs/3_tm_linux/as_array/GCC_tm2.c}

\myindex{\CStandardLibrary!localtime\_r()}
N.B. 
The pointer to the \TT{tm\_sec} field is passed into \TT{localtime\_r}, i.e., 
to the first element of the \q{structure}.

The compiler warns us:

\begin{lstlisting}[caption=GCC 4.7.3]
GCC_tm2.c: In function 'main':
GCC_tm2.c:11:5: warning: passing argument 2 of 'localtime_r' from incompatible pointer type [enabled by default]
In file included from GCC_tm2.c:2:0:
/usr/include/time.h:59:12: note: expected 'struct tm *' but argument is of type 'int *'
\end{lstlisting}

But nevertheless, it generates this:

\lstinputlisting[caption=GCC 4.7.3,style=customasmx86]{patterns/15_structs/3_tm_linux/as_array/GCC_tm2.asm}

This code is identical to what we saw previously and it is
not possible to say, was it a structure in original source code or just a pack of variables.

And this works. 
However, it is not recommended to do this in practice. 

Usually, non-optimizing compilers allocates variables in the local stack in the 
same order as they were declared in the function.

Nevertheless, there is no guarantee.

By the way, some other compiler may warn about the \TT{tm\_year}, \TT{tm\_mon}, \TT{tm\_mday},
\TT{tm\_hour}, \TT{tm\_min} variables, but not \TT{tm\_sec}
 are used without being initialized.

Indeed, the compiler is not aware that these are to be filled by\\
\TT{localtime\_r()} function.

We chose this example, since all structure fields are of type \Tint.%

This would not work if structure fields are 16-bit (\TT{WORD}), 
like in the case of the \TT{SYSTEMTIME} structure---\TT{GetSystemTime()} 
will fill them incorrectly 
(because the local variables are aligned on a 32-bit boundary).
Read more about it in next section: 
\q{\StructurePackingSectionName} (\myref{structure_packing}).

So, a structure is just a pack of variables laying in one place, side-by-side.
We could say that the structure is the instruction to the compiler, directing it to hold variables in one place.
By the way, in some very early C versions (before 1972), there were no structures at all \RitchieDevC.

There is no debugger example here: it is just the same as you already saw.

\subsubsection{Structure as an array of 32-bit words}

\lstinputlisting[style=customc]{patterns/15_structs/3_tm_linux/as_array/GCC_tm3.c}

We just \IT{cast} a pointer to structure to an array of \Tint{}'s.
And that works!
We run the example at 23:51:45 26-July-2014.

\begin{lstlisting}[label=GCC_tm3_output]
0x0000002D (45)
0x00000033 (51)
0x00000017 (23)
0x0000001A (26)
0x00000006 (6)
0x00000072 (114)
0x00000006 (6)
0x000000CE (206)
0x00000001 (1)
\end{lstlisting}

The variables here 
are in the same order as they are enumerated in the definition of the structure: \myref{struct_tm}.

Here is how it gets compiled:

\lstinputlisting[caption=\Optimizing GCC 4.8.1,style=customasmx86]{patterns/15_structs/3_tm_linux/as_array/GCC_tm3_JPN.lst}

Indeed: the space in the local stack is first treated as a structure, and then it's treated as an array.

It's even possible to modify the fields of the structure through this pointer.

And again, it's dubiously hackish way to do things, not recommended for use in production code.

\mysubparagraph{\Exercise}

As an exercise, try to modify (increase by 1) the current month number, treating the structure as 
an array.

\subsubsection{Structure as an array of bytes}

We can go even further. Let's \IT{cast} the pointer to an array of bytes and dump it:

\lstinputlisting[style=customc]{patterns/15_structs/3_tm_linux/as_array/GCC_tm4.c}

\begin{lstlisting}
0x2D 0x00 0x00 0x00 
0x33 0x00 0x00 0x00 
0x17 0x00 0x00 0x00 
0x1A 0x00 0x00 0x00 
0x06 0x00 0x00 0x00 
0x72 0x00 0x00 0x00 
0x06 0x00 0x00 0x00 
0xCE 0x00 0x00 0x00 
0x01 0x00 0x00 0x00 
\end{lstlisting}

We also run this example at 23:51:45 26-July-2014
\footnote{The time and date are the same for demonstration purposes. Byte values are fixed up.}.
The values are just the same as in the previous dump 
(\myref{GCC_tm3_output}), and of course, the lowest byte goes first, because this is a little-endian architecture 
(\myref{sec:endianness}).

\lstinputlisting[caption=\Optimizing GCC 4.8.1,style=customasmx86]{patterns/15_structs/3_tm_linux/as_array/GCC_tm4_JPN.lst}
}


\subsection{UNIX: struct tm}

% subsections here:
\EN{\input{patterns/15_structs/3_tm_linux/linux_EN}}
\RU{\input{patterns/15_structs/3_tm_linux/linux_RU}}
\DE{\input{patterns/15_structs/3_tm_linux/linux_DE}}
\FR{\input{patterns/15_structs/3_tm_linux/linux_FR}}
\JPN{\subsubsection{Linux}

Let's take the \TT{tm} structure from \TT{time.h} in Linux for example:

\lstinputlisting[style=customc]{patterns/15_structs/3_tm_linux/GCC_tm.c}

Let's compile it in GCC 4.4.1:

\lstinputlisting[caption=GCC 4.4.1,style=customasmx86]{patterns/15_structs/3_tm_linux/GCC_tm_EN.asm}

Somehow, \IDA did not write the local variables' names in the local stack.
But since we already are experienced reverse engineers :-) we may do it without this information in 
this simple example.

\myindex{x86!\Instructions!LEA}

Please also pay attention to the \TT{lea edx, [eax+76Ch]}~---this instruction just adds \TT{0x76C} (1900) to value in \EAX,
but doesn't modify any flags. See also the relevant section about \LEA{}~(\myref{sec:LEA}).

\myparagraph{GDB}

Let's try to load the example into GDB
\footnote{The \IT{date} result is slightly corrected for demonstration purposes.
Of course, it's not possible to run GDB that quickly, in the same second.}:

\lstinputlisting[caption=GDB]{patterns/15_structs/3_tm_linux/GCC_tm_GDB.txt}

We can easily find our structure in the stack.
First, let's see how it's defined in \IT{time.h}:

\begin{lstlisting}[caption=time.h, label=struct_tm,style=customc]
struct tm
{
  int	tm_sec;
  int	tm_min;
  int	tm_hour;
  int	tm_mday;
  int	tm_mon;
  int	tm_year;
  int	tm_wday;
  int	tm_yday;
  int	tm_isdst;
};
\end{lstlisting}

Pay attention that
32-bit \Tint is used here instead of WORD in SYSTEMTIME.
So, each field occupies 32-bit.

Here are the fields of our structure in the stack:

\begin{lstlisting}
0xbffff0dc:	0x080484c3	0x080485c0	0x000007de	0x00000000
0xbffff0ec:	0x08048301	0x538c93ed	0x00000025 sec	0x0000000a min
0xbffff0fc:	0x00000012 hour	0x00000002 mday	0x00000005 mon 	0x00000072 year
0xbffff10c:	0x00000001 wday	0x00000098 yday	0x00000001 isdst0x00002a30
0xbffff11c:	0x0804b090	0x08048530	0x00000000	0x00000000
\end{lstlisting}

Or as a table:

\begin{center}
\begin{tabular}{ | l | l | l | }
\hline
\headercolor{} Hexadecimal number & 
\headercolor{} decimal number & 
\headercolor{} field name \\
\hline
0x00000025 & 37 	& tm\_sec \\
\hline
0x0000000a & 10 	& tm\_min \\
\hline
0x00000012 & 18 	& tm\_hour \\	
\hline
0x00000002 & 2 		& tm\_mday \\	
\hline
0x00000005 & 5 		& tm\_mon \\	
\hline
0x00000072 & 114 	& tm\_year \\
\hline
0x00000001 & 1 		& tm\_wday \\	
\hline
0x00000098 & 152 	& tm\_yday \\	
\hline
0x00000001 & 1 		& tm\_isdst \\
\hline
\end{tabular}
\end{center}

Just like SYSTEMTIME (\myref{sec:SYSTEMTIME}), 

there are also other fields available that are not used, like tm\_wday, tm\_yday, tm\_isdst.
}

\EN{\input{patterns/15_structs/3_tm_linux/ARM/main_EN}}
\RU{\input{patterns/15_structs/3_tm_linux/ARM/main_RU}}
\DE{\input{patterns/15_structs/3_tm_linux/ARM/main_DE}}
\FR{\input{patterns/15_structs/3_tm_linux/ARM/main_FR}}
\JPN{\subsubsection{ARM}

\myparagraph{\OptimizingKeilVI (\ThumbMode)}

Same example:

\lstinputlisting[caption=\OptimizingKeilVI (\ThumbMode),style=customasmARM]{patterns/15_structs/3_tm_linux/ARM/tm_ARM_keil_thumb.asm}

\myparagraph{\OptimizingXcodeIV (\ThumbTwoMode)}

\IDA \q{knows} the \TT{tm} structure 
(because \IDA \q{knows} the types of the arguments of library functions like \TT{localtime\_r()}), 

so it shows here structure elements accesses and their names.

\lstinputlisting[caption=\OptimizingXcodeIV (\ThumbTwoMode),style=customasmARM]{patterns/15_structs/3_tm_linux/ARM/tm_ARM_xcode_thumb.asm}
}

\EN{\input{patterns/15_structs/3_tm_linux/MIPS/main_EN}}
\RU{\input{patterns/15_structs/3_tm_linux/MIPS/main_RU}}
\DE{\input{patterns/15_structs/3_tm_linux/MIPS/main_DE}}
\FR{\input{patterns/15_structs/3_tm_linux/MIPS/main_FR}}
\JPN{\subsubsection{MIPS}

\lstinputlisting[caption=\Optimizing GCC 4.4.5 (IDA),numbers=left,style=customasmMIPS]{patterns/15_structs/3_tm_linux/MIPS/MIPS_O3_IDA_EN.lst}

This is an example where the branch delay slots can confuse us.

For example, there is the instruction \INS{addiu \$a1, 1900} at line 35 which adds 1900 to the year number.

It's executed before the corresponding \INS{JALR} at line 34, do not forget about it.

}

% subsection:
\EN{\input{patterns/15_structs/3_tm_linux/as_array/main_EN}}
\RU{\input{patterns/15_structs/3_tm_linux/as_array/main_RU}}
\DE{\input{patterns/15_structs/3_tm_linux/as_array/main_DE}}
\FR{\input{patterns/15_structs/3_tm_linux/as_array/main_FR}}
\JPN{\subsubsection{Structure as a set of values}

In order to illustrate that the structure is just variables laying side-by-side in one place, 
let's rework our example while looking at the \IT{tm} structure definition again: \lstref{struct_tm}.

\lstinputlisting[style=customc]{patterns/15_structs/3_tm_linux/as_array/GCC_tm2.c}

\myindex{\CStandardLibrary!localtime\_r()}
N.B. 
The pointer to the \TT{tm\_sec} field is passed into \TT{localtime\_r}, i.e., 
to the first element of the \q{structure}.

The compiler warns us:

\begin{lstlisting}[caption=GCC 4.7.3]
GCC_tm2.c: In function 'main':
GCC_tm2.c:11:5: warning: passing argument 2 of 'localtime_r' from incompatible pointer type [enabled by default]
In file included from GCC_tm2.c:2:0:
/usr/include/time.h:59:12: note: expected 'struct tm *' but argument is of type 'int *'
\end{lstlisting}

But nevertheless, it generates this:

\lstinputlisting[caption=GCC 4.7.3,style=customasmx86]{patterns/15_structs/3_tm_linux/as_array/GCC_tm2.asm}

This code is identical to what we saw previously and it is
not possible to say, was it a structure in original source code or just a pack of variables.

And this works. 
However, it is not recommended to do this in practice. 

Usually, non-optimizing compilers allocates variables in the local stack in the 
same order as they were declared in the function.

Nevertheless, there is no guarantee.

By the way, some other compiler may warn about the \TT{tm\_year}, \TT{tm\_mon}, \TT{tm\_mday},
\TT{tm\_hour}, \TT{tm\_min} variables, but not \TT{tm\_sec}
 are used without being initialized.

Indeed, the compiler is not aware that these are to be filled by\\
\TT{localtime\_r()} function.

We chose this example, since all structure fields are of type \Tint.%

This would not work if structure fields are 16-bit (\TT{WORD}), 
like in the case of the \TT{SYSTEMTIME} structure---\TT{GetSystemTime()} 
will fill them incorrectly 
(because the local variables are aligned on a 32-bit boundary).
Read more about it in next section: 
\q{\StructurePackingSectionName} (\myref{structure_packing}).

So, a structure is just a pack of variables laying in one place, side-by-side.
We could say that the structure is the instruction to the compiler, directing it to hold variables in one place.
By the way, in some very early C versions (before 1972), there were no structures at all \RitchieDevC.

There is no debugger example here: it is just the same as you already saw.

\subsubsection{Structure as an array of 32-bit words}

\lstinputlisting[style=customc]{patterns/15_structs/3_tm_linux/as_array/GCC_tm3.c}

We just \IT{cast} a pointer to structure to an array of \Tint{}'s.
And that works!
We run the example at 23:51:45 26-July-2014.

\begin{lstlisting}[label=GCC_tm3_output]
0x0000002D (45)
0x00000033 (51)
0x00000017 (23)
0x0000001A (26)
0x00000006 (6)
0x00000072 (114)
0x00000006 (6)
0x000000CE (206)
0x00000001 (1)
\end{lstlisting}

The variables here 
are in the same order as they are enumerated in the definition of the structure: \myref{struct_tm}.

Here is how it gets compiled:

\lstinputlisting[caption=\Optimizing GCC 4.8.1,style=customasmx86]{patterns/15_structs/3_tm_linux/as_array/GCC_tm3_JPN.lst}

Indeed: the space in the local stack is first treated as a structure, and then it's treated as an array.

It's even possible to modify the fields of the structure through this pointer.

And again, it's dubiously hackish way to do things, not recommended for use in production code.

\mysubparagraph{\Exercise}

As an exercise, try to modify (increase by 1) the current month number, treating the structure as 
an array.

\subsubsection{Structure as an array of bytes}

We can go even further. Let's \IT{cast} the pointer to an array of bytes and dump it:

\lstinputlisting[style=customc]{patterns/15_structs/3_tm_linux/as_array/GCC_tm4.c}

\begin{lstlisting}
0x2D 0x00 0x00 0x00 
0x33 0x00 0x00 0x00 
0x17 0x00 0x00 0x00 
0x1A 0x00 0x00 0x00 
0x06 0x00 0x00 0x00 
0x72 0x00 0x00 0x00 
0x06 0x00 0x00 0x00 
0xCE 0x00 0x00 0x00 
0x01 0x00 0x00 0x00 
\end{lstlisting}

We also run this example at 23:51:45 26-July-2014
\footnote{The time and date are the same for demonstration purposes. Byte values are fixed up.}.
The values are just the same as in the previous dump 
(\myref{GCC_tm3_output}), and of course, the lowest byte goes first, because this is a little-endian architecture 
(\myref{sec:endianness}).

\lstinputlisting[caption=\Optimizing GCC 4.8.1,style=customasmx86]{patterns/15_structs/3_tm_linux/as_array/GCC_tm4_JPN.lst}
}


\input{appendix/GCC_library}
\input{appendix/MSVC_library}
\input{appendix/cheatsheets}
}
\RU{\part*{\RU{Приложение}\EN{Appendix}\DE{Anhang}\FR{Appendice}\JPN{付録}}
\appendix
\addcontentsline{toc}{part}{\RU{Приложение}\EN{Appendix}\DE{Anhang}\FR{Appendice}\JPN{付録}}

% chapters
\EN{\input{appendix/x86/main_EN}}
\RU{\input{appendix/x86/main_RU}}
\DE{\input{appendix/x86/main_DE}}
\FR{\input{appendix/x86/main_FR}}
\JPN{\mysection{x86}

\subsection{Terminology}

Common for 16-bit (8086/80286), 32-bit (80386, etc.), 64-bit.

\myindex{IEEE 754}
\myindex{MS-DOS}
\begin{description}
	\item[byte] 8-bit.
		The DB assembly directive is used for defining variables and arrays of bytes.
		Bytes are passed in the 8-bit part of registers: \TT{AL/BL/CL/DL/AH/BH/CH/DH/SIL/DIL/R*L}.
	\item[word] 16-bit. 
		DW assembly directive \dittoclosing.
		Words are passed in the 16-bit part of the registers:\\
			\TT{AX/BX/CX/DX/SI/DI/R*W}.
	\item[double word] (\q{dword}) 32-bit.
		DD assembly directive \dittoclosing.
		Double words are passed in registers (x86) or in the 32-bit part of registers (x64). 
		In 16-bit code, double words are passed in 16-bit register pairs.
	\item[quad word] (\q{qword}) 64-bit.
		DQ assembly directive \dittoclosing.
		In 32-bit environment, quad words are passed in 32-bit register pairs.
	\item[tbyte] (10 bytes) 80-bit or 10 bytes (used for IEEE 754 FPU registers).
	\item[paragraph] (16 bytes)---term was popular in MS-DOS environment. % TODO link to a part about 8086 memory model...
\end{description}

\myindex{Windows!API}

Data types of the same width (BYTE, WORD, DWORD) are also the same in Windows \ac{API}.

\input{appendix/x86/registers} % subsection
\subsection{\RU{Инструкции}\EN{Instructions}}
\label{sec:x86_instructions}

\RU{Инструкции, отмеченные как (M) обычно не генерируются компилятором: если вы видите её, очень может быть
это вручную написанный фрагмент кода, либо это т.н. compiler intrinsic}
\EN{Instructions marked as (M) are not usually generated by the compiler: if you see one of them, it is probably
a hand-written piece of assembly code, or a compiler intrinsic} (\myref{sec:compiler_intrinsic}).

% TODO ? обратные инструкции

\RU{Только наиболее используемые инструкции перечислены здесь}
\EN{Only the most frequently used instructions are listed here}.
\EN{You can read \myref{x86_manuals} for a full documentation.}%
\RU{Обращайтесь к \myref{x86_manuals} для полной документации.}

\RU{Нужно ли заучивать опкоды инструкций на память?}\EN{Do you have to know all instruction's opcodes by heart?}
\RU{Нет, только те, которые часто используются для модификации кода}\EN{No, only those
which are used for code patching} (\myref{x86_patching}).
\RU{Остальные запоминать нет смысла.}\EN{All the rest of the opcodes don't need to be memorized.}

\subsubsection{\RU{Префиксы}\EN{Prefixes}}

\myindex{x86!\Prefixes!LOCK}
\myindex{x86!\Prefixes!REP}
\myindex{x86!\Prefixes!REPE/REPNE}
\begin{description}
\label{x86_lock}
\item[LOCK] \RU{используется чтобы предоставить эксклюзивный доступ к памяти в многопроцессорной среде}
\EN{forces CPU to make exclusive access to the RAM in multiprocessor environment}.
\RU{Для упрощения, можно сказать, что когда исполняется инструкция с этим префиксом, остальные процессоры
в системе останавливаются}\EN{For the sake of simplification, it can be said that when an instruction
with this prefix is executed, all other CPUs in a multiprocessor system are stopped}.
\RU{Чаще все это используется для критических секций, семафоров, мьютексов}\EN{Most often
it is used for critical sections, semaphores, mutexes}.
\RU{Обычно используется с}\EN{Commonly used with} ADD, AND, BTR, BTS, CMPXCHG, OR, XADD, XOR.
\RU{Читайте больше о критических секциях}\EN{You can read more about critical sections here} (\myref{critical_sections}).

\item[REP] \RU{используется с инструкциями}\EN{is used with the} MOVSx \AndENRU STOSx\EN{ instructions}:
\RU{инструкция будет исполняться в цикле, счетчик расположен в регистре CX/ECX/RCX}
\EN{execute the instruction in a loop, the counter is located in the CX/ECX/RCX register}.
\RU{Для более детального описания, читайте больше об инструкциях}
\EN{For a detailed description, read more about the} MOVSx (\myref{REP_MOVSx}) 
\AndENRU STOSx (\myref{REP_STOSx})\EN{ instructions}.

\RU{Работа инструкций с префиксом REP зависит от флага DF, он задает направление}
\EN{The instructions prefixed by REP are sensitive to the DF flag, which is used to set the direction}.

\item[REPE/REPNE] (\ac{AKA} REPZ/REPNZ) \RU{используется с инструкциями}\EN{used with} CMPSx \AndENRU
SCASx\EN{ instructions}:
\RU{инструкция будет исполняться в цикле, счетчик расположен в регистре \TT{CX}/\TT{ECX}/\TT{RCX}}
\EN{execute the last instruction in a loop, the count is set in the \TT{CX}/\TT{ECX}/\TT{RCX} register}. 
\RU{Выполнение будет прервано если ZF будет 0 (REPE) либо если ZF будет 1 (REPNE)}
\EN{It terminates prematurely if ZF is 0 (REPE) or if ZF is 1 (REPNE)}.

\RU{Для более детального описания, читайте больше об инструкциях}
\EN{For a detailed description, you can read more about the} CMPSx (\myref{REPE_CMPSx}) 
\AndENRU SCASx (\myref{REPNE_SCASx})\EN{ instructions}.

\RU{Работа инструкций с префиксами REPE/REPNE зависит от флага DF, он задает направление}
\EN{Instructions prefixed by REPE/REPNE are sensitive to the DF flag, which is used to set the direction}.

\end{description}

\subsubsection{\RU{Наиболее часто используемые инструкции}\EN{Most frequently used instructions}}

\RU{Их можно заучить в первую очередь}\EN{These can be memorized in the first place}.

\begin{description}
% in order to keep them easily sorted...
\input{appendix/x86/instructions/ADC}
\input{appendix/x86/instructions/ADD}
\input{appendix/x86/instructions/AND}
\input{appendix/x86/instructions/CALL}
\input{appendix/x86/instructions/CMP}
\input{appendix/x86/instructions/DEC}
\input{appendix/x86/instructions/IMUL}
\input{appendix/x86/instructions/INC}
\input{appendix/x86/instructions/JCXZ}
\input{appendix/x86/instructions/JMP}
\input{appendix/x86/instructions/Jcc}
\input{appendix/x86/instructions/LAHF}
\input{appendix/x86/instructions/LEAVE}
\input{appendix/x86/instructions/LEA}
\input{appendix/x86/instructions/MOVSB_W_D_Q}
\input{appendix/x86/instructions/MOVSX}
\input{appendix/x86/instructions/MOVZX}
\input{appendix/x86/instructions/MOV}
\input{appendix/x86/instructions/MUL}
\input{appendix/x86/instructions/NEG}
\input{appendix/x86/instructions/NOP}
\input{appendix/x86/instructions/NOT}
\input{appendix/x86/instructions/OR}
\input{appendix/x86/instructions/POP}
\input{appendix/x86/instructions/PUSH}
\input{appendix/x86/instructions/RET}
\input{appendix/x86/instructions/SAHF}
\input{appendix/x86/instructions/SBB}
\input{appendix/x86/instructions/SCASB_W_D_Q}
\input{appendix/x86/instructions/SHx}
\input{appendix/x86/instructions/SHRD}
\input{appendix/x86/instructions/STOSB_W_D_Q}
\input{appendix/x86/instructions/SUB}
\input{appendix/x86/instructions/TEST}
\input{appendix/x86/instructions/XOR}
\end{description}

\subsubsection{\RU{Реже используемые инструкции}\EN{Less frequently used instructions}}

\begin{description}
\input{appendix/x86/instructions/BSF}
\input{appendix/x86/instructions/BSR}
\input{appendix/x86/instructions/BSWAP}
\input{appendix/x86/instructions/BTC}
\input{appendix/x86/instructions/BTR}
\input{appendix/x86/instructions/BTS}
\input{appendix/x86/instructions/BT}
\input{appendix/x86/instructions/CBW_CWDE_CDQ}
\input{appendix/x86/instructions/CLD}
\input{appendix/x86/instructions/CLI}
\input{appendix/x86/instructions/CMC}
\input{appendix/x86/instructions/CMOVcc}
\input{appendix/x86/instructions/CMPSB_W_D_Q}
\input{appendix/x86/instructions/CPUID}
\input{appendix/x86/instructions/DIV}
\input{appendix/x86/instructions/IDIV}
\input{appendix/x86/instructions/INT}
\input{appendix/x86/instructions/IN}
\input{appendix/x86/instructions/IRET}
\input{appendix/x86/instructions/LOOP}
\input{appendix/x86/instructions/OUT}
\input{appendix/x86/instructions/POPA}
\input{appendix/x86/instructions/POPCNT}
\input{appendix/x86/instructions/POPF}
\input{appendix/x86/instructions/PUSHA}
\input{appendix/x86/instructions/PUSHF}
\input{appendix/x86/instructions/RCx}
\input{appendix/x86/instructions/ROx}
\input{appendix/x86/instructions/SAL}
\input{appendix/x86/instructions/SAR}
\input{appendix/x86/instructions/SETcc}
\input{appendix/x86/instructions/STC}
\input{appendix/x86/instructions/STD}
\input{appendix/x86/instructions/STI}
\input{appendix/x86/instructions/SYSCALL}
\input{appendix/x86/instructions/SYSENTER}
\input{appendix/x86/instructions/UD2}
\input{appendix/x86/instructions/XCHG}
\end{description}

\subsubsection{\RU{Инструкции FPU}\EN{FPU instructions}}

\RU{Суффикс \TT{-R} в названии инструкции обычно означает, что операнды поменяны местами, суффикс \TT{-P} означает
что один элемент выталкивается из стека после исполнения инструкции, суффикс \TT{-PP} означает, что
выталкиваются два элемента}%
\EN{\TT{-R} suffix in the mnemonic usually implies that the operands are reversed,
\TT{-P} suffix implies that one element is popped
from the stack after the instruction's execution, \TT{-PP} suffix implies that two elements are popped}.

\TT{-P} \RU{инструкции часто бывают полезны, когда нам уже больше не нужно хранить значение в 
FPU-стеке после операции.}%
\EN{instructions are often useful when we do not need the value in the FPU stack to be 
present anymore after the operation.}

\begin{description}
\input{appendix/x86/instructions/FABS}
\input{appendix/x86/instructions/FADD} % + FADDP
\input{appendix/x86/instructions/FCHS}
\input{appendix/x86/instructions/FCOM} % + FCOMP + FCOMPP
\input{appendix/x86/instructions/FDIVR} % + FDIVRP
\input{appendix/x86/instructions/FDIV} % + FDIVP
\input{appendix/x86/instructions/FILD}
\input{appendix/x86/instructions/FIST} % + FISTP
\input{appendix/x86/instructions/FLD1}
\input{appendix/x86/instructions/FLDCW}
\input{appendix/x86/instructions/FLDZ}
\input{appendix/x86/instructions/FLD}
\input{appendix/x86/instructions/FMUL} % + FMULP
\input{appendix/x86/instructions/FSINCOS}
\input{appendix/x86/instructions/FSQRT}
\input{appendix/x86/instructions/FSTCW} % + FNSTCW
\input{appendix/x86/instructions/FSTSW} % + FNSTSW
\input{appendix/x86/instructions/FST}
\input{appendix/x86/instructions/FSUBR} % + FSUBRP
\input{appendix/x86/instructions/FSUB} % + FSUBP
\input{appendix/x86/instructions/FUCOM} % + FUCOMP + FUCOMPP
\input{appendix/x86/instructions/FXCH}
\end{description}

%\subsubsection{\RU{SIMD-инструкции}\EN{SIMD instructions}}

% TODO

%\begin{description}
%\input{appendix/x86/instructions/DIVSD}
%\input{appendix/x86/instructions/MOVDQA}
%\input{appendix/x86/instructions/MOVDQU}
%\input{appendix/x86/instructions/PADDD}
%\input{appendix/x86/instructions/PCMPEQB}
%\input{appendix/x86/instructions/PLMULHW}
%\input{appendix/x86/instructions/PLMULLD}
%\input{appendix/x86/instructions/PMOVMSKB}
%\input{appendix/x86/instructions/PXOR}
%\end{description}

% SHLD !
% SHRD !
% BSWAP !
% CMPXCHG
% XADD !
% CMPXCHG8B
% RDTSC !
% PAUSE!

% xsave
% fnclex, fnsave
% movsxd, movaps, wait, sfence, lfence, pushfq
% prefetchw
% REP RETN
% REP BSF
% movnti, movntdq, rdmsr, wrmsr
% ldmxcsr, stmxcsr, invlpg
% swapgs
% movq, movd
% mulsd
% POR
% IRETQ
% pslldq
% psrldq
% cqo, fxrstor, comisd, xrstor, wbinvd, movntq
% fprem
% addsb, subsd, frndint

% rare:
%\item[ENTER]
%\item[LES]
% LDS
% XLAT

\subsubsection{\RU{Инструкции с печатаемым ASCII-опкодом}\EN{Instructions having printable ASCII opcode}}

(\RU{В 32-битном режиме}\EN{In 32-bit mode}).

\label{printable_x86_opcodes}
\myindex{Shellcode}
\RU{Это может пригодиться для создания шеллкодов}\EN{These can be suitable for shellcode construction}.
\RU{См. также}\EN{See also}: \myref{subsec:EICAR}.

% FIXME: break table
% FIXME: start at 0x20...
\begin{center}
\begin{longtable}{ | l | l | l | }
\hline
\HeaderColor ASCII\RU{-символ}\EN{ character} & 
\HeaderColor \RU{шестнадцатеричный код}\EN{hexadecimal code} & 
\HeaderColor x86\RU{-инструкция}\EN{ instruction} \\
\hline
0	 &30	 &XOR \\
1	 &31	 &XOR \\
2	 &32	 &XOR \\
3	 &33	 &XOR \\
4	 &34	 &XOR \\
5	 &35	 &XOR \\
7	 &37	 &AAA \\
8	 &38	 &CMP \\
9	 &39	 &CMP \\
:	 &3a	 &CMP \\
;	 &3b	 &CMP \\
<	 &3c	 &CMP \\
=	 &3d	 &CMP \\
?	 &3f	 &AAS \\
@	 &40	 &INC \\
A	 &41	 &INC \\
B	 &42	 &INC \\
C	 &43	 &INC \\
D	 &44	 &INC \\
E	 &45	 &INC \\
F	 &46	 &INC \\
G	 &47	 &INC \\
H	 &48	 &DEC \\
I	 &49	 &DEC \\
J	 &4a	 &DEC \\
K	 &4b	 &DEC \\
L	 &4c	 &DEC \\
M	 &4d	 &DEC \\
N	 &4e	 &DEC \\
O	 &4f	 &DEC \\
P	 &50	 &PUSH \\
Q	 &51	 &PUSH \\
R	 &52	 &PUSH \\
S	 &53	 &PUSH \\
T	 &54	 &PUSH \\
U	 &55	 &PUSH \\
V	 &56	 &PUSH \\
W	 &57	 &PUSH \\
X	 &58	 &POP \\
Y	 &59	 &POP \\
Z	 &5a	 &POP \\
\lbrack{}	 &5b	 &POP \\
\textbackslash{}	 &5c	 &POP \\
\rbrack{}	 &5d	 &POP \\
\verb|^|	 &5e	 &POP \\
\_	 &5f	 &POP \\
\verb|`|	 &60	 &PUSHA \\
a	 &61	 &POPA \\
f	 &66	 &\RU{(в 32-битном режиме) переключиться на}\EN{(in 32-bit mode) switch to}\\
   & & \RU{16-битный размер операнда}\EN{16-bit operand size} \\
g	 &67	 &\RU{(в 32-битном режиме) переключиться на}\EN{in 32-bit mode) switch to}\\
   & & \RU{16-битный размер адреса}\EN{16-bit address size} \\
h	 &68	 &PUSH\\
i	 &69	 &IMUL\\
j	 &6a	 &PUSH\\
k	 &6b	 &IMUL\\
p	 &70	 &JO\\
q	 &71	 &JNO\\
r	 &72	 &JB\\
s	 &73	 &JAE\\
t	 &74	 &JE\\
u	 &75	 &JNE\\
v	 &76	 &JBE\\
w	 &77	 &JA\\
x	 &78	 &JS\\
y	 &79	 &JNS\\
z	 &7a	 &JP\\
\hline
\end{longtable}
\end{center}

\myindex{x86!\Instructions!AAA}
\myindex{x86!\Instructions!AAS}
\myindex{x86!\Instructions!CMP}
\myindex{x86!\Instructions!DEC}
\myindex{x86!\Instructions!IMUL}
\myindex{x86!\Instructions!INC}
\myindex{x86!\Instructions!JA}
\myindex{x86!\Instructions!JAE}
\myindex{x86!\Instructions!JB}
\myindex{x86!\Instructions!JBE}
\myindex{x86!\Instructions!JE}
\myindex{x86!\Instructions!JNE}
\myindex{x86!\Instructions!JNO}
\myindex{x86!\Instructions!JNS}
\myindex{x86!\Instructions!JO}
\myindex{x86!\Instructions!JP}
\myindex{x86!\Instructions!JS}
\myindex{x86!\Instructions!POP}
\myindex{x86!\Instructions!POPA}
\myindex{x86!\Instructions!PUSH}
\myindex{x86!\Instructions!PUSHA}
\myindex{x86!\Instructions!XOR}

\RU{В итоге}\EN{In summary}:
AAA, AAS, CMP, DEC, IMUL, INC, JA, JAE, JB, JBE, JE, JNE, JNO, JNS, JO, JP, JS, POP, POPA, PUSH, PUSHA, 
XOR.

 % subsection
\subsection{npad}
\label{sec:npad}

\RU{Это макрос в ассемблере, для выравнивания некоторой метки по некоторой границе.}
\EN{It is an assembly language macro for aligning labels on a specific boundary.}
\DE{Dies ist ein Assembler-Makro um Labels an bestimmten Grenzen auszurichten.}
\FR{C'est une macro du langage d'assemblage pour aligner les labels sur une limite
spécifique.}
\JPN{It is an assembly language macro for aligning labels on a specific boundary.}

\RU{Это нужно для тех \IT{нагруженных} меток, куда чаще всего передается управление, например, 
начало тела цикла. 
Для того чтобы процессор мог эффективнее вытягивать данные или код из памяти, через шину с памятью, 
кэширование, итд.}
\EN{That's often needed for the busy labels to where the control flow is often passed, e.g., loop body starts.
So the CPU can load the data or code from the memory effectively, through the memory bus, cache lines, etc.}
\DE{Dies ist oft nützlich Labels, die oft Ziel einer Kotrollstruktur sind, wie Schleifenköpfe.
Somit kann die CPU Daten oder Code sehr effizient vom Speicher durch den Bus, den Cache, usw. laden.}
\FR{C'est souvent necessaire pour des labels très utilisés, comme par exemple le
début d'un corps de boucle. Ainsi, le CPU peut charger les données ou le code depuis
la mémoire efficacement, à travers le bus mémoire, les caches, etc.}

\RU{Взято из}\EN{Taken from}\DE{Entnommen von}\FR{Pris de} \TT{listing.inc} (MSVC):

\myindex{x86!\Instructions!NOP}
\RU{Это, кстати, любопытный пример различных вариантов \NOP{}-ов. 
Все эти инструкции не дают никакого эффекта, но отличаются разной длиной.}
\EN{By the way, it is a curious example of the different \NOP variations.
All these instructions have no effects whatsoever, but have a different size.}
\DE{Dies ist übrigens ein Beispiel für die unterschiedlichen \NOP-Variationen.
Keine dieser Anweisungen hat eine Auswirkung, aber alle haben eine unterschiedliche Größe.}
\FR{À propos, c'est un exemple curieux des différentes variations de \NOP. Toutes
ces instructions n'ont pas d'effet, mais ont une taille différente.}

\RU{Цель в том, чтобы была только одна инструкция, а не набор NOP-ов, 
считается что так лучше для производительности CPU.}
\EN{Having a single idle instruction instead of couple of NOP-s,
is accepted to be better for CPU performance.}
\DE{Eine einzelne Idle-Anweisung anstatt mehrerer NOPs hat positive Auswirkungen
auf die CPU-Performance.}
\FR{Avoir une seule instruction sans effet au lieu de plusieurs est accepté comme
étant meilleur pour la performance du CPU.}

\begin{lstlisting}[style=customasmx86]
;; LISTING.INC
;;
;; This file contains assembler macros and is included by the files created
;; with the -FA compiler switch to be assembled by MASM (Microsoft Macro
;; Assembler).
;;
;; Copyright (c) 1993-2003, Microsoft Corporation. All rights reserved.

;; non destructive nops
npad macro size
if size eq 1
  nop
else
 if size eq 2
   mov edi, edi
 else
  if size eq 3
    ; lea ecx, [ecx+00]
    DB 8DH, 49H, 00H
  else
   if size eq 4
     ; lea esp, [esp+00]
     DB 8DH, 64H, 24H, 00H
   else
    if size eq 5
      add eax, DWORD PTR 0
    else
     if size eq 6
       ; lea ebx, [ebx+00000000]
       DB 8DH, 9BH, 00H, 00H, 00H, 00H
     else
      if size eq 7
	; lea esp, [esp+00000000]
	DB 8DH, 0A4H, 24H, 00H, 00H, 00H, 00H 
      else
       if size eq 8
        ; jmp .+8; .npad 6
	DB 0EBH, 06H, 8DH, 9BH, 00H, 00H, 00H, 00H
       else
        if size eq 9
         ; jmp .+9; .npad 7
         DB 0EBH, 07H, 8DH, 0A4H, 24H, 00H, 00H, 00H, 00H
        else
         if size eq 10
          ; jmp .+A; .npad 7; .npad 1
          DB 0EBH, 08H, 8DH, 0A4H, 24H, 00H, 00H, 00H, 00H, 90H
         else
          if size eq 11
           ; jmp .+B; .npad 7; .npad 2
           DB 0EBH, 09H, 8DH, 0A4H, 24H, 00H, 00H, 00H, 00H, 8BH, 0FFH
          else
           if size eq 12
            ; jmp .+C; .npad 7; .npad 3
            DB 0EBH, 0AH, 8DH, 0A4H, 24H, 00H, 00H, 00H, 00H, 8DH, 49H, 00H
           else
            if size eq 13
             ; jmp .+D; .npad 7; .npad 4
             DB 0EBH, 0BH, 8DH, 0A4H, 24H, 00H, 00H, 00H, 00H, 8DH, 64H, 24H, 00H
            else
             if size eq 14
              ; jmp .+E; .npad 7; .npad 5
              DB 0EBH, 0CH, 8DH, 0A4H, 24H, 00H, 00H, 00H, 00H, 05H, 00H, 00H, 00H, 00H
             else
              if size eq 15
               ; jmp .+F; .npad 7; .npad 6
               DB 0EBH, 0DH, 8DH, 0A4H, 24H, 00H, 00H, 00H, 00H, 8DH, 9BH, 00H, 00H, 00H, 00H
              else
	       %out error: unsupported npad size
               .err
              endif
             endif
            endif
           endif
          endif
         endif
        endif
       endif
      endif
     endif
    endif
   endif
  endif
 endif
endif
endm
\end{lstlisting}
 % subsection

}
\subsection{UNIX: struct tm}

% subsections here:
\EN{\input{patterns/15_structs/3_tm_linux/linux_EN}}
\RU{\input{patterns/15_structs/3_tm_linux/linux_RU}}
\DE{\input{patterns/15_structs/3_tm_linux/linux_DE}}
\FR{\input{patterns/15_structs/3_tm_linux/linux_FR}}
\JPN{\subsubsection{Linux}

Let's take the \TT{tm} structure from \TT{time.h} in Linux for example:

\lstinputlisting[style=customc]{patterns/15_structs/3_tm_linux/GCC_tm.c}

Let's compile it in GCC 4.4.1:

\lstinputlisting[caption=GCC 4.4.1,style=customasmx86]{patterns/15_structs/3_tm_linux/GCC_tm_EN.asm}

Somehow, \IDA did not write the local variables' names in the local stack.
But since we already are experienced reverse engineers :-) we may do it without this information in 
this simple example.

\myindex{x86!\Instructions!LEA}

Please also pay attention to the \TT{lea edx, [eax+76Ch]}~---this instruction just adds \TT{0x76C} (1900) to value in \EAX,
but doesn't modify any flags. See also the relevant section about \LEA{}~(\myref{sec:LEA}).

\myparagraph{GDB}

Let's try to load the example into GDB
\footnote{The \IT{date} result is slightly corrected for demonstration purposes.
Of course, it's not possible to run GDB that quickly, in the same second.}:

\lstinputlisting[caption=GDB]{patterns/15_structs/3_tm_linux/GCC_tm_GDB.txt}

We can easily find our structure in the stack.
First, let's see how it's defined in \IT{time.h}:

\begin{lstlisting}[caption=time.h, label=struct_tm,style=customc]
struct tm
{
  int	tm_sec;
  int	tm_min;
  int	tm_hour;
  int	tm_mday;
  int	tm_mon;
  int	tm_year;
  int	tm_wday;
  int	tm_yday;
  int	tm_isdst;
};
\end{lstlisting}

Pay attention that
32-bit \Tint is used here instead of WORD in SYSTEMTIME.
So, each field occupies 32-bit.

Here are the fields of our structure in the stack:

\begin{lstlisting}
0xbffff0dc:	0x080484c3	0x080485c0	0x000007de	0x00000000
0xbffff0ec:	0x08048301	0x538c93ed	0x00000025 sec	0x0000000a min
0xbffff0fc:	0x00000012 hour	0x00000002 mday	0x00000005 mon 	0x00000072 year
0xbffff10c:	0x00000001 wday	0x00000098 yday	0x00000001 isdst0x00002a30
0xbffff11c:	0x0804b090	0x08048530	0x00000000	0x00000000
\end{lstlisting}

Or as a table:

\begin{center}
\begin{tabular}{ | l | l | l | }
\hline
\headercolor{} Hexadecimal number & 
\headercolor{} decimal number & 
\headercolor{} field name \\
\hline
0x00000025 & 37 	& tm\_sec \\
\hline
0x0000000a & 10 	& tm\_min \\
\hline
0x00000012 & 18 	& tm\_hour \\	
\hline
0x00000002 & 2 		& tm\_mday \\	
\hline
0x00000005 & 5 		& tm\_mon \\	
\hline
0x00000072 & 114 	& tm\_year \\
\hline
0x00000001 & 1 		& tm\_wday \\	
\hline
0x00000098 & 152 	& tm\_yday \\	
\hline
0x00000001 & 1 		& tm\_isdst \\
\hline
\end{tabular}
\end{center}

Just like SYSTEMTIME (\myref{sec:SYSTEMTIME}), 

there are also other fields available that are not used, like tm\_wday, tm\_yday, tm\_isdst.
}

\EN{\input{patterns/15_structs/3_tm_linux/ARM/main_EN}}
\RU{\input{patterns/15_structs/3_tm_linux/ARM/main_RU}}
\DE{\input{patterns/15_structs/3_tm_linux/ARM/main_DE}}
\FR{\input{patterns/15_structs/3_tm_linux/ARM/main_FR}}
\JPN{\subsubsection{ARM}

\myparagraph{\OptimizingKeilVI (\ThumbMode)}

Same example:

\lstinputlisting[caption=\OptimizingKeilVI (\ThumbMode),style=customasmARM]{patterns/15_structs/3_tm_linux/ARM/tm_ARM_keil_thumb.asm}

\myparagraph{\OptimizingXcodeIV (\ThumbTwoMode)}

\IDA \q{knows} the \TT{tm} structure 
(because \IDA \q{knows} the types of the arguments of library functions like \TT{localtime\_r()}), 

so it shows here structure elements accesses and their names.

\lstinputlisting[caption=\OptimizingXcodeIV (\ThumbTwoMode),style=customasmARM]{patterns/15_structs/3_tm_linux/ARM/tm_ARM_xcode_thumb.asm}
}

\EN{\input{patterns/15_structs/3_tm_linux/MIPS/main_EN}}
\RU{\input{patterns/15_structs/3_tm_linux/MIPS/main_RU}}
\DE{\input{patterns/15_structs/3_tm_linux/MIPS/main_DE}}
\FR{\input{patterns/15_structs/3_tm_linux/MIPS/main_FR}}
\JPN{\subsubsection{MIPS}

\lstinputlisting[caption=\Optimizing GCC 4.4.5 (IDA),numbers=left,style=customasmMIPS]{patterns/15_structs/3_tm_linux/MIPS/MIPS_O3_IDA_EN.lst}

This is an example where the branch delay slots can confuse us.

For example, there is the instruction \INS{addiu \$a1, 1900} at line 35 which adds 1900 to the year number.

It's executed before the corresponding \INS{JALR} at line 34, do not forget about it.

}

% subsection:
\EN{\input{patterns/15_structs/3_tm_linux/as_array/main_EN}}
\RU{\input{patterns/15_structs/3_tm_linux/as_array/main_RU}}
\DE{\input{patterns/15_structs/3_tm_linux/as_array/main_DE}}
\FR{\input{patterns/15_structs/3_tm_linux/as_array/main_FR}}
\JPN{\subsubsection{Structure as a set of values}

In order to illustrate that the structure is just variables laying side-by-side in one place, 
let's rework our example while looking at the \IT{tm} structure definition again: \lstref{struct_tm}.

\lstinputlisting[style=customc]{patterns/15_structs/3_tm_linux/as_array/GCC_tm2.c}

\myindex{\CStandardLibrary!localtime\_r()}
N.B. 
The pointer to the \TT{tm\_sec} field is passed into \TT{localtime\_r}, i.e., 
to the first element of the \q{structure}.

The compiler warns us:

\begin{lstlisting}[caption=GCC 4.7.3]
GCC_tm2.c: In function 'main':
GCC_tm2.c:11:5: warning: passing argument 2 of 'localtime_r' from incompatible pointer type [enabled by default]
In file included from GCC_tm2.c:2:0:
/usr/include/time.h:59:12: note: expected 'struct tm *' but argument is of type 'int *'
\end{lstlisting}

But nevertheless, it generates this:

\lstinputlisting[caption=GCC 4.7.3,style=customasmx86]{patterns/15_structs/3_tm_linux/as_array/GCC_tm2.asm}

This code is identical to what we saw previously and it is
not possible to say, was it a structure in original source code or just a pack of variables.

And this works. 
However, it is not recommended to do this in practice. 

Usually, non-optimizing compilers allocates variables in the local stack in the 
same order as they were declared in the function.

Nevertheless, there is no guarantee.

By the way, some other compiler may warn about the \TT{tm\_year}, \TT{tm\_mon}, \TT{tm\_mday},
\TT{tm\_hour}, \TT{tm\_min} variables, but not \TT{tm\_sec}
 are used without being initialized.

Indeed, the compiler is not aware that these are to be filled by\\
\TT{localtime\_r()} function.

We chose this example, since all structure fields are of type \Tint.%

This would not work if structure fields are 16-bit (\TT{WORD}), 
like in the case of the \TT{SYSTEMTIME} structure---\TT{GetSystemTime()} 
will fill them incorrectly 
(because the local variables are aligned on a 32-bit boundary).
Read more about it in next section: 
\q{\StructurePackingSectionName} (\myref{structure_packing}).

So, a structure is just a pack of variables laying in one place, side-by-side.
We could say that the structure is the instruction to the compiler, directing it to hold variables in one place.
By the way, in some very early C versions (before 1972), there were no structures at all \RitchieDevC.

There is no debugger example here: it is just the same as you already saw.

\subsubsection{Structure as an array of 32-bit words}

\lstinputlisting[style=customc]{patterns/15_structs/3_tm_linux/as_array/GCC_tm3.c}

We just \IT{cast} a pointer to structure to an array of \Tint{}'s.
And that works!
We run the example at 23:51:45 26-July-2014.

\begin{lstlisting}[label=GCC_tm3_output]
0x0000002D (45)
0x00000033 (51)
0x00000017 (23)
0x0000001A (26)
0x00000006 (6)
0x00000072 (114)
0x00000006 (6)
0x000000CE (206)
0x00000001 (1)
\end{lstlisting}

The variables here 
are in the same order as they are enumerated in the definition of the structure: \myref{struct_tm}.

Here is how it gets compiled:

\lstinputlisting[caption=\Optimizing GCC 4.8.1,style=customasmx86]{patterns/15_structs/3_tm_linux/as_array/GCC_tm3_JPN.lst}

Indeed: the space in the local stack is first treated as a structure, and then it's treated as an array.

It's even possible to modify the fields of the structure through this pointer.

And again, it's dubiously hackish way to do things, not recommended for use in production code.

\mysubparagraph{\Exercise}

As an exercise, try to modify (increase by 1) the current month number, treating the structure as 
an array.

\subsubsection{Structure as an array of bytes}

We can go even further. Let's \IT{cast} the pointer to an array of bytes and dump it:

\lstinputlisting[style=customc]{patterns/15_structs/3_tm_linux/as_array/GCC_tm4.c}

\begin{lstlisting}
0x2D 0x00 0x00 0x00 
0x33 0x00 0x00 0x00 
0x17 0x00 0x00 0x00 
0x1A 0x00 0x00 0x00 
0x06 0x00 0x00 0x00 
0x72 0x00 0x00 0x00 
0x06 0x00 0x00 0x00 
0xCE 0x00 0x00 0x00 
0x01 0x00 0x00 0x00 
\end{lstlisting}

We also run this example at 23:51:45 26-July-2014
\footnote{The time and date are the same for demonstration purposes. Byte values are fixed up.}.
The values are just the same as in the previous dump 
(\myref{GCC_tm3_output}), and of course, the lowest byte goes first, because this is a little-endian architecture 
(\myref{sec:endianness}).

\lstinputlisting[caption=\Optimizing GCC 4.8.1,style=customasmx86]{patterns/15_structs/3_tm_linux/as_array/GCC_tm4_JPN.lst}
}


\subsection{UNIX: struct tm}

% subsections here:
\EN{\input{patterns/15_structs/3_tm_linux/linux_EN}}
\RU{\input{patterns/15_structs/3_tm_linux/linux_RU}}
\DE{\input{patterns/15_structs/3_tm_linux/linux_DE}}
\FR{\input{patterns/15_structs/3_tm_linux/linux_FR}}
\JPN{\subsubsection{Linux}

Let's take the \TT{tm} structure from \TT{time.h} in Linux for example:

\lstinputlisting[style=customc]{patterns/15_structs/3_tm_linux/GCC_tm.c}

Let's compile it in GCC 4.4.1:

\lstinputlisting[caption=GCC 4.4.1,style=customasmx86]{patterns/15_structs/3_tm_linux/GCC_tm_EN.asm}

Somehow, \IDA did not write the local variables' names in the local stack.
But since we already are experienced reverse engineers :-) we may do it without this information in 
this simple example.

\myindex{x86!\Instructions!LEA}

Please also pay attention to the \TT{lea edx, [eax+76Ch]}~---this instruction just adds \TT{0x76C} (1900) to value in \EAX,
but doesn't modify any flags. See also the relevant section about \LEA{}~(\myref{sec:LEA}).

\myparagraph{GDB}

Let's try to load the example into GDB
\footnote{The \IT{date} result is slightly corrected for demonstration purposes.
Of course, it's not possible to run GDB that quickly, in the same second.}:

\lstinputlisting[caption=GDB]{patterns/15_structs/3_tm_linux/GCC_tm_GDB.txt}

We can easily find our structure in the stack.
First, let's see how it's defined in \IT{time.h}:

\begin{lstlisting}[caption=time.h, label=struct_tm,style=customc]
struct tm
{
  int	tm_sec;
  int	tm_min;
  int	tm_hour;
  int	tm_mday;
  int	tm_mon;
  int	tm_year;
  int	tm_wday;
  int	tm_yday;
  int	tm_isdst;
};
\end{lstlisting}

Pay attention that
32-bit \Tint is used here instead of WORD in SYSTEMTIME.
So, each field occupies 32-bit.

Here are the fields of our structure in the stack:

\begin{lstlisting}
0xbffff0dc:	0x080484c3	0x080485c0	0x000007de	0x00000000
0xbffff0ec:	0x08048301	0x538c93ed	0x00000025 sec	0x0000000a min
0xbffff0fc:	0x00000012 hour	0x00000002 mday	0x00000005 mon 	0x00000072 year
0xbffff10c:	0x00000001 wday	0x00000098 yday	0x00000001 isdst0x00002a30
0xbffff11c:	0x0804b090	0x08048530	0x00000000	0x00000000
\end{lstlisting}

Or as a table:

\begin{center}
\begin{tabular}{ | l | l | l | }
\hline
\headercolor{} Hexadecimal number & 
\headercolor{} decimal number & 
\headercolor{} field name \\
\hline
0x00000025 & 37 	& tm\_sec \\
\hline
0x0000000a & 10 	& tm\_min \\
\hline
0x00000012 & 18 	& tm\_hour \\	
\hline
0x00000002 & 2 		& tm\_mday \\	
\hline
0x00000005 & 5 		& tm\_mon \\	
\hline
0x00000072 & 114 	& tm\_year \\
\hline
0x00000001 & 1 		& tm\_wday \\	
\hline
0x00000098 & 152 	& tm\_yday \\	
\hline
0x00000001 & 1 		& tm\_isdst \\
\hline
\end{tabular}
\end{center}

Just like SYSTEMTIME (\myref{sec:SYSTEMTIME}), 

there are also other fields available that are not used, like tm\_wday, tm\_yday, tm\_isdst.
}

\EN{\input{patterns/15_structs/3_tm_linux/ARM/main_EN}}
\RU{\input{patterns/15_structs/3_tm_linux/ARM/main_RU}}
\DE{\input{patterns/15_structs/3_tm_linux/ARM/main_DE}}
\FR{\input{patterns/15_structs/3_tm_linux/ARM/main_FR}}
\JPN{\subsubsection{ARM}

\myparagraph{\OptimizingKeilVI (\ThumbMode)}

Same example:

\lstinputlisting[caption=\OptimizingKeilVI (\ThumbMode),style=customasmARM]{patterns/15_structs/3_tm_linux/ARM/tm_ARM_keil_thumb.asm}

\myparagraph{\OptimizingXcodeIV (\ThumbTwoMode)}

\IDA \q{knows} the \TT{tm} structure 
(because \IDA \q{knows} the types of the arguments of library functions like \TT{localtime\_r()}), 

so it shows here structure elements accesses and their names.

\lstinputlisting[caption=\OptimizingXcodeIV (\ThumbTwoMode),style=customasmARM]{patterns/15_structs/3_tm_linux/ARM/tm_ARM_xcode_thumb.asm}
}

\EN{\input{patterns/15_structs/3_tm_linux/MIPS/main_EN}}
\RU{\input{patterns/15_structs/3_tm_linux/MIPS/main_RU}}
\DE{\input{patterns/15_structs/3_tm_linux/MIPS/main_DE}}
\FR{\input{patterns/15_structs/3_tm_linux/MIPS/main_FR}}
\JPN{\subsubsection{MIPS}

\lstinputlisting[caption=\Optimizing GCC 4.4.5 (IDA),numbers=left,style=customasmMIPS]{patterns/15_structs/3_tm_linux/MIPS/MIPS_O3_IDA_EN.lst}

This is an example where the branch delay slots can confuse us.

For example, there is the instruction \INS{addiu \$a1, 1900} at line 35 which adds 1900 to the year number.

It's executed before the corresponding \INS{JALR} at line 34, do not forget about it.

}

% subsection:
\EN{\input{patterns/15_structs/3_tm_linux/as_array/main_EN}}
\RU{\input{patterns/15_structs/3_tm_linux/as_array/main_RU}}
\DE{\input{patterns/15_structs/3_tm_linux/as_array/main_DE}}
\FR{\input{patterns/15_structs/3_tm_linux/as_array/main_FR}}
\JPN{\subsubsection{Structure as a set of values}

In order to illustrate that the structure is just variables laying side-by-side in one place, 
let's rework our example while looking at the \IT{tm} structure definition again: \lstref{struct_tm}.

\lstinputlisting[style=customc]{patterns/15_structs/3_tm_linux/as_array/GCC_tm2.c}

\myindex{\CStandardLibrary!localtime\_r()}
N.B. 
The pointer to the \TT{tm\_sec} field is passed into \TT{localtime\_r}, i.e., 
to the first element of the \q{structure}.

The compiler warns us:

\begin{lstlisting}[caption=GCC 4.7.3]
GCC_tm2.c: In function 'main':
GCC_tm2.c:11:5: warning: passing argument 2 of 'localtime_r' from incompatible pointer type [enabled by default]
In file included from GCC_tm2.c:2:0:
/usr/include/time.h:59:12: note: expected 'struct tm *' but argument is of type 'int *'
\end{lstlisting}

But nevertheless, it generates this:

\lstinputlisting[caption=GCC 4.7.3,style=customasmx86]{patterns/15_structs/3_tm_linux/as_array/GCC_tm2.asm}

This code is identical to what we saw previously and it is
not possible to say, was it a structure in original source code or just a pack of variables.

And this works. 
However, it is not recommended to do this in practice. 

Usually, non-optimizing compilers allocates variables in the local stack in the 
same order as they were declared in the function.

Nevertheless, there is no guarantee.

By the way, some other compiler may warn about the \TT{tm\_year}, \TT{tm\_mon}, \TT{tm\_mday},
\TT{tm\_hour}, \TT{tm\_min} variables, but not \TT{tm\_sec}
 are used without being initialized.

Indeed, the compiler is not aware that these are to be filled by\\
\TT{localtime\_r()} function.

We chose this example, since all structure fields are of type \Tint.%

This would not work if structure fields are 16-bit (\TT{WORD}), 
like in the case of the \TT{SYSTEMTIME} structure---\TT{GetSystemTime()} 
will fill them incorrectly 
(because the local variables are aligned on a 32-bit boundary).
Read more about it in next section: 
\q{\StructurePackingSectionName} (\myref{structure_packing}).

So, a structure is just a pack of variables laying in one place, side-by-side.
We could say that the structure is the instruction to the compiler, directing it to hold variables in one place.
By the way, in some very early C versions (before 1972), there were no structures at all \RitchieDevC.

There is no debugger example here: it is just the same as you already saw.

\subsubsection{Structure as an array of 32-bit words}

\lstinputlisting[style=customc]{patterns/15_structs/3_tm_linux/as_array/GCC_tm3.c}

We just \IT{cast} a pointer to structure to an array of \Tint{}'s.
And that works!
We run the example at 23:51:45 26-July-2014.

\begin{lstlisting}[label=GCC_tm3_output]
0x0000002D (45)
0x00000033 (51)
0x00000017 (23)
0x0000001A (26)
0x00000006 (6)
0x00000072 (114)
0x00000006 (6)
0x000000CE (206)
0x00000001 (1)
\end{lstlisting}

The variables here 
are in the same order as they are enumerated in the definition of the structure: \myref{struct_tm}.

Here is how it gets compiled:

\lstinputlisting[caption=\Optimizing GCC 4.8.1,style=customasmx86]{patterns/15_structs/3_tm_linux/as_array/GCC_tm3_JPN.lst}

Indeed: the space in the local stack is first treated as a structure, and then it's treated as an array.

It's even possible to modify the fields of the structure through this pointer.

And again, it's dubiously hackish way to do things, not recommended for use in production code.

\mysubparagraph{\Exercise}

As an exercise, try to modify (increase by 1) the current month number, treating the structure as 
an array.

\subsubsection{Structure as an array of bytes}

We can go even further. Let's \IT{cast} the pointer to an array of bytes and dump it:

\lstinputlisting[style=customc]{patterns/15_structs/3_tm_linux/as_array/GCC_tm4.c}

\begin{lstlisting}
0x2D 0x00 0x00 0x00 
0x33 0x00 0x00 0x00 
0x17 0x00 0x00 0x00 
0x1A 0x00 0x00 0x00 
0x06 0x00 0x00 0x00 
0x72 0x00 0x00 0x00 
0x06 0x00 0x00 0x00 
0xCE 0x00 0x00 0x00 
0x01 0x00 0x00 0x00 
\end{lstlisting}

We also run this example at 23:51:45 26-July-2014
\footnote{The time and date are the same for demonstration purposes. Byte values are fixed up.}.
The values are just the same as in the previous dump 
(\myref{GCC_tm3_output}), and of course, the lowest byte goes first, because this is a little-endian architecture 
(\myref{sec:endianness}).

\lstinputlisting[caption=\Optimizing GCC 4.8.1,style=customasmx86]{patterns/15_structs/3_tm_linux/as_array/GCC_tm4_JPN.lst}
}


\input{appendix/GCC_library}
\input{appendix/MSVC_library}
\input{appendix/cheatsheets}
}
\DE{\part*{\RU{Приложение}\EN{Appendix}\DE{Anhang}\FR{Appendice}\JPN{付録}}
\appendix
\addcontentsline{toc}{part}{\RU{Приложение}\EN{Appendix}\DE{Anhang}\FR{Appendice}\JPN{付録}}

% chapters
\EN{\input{appendix/x86/main_EN}}
\RU{\input{appendix/x86/main_RU}}
\DE{\input{appendix/x86/main_DE}}
\FR{\input{appendix/x86/main_FR}}
\JPN{\mysection{x86}

\subsection{Terminology}

Common for 16-bit (8086/80286), 32-bit (80386, etc.), 64-bit.

\myindex{IEEE 754}
\myindex{MS-DOS}
\begin{description}
	\item[byte] 8-bit.
		The DB assembly directive is used for defining variables and arrays of bytes.
		Bytes are passed in the 8-bit part of registers: \TT{AL/BL/CL/DL/AH/BH/CH/DH/SIL/DIL/R*L}.
	\item[word] 16-bit. 
		DW assembly directive \dittoclosing.
		Words are passed in the 16-bit part of the registers:\\
			\TT{AX/BX/CX/DX/SI/DI/R*W}.
	\item[double word] (\q{dword}) 32-bit.
		DD assembly directive \dittoclosing.
		Double words are passed in registers (x86) or in the 32-bit part of registers (x64). 
		In 16-bit code, double words are passed in 16-bit register pairs.
	\item[quad word] (\q{qword}) 64-bit.
		DQ assembly directive \dittoclosing.
		In 32-bit environment, quad words are passed in 32-bit register pairs.
	\item[tbyte] (10 bytes) 80-bit or 10 bytes (used for IEEE 754 FPU registers).
	\item[paragraph] (16 bytes)---term was popular in MS-DOS environment. % TODO link to a part about 8086 memory model...
\end{description}

\myindex{Windows!API}

Data types of the same width (BYTE, WORD, DWORD) are also the same in Windows \ac{API}.

\input{appendix/x86/registers} % subsection
\subsection{\RU{Инструкции}\EN{Instructions}}
\label{sec:x86_instructions}

\RU{Инструкции, отмеченные как (M) обычно не генерируются компилятором: если вы видите её, очень может быть
это вручную написанный фрагмент кода, либо это т.н. compiler intrinsic}
\EN{Instructions marked as (M) are not usually generated by the compiler: if you see one of them, it is probably
a hand-written piece of assembly code, or a compiler intrinsic} (\myref{sec:compiler_intrinsic}).

% TODO ? обратные инструкции

\RU{Только наиболее используемые инструкции перечислены здесь}
\EN{Only the most frequently used instructions are listed here}.
\EN{You can read \myref{x86_manuals} for a full documentation.}%
\RU{Обращайтесь к \myref{x86_manuals} для полной документации.}

\RU{Нужно ли заучивать опкоды инструкций на память?}\EN{Do you have to know all instruction's opcodes by heart?}
\RU{Нет, только те, которые часто используются для модификации кода}\EN{No, only those
which are used for code patching} (\myref{x86_patching}).
\RU{Остальные запоминать нет смысла.}\EN{All the rest of the opcodes don't need to be memorized.}

\subsubsection{\RU{Префиксы}\EN{Prefixes}}

\myindex{x86!\Prefixes!LOCK}
\myindex{x86!\Prefixes!REP}
\myindex{x86!\Prefixes!REPE/REPNE}
\begin{description}
\label{x86_lock}
\item[LOCK] \RU{используется чтобы предоставить эксклюзивный доступ к памяти в многопроцессорной среде}
\EN{forces CPU to make exclusive access to the RAM in multiprocessor environment}.
\RU{Для упрощения, можно сказать, что когда исполняется инструкция с этим префиксом, остальные процессоры
в системе останавливаются}\EN{For the sake of simplification, it can be said that when an instruction
with this prefix is executed, all other CPUs in a multiprocessor system are stopped}.
\RU{Чаще все это используется для критических секций, семафоров, мьютексов}\EN{Most often
it is used for critical sections, semaphores, mutexes}.
\RU{Обычно используется с}\EN{Commonly used with} ADD, AND, BTR, BTS, CMPXCHG, OR, XADD, XOR.
\RU{Читайте больше о критических секциях}\EN{You can read more about critical sections here} (\myref{critical_sections}).

\item[REP] \RU{используется с инструкциями}\EN{is used with the} MOVSx \AndENRU STOSx\EN{ instructions}:
\RU{инструкция будет исполняться в цикле, счетчик расположен в регистре CX/ECX/RCX}
\EN{execute the instruction in a loop, the counter is located in the CX/ECX/RCX register}.
\RU{Для более детального описания, читайте больше об инструкциях}
\EN{For a detailed description, read more about the} MOVSx (\myref{REP_MOVSx}) 
\AndENRU STOSx (\myref{REP_STOSx})\EN{ instructions}.

\RU{Работа инструкций с префиксом REP зависит от флага DF, он задает направление}
\EN{The instructions prefixed by REP are sensitive to the DF flag, which is used to set the direction}.

\item[REPE/REPNE] (\ac{AKA} REPZ/REPNZ) \RU{используется с инструкциями}\EN{used with} CMPSx \AndENRU
SCASx\EN{ instructions}:
\RU{инструкция будет исполняться в цикле, счетчик расположен в регистре \TT{CX}/\TT{ECX}/\TT{RCX}}
\EN{execute the last instruction in a loop, the count is set in the \TT{CX}/\TT{ECX}/\TT{RCX} register}. 
\RU{Выполнение будет прервано если ZF будет 0 (REPE) либо если ZF будет 1 (REPNE)}
\EN{It terminates prematurely if ZF is 0 (REPE) or if ZF is 1 (REPNE)}.

\RU{Для более детального описания, читайте больше об инструкциях}
\EN{For a detailed description, you can read more about the} CMPSx (\myref{REPE_CMPSx}) 
\AndENRU SCASx (\myref{REPNE_SCASx})\EN{ instructions}.

\RU{Работа инструкций с префиксами REPE/REPNE зависит от флага DF, он задает направление}
\EN{Instructions prefixed by REPE/REPNE are sensitive to the DF flag, which is used to set the direction}.

\end{description}

\subsubsection{\RU{Наиболее часто используемые инструкции}\EN{Most frequently used instructions}}

\RU{Их можно заучить в первую очередь}\EN{These can be memorized in the first place}.

\begin{description}
% in order to keep them easily sorted...
\input{appendix/x86/instructions/ADC}
\input{appendix/x86/instructions/ADD}
\input{appendix/x86/instructions/AND}
\input{appendix/x86/instructions/CALL}
\input{appendix/x86/instructions/CMP}
\input{appendix/x86/instructions/DEC}
\input{appendix/x86/instructions/IMUL}
\input{appendix/x86/instructions/INC}
\input{appendix/x86/instructions/JCXZ}
\input{appendix/x86/instructions/JMP}
\input{appendix/x86/instructions/Jcc}
\input{appendix/x86/instructions/LAHF}
\input{appendix/x86/instructions/LEAVE}
\input{appendix/x86/instructions/LEA}
\input{appendix/x86/instructions/MOVSB_W_D_Q}
\input{appendix/x86/instructions/MOVSX}
\input{appendix/x86/instructions/MOVZX}
\input{appendix/x86/instructions/MOV}
\input{appendix/x86/instructions/MUL}
\input{appendix/x86/instructions/NEG}
\input{appendix/x86/instructions/NOP}
\input{appendix/x86/instructions/NOT}
\input{appendix/x86/instructions/OR}
\input{appendix/x86/instructions/POP}
\input{appendix/x86/instructions/PUSH}
\input{appendix/x86/instructions/RET}
\input{appendix/x86/instructions/SAHF}
\input{appendix/x86/instructions/SBB}
\input{appendix/x86/instructions/SCASB_W_D_Q}
\input{appendix/x86/instructions/SHx}
\input{appendix/x86/instructions/SHRD}
\input{appendix/x86/instructions/STOSB_W_D_Q}
\input{appendix/x86/instructions/SUB}
\input{appendix/x86/instructions/TEST}
\input{appendix/x86/instructions/XOR}
\end{description}

\subsubsection{\RU{Реже используемые инструкции}\EN{Less frequently used instructions}}

\begin{description}
\input{appendix/x86/instructions/BSF}
\input{appendix/x86/instructions/BSR}
\input{appendix/x86/instructions/BSWAP}
\input{appendix/x86/instructions/BTC}
\input{appendix/x86/instructions/BTR}
\input{appendix/x86/instructions/BTS}
\input{appendix/x86/instructions/BT}
\input{appendix/x86/instructions/CBW_CWDE_CDQ}
\input{appendix/x86/instructions/CLD}
\input{appendix/x86/instructions/CLI}
\input{appendix/x86/instructions/CMC}
\input{appendix/x86/instructions/CMOVcc}
\input{appendix/x86/instructions/CMPSB_W_D_Q}
\input{appendix/x86/instructions/CPUID}
\input{appendix/x86/instructions/DIV}
\input{appendix/x86/instructions/IDIV}
\input{appendix/x86/instructions/INT}
\input{appendix/x86/instructions/IN}
\input{appendix/x86/instructions/IRET}
\input{appendix/x86/instructions/LOOP}
\input{appendix/x86/instructions/OUT}
\input{appendix/x86/instructions/POPA}
\input{appendix/x86/instructions/POPCNT}
\input{appendix/x86/instructions/POPF}
\input{appendix/x86/instructions/PUSHA}
\input{appendix/x86/instructions/PUSHF}
\input{appendix/x86/instructions/RCx}
\input{appendix/x86/instructions/ROx}
\input{appendix/x86/instructions/SAL}
\input{appendix/x86/instructions/SAR}
\input{appendix/x86/instructions/SETcc}
\input{appendix/x86/instructions/STC}
\input{appendix/x86/instructions/STD}
\input{appendix/x86/instructions/STI}
\input{appendix/x86/instructions/SYSCALL}
\input{appendix/x86/instructions/SYSENTER}
\input{appendix/x86/instructions/UD2}
\input{appendix/x86/instructions/XCHG}
\end{description}

\subsubsection{\RU{Инструкции FPU}\EN{FPU instructions}}

\RU{Суффикс \TT{-R} в названии инструкции обычно означает, что операнды поменяны местами, суффикс \TT{-P} означает
что один элемент выталкивается из стека после исполнения инструкции, суффикс \TT{-PP} означает, что
выталкиваются два элемента}%
\EN{\TT{-R} suffix in the mnemonic usually implies that the operands are reversed,
\TT{-P} suffix implies that one element is popped
from the stack after the instruction's execution, \TT{-PP} suffix implies that two elements are popped}.

\TT{-P} \RU{инструкции часто бывают полезны, когда нам уже больше не нужно хранить значение в 
FPU-стеке после операции.}%
\EN{instructions are often useful when we do not need the value in the FPU stack to be 
present anymore after the operation.}

\begin{description}
\input{appendix/x86/instructions/FABS}
\input{appendix/x86/instructions/FADD} % + FADDP
\input{appendix/x86/instructions/FCHS}
\input{appendix/x86/instructions/FCOM} % + FCOMP + FCOMPP
\input{appendix/x86/instructions/FDIVR} % + FDIVRP
\input{appendix/x86/instructions/FDIV} % + FDIVP
\input{appendix/x86/instructions/FILD}
\input{appendix/x86/instructions/FIST} % + FISTP
\input{appendix/x86/instructions/FLD1}
\input{appendix/x86/instructions/FLDCW}
\input{appendix/x86/instructions/FLDZ}
\input{appendix/x86/instructions/FLD}
\input{appendix/x86/instructions/FMUL} % + FMULP
\input{appendix/x86/instructions/FSINCOS}
\input{appendix/x86/instructions/FSQRT}
\input{appendix/x86/instructions/FSTCW} % + FNSTCW
\input{appendix/x86/instructions/FSTSW} % + FNSTSW
\input{appendix/x86/instructions/FST}
\input{appendix/x86/instructions/FSUBR} % + FSUBRP
\input{appendix/x86/instructions/FSUB} % + FSUBP
\input{appendix/x86/instructions/FUCOM} % + FUCOMP + FUCOMPP
\input{appendix/x86/instructions/FXCH}
\end{description}

%\subsubsection{\RU{SIMD-инструкции}\EN{SIMD instructions}}

% TODO

%\begin{description}
%\input{appendix/x86/instructions/DIVSD}
%\input{appendix/x86/instructions/MOVDQA}
%\input{appendix/x86/instructions/MOVDQU}
%\input{appendix/x86/instructions/PADDD}
%\input{appendix/x86/instructions/PCMPEQB}
%\input{appendix/x86/instructions/PLMULHW}
%\input{appendix/x86/instructions/PLMULLD}
%\input{appendix/x86/instructions/PMOVMSKB}
%\input{appendix/x86/instructions/PXOR}
%\end{description}

% SHLD !
% SHRD !
% BSWAP !
% CMPXCHG
% XADD !
% CMPXCHG8B
% RDTSC !
% PAUSE!

% xsave
% fnclex, fnsave
% movsxd, movaps, wait, sfence, lfence, pushfq
% prefetchw
% REP RETN
% REP BSF
% movnti, movntdq, rdmsr, wrmsr
% ldmxcsr, stmxcsr, invlpg
% swapgs
% movq, movd
% mulsd
% POR
% IRETQ
% pslldq
% psrldq
% cqo, fxrstor, comisd, xrstor, wbinvd, movntq
% fprem
% addsb, subsd, frndint

% rare:
%\item[ENTER]
%\item[LES]
% LDS
% XLAT

\subsubsection{\RU{Инструкции с печатаемым ASCII-опкодом}\EN{Instructions having printable ASCII opcode}}

(\RU{В 32-битном режиме}\EN{In 32-bit mode}).

\label{printable_x86_opcodes}
\myindex{Shellcode}
\RU{Это может пригодиться для создания шеллкодов}\EN{These can be suitable for shellcode construction}.
\RU{См. также}\EN{See also}: \myref{subsec:EICAR}.

% FIXME: break table
% FIXME: start at 0x20...
\begin{center}
\begin{longtable}{ | l | l | l | }
\hline
\HeaderColor ASCII\RU{-символ}\EN{ character} & 
\HeaderColor \RU{шестнадцатеричный код}\EN{hexadecimal code} & 
\HeaderColor x86\RU{-инструкция}\EN{ instruction} \\
\hline
0	 &30	 &XOR \\
1	 &31	 &XOR \\
2	 &32	 &XOR \\
3	 &33	 &XOR \\
4	 &34	 &XOR \\
5	 &35	 &XOR \\
7	 &37	 &AAA \\
8	 &38	 &CMP \\
9	 &39	 &CMP \\
:	 &3a	 &CMP \\
;	 &3b	 &CMP \\
<	 &3c	 &CMP \\
=	 &3d	 &CMP \\
?	 &3f	 &AAS \\
@	 &40	 &INC \\
A	 &41	 &INC \\
B	 &42	 &INC \\
C	 &43	 &INC \\
D	 &44	 &INC \\
E	 &45	 &INC \\
F	 &46	 &INC \\
G	 &47	 &INC \\
H	 &48	 &DEC \\
I	 &49	 &DEC \\
J	 &4a	 &DEC \\
K	 &4b	 &DEC \\
L	 &4c	 &DEC \\
M	 &4d	 &DEC \\
N	 &4e	 &DEC \\
O	 &4f	 &DEC \\
P	 &50	 &PUSH \\
Q	 &51	 &PUSH \\
R	 &52	 &PUSH \\
S	 &53	 &PUSH \\
T	 &54	 &PUSH \\
U	 &55	 &PUSH \\
V	 &56	 &PUSH \\
W	 &57	 &PUSH \\
X	 &58	 &POP \\
Y	 &59	 &POP \\
Z	 &5a	 &POP \\
\lbrack{}	 &5b	 &POP \\
\textbackslash{}	 &5c	 &POP \\
\rbrack{}	 &5d	 &POP \\
\verb|^|	 &5e	 &POP \\
\_	 &5f	 &POP \\
\verb|`|	 &60	 &PUSHA \\
a	 &61	 &POPA \\
f	 &66	 &\RU{(в 32-битном режиме) переключиться на}\EN{(in 32-bit mode) switch to}\\
   & & \RU{16-битный размер операнда}\EN{16-bit operand size} \\
g	 &67	 &\RU{(в 32-битном режиме) переключиться на}\EN{in 32-bit mode) switch to}\\
   & & \RU{16-битный размер адреса}\EN{16-bit address size} \\
h	 &68	 &PUSH\\
i	 &69	 &IMUL\\
j	 &6a	 &PUSH\\
k	 &6b	 &IMUL\\
p	 &70	 &JO\\
q	 &71	 &JNO\\
r	 &72	 &JB\\
s	 &73	 &JAE\\
t	 &74	 &JE\\
u	 &75	 &JNE\\
v	 &76	 &JBE\\
w	 &77	 &JA\\
x	 &78	 &JS\\
y	 &79	 &JNS\\
z	 &7a	 &JP\\
\hline
\end{longtable}
\end{center}

\myindex{x86!\Instructions!AAA}
\myindex{x86!\Instructions!AAS}
\myindex{x86!\Instructions!CMP}
\myindex{x86!\Instructions!DEC}
\myindex{x86!\Instructions!IMUL}
\myindex{x86!\Instructions!INC}
\myindex{x86!\Instructions!JA}
\myindex{x86!\Instructions!JAE}
\myindex{x86!\Instructions!JB}
\myindex{x86!\Instructions!JBE}
\myindex{x86!\Instructions!JE}
\myindex{x86!\Instructions!JNE}
\myindex{x86!\Instructions!JNO}
\myindex{x86!\Instructions!JNS}
\myindex{x86!\Instructions!JO}
\myindex{x86!\Instructions!JP}
\myindex{x86!\Instructions!JS}
\myindex{x86!\Instructions!POP}
\myindex{x86!\Instructions!POPA}
\myindex{x86!\Instructions!PUSH}
\myindex{x86!\Instructions!PUSHA}
\myindex{x86!\Instructions!XOR}

\RU{В итоге}\EN{In summary}:
AAA, AAS, CMP, DEC, IMUL, INC, JA, JAE, JB, JBE, JE, JNE, JNO, JNS, JO, JP, JS, POP, POPA, PUSH, PUSHA, 
XOR.

 % subsection
\subsection{npad}
\label{sec:npad}

\RU{Это макрос в ассемблере, для выравнивания некоторой метки по некоторой границе.}
\EN{It is an assembly language macro for aligning labels on a specific boundary.}
\DE{Dies ist ein Assembler-Makro um Labels an bestimmten Grenzen auszurichten.}
\FR{C'est une macro du langage d'assemblage pour aligner les labels sur une limite
spécifique.}
\JPN{It is an assembly language macro for aligning labels on a specific boundary.}

\RU{Это нужно для тех \IT{нагруженных} меток, куда чаще всего передается управление, например, 
начало тела цикла. 
Для того чтобы процессор мог эффективнее вытягивать данные или код из памяти, через шину с памятью, 
кэширование, итд.}
\EN{That's often needed for the busy labels to where the control flow is often passed, e.g., loop body starts.
So the CPU can load the data or code from the memory effectively, through the memory bus, cache lines, etc.}
\DE{Dies ist oft nützlich Labels, die oft Ziel einer Kotrollstruktur sind, wie Schleifenköpfe.
Somit kann die CPU Daten oder Code sehr effizient vom Speicher durch den Bus, den Cache, usw. laden.}
\FR{C'est souvent necessaire pour des labels très utilisés, comme par exemple le
début d'un corps de boucle. Ainsi, le CPU peut charger les données ou le code depuis
la mémoire efficacement, à travers le bus mémoire, les caches, etc.}

\RU{Взято из}\EN{Taken from}\DE{Entnommen von}\FR{Pris de} \TT{listing.inc} (MSVC):

\myindex{x86!\Instructions!NOP}
\RU{Это, кстати, любопытный пример различных вариантов \NOP{}-ов. 
Все эти инструкции не дают никакого эффекта, но отличаются разной длиной.}
\EN{By the way, it is a curious example of the different \NOP variations.
All these instructions have no effects whatsoever, but have a different size.}
\DE{Dies ist übrigens ein Beispiel für die unterschiedlichen \NOP-Variationen.
Keine dieser Anweisungen hat eine Auswirkung, aber alle haben eine unterschiedliche Größe.}
\FR{À propos, c'est un exemple curieux des différentes variations de \NOP. Toutes
ces instructions n'ont pas d'effet, mais ont une taille différente.}

\RU{Цель в том, чтобы была только одна инструкция, а не набор NOP-ов, 
считается что так лучше для производительности CPU.}
\EN{Having a single idle instruction instead of couple of NOP-s,
is accepted to be better for CPU performance.}
\DE{Eine einzelne Idle-Anweisung anstatt mehrerer NOPs hat positive Auswirkungen
auf die CPU-Performance.}
\FR{Avoir une seule instruction sans effet au lieu de plusieurs est accepté comme
étant meilleur pour la performance du CPU.}

\begin{lstlisting}[style=customasmx86]
;; LISTING.INC
;;
;; This file contains assembler macros and is included by the files created
;; with the -FA compiler switch to be assembled by MASM (Microsoft Macro
;; Assembler).
;;
;; Copyright (c) 1993-2003, Microsoft Corporation. All rights reserved.

;; non destructive nops
npad macro size
if size eq 1
  nop
else
 if size eq 2
   mov edi, edi
 else
  if size eq 3
    ; lea ecx, [ecx+00]
    DB 8DH, 49H, 00H
  else
   if size eq 4
     ; lea esp, [esp+00]
     DB 8DH, 64H, 24H, 00H
   else
    if size eq 5
      add eax, DWORD PTR 0
    else
     if size eq 6
       ; lea ebx, [ebx+00000000]
       DB 8DH, 9BH, 00H, 00H, 00H, 00H
     else
      if size eq 7
	; lea esp, [esp+00000000]
	DB 8DH, 0A4H, 24H, 00H, 00H, 00H, 00H 
      else
       if size eq 8
        ; jmp .+8; .npad 6
	DB 0EBH, 06H, 8DH, 9BH, 00H, 00H, 00H, 00H
       else
        if size eq 9
         ; jmp .+9; .npad 7
         DB 0EBH, 07H, 8DH, 0A4H, 24H, 00H, 00H, 00H, 00H
        else
         if size eq 10
          ; jmp .+A; .npad 7; .npad 1
          DB 0EBH, 08H, 8DH, 0A4H, 24H, 00H, 00H, 00H, 00H, 90H
         else
          if size eq 11
           ; jmp .+B; .npad 7; .npad 2
           DB 0EBH, 09H, 8DH, 0A4H, 24H, 00H, 00H, 00H, 00H, 8BH, 0FFH
          else
           if size eq 12
            ; jmp .+C; .npad 7; .npad 3
            DB 0EBH, 0AH, 8DH, 0A4H, 24H, 00H, 00H, 00H, 00H, 8DH, 49H, 00H
           else
            if size eq 13
             ; jmp .+D; .npad 7; .npad 4
             DB 0EBH, 0BH, 8DH, 0A4H, 24H, 00H, 00H, 00H, 00H, 8DH, 64H, 24H, 00H
            else
             if size eq 14
              ; jmp .+E; .npad 7; .npad 5
              DB 0EBH, 0CH, 8DH, 0A4H, 24H, 00H, 00H, 00H, 00H, 05H, 00H, 00H, 00H, 00H
             else
              if size eq 15
               ; jmp .+F; .npad 7; .npad 6
               DB 0EBH, 0DH, 8DH, 0A4H, 24H, 00H, 00H, 00H, 00H, 8DH, 9BH, 00H, 00H, 00H, 00H
              else
	       %out error: unsupported npad size
               .err
              endif
             endif
            endif
           endif
          endif
         endif
        endif
       endif
      endif
     endif
    endif
   endif
  endif
 endif
endif
endm
\end{lstlisting}
 % subsection

}
\subsection{UNIX: struct tm}

% subsections here:
\EN{\input{patterns/15_structs/3_tm_linux/linux_EN}}
\RU{\input{patterns/15_structs/3_tm_linux/linux_RU}}
\DE{\input{patterns/15_structs/3_tm_linux/linux_DE}}
\FR{\input{patterns/15_structs/3_tm_linux/linux_FR}}
\JPN{\subsubsection{Linux}

Let's take the \TT{tm} structure from \TT{time.h} in Linux for example:

\lstinputlisting[style=customc]{patterns/15_structs/3_tm_linux/GCC_tm.c}

Let's compile it in GCC 4.4.1:

\lstinputlisting[caption=GCC 4.4.1,style=customasmx86]{patterns/15_structs/3_tm_linux/GCC_tm_EN.asm}

Somehow, \IDA did not write the local variables' names in the local stack.
But since we already are experienced reverse engineers :-) we may do it without this information in 
this simple example.

\myindex{x86!\Instructions!LEA}

Please also pay attention to the \TT{lea edx, [eax+76Ch]}~---this instruction just adds \TT{0x76C} (1900) to value in \EAX,
but doesn't modify any flags. See also the relevant section about \LEA{}~(\myref{sec:LEA}).

\myparagraph{GDB}

Let's try to load the example into GDB
\footnote{The \IT{date} result is slightly corrected for demonstration purposes.
Of course, it's not possible to run GDB that quickly, in the same second.}:

\lstinputlisting[caption=GDB]{patterns/15_structs/3_tm_linux/GCC_tm_GDB.txt}

We can easily find our structure in the stack.
First, let's see how it's defined in \IT{time.h}:

\begin{lstlisting}[caption=time.h, label=struct_tm,style=customc]
struct tm
{
  int	tm_sec;
  int	tm_min;
  int	tm_hour;
  int	tm_mday;
  int	tm_mon;
  int	tm_year;
  int	tm_wday;
  int	tm_yday;
  int	tm_isdst;
};
\end{lstlisting}

Pay attention that
32-bit \Tint is used here instead of WORD in SYSTEMTIME.
So, each field occupies 32-bit.

Here are the fields of our structure in the stack:

\begin{lstlisting}
0xbffff0dc:	0x080484c3	0x080485c0	0x000007de	0x00000000
0xbffff0ec:	0x08048301	0x538c93ed	0x00000025 sec	0x0000000a min
0xbffff0fc:	0x00000012 hour	0x00000002 mday	0x00000005 mon 	0x00000072 year
0xbffff10c:	0x00000001 wday	0x00000098 yday	0x00000001 isdst0x00002a30
0xbffff11c:	0x0804b090	0x08048530	0x00000000	0x00000000
\end{lstlisting}

Or as a table:

\begin{center}
\begin{tabular}{ | l | l | l | }
\hline
\headercolor{} Hexadecimal number & 
\headercolor{} decimal number & 
\headercolor{} field name \\
\hline
0x00000025 & 37 	& tm\_sec \\
\hline
0x0000000a & 10 	& tm\_min \\
\hline
0x00000012 & 18 	& tm\_hour \\	
\hline
0x00000002 & 2 		& tm\_mday \\	
\hline
0x00000005 & 5 		& tm\_mon \\	
\hline
0x00000072 & 114 	& tm\_year \\
\hline
0x00000001 & 1 		& tm\_wday \\	
\hline
0x00000098 & 152 	& tm\_yday \\	
\hline
0x00000001 & 1 		& tm\_isdst \\
\hline
\end{tabular}
\end{center}

Just like SYSTEMTIME (\myref{sec:SYSTEMTIME}), 

there are also other fields available that are not used, like tm\_wday, tm\_yday, tm\_isdst.
}

\EN{\input{patterns/15_structs/3_tm_linux/ARM/main_EN}}
\RU{\input{patterns/15_structs/3_tm_linux/ARM/main_RU}}
\DE{\input{patterns/15_structs/3_tm_linux/ARM/main_DE}}
\FR{\input{patterns/15_structs/3_tm_linux/ARM/main_FR}}
\JPN{\subsubsection{ARM}

\myparagraph{\OptimizingKeilVI (\ThumbMode)}

Same example:

\lstinputlisting[caption=\OptimizingKeilVI (\ThumbMode),style=customasmARM]{patterns/15_structs/3_tm_linux/ARM/tm_ARM_keil_thumb.asm}

\myparagraph{\OptimizingXcodeIV (\ThumbTwoMode)}

\IDA \q{knows} the \TT{tm} structure 
(because \IDA \q{knows} the types of the arguments of library functions like \TT{localtime\_r()}), 

so it shows here structure elements accesses and their names.

\lstinputlisting[caption=\OptimizingXcodeIV (\ThumbTwoMode),style=customasmARM]{patterns/15_structs/3_tm_linux/ARM/tm_ARM_xcode_thumb.asm}
}

\EN{\input{patterns/15_structs/3_tm_linux/MIPS/main_EN}}
\RU{\input{patterns/15_structs/3_tm_linux/MIPS/main_RU}}
\DE{\input{patterns/15_structs/3_tm_linux/MIPS/main_DE}}
\FR{\input{patterns/15_structs/3_tm_linux/MIPS/main_FR}}
\JPN{\subsubsection{MIPS}

\lstinputlisting[caption=\Optimizing GCC 4.4.5 (IDA),numbers=left,style=customasmMIPS]{patterns/15_structs/3_tm_linux/MIPS/MIPS_O3_IDA_EN.lst}

This is an example where the branch delay slots can confuse us.

For example, there is the instruction \INS{addiu \$a1, 1900} at line 35 which adds 1900 to the year number.

It's executed before the corresponding \INS{JALR} at line 34, do not forget about it.

}

% subsection:
\EN{\input{patterns/15_structs/3_tm_linux/as_array/main_EN}}
\RU{\input{patterns/15_structs/3_tm_linux/as_array/main_RU}}
\DE{\input{patterns/15_structs/3_tm_linux/as_array/main_DE}}
\FR{\input{patterns/15_structs/3_tm_linux/as_array/main_FR}}
\JPN{\subsubsection{Structure as a set of values}

In order to illustrate that the structure is just variables laying side-by-side in one place, 
let's rework our example while looking at the \IT{tm} structure definition again: \lstref{struct_tm}.

\lstinputlisting[style=customc]{patterns/15_structs/3_tm_linux/as_array/GCC_tm2.c}

\myindex{\CStandardLibrary!localtime\_r()}
N.B. 
The pointer to the \TT{tm\_sec} field is passed into \TT{localtime\_r}, i.e., 
to the first element of the \q{structure}.

The compiler warns us:

\begin{lstlisting}[caption=GCC 4.7.3]
GCC_tm2.c: In function 'main':
GCC_tm2.c:11:5: warning: passing argument 2 of 'localtime_r' from incompatible pointer type [enabled by default]
In file included from GCC_tm2.c:2:0:
/usr/include/time.h:59:12: note: expected 'struct tm *' but argument is of type 'int *'
\end{lstlisting}

But nevertheless, it generates this:

\lstinputlisting[caption=GCC 4.7.3,style=customasmx86]{patterns/15_structs/3_tm_linux/as_array/GCC_tm2.asm}

This code is identical to what we saw previously and it is
not possible to say, was it a structure in original source code or just a pack of variables.

And this works. 
However, it is not recommended to do this in practice. 

Usually, non-optimizing compilers allocates variables in the local stack in the 
same order as they were declared in the function.

Nevertheless, there is no guarantee.

By the way, some other compiler may warn about the \TT{tm\_year}, \TT{tm\_mon}, \TT{tm\_mday},
\TT{tm\_hour}, \TT{tm\_min} variables, but not \TT{tm\_sec}
 are used without being initialized.

Indeed, the compiler is not aware that these are to be filled by\\
\TT{localtime\_r()} function.

We chose this example, since all structure fields are of type \Tint.%

This would not work if structure fields are 16-bit (\TT{WORD}), 
like in the case of the \TT{SYSTEMTIME} structure---\TT{GetSystemTime()} 
will fill them incorrectly 
(because the local variables are aligned on a 32-bit boundary).
Read more about it in next section: 
\q{\StructurePackingSectionName} (\myref{structure_packing}).

So, a structure is just a pack of variables laying in one place, side-by-side.
We could say that the structure is the instruction to the compiler, directing it to hold variables in one place.
By the way, in some very early C versions (before 1972), there were no structures at all \RitchieDevC.

There is no debugger example here: it is just the same as you already saw.

\subsubsection{Structure as an array of 32-bit words}

\lstinputlisting[style=customc]{patterns/15_structs/3_tm_linux/as_array/GCC_tm3.c}

We just \IT{cast} a pointer to structure to an array of \Tint{}'s.
And that works!
We run the example at 23:51:45 26-July-2014.

\begin{lstlisting}[label=GCC_tm3_output]
0x0000002D (45)
0x00000033 (51)
0x00000017 (23)
0x0000001A (26)
0x00000006 (6)
0x00000072 (114)
0x00000006 (6)
0x000000CE (206)
0x00000001 (1)
\end{lstlisting}

The variables here 
are in the same order as they are enumerated in the definition of the structure: \myref{struct_tm}.

Here is how it gets compiled:

\lstinputlisting[caption=\Optimizing GCC 4.8.1,style=customasmx86]{patterns/15_structs/3_tm_linux/as_array/GCC_tm3_JPN.lst}

Indeed: the space in the local stack is first treated as a structure, and then it's treated as an array.

It's even possible to modify the fields of the structure through this pointer.

And again, it's dubiously hackish way to do things, not recommended for use in production code.

\mysubparagraph{\Exercise}

As an exercise, try to modify (increase by 1) the current month number, treating the structure as 
an array.

\subsubsection{Structure as an array of bytes}

We can go even further. Let's \IT{cast} the pointer to an array of bytes and dump it:

\lstinputlisting[style=customc]{patterns/15_structs/3_tm_linux/as_array/GCC_tm4.c}

\begin{lstlisting}
0x2D 0x00 0x00 0x00 
0x33 0x00 0x00 0x00 
0x17 0x00 0x00 0x00 
0x1A 0x00 0x00 0x00 
0x06 0x00 0x00 0x00 
0x72 0x00 0x00 0x00 
0x06 0x00 0x00 0x00 
0xCE 0x00 0x00 0x00 
0x01 0x00 0x00 0x00 
\end{lstlisting}

We also run this example at 23:51:45 26-July-2014
\footnote{The time and date are the same for demonstration purposes. Byte values are fixed up.}.
The values are just the same as in the previous dump 
(\myref{GCC_tm3_output}), and of course, the lowest byte goes first, because this is a little-endian architecture 
(\myref{sec:endianness}).

\lstinputlisting[caption=\Optimizing GCC 4.8.1,style=customasmx86]{patterns/15_structs/3_tm_linux/as_array/GCC_tm4_JPN.lst}
}


\subsection{UNIX: struct tm}

% subsections here:
\EN{\input{patterns/15_structs/3_tm_linux/linux_EN}}
\RU{\input{patterns/15_structs/3_tm_linux/linux_RU}}
\DE{\input{patterns/15_structs/3_tm_linux/linux_DE}}
\FR{\input{patterns/15_structs/3_tm_linux/linux_FR}}
\JPN{\subsubsection{Linux}

Let's take the \TT{tm} structure from \TT{time.h} in Linux for example:

\lstinputlisting[style=customc]{patterns/15_structs/3_tm_linux/GCC_tm.c}

Let's compile it in GCC 4.4.1:

\lstinputlisting[caption=GCC 4.4.1,style=customasmx86]{patterns/15_structs/3_tm_linux/GCC_tm_EN.asm}

Somehow, \IDA did not write the local variables' names in the local stack.
But since we already are experienced reverse engineers :-) we may do it without this information in 
this simple example.

\myindex{x86!\Instructions!LEA}

Please also pay attention to the \TT{lea edx, [eax+76Ch]}~---this instruction just adds \TT{0x76C} (1900) to value in \EAX,
but doesn't modify any flags. See also the relevant section about \LEA{}~(\myref{sec:LEA}).

\myparagraph{GDB}

Let's try to load the example into GDB
\footnote{The \IT{date} result is slightly corrected for demonstration purposes.
Of course, it's not possible to run GDB that quickly, in the same second.}:

\lstinputlisting[caption=GDB]{patterns/15_structs/3_tm_linux/GCC_tm_GDB.txt}

We can easily find our structure in the stack.
First, let's see how it's defined in \IT{time.h}:

\begin{lstlisting}[caption=time.h, label=struct_tm,style=customc]
struct tm
{
  int	tm_sec;
  int	tm_min;
  int	tm_hour;
  int	tm_mday;
  int	tm_mon;
  int	tm_year;
  int	tm_wday;
  int	tm_yday;
  int	tm_isdst;
};
\end{lstlisting}

Pay attention that
32-bit \Tint is used here instead of WORD in SYSTEMTIME.
So, each field occupies 32-bit.

Here are the fields of our structure in the stack:

\begin{lstlisting}
0xbffff0dc:	0x080484c3	0x080485c0	0x000007de	0x00000000
0xbffff0ec:	0x08048301	0x538c93ed	0x00000025 sec	0x0000000a min
0xbffff0fc:	0x00000012 hour	0x00000002 mday	0x00000005 mon 	0x00000072 year
0xbffff10c:	0x00000001 wday	0x00000098 yday	0x00000001 isdst0x00002a30
0xbffff11c:	0x0804b090	0x08048530	0x00000000	0x00000000
\end{lstlisting}

Or as a table:

\begin{center}
\begin{tabular}{ | l | l | l | }
\hline
\headercolor{} Hexadecimal number & 
\headercolor{} decimal number & 
\headercolor{} field name \\
\hline
0x00000025 & 37 	& tm\_sec \\
\hline
0x0000000a & 10 	& tm\_min \\
\hline
0x00000012 & 18 	& tm\_hour \\	
\hline
0x00000002 & 2 		& tm\_mday \\	
\hline
0x00000005 & 5 		& tm\_mon \\	
\hline
0x00000072 & 114 	& tm\_year \\
\hline
0x00000001 & 1 		& tm\_wday \\	
\hline
0x00000098 & 152 	& tm\_yday \\	
\hline
0x00000001 & 1 		& tm\_isdst \\
\hline
\end{tabular}
\end{center}

Just like SYSTEMTIME (\myref{sec:SYSTEMTIME}), 

there are also other fields available that are not used, like tm\_wday, tm\_yday, tm\_isdst.
}

\EN{\input{patterns/15_structs/3_tm_linux/ARM/main_EN}}
\RU{\input{patterns/15_structs/3_tm_linux/ARM/main_RU}}
\DE{\input{patterns/15_structs/3_tm_linux/ARM/main_DE}}
\FR{\input{patterns/15_structs/3_tm_linux/ARM/main_FR}}
\JPN{\subsubsection{ARM}

\myparagraph{\OptimizingKeilVI (\ThumbMode)}

Same example:

\lstinputlisting[caption=\OptimizingKeilVI (\ThumbMode),style=customasmARM]{patterns/15_structs/3_tm_linux/ARM/tm_ARM_keil_thumb.asm}

\myparagraph{\OptimizingXcodeIV (\ThumbTwoMode)}

\IDA \q{knows} the \TT{tm} structure 
(because \IDA \q{knows} the types of the arguments of library functions like \TT{localtime\_r()}), 

so it shows here structure elements accesses and their names.

\lstinputlisting[caption=\OptimizingXcodeIV (\ThumbTwoMode),style=customasmARM]{patterns/15_structs/3_tm_linux/ARM/tm_ARM_xcode_thumb.asm}
}

\EN{\input{patterns/15_structs/3_tm_linux/MIPS/main_EN}}
\RU{\input{patterns/15_structs/3_tm_linux/MIPS/main_RU}}
\DE{\input{patterns/15_structs/3_tm_linux/MIPS/main_DE}}
\FR{\input{patterns/15_structs/3_tm_linux/MIPS/main_FR}}
\JPN{\subsubsection{MIPS}

\lstinputlisting[caption=\Optimizing GCC 4.4.5 (IDA),numbers=left,style=customasmMIPS]{patterns/15_structs/3_tm_linux/MIPS/MIPS_O3_IDA_EN.lst}

This is an example where the branch delay slots can confuse us.

For example, there is the instruction \INS{addiu \$a1, 1900} at line 35 which adds 1900 to the year number.

It's executed before the corresponding \INS{JALR} at line 34, do not forget about it.

}

% subsection:
\EN{\input{patterns/15_structs/3_tm_linux/as_array/main_EN}}
\RU{\input{patterns/15_structs/3_tm_linux/as_array/main_RU}}
\DE{\input{patterns/15_structs/3_tm_linux/as_array/main_DE}}
\FR{\input{patterns/15_structs/3_tm_linux/as_array/main_FR}}
\JPN{\subsubsection{Structure as a set of values}

In order to illustrate that the structure is just variables laying side-by-side in one place, 
let's rework our example while looking at the \IT{tm} structure definition again: \lstref{struct_tm}.

\lstinputlisting[style=customc]{patterns/15_structs/3_tm_linux/as_array/GCC_tm2.c}

\myindex{\CStandardLibrary!localtime\_r()}
N.B. 
The pointer to the \TT{tm\_sec} field is passed into \TT{localtime\_r}, i.e., 
to the first element of the \q{structure}.

The compiler warns us:

\begin{lstlisting}[caption=GCC 4.7.3]
GCC_tm2.c: In function 'main':
GCC_tm2.c:11:5: warning: passing argument 2 of 'localtime_r' from incompatible pointer type [enabled by default]
In file included from GCC_tm2.c:2:0:
/usr/include/time.h:59:12: note: expected 'struct tm *' but argument is of type 'int *'
\end{lstlisting}

But nevertheless, it generates this:

\lstinputlisting[caption=GCC 4.7.3,style=customasmx86]{patterns/15_structs/3_tm_linux/as_array/GCC_tm2.asm}

This code is identical to what we saw previously and it is
not possible to say, was it a structure in original source code or just a pack of variables.

And this works. 
However, it is not recommended to do this in practice. 

Usually, non-optimizing compilers allocates variables in the local stack in the 
same order as they were declared in the function.

Nevertheless, there is no guarantee.

By the way, some other compiler may warn about the \TT{tm\_year}, \TT{tm\_mon}, \TT{tm\_mday},
\TT{tm\_hour}, \TT{tm\_min} variables, but not \TT{tm\_sec}
 are used without being initialized.

Indeed, the compiler is not aware that these are to be filled by\\
\TT{localtime\_r()} function.

We chose this example, since all structure fields are of type \Tint.%

This would not work if structure fields are 16-bit (\TT{WORD}), 
like in the case of the \TT{SYSTEMTIME} structure---\TT{GetSystemTime()} 
will fill them incorrectly 
(because the local variables are aligned on a 32-bit boundary).
Read more about it in next section: 
\q{\StructurePackingSectionName} (\myref{structure_packing}).

So, a structure is just a pack of variables laying in one place, side-by-side.
We could say that the structure is the instruction to the compiler, directing it to hold variables in one place.
By the way, in some very early C versions (before 1972), there were no structures at all \RitchieDevC.

There is no debugger example here: it is just the same as you already saw.

\subsubsection{Structure as an array of 32-bit words}

\lstinputlisting[style=customc]{patterns/15_structs/3_tm_linux/as_array/GCC_tm3.c}

We just \IT{cast} a pointer to structure to an array of \Tint{}'s.
And that works!
We run the example at 23:51:45 26-July-2014.

\begin{lstlisting}[label=GCC_tm3_output]
0x0000002D (45)
0x00000033 (51)
0x00000017 (23)
0x0000001A (26)
0x00000006 (6)
0x00000072 (114)
0x00000006 (6)
0x000000CE (206)
0x00000001 (1)
\end{lstlisting}

The variables here 
are in the same order as they are enumerated in the definition of the structure: \myref{struct_tm}.

Here is how it gets compiled:

\lstinputlisting[caption=\Optimizing GCC 4.8.1,style=customasmx86]{patterns/15_structs/3_tm_linux/as_array/GCC_tm3_JPN.lst}

Indeed: the space in the local stack is first treated as a structure, and then it's treated as an array.

It's even possible to modify the fields of the structure through this pointer.

And again, it's dubiously hackish way to do things, not recommended for use in production code.

\mysubparagraph{\Exercise}

As an exercise, try to modify (increase by 1) the current month number, treating the structure as 
an array.

\subsubsection{Structure as an array of bytes}

We can go even further. Let's \IT{cast} the pointer to an array of bytes and dump it:

\lstinputlisting[style=customc]{patterns/15_structs/3_tm_linux/as_array/GCC_tm4.c}

\begin{lstlisting}
0x2D 0x00 0x00 0x00 
0x33 0x00 0x00 0x00 
0x17 0x00 0x00 0x00 
0x1A 0x00 0x00 0x00 
0x06 0x00 0x00 0x00 
0x72 0x00 0x00 0x00 
0x06 0x00 0x00 0x00 
0xCE 0x00 0x00 0x00 
0x01 0x00 0x00 0x00 
\end{lstlisting}

We also run this example at 23:51:45 26-July-2014
\footnote{The time and date are the same for demonstration purposes. Byte values are fixed up.}.
The values are just the same as in the previous dump 
(\myref{GCC_tm3_output}), and of course, the lowest byte goes first, because this is a little-endian architecture 
(\myref{sec:endianness}).

\lstinputlisting[caption=\Optimizing GCC 4.8.1,style=customasmx86]{patterns/15_structs/3_tm_linux/as_array/GCC_tm4_JPN.lst}
}


\input{appendix/GCC_library}
\input{appendix/MSVC_library}
\input{appendix/cheatsheets}
}
\FR{\part*{\RU{Приложение}\EN{Appendix}\DE{Anhang}\FR{Appendice}\JPN{付録}}
\appendix
\addcontentsline{toc}{part}{\RU{Приложение}\EN{Appendix}\DE{Anhang}\FR{Appendice}\JPN{付録}}

% chapters
\EN{\input{appendix/x86/main_EN}}
\RU{\input{appendix/x86/main_RU}}
\DE{\input{appendix/x86/main_DE}}
\FR{\input{appendix/x86/main_FR}}
\JPN{\mysection{x86}

\subsection{Terminology}

Common for 16-bit (8086/80286), 32-bit (80386, etc.), 64-bit.

\myindex{IEEE 754}
\myindex{MS-DOS}
\begin{description}
	\item[byte] 8-bit.
		The DB assembly directive is used for defining variables and arrays of bytes.
		Bytes are passed in the 8-bit part of registers: \TT{AL/BL/CL/DL/AH/BH/CH/DH/SIL/DIL/R*L}.
	\item[word] 16-bit. 
		DW assembly directive \dittoclosing.
		Words are passed in the 16-bit part of the registers:\\
			\TT{AX/BX/CX/DX/SI/DI/R*W}.
	\item[double word] (\q{dword}) 32-bit.
		DD assembly directive \dittoclosing.
		Double words are passed in registers (x86) or in the 32-bit part of registers (x64). 
		In 16-bit code, double words are passed in 16-bit register pairs.
	\item[quad word] (\q{qword}) 64-bit.
		DQ assembly directive \dittoclosing.
		In 32-bit environment, quad words are passed in 32-bit register pairs.
	\item[tbyte] (10 bytes) 80-bit or 10 bytes (used for IEEE 754 FPU registers).
	\item[paragraph] (16 bytes)---term was popular in MS-DOS environment. % TODO link to a part about 8086 memory model...
\end{description}

\myindex{Windows!API}

Data types of the same width (BYTE, WORD, DWORD) are also the same in Windows \ac{API}.

\input{appendix/x86/registers} % subsection
\subsection{\RU{Инструкции}\EN{Instructions}}
\label{sec:x86_instructions}

\RU{Инструкции, отмеченные как (M) обычно не генерируются компилятором: если вы видите её, очень может быть
это вручную написанный фрагмент кода, либо это т.н. compiler intrinsic}
\EN{Instructions marked as (M) are not usually generated by the compiler: if you see one of them, it is probably
a hand-written piece of assembly code, or a compiler intrinsic} (\myref{sec:compiler_intrinsic}).

% TODO ? обратные инструкции

\RU{Только наиболее используемые инструкции перечислены здесь}
\EN{Only the most frequently used instructions are listed here}.
\EN{You can read \myref{x86_manuals} for a full documentation.}%
\RU{Обращайтесь к \myref{x86_manuals} для полной документации.}

\RU{Нужно ли заучивать опкоды инструкций на память?}\EN{Do you have to know all instruction's opcodes by heart?}
\RU{Нет, только те, которые часто используются для модификации кода}\EN{No, only those
which are used for code patching} (\myref{x86_patching}).
\RU{Остальные запоминать нет смысла.}\EN{All the rest of the opcodes don't need to be memorized.}

\subsubsection{\RU{Префиксы}\EN{Prefixes}}

\myindex{x86!\Prefixes!LOCK}
\myindex{x86!\Prefixes!REP}
\myindex{x86!\Prefixes!REPE/REPNE}
\begin{description}
\label{x86_lock}
\item[LOCK] \RU{используется чтобы предоставить эксклюзивный доступ к памяти в многопроцессорной среде}
\EN{forces CPU to make exclusive access to the RAM in multiprocessor environment}.
\RU{Для упрощения, можно сказать, что когда исполняется инструкция с этим префиксом, остальные процессоры
в системе останавливаются}\EN{For the sake of simplification, it can be said that when an instruction
with this prefix is executed, all other CPUs in a multiprocessor system are stopped}.
\RU{Чаще все это используется для критических секций, семафоров, мьютексов}\EN{Most often
it is used for critical sections, semaphores, mutexes}.
\RU{Обычно используется с}\EN{Commonly used with} ADD, AND, BTR, BTS, CMPXCHG, OR, XADD, XOR.
\RU{Читайте больше о критических секциях}\EN{You can read more about critical sections here} (\myref{critical_sections}).

\item[REP] \RU{используется с инструкциями}\EN{is used with the} MOVSx \AndENRU STOSx\EN{ instructions}:
\RU{инструкция будет исполняться в цикле, счетчик расположен в регистре CX/ECX/RCX}
\EN{execute the instruction in a loop, the counter is located in the CX/ECX/RCX register}.
\RU{Для более детального описания, читайте больше об инструкциях}
\EN{For a detailed description, read more about the} MOVSx (\myref{REP_MOVSx}) 
\AndENRU STOSx (\myref{REP_STOSx})\EN{ instructions}.

\RU{Работа инструкций с префиксом REP зависит от флага DF, он задает направление}
\EN{The instructions prefixed by REP are sensitive to the DF flag, which is used to set the direction}.

\item[REPE/REPNE] (\ac{AKA} REPZ/REPNZ) \RU{используется с инструкциями}\EN{used with} CMPSx \AndENRU
SCASx\EN{ instructions}:
\RU{инструкция будет исполняться в цикле, счетчик расположен в регистре \TT{CX}/\TT{ECX}/\TT{RCX}}
\EN{execute the last instruction in a loop, the count is set in the \TT{CX}/\TT{ECX}/\TT{RCX} register}. 
\RU{Выполнение будет прервано если ZF будет 0 (REPE) либо если ZF будет 1 (REPNE)}
\EN{It terminates prematurely if ZF is 0 (REPE) or if ZF is 1 (REPNE)}.

\RU{Для более детального описания, читайте больше об инструкциях}
\EN{For a detailed description, you can read more about the} CMPSx (\myref{REPE_CMPSx}) 
\AndENRU SCASx (\myref{REPNE_SCASx})\EN{ instructions}.

\RU{Работа инструкций с префиксами REPE/REPNE зависит от флага DF, он задает направление}
\EN{Instructions prefixed by REPE/REPNE are sensitive to the DF flag, which is used to set the direction}.

\end{description}

\subsubsection{\RU{Наиболее часто используемые инструкции}\EN{Most frequently used instructions}}

\RU{Их можно заучить в первую очередь}\EN{These can be memorized in the first place}.

\begin{description}
% in order to keep them easily sorted...
\input{appendix/x86/instructions/ADC}
\input{appendix/x86/instructions/ADD}
\input{appendix/x86/instructions/AND}
\input{appendix/x86/instructions/CALL}
\input{appendix/x86/instructions/CMP}
\input{appendix/x86/instructions/DEC}
\input{appendix/x86/instructions/IMUL}
\input{appendix/x86/instructions/INC}
\input{appendix/x86/instructions/JCXZ}
\input{appendix/x86/instructions/JMP}
\input{appendix/x86/instructions/Jcc}
\input{appendix/x86/instructions/LAHF}
\input{appendix/x86/instructions/LEAVE}
\input{appendix/x86/instructions/LEA}
\input{appendix/x86/instructions/MOVSB_W_D_Q}
\input{appendix/x86/instructions/MOVSX}
\input{appendix/x86/instructions/MOVZX}
\input{appendix/x86/instructions/MOV}
\input{appendix/x86/instructions/MUL}
\input{appendix/x86/instructions/NEG}
\input{appendix/x86/instructions/NOP}
\input{appendix/x86/instructions/NOT}
\input{appendix/x86/instructions/OR}
\input{appendix/x86/instructions/POP}
\input{appendix/x86/instructions/PUSH}
\input{appendix/x86/instructions/RET}
\input{appendix/x86/instructions/SAHF}
\input{appendix/x86/instructions/SBB}
\input{appendix/x86/instructions/SCASB_W_D_Q}
\input{appendix/x86/instructions/SHx}
\input{appendix/x86/instructions/SHRD}
\input{appendix/x86/instructions/STOSB_W_D_Q}
\input{appendix/x86/instructions/SUB}
\input{appendix/x86/instructions/TEST}
\input{appendix/x86/instructions/XOR}
\end{description}

\subsubsection{\RU{Реже используемые инструкции}\EN{Less frequently used instructions}}

\begin{description}
\input{appendix/x86/instructions/BSF}
\input{appendix/x86/instructions/BSR}
\input{appendix/x86/instructions/BSWAP}
\input{appendix/x86/instructions/BTC}
\input{appendix/x86/instructions/BTR}
\input{appendix/x86/instructions/BTS}
\input{appendix/x86/instructions/BT}
\input{appendix/x86/instructions/CBW_CWDE_CDQ}
\input{appendix/x86/instructions/CLD}
\input{appendix/x86/instructions/CLI}
\input{appendix/x86/instructions/CMC}
\input{appendix/x86/instructions/CMOVcc}
\input{appendix/x86/instructions/CMPSB_W_D_Q}
\input{appendix/x86/instructions/CPUID}
\input{appendix/x86/instructions/DIV}
\input{appendix/x86/instructions/IDIV}
\input{appendix/x86/instructions/INT}
\input{appendix/x86/instructions/IN}
\input{appendix/x86/instructions/IRET}
\input{appendix/x86/instructions/LOOP}
\input{appendix/x86/instructions/OUT}
\input{appendix/x86/instructions/POPA}
\input{appendix/x86/instructions/POPCNT}
\input{appendix/x86/instructions/POPF}
\input{appendix/x86/instructions/PUSHA}
\input{appendix/x86/instructions/PUSHF}
\input{appendix/x86/instructions/RCx}
\input{appendix/x86/instructions/ROx}
\input{appendix/x86/instructions/SAL}
\input{appendix/x86/instructions/SAR}
\input{appendix/x86/instructions/SETcc}
\input{appendix/x86/instructions/STC}
\input{appendix/x86/instructions/STD}
\input{appendix/x86/instructions/STI}
\input{appendix/x86/instructions/SYSCALL}
\input{appendix/x86/instructions/SYSENTER}
\input{appendix/x86/instructions/UD2}
\input{appendix/x86/instructions/XCHG}
\end{description}

\subsubsection{\RU{Инструкции FPU}\EN{FPU instructions}}

\RU{Суффикс \TT{-R} в названии инструкции обычно означает, что операнды поменяны местами, суффикс \TT{-P} означает
что один элемент выталкивается из стека после исполнения инструкции, суффикс \TT{-PP} означает, что
выталкиваются два элемента}%
\EN{\TT{-R} suffix in the mnemonic usually implies that the operands are reversed,
\TT{-P} suffix implies that one element is popped
from the stack after the instruction's execution, \TT{-PP} suffix implies that two elements are popped}.

\TT{-P} \RU{инструкции часто бывают полезны, когда нам уже больше не нужно хранить значение в 
FPU-стеке после операции.}%
\EN{instructions are often useful when we do not need the value in the FPU stack to be 
present anymore after the operation.}

\begin{description}
\input{appendix/x86/instructions/FABS}
\input{appendix/x86/instructions/FADD} % + FADDP
\input{appendix/x86/instructions/FCHS}
\input{appendix/x86/instructions/FCOM} % + FCOMP + FCOMPP
\input{appendix/x86/instructions/FDIVR} % + FDIVRP
\input{appendix/x86/instructions/FDIV} % + FDIVP
\input{appendix/x86/instructions/FILD}
\input{appendix/x86/instructions/FIST} % + FISTP
\input{appendix/x86/instructions/FLD1}
\input{appendix/x86/instructions/FLDCW}
\input{appendix/x86/instructions/FLDZ}
\input{appendix/x86/instructions/FLD}
\input{appendix/x86/instructions/FMUL} % + FMULP
\input{appendix/x86/instructions/FSINCOS}
\input{appendix/x86/instructions/FSQRT}
\input{appendix/x86/instructions/FSTCW} % + FNSTCW
\input{appendix/x86/instructions/FSTSW} % + FNSTSW
\input{appendix/x86/instructions/FST}
\input{appendix/x86/instructions/FSUBR} % + FSUBRP
\input{appendix/x86/instructions/FSUB} % + FSUBP
\input{appendix/x86/instructions/FUCOM} % + FUCOMP + FUCOMPP
\input{appendix/x86/instructions/FXCH}
\end{description}

%\subsubsection{\RU{SIMD-инструкции}\EN{SIMD instructions}}

% TODO

%\begin{description}
%\input{appendix/x86/instructions/DIVSD}
%\input{appendix/x86/instructions/MOVDQA}
%\input{appendix/x86/instructions/MOVDQU}
%\input{appendix/x86/instructions/PADDD}
%\input{appendix/x86/instructions/PCMPEQB}
%\input{appendix/x86/instructions/PLMULHW}
%\input{appendix/x86/instructions/PLMULLD}
%\input{appendix/x86/instructions/PMOVMSKB}
%\input{appendix/x86/instructions/PXOR}
%\end{description}

% SHLD !
% SHRD !
% BSWAP !
% CMPXCHG
% XADD !
% CMPXCHG8B
% RDTSC !
% PAUSE!

% xsave
% fnclex, fnsave
% movsxd, movaps, wait, sfence, lfence, pushfq
% prefetchw
% REP RETN
% REP BSF
% movnti, movntdq, rdmsr, wrmsr
% ldmxcsr, stmxcsr, invlpg
% swapgs
% movq, movd
% mulsd
% POR
% IRETQ
% pslldq
% psrldq
% cqo, fxrstor, comisd, xrstor, wbinvd, movntq
% fprem
% addsb, subsd, frndint

% rare:
%\item[ENTER]
%\item[LES]
% LDS
% XLAT

\subsubsection{\RU{Инструкции с печатаемым ASCII-опкодом}\EN{Instructions having printable ASCII opcode}}

(\RU{В 32-битном режиме}\EN{In 32-bit mode}).

\label{printable_x86_opcodes}
\myindex{Shellcode}
\RU{Это может пригодиться для создания шеллкодов}\EN{These can be suitable for shellcode construction}.
\RU{См. также}\EN{See also}: \myref{subsec:EICAR}.

% FIXME: break table
% FIXME: start at 0x20...
\begin{center}
\begin{longtable}{ | l | l | l | }
\hline
\HeaderColor ASCII\RU{-символ}\EN{ character} & 
\HeaderColor \RU{шестнадцатеричный код}\EN{hexadecimal code} & 
\HeaderColor x86\RU{-инструкция}\EN{ instruction} \\
\hline
0	 &30	 &XOR \\
1	 &31	 &XOR \\
2	 &32	 &XOR \\
3	 &33	 &XOR \\
4	 &34	 &XOR \\
5	 &35	 &XOR \\
7	 &37	 &AAA \\
8	 &38	 &CMP \\
9	 &39	 &CMP \\
:	 &3a	 &CMP \\
;	 &3b	 &CMP \\
<	 &3c	 &CMP \\
=	 &3d	 &CMP \\
?	 &3f	 &AAS \\
@	 &40	 &INC \\
A	 &41	 &INC \\
B	 &42	 &INC \\
C	 &43	 &INC \\
D	 &44	 &INC \\
E	 &45	 &INC \\
F	 &46	 &INC \\
G	 &47	 &INC \\
H	 &48	 &DEC \\
I	 &49	 &DEC \\
J	 &4a	 &DEC \\
K	 &4b	 &DEC \\
L	 &4c	 &DEC \\
M	 &4d	 &DEC \\
N	 &4e	 &DEC \\
O	 &4f	 &DEC \\
P	 &50	 &PUSH \\
Q	 &51	 &PUSH \\
R	 &52	 &PUSH \\
S	 &53	 &PUSH \\
T	 &54	 &PUSH \\
U	 &55	 &PUSH \\
V	 &56	 &PUSH \\
W	 &57	 &PUSH \\
X	 &58	 &POP \\
Y	 &59	 &POP \\
Z	 &5a	 &POP \\
\lbrack{}	 &5b	 &POP \\
\textbackslash{}	 &5c	 &POP \\
\rbrack{}	 &5d	 &POP \\
\verb|^|	 &5e	 &POP \\
\_	 &5f	 &POP \\
\verb|`|	 &60	 &PUSHA \\
a	 &61	 &POPA \\
f	 &66	 &\RU{(в 32-битном режиме) переключиться на}\EN{(in 32-bit mode) switch to}\\
   & & \RU{16-битный размер операнда}\EN{16-bit operand size} \\
g	 &67	 &\RU{(в 32-битном режиме) переключиться на}\EN{in 32-bit mode) switch to}\\
   & & \RU{16-битный размер адреса}\EN{16-bit address size} \\
h	 &68	 &PUSH\\
i	 &69	 &IMUL\\
j	 &6a	 &PUSH\\
k	 &6b	 &IMUL\\
p	 &70	 &JO\\
q	 &71	 &JNO\\
r	 &72	 &JB\\
s	 &73	 &JAE\\
t	 &74	 &JE\\
u	 &75	 &JNE\\
v	 &76	 &JBE\\
w	 &77	 &JA\\
x	 &78	 &JS\\
y	 &79	 &JNS\\
z	 &7a	 &JP\\
\hline
\end{longtable}
\end{center}

\myindex{x86!\Instructions!AAA}
\myindex{x86!\Instructions!AAS}
\myindex{x86!\Instructions!CMP}
\myindex{x86!\Instructions!DEC}
\myindex{x86!\Instructions!IMUL}
\myindex{x86!\Instructions!INC}
\myindex{x86!\Instructions!JA}
\myindex{x86!\Instructions!JAE}
\myindex{x86!\Instructions!JB}
\myindex{x86!\Instructions!JBE}
\myindex{x86!\Instructions!JE}
\myindex{x86!\Instructions!JNE}
\myindex{x86!\Instructions!JNO}
\myindex{x86!\Instructions!JNS}
\myindex{x86!\Instructions!JO}
\myindex{x86!\Instructions!JP}
\myindex{x86!\Instructions!JS}
\myindex{x86!\Instructions!POP}
\myindex{x86!\Instructions!POPA}
\myindex{x86!\Instructions!PUSH}
\myindex{x86!\Instructions!PUSHA}
\myindex{x86!\Instructions!XOR}

\RU{В итоге}\EN{In summary}:
AAA, AAS, CMP, DEC, IMUL, INC, JA, JAE, JB, JBE, JE, JNE, JNO, JNS, JO, JP, JS, POP, POPA, PUSH, PUSHA, 
XOR.

 % subsection
\subsection{npad}
\label{sec:npad}

\RU{Это макрос в ассемблере, для выравнивания некоторой метки по некоторой границе.}
\EN{It is an assembly language macro for aligning labels on a specific boundary.}
\DE{Dies ist ein Assembler-Makro um Labels an bestimmten Grenzen auszurichten.}
\FR{C'est une macro du langage d'assemblage pour aligner les labels sur une limite
spécifique.}
\JPN{It is an assembly language macro for aligning labels on a specific boundary.}

\RU{Это нужно для тех \IT{нагруженных} меток, куда чаще всего передается управление, например, 
начало тела цикла. 
Для того чтобы процессор мог эффективнее вытягивать данные или код из памяти, через шину с памятью, 
кэширование, итд.}
\EN{That's often needed for the busy labels to where the control flow is often passed, e.g., loop body starts.
So the CPU can load the data or code from the memory effectively, through the memory bus, cache lines, etc.}
\DE{Dies ist oft nützlich Labels, die oft Ziel einer Kotrollstruktur sind, wie Schleifenköpfe.
Somit kann die CPU Daten oder Code sehr effizient vom Speicher durch den Bus, den Cache, usw. laden.}
\FR{C'est souvent necessaire pour des labels très utilisés, comme par exemple le
début d'un corps de boucle. Ainsi, le CPU peut charger les données ou le code depuis
la mémoire efficacement, à travers le bus mémoire, les caches, etc.}

\RU{Взято из}\EN{Taken from}\DE{Entnommen von}\FR{Pris de} \TT{listing.inc} (MSVC):

\myindex{x86!\Instructions!NOP}
\RU{Это, кстати, любопытный пример различных вариантов \NOP{}-ов. 
Все эти инструкции не дают никакого эффекта, но отличаются разной длиной.}
\EN{By the way, it is a curious example of the different \NOP variations.
All these instructions have no effects whatsoever, but have a different size.}
\DE{Dies ist übrigens ein Beispiel für die unterschiedlichen \NOP-Variationen.
Keine dieser Anweisungen hat eine Auswirkung, aber alle haben eine unterschiedliche Größe.}
\FR{À propos, c'est un exemple curieux des différentes variations de \NOP. Toutes
ces instructions n'ont pas d'effet, mais ont une taille différente.}

\RU{Цель в том, чтобы была только одна инструкция, а не набор NOP-ов, 
считается что так лучше для производительности CPU.}
\EN{Having a single idle instruction instead of couple of NOP-s,
is accepted to be better for CPU performance.}
\DE{Eine einzelne Idle-Anweisung anstatt mehrerer NOPs hat positive Auswirkungen
auf die CPU-Performance.}
\FR{Avoir une seule instruction sans effet au lieu de plusieurs est accepté comme
étant meilleur pour la performance du CPU.}

\begin{lstlisting}[style=customasmx86]
;; LISTING.INC
;;
;; This file contains assembler macros and is included by the files created
;; with the -FA compiler switch to be assembled by MASM (Microsoft Macro
;; Assembler).
;;
;; Copyright (c) 1993-2003, Microsoft Corporation. All rights reserved.

;; non destructive nops
npad macro size
if size eq 1
  nop
else
 if size eq 2
   mov edi, edi
 else
  if size eq 3
    ; lea ecx, [ecx+00]
    DB 8DH, 49H, 00H
  else
   if size eq 4
     ; lea esp, [esp+00]
     DB 8DH, 64H, 24H, 00H
   else
    if size eq 5
      add eax, DWORD PTR 0
    else
     if size eq 6
       ; lea ebx, [ebx+00000000]
       DB 8DH, 9BH, 00H, 00H, 00H, 00H
     else
      if size eq 7
	; lea esp, [esp+00000000]
	DB 8DH, 0A4H, 24H, 00H, 00H, 00H, 00H 
      else
       if size eq 8
        ; jmp .+8; .npad 6
	DB 0EBH, 06H, 8DH, 9BH, 00H, 00H, 00H, 00H
       else
        if size eq 9
         ; jmp .+9; .npad 7
         DB 0EBH, 07H, 8DH, 0A4H, 24H, 00H, 00H, 00H, 00H
        else
         if size eq 10
          ; jmp .+A; .npad 7; .npad 1
          DB 0EBH, 08H, 8DH, 0A4H, 24H, 00H, 00H, 00H, 00H, 90H
         else
          if size eq 11
           ; jmp .+B; .npad 7; .npad 2
           DB 0EBH, 09H, 8DH, 0A4H, 24H, 00H, 00H, 00H, 00H, 8BH, 0FFH
          else
           if size eq 12
            ; jmp .+C; .npad 7; .npad 3
            DB 0EBH, 0AH, 8DH, 0A4H, 24H, 00H, 00H, 00H, 00H, 8DH, 49H, 00H
           else
            if size eq 13
             ; jmp .+D; .npad 7; .npad 4
             DB 0EBH, 0BH, 8DH, 0A4H, 24H, 00H, 00H, 00H, 00H, 8DH, 64H, 24H, 00H
            else
             if size eq 14
              ; jmp .+E; .npad 7; .npad 5
              DB 0EBH, 0CH, 8DH, 0A4H, 24H, 00H, 00H, 00H, 00H, 05H, 00H, 00H, 00H, 00H
             else
              if size eq 15
               ; jmp .+F; .npad 7; .npad 6
               DB 0EBH, 0DH, 8DH, 0A4H, 24H, 00H, 00H, 00H, 00H, 8DH, 9BH, 00H, 00H, 00H, 00H
              else
	       %out error: unsupported npad size
               .err
              endif
             endif
            endif
           endif
          endif
         endif
        endif
       endif
      endif
     endif
    endif
   endif
  endif
 endif
endif
endm
\end{lstlisting}
 % subsection

}
\subsection{UNIX: struct tm}

% subsections here:
\EN{\input{patterns/15_structs/3_tm_linux/linux_EN}}
\RU{\input{patterns/15_structs/3_tm_linux/linux_RU}}
\DE{\input{patterns/15_structs/3_tm_linux/linux_DE}}
\FR{\input{patterns/15_structs/3_tm_linux/linux_FR}}
\JPN{\subsubsection{Linux}

Let's take the \TT{tm} structure from \TT{time.h} in Linux for example:

\lstinputlisting[style=customc]{patterns/15_structs/3_tm_linux/GCC_tm.c}

Let's compile it in GCC 4.4.1:

\lstinputlisting[caption=GCC 4.4.1,style=customasmx86]{patterns/15_structs/3_tm_linux/GCC_tm_EN.asm}

Somehow, \IDA did not write the local variables' names in the local stack.
But since we already are experienced reverse engineers :-) we may do it without this information in 
this simple example.

\myindex{x86!\Instructions!LEA}

Please also pay attention to the \TT{lea edx, [eax+76Ch]}~---this instruction just adds \TT{0x76C} (1900) to value in \EAX,
but doesn't modify any flags. See also the relevant section about \LEA{}~(\myref{sec:LEA}).

\myparagraph{GDB}

Let's try to load the example into GDB
\footnote{The \IT{date} result is slightly corrected for demonstration purposes.
Of course, it's not possible to run GDB that quickly, in the same second.}:

\lstinputlisting[caption=GDB]{patterns/15_structs/3_tm_linux/GCC_tm_GDB.txt}

We can easily find our structure in the stack.
First, let's see how it's defined in \IT{time.h}:

\begin{lstlisting}[caption=time.h, label=struct_tm,style=customc]
struct tm
{
  int	tm_sec;
  int	tm_min;
  int	tm_hour;
  int	tm_mday;
  int	tm_mon;
  int	tm_year;
  int	tm_wday;
  int	tm_yday;
  int	tm_isdst;
};
\end{lstlisting}

Pay attention that
32-bit \Tint is used here instead of WORD in SYSTEMTIME.
So, each field occupies 32-bit.

Here are the fields of our structure in the stack:

\begin{lstlisting}
0xbffff0dc:	0x080484c3	0x080485c0	0x000007de	0x00000000
0xbffff0ec:	0x08048301	0x538c93ed	0x00000025 sec	0x0000000a min
0xbffff0fc:	0x00000012 hour	0x00000002 mday	0x00000005 mon 	0x00000072 year
0xbffff10c:	0x00000001 wday	0x00000098 yday	0x00000001 isdst0x00002a30
0xbffff11c:	0x0804b090	0x08048530	0x00000000	0x00000000
\end{lstlisting}

Or as a table:

\begin{center}
\begin{tabular}{ | l | l | l | }
\hline
\headercolor{} Hexadecimal number & 
\headercolor{} decimal number & 
\headercolor{} field name \\
\hline
0x00000025 & 37 	& tm\_sec \\
\hline
0x0000000a & 10 	& tm\_min \\
\hline
0x00000012 & 18 	& tm\_hour \\	
\hline
0x00000002 & 2 		& tm\_mday \\	
\hline
0x00000005 & 5 		& tm\_mon \\	
\hline
0x00000072 & 114 	& tm\_year \\
\hline
0x00000001 & 1 		& tm\_wday \\	
\hline
0x00000098 & 152 	& tm\_yday \\	
\hline
0x00000001 & 1 		& tm\_isdst \\
\hline
\end{tabular}
\end{center}

Just like SYSTEMTIME (\myref{sec:SYSTEMTIME}), 

there are also other fields available that are not used, like tm\_wday, tm\_yday, tm\_isdst.
}

\EN{\input{patterns/15_structs/3_tm_linux/ARM/main_EN}}
\RU{\input{patterns/15_structs/3_tm_linux/ARM/main_RU}}
\DE{\input{patterns/15_structs/3_tm_linux/ARM/main_DE}}
\FR{\input{patterns/15_structs/3_tm_linux/ARM/main_FR}}
\JPN{\subsubsection{ARM}

\myparagraph{\OptimizingKeilVI (\ThumbMode)}

Same example:

\lstinputlisting[caption=\OptimizingKeilVI (\ThumbMode),style=customasmARM]{patterns/15_structs/3_tm_linux/ARM/tm_ARM_keil_thumb.asm}

\myparagraph{\OptimizingXcodeIV (\ThumbTwoMode)}

\IDA \q{knows} the \TT{tm} structure 
(because \IDA \q{knows} the types of the arguments of library functions like \TT{localtime\_r()}), 

so it shows here structure elements accesses and their names.

\lstinputlisting[caption=\OptimizingXcodeIV (\ThumbTwoMode),style=customasmARM]{patterns/15_structs/3_tm_linux/ARM/tm_ARM_xcode_thumb.asm}
}

\EN{\input{patterns/15_structs/3_tm_linux/MIPS/main_EN}}
\RU{\input{patterns/15_structs/3_tm_linux/MIPS/main_RU}}
\DE{\input{patterns/15_structs/3_tm_linux/MIPS/main_DE}}
\FR{\input{patterns/15_structs/3_tm_linux/MIPS/main_FR}}
\JPN{\subsubsection{MIPS}

\lstinputlisting[caption=\Optimizing GCC 4.4.5 (IDA),numbers=left,style=customasmMIPS]{patterns/15_structs/3_tm_linux/MIPS/MIPS_O3_IDA_EN.lst}

This is an example where the branch delay slots can confuse us.

For example, there is the instruction \INS{addiu \$a1, 1900} at line 35 which adds 1900 to the year number.

It's executed before the corresponding \INS{JALR} at line 34, do not forget about it.

}

% subsection:
\EN{\input{patterns/15_structs/3_tm_linux/as_array/main_EN}}
\RU{\input{patterns/15_structs/3_tm_linux/as_array/main_RU}}
\DE{\input{patterns/15_structs/3_tm_linux/as_array/main_DE}}
\FR{\input{patterns/15_structs/3_tm_linux/as_array/main_FR}}
\JPN{\subsubsection{Structure as a set of values}

In order to illustrate that the structure is just variables laying side-by-side in one place, 
let's rework our example while looking at the \IT{tm} structure definition again: \lstref{struct_tm}.

\lstinputlisting[style=customc]{patterns/15_structs/3_tm_linux/as_array/GCC_tm2.c}

\myindex{\CStandardLibrary!localtime\_r()}
N.B. 
The pointer to the \TT{tm\_sec} field is passed into \TT{localtime\_r}, i.e., 
to the first element of the \q{structure}.

The compiler warns us:

\begin{lstlisting}[caption=GCC 4.7.3]
GCC_tm2.c: In function 'main':
GCC_tm2.c:11:5: warning: passing argument 2 of 'localtime_r' from incompatible pointer type [enabled by default]
In file included from GCC_tm2.c:2:0:
/usr/include/time.h:59:12: note: expected 'struct tm *' but argument is of type 'int *'
\end{lstlisting}

But nevertheless, it generates this:

\lstinputlisting[caption=GCC 4.7.3,style=customasmx86]{patterns/15_structs/3_tm_linux/as_array/GCC_tm2.asm}

This code is identical to what we saw previously and it is
not possible to say, was it a structure in original source code or just a pack of variables.

And this works. 
However, it is not recommended to do this in practice. 

Usually, non-optimizing compilers allocates variables in the local stack in the 
same order as they were declared in the function.

Nevertheless, there is no guarantee.

By the way, some other compiler may warn about the \TT{tm\_year}, \TT{tm\_mon}, \TT{tm\_mday},
\TT{tm\_hour}, \TT{tm\_min} variables, but not \TT{tm\_sec}
 are used without being initialized.

Indeed, the compiler is not aware that these are to be filled by\\
\TT{localtime\_r()} function.

We chose this example, since all structure fields are of type \Tint.%

This would not work if structure fields are 16-bit (\TT{WORD}), 
like in the case of the \TT{SYSTEMTIME} structure---\TT{GetSystemTime()} 
will fill them incorrectly 
(because the local variables are aligned on a 32-bit boundary).
Read more about it in next section: 
\q{\StructurePackingSectionName} (\myref{structure_packing}).

So, a structure is just a pack of variables laying in one place, side-by-side.
We could say that the structure is the instruction to the compiler, directing it to hold variables in one place.
By the way, in some very early C versions (before 1972), there were no structures at all \RitchieDevC.

There is no debugger example here: it is just the same as you already saw.

\subsubsection{Structure as an array of 32-bit words}

\lstinputlisting[style=customc]{patterns/15_structs/3_tm_linux/as_array/GCC_tm3.c}

We just \IT{cast} a pointer to structure to an array of \Tint{}'s.
And that works!
We run the example at 23:51:45 26-July-2014.

\begin{lstlisting}[label=GCC_tm3_output]
0x0000002D (45)
0x00000033 (51)
0x00000017 (23)
0x0000001A (26)
0x00000006 (6)
0x00000072 (114)
0x00000006 (6)
0x000000CE (206)
0x00000001 (1)
\end{lstlisting}

The variables here 
are in the same order as they are enumerated in the definition of the structure: \myref{struct_tm}.

Here is how it gets compiled:

\lstinputlisting[caption=\Optimizing GCC 4.8.1,style=customasmx86]{patterns/15_structs/3_tm_linux/as_array/GCC_tm3_JPN.lst}

Indeed: the space in the local stack is first treated as a structure, and then it's treated as an array.

It's even possible to modify the fields of the structure through this pointer.

And again, it's dubiously hackish way to do things, not recommended for use in production code.

\mysubparagraph{\Exercise}

As an exercise, try to modify (increase by 1) the current month number, treating the structure as 
an array.

\subsubsection{Structure as an array of bytes}

We can go even further. Let's \IT{cast} the pointer to an array of bytes and dump it:

\lstinputlisting[style=customc]{patterns/15_structs/3_tm_linux/as_array/GCC_tm4.c}

\begin{lstlisting}
0x2D 0x00 0x00 0x00 
0x33 0x00 0x00 0x00 
0x17 0x00 0x00 0x00 
0x1A 0x00 0x00 0x00 
0x06 0x00 0x00 0x00 
0x72 0x00 0x00 0x00 
0x06 0x00 0x00 0x00 
0xCE 0x00 0x00 0x00 
0x01 0x00 0x00 0x00 
\end{lstlisting}

We also run this example at 23:51:45 26-July-2014
\footnote{The time and date are the same for demonstration purposes. Byte values are fixed up.}.
The values are just the same as in the previous dump 
(\myref{GCC_tm3_output}), and of course, the lowest byte goes first, because this is a little-endian architecture 
(\myref{sec:endianness}).

\lstinputlisting[caption=\Optimizing GCC 4.8.1,style=customasmx86]{patterns/15_structs/3_tm_linux/as_array/GCC_tm4_JPN.lst}
}


\subsection{UNIX: struct tm}

% subsections here:
\EN{\input{patterns/15_structs/3_tm_linux/linux_EN}}
\RU{\input{patterns/15_structs/3_tm_linux/linux_RU}}
\DE{\input{patterns/15_structs/3_tm_linux/linux_DE}}
\FR{\input{patterns/15_structs/3_tm_linux/linux_FR}}
\JPN{\subsubsection{Linux}

Let's take the \TT{tm} structure from \TT{time.h} in Linux for example:

\lstinputlisting[style=customc]{patterns/15_structs/3_tm_linux/GCC_tm.c}

Let's compile it in GCC 4.4.1:

\lstinputlisting[caption=GCC 4.4.1,style=customasmx86]{patterns/15_structs/3_tm_linux/GCC_tm_EN.asm}

Somehow, \IDA did not write the local variables' names in the local stack.
But since we already are experienced reverse engineers :-) we may do it without this information in 
this simple example.

\myindex{x86!\Instructions!LEA}

Please also pay attention to the \TT{lea edx, [eax+76Ch]}~---this instruction just adds \TT{0x76C} (1900) to value in \EAX,
but doesn't modify any flags. See also the relevant section about \LEA{}~(\myref{sec:LEA}).

\myparagraph{GDB}

Let's try to load the example into GDB
\footnote{The \IT{date} result is slightly corrected for demonstration purposes.
Of course, it's not possible to run GDB that quickly, in the same second.}:

\lstinputlisting[caption=GDB]{patterns/15_structs/3_tm_linux/GCC_tm_GDB.txt}

We can easily find our structure in the stack.
First, let's see how it's defined in \IT{time.h}:

\begin{lstlisting}[caption=time.h, label=struct_tm,style=customc]
struct tm
{
  int	tm_sec;
  int	tm_min;
  int	tm_hour;
  int	tm_mday;
  int	tm_mon;
  int	tm_year;
  int	tm_wday;
  int	tm_yday;
  int	tm_isdst;
};
\end{lstlisting}

Pay attention that
32-bit \Tint is used here instead of WORD in SYSTEMTIME.
So, each field occupies 32-bit.

Here are the fields of our structure in the stack:

\begin{lstlisting}
0xbffff0dc:	0x080484c3	0x080485c0	0x000007de	0x00000000
0xbffff0ec:	0x08048301	0x538c93ed	0x00000025 sec	0x0000000a min
0xbffff0fc:	0x00000012 hour	0x00000002 mday	0x00000005 mon 	0x00000072 year
0xbffff10c:	0x00000001 wday	0x00000098 yday	0x00000001 isdst0x00002a30
0xbffff11c:	0x0804b090	0x08048530	0x00000000	0x00000000
\end{lstlisting}

Or as a table:

\begin{center}
\begin{tabular}{ | l | l | l | }
\hline
\headercolor{} Hexadecimal number & 
\headercolor{} decimal number & 
\headercolor{} field name \\
\hline
0x00000025 & 37 	& tm\_sec \\
\hline
0x0000000a & 10 	& tm\_min \\
\hline
0x00000012 & 18 	& tm\_hour \\	
\hline
0x00000002 & 2 		& tm\_mday \\	
\hline
0x00000005 & 5 		& tm\_mon \\	
\hline
0x00000072 & 114 	& tm\_year \\
\hline
0x00000001 & 1 		& tm\_wday \\	
\hline
0x00000098 & 152 	& tm\_yday \\	
\hline
0x00000001 & 1 		& tm\_isdst \\
\hline
\end{tabular}
\end{center}

Just like SYSTEMTIME (\myref{sec:SYSTEMTIME}), 

there are also other fields available that are not used, like tm\_wday, tm\_yday, tm\_isdst.
}

\EN{\input{patterns/15_structs/3_tm_linux/ARM/main_EN}}
\RU{\input{patterns/15_structs/3_tm_linux/ARM/main_RU}}
\DE{\input{patterns/15_structs/3_tm_linux/ARM/main_DE}}
\FR{\input{patterns/15_structs/3_tm_linux/ARM/main_FR}}
\JPN{\subsubsection{ARM}

\myparagraph{\OptimizingKeilVI (\ThumbMode)}

Same example:

\lstinputlisting[caption=\OptimizingKeilVI (\ThumbMode),style=customasmARM]{patterns/15_structs/3_tm_linux/ARM/tm_ARM_keil_thumb.asm}

\myparagraph{\OptimizingXcodeIV (\ThumbTwoMode)}

\IDA \q{knows} the \TT{tm} structure 
(because \IDA \q{knows} the types of the arguments of library functions like \TT{localtime\_r()}), 

so it shows here structure elements accesses and their names.

\lstinputlisting[caption=\OptimizingXcodeIV (\ThumbTwoMode),style=customasmARM]{patterns/15_structs/3_tm_linux/ARM/tm_ARM_xcode_thumb.asm}
}

\EN{\input{patterns/15_structs/3_tm_linux/MIPS/main_EN}}
\RU{\input{patterns/15_structs/3_tm_linux/MIPS/main_RU}}
\DE{\input{patterns/15_structs/3_tm_linux/MIPS/main_DE}}
\FR{\input{patterns/15_structs/3_tm_linux/MIPS/main_FR}}
\JPN{\subsubsection{MIPS}

\lstinputlisting[caption=\Optimizing GCC 4.4.5 (IDA),numbers=left,style=customasmMIPS]{patterns/15_structs/3_tm_linux/MIPS/MIPS_O3_IDA_EN.lst}

This is an example where the branch delay slots can confuse us.

For example, there is the instruction \INS{addiu \$a1, 1900} at line 35 which adds 1900 to the year number.

It's executed before the corresponding \INS{JALR} at line 34, do not forget about it.

}

% subsection:
\EN{\input{patterns/15_structs/3_tm_linux/as_array/main_EN}}
\RU{\input{patterns/15_structs/3_tm_linux/as_array/main_RU}}
\DE{\input{patterns/15_structs/3_tm_linux/as_array/main_DE}}
\FR{\input{patterns/15_structs/3_tm_linux/as_array/main_FR}}
\JPN{\subsubsection{Structure as a set of values}

In order to illustrate that the structure is just variables laying side-by-side in one place, 
let's rework our example while looking at the \IT{tm} structure definition again: \lstref{struct_tm}.

\lstinputlisting[style=customc]{patterns/15_structs/3_tm_linux/as_array/GCC_tm2.c}

\myindex{\CStandardLibrary!localtime\_r()}
N.B. 
The pointer to the \TT{tm\_sec} field is passed into \TT{localtime\_r}, i.e., 
to the first element of the \q{structure}.

The compiler warns us:

\begin{lstlisting}[caption=GCC 4.7.3]
GCC_tm2.c: In function 'main':
GCC_tm2.c:11:5: warning: passing argument 2 of 'localtime_r' from incompatible pointer type [enabled by default]
In file included from GCC_tm2.c:2:0:
/usr/include/time.h:59:12: note: expected 'struct tm *' but argument is of type 'int *'
\end{lstlisting}

But nevertheless, it generates this:

\lstinputlisting[caption=GCC 4.7.3,style=customasmx86]{patterns/15_structs/3_tm_linux/as_array/GCC_tm2.asm}

This code is identical to what we saw previously and it is
not possible to say, was it a structure in original source code or just a pack of variables.

And this works. 
However, it is not recommended to do this in practice. 

Usually, non-optimizing compilers allocates variables in the local stack in the 
same order as they were declared in the function.

Nevertheless, there is no guarantee.

By the way, some other compiler may warn about the \TT{tm\_year}, \TT{tm\_mon}, \TT{tm\_mday},
\TT{tm\_hour}, \TT{tm\_min} variables, but not \TT{tm\_sec}
 are used without being initialized.

Indeed, the compiler is not aware that these are to be filled by\\
\TT{localtime\_r()} function.

We chose this example, since all structure fields are of type \Tint.%

This would not work if structure fields are 16-bit (\TT{WORD}), 
like in the case of the \TT{SYSTEMTIME} structure---\TT{GetSystemTime()} 
will fill them incorrectly 
(because the local variables are aligned on a 32-bit boundary).
Read more about it in next section: 
\q{\StructurePackingSectionName} (\myref{structure_packing}).

So, a structure is just a pack of variables laying in one place, side-by-side.
We could say that the structure is the instruction to the compiler, directing it to hold variables in one place.
By the way, in some very early C versions (before 1972), there were no structures at all \RitchieDevC.

There is no debugger example here: it is just the same as you already saw.

\subsubsection{Structure as an array of 32-bit words}

\lstinputlisting[style=customc]{patterns/15_structs/3_tm_linux/as_array/GCC_tm3.c}

We just \IT{cast} a pointer to structure to an array of \Tint{}'s.
And that works!
We run the example at 23:51:45 26-July-2014.

\begin{lstlisting}[label=GCC_tm3_output]
0x0000002D (45)
0x00000033 (51)
0x00000017 (23)
0x0000001A (26)
0x00000006 (6)
0x00000072 (114)
0x00000006 (6)
0x000000CE (206)
0x00000001 (1)
\end{lstlisting}

The variables here 
are in the same order as they are enumerated in the definition of the structure: \myref{struct_tm}.

Here is how it gets compiled:

\lstinputlisting[caption=\Optimizing GCC 4.8.1,style=customasmx86]{patterns/15_structs/3_tm_linux/as_array/GCC_tm3_JPN.lst}

Indeed: the space in the local stack is first treated as a structure, and then it's treated as an array.

It's even possible to modify the fields of the structure through this pointer.

And again, it's dubiously hackish way to do things, not recommended for use in production code.

\mysubparagraph{\Exercise}

As an exercise, try to modify (increase by 1) the current month number, treating the structure as 
an array.

\subsubsection{Structure as an array of bytes}

We can go even further. Let's \IT{cast} the pointer to an array of bytes and dump it:

\lstinputlisting[style=customc]{patterns/15_structs/3_tm_linux/as_array/GCC_tm4.c}

\begin{lstlisting}
0x2D 0x00 0x00 0x00 
0x33 0x00 0x00 0x00 
0x17 0x00 0x00 0x00 
0x1A 0x00 0x00 0x00 
0x06 0x00 0x00 0x00 
0x72 0x00 0x00 0x00 
0x06 0x00 0x00 0x00 
0xCE 0x00 0x00 0x00 
0x01 0x00 0x00 0x00 
\end{lstlisting}

We also run this example at 23:51:45 26-July-2014
\footnote{The time and date are the same for demonstration purposes. Byte values are fixed up.}.
The values are just the same as in the previous dump 
(\myref{GCC_tm3_output}), and of course, the lowest byte goes first, because this is a little-endian architecture 
(\myref{sec:endianness}).

\lstinputlisting[caption=\Optimizing GCC 4.8.1,style=customasmx86]{patterns/15_structs/3_tm_linux/as_array/GCC_tm4_JPN.lst}
}


\input{appendix/GCC_library}
\input{appendix/MSVC_library}
\input{appendix/cheatsheets}
}
\JPN{\part*{\RU{Приложение}\EN{Appendix}\DE{Anhang}\FR{Appendice}\JPN{付録}}
\appendix
\addcontentsline{toc}{part}{\RU{Приложение}\EN{Appendix}\DE{Anhang}\FR{Appendice}\JPN{付録}}

% chapters
\EN{\input{appendix/x86/main_EN}}
\RU{\input{appendix/x86/main_RU}}
\DE{\input{appendix/x86/main_DE}}
\FR{\input{appendix/x86/main_FR}}
\JPN{\mysection{x86}

\subsection{Terminology}

Common for 16-bit (8086/80286), 32-bit (80386, etc.), 64-bit.

\myindex{IEEE 754}
\myindex{MS-DOS}
\begin{description}
	\item[byte] 8-bit.
		The DB assembly directive is used for defining variables and arrays of bytes.
		Bytes are passed in the 8-bit part of registers: \TT{AL/BL/CL/DL/AH/BH/CH/DH/SIL/DIL/R*L}.
	\item[word] 16-bit. 
		DW assembly directive \dittoclosing.
		Words are passed in the 16-bit part of the registers:\\
			\TT{AX/BX/CX/DX/SI/DI/R*W}.
	\item[double word] (\q{dword}) 32-bit.
		DD assembly directive \dittoclosing.
		Double words are passed in registers (x86) or in the 32-bit part of registers (x64). 
		In 16-bit code, double words are passed in 16-bit register pairs.
	\item[quad word] (\q{qword}) 64-bit.
		DQ assembly directive \dittoclosing.
		In 32-bit environment, quad words are passed in 32-bit register pairs.
	\item[tbyte] (10 bytes) 80-bit or 10 bytes (used for IEEE 754 FPU registers).
	\item[paragraph] (16 bytes)---term was popular in MS-DOS environment. % TODO link to a part about 8086 memory model...
\end{description}

\myindex{Windows!API}

Data types of the same width (BYTE, WORD, DWORD) are also the same in Windows \ac{API}.

\input{appendix/x86/registers} % subsection
\subsection{\RU{Инструкции}\EN{Instructions}}
\label{sec:x86_instructions}

\RU{Инструкции, отмеченные как (M) обычно не генерируются компилятором: если вы видите её, очень может быть
это вручную написанный фрагмент кода, либо это т.н. compiler intrinsic}
\EN{Instructions marked as (M) are not usually generated by the compiler: if you see one of them, it is probably
a hand-written piece of assembly code, or a compiler intrinsic} (\myref{sec:compiler_intrinsic}).

% TODO ? обратные инструкции

\RU{Только наиболее используемые инструкции перечислены здесь}
\EN{Only the most frequently used instructions are listed here}.
\EN{You can read \myref{x86_manuals} for a full documentation.}%
\RU{Обращайтесь к \myref{x86_manuals} для полной документации.}

\RU{Нужно ли заучивать опкоды инструкций на память?}\EN{Do you have to know all instruction's opcodes by heart?}
\RU{Нет, только те, которые часто используются для модификации кода}\EN{No, only those
which are used for code patching} (\myref{x86_patching}).
\RU{Остальные запоминать нет смысла.}\EN{All the rest of the opcodes don't need to be memorized.}

\subsubsection{\RU{Префиксы}\EN{Prefixes}}

\myindex{x86!\Prefixes!LOCK}
\myindex{x86!\Prefixes!REP}
\myindex{x86!\Prefixes!REPE/REPNE}
\begin{description}
\label{x86_lock}
\item[LOCK] \RU{используется чтобы предоставить эксклюзивный доступ к памяти в многопроцессорной среде}
\EN{forces CPU to make exclusive access to the RAM in multiprocessor environment}.
\RU{Для упрощения, можно сказать, что когда исполняется инструкция с этим префиксом, остальные процессоры
в системе останавливаются}\EN{For the sake of simplification, it can be said that when an instruction
with this prefix is executed, all other CPUs in a multiprocessor system are stopped}.
\RU{Чаще все это используется для критических секций, семафоров, мьютексов}\EN{Most often
it is used for critical sections, semaphores, mutexes}.
\RU{Обычно используется с}\EN{Commonly used with} ADD, AND, BTR, BTS, CMPXCHG, OR, XADD, XOR.
\RU{Читайте больше о критических секциях}\EN{You can read more about critical sections here} (\myref{critical_sections}).

\item[REP] \RU{используется с инструкциями}\EN{is used with the} MOVSx \AndENRU STOSx\EN{ instructions}:
\RU{инструкция будет исполняться в цикле, счетчик расположен в регистре CX/ECX/RCX}
\EN{execute the instruction in a loop, the counter is located in the CX/ECX/RCX register}.
\RU{Для более детального описания, читайте больше об инструкциях}
\EN{For a detailed description, read more about the} MOVSx (\myref{REP_MOVSx}) 
\AndENRU STOSx (\myref{REP_STOSx})\EN{ instructions}.

\RU{Работа инструкций с префиксом REP зависит от флага DF, он задает направление}
\EN{The instructions prefixed by REP are sensitive to the DF flag, which is used to set the direction}.

\item[REPE/REPNE] (\ac{AKA} REPZ/REPNZ) \RU{используется с инструкциями}\EN{used with} CMPSx \AndENRU
SCASx\EN{ instructions}:
\RU{инструкция будет исполняться в цикле, счетчик расположен в регистре \TT{CX}/\TT{ECX}/\TT{RCX}}
\EN{execute the last instruction in a loop, the count is set in the \TT{CX}/\TT{ECX}/\TT{RCX} register}. 
\RU{Выполнение будет прервано если ZF будет 0 (REPE) либо если ZF будет 1 (REPNE)}
\EN{It terminates prematurely if ZF is 0 (REPE) or if ZF is 1 (REPNE)}.

\RU{Для более детального описания, читайте больше об инструкциях}
\EN{For a detailed description, you can read more about the} CMPSx (\myref{REPE_CMPSx}) 
\AndENRU SCASx (\myref{REPNE_SCASx})\EN{ instructions}.

\RU{Работа инструкций с префиксами REPE/REPNE зависит от флага DF, он задает направление}
\EN{Instructions prefixed by REPE/REPNE are sensitive to the DF flag, which is used to set the direction}.

\end{description}

\subsubsection{\RU{Наиболее часто используемые инструкции}\EN{Most frequently used instructions}}

\RU{Их можно заучить в первую очередь}\EN{These can be memorized in the first place}.

\begin{description}
% in order to keep them easily sorted...
\input{appendix/x86/instructions/ADC}
\input{appendix/x86/instructions/ADD}
\input{appendix/x86/instructions/AND}
\input{appendix/x86/instructions/CALL}
\input{appendix/x86/instructions/CMP}
\input{appendix/x86/instructions/DEC}
\input{appendix/x86/instructions/IMUL}
\input{appendix/x86/instructions/INC}
\input{appendix/x86/instructions/JCXZ}
\input{appendix/x86/instructions/JMP}
\input{appendix/x86/instructions/Jcc}
\input{appendix/x86/instructions/LAHF}
\input{appendix/x86/instructions/LEAVE}
\input{appendix/x86/instructions/LEA}
\input{appendix/x86/instructions/MOVSB_W_D_Q}
\input{appendix/x86/instructions/MOVSX}
\input{appendix/x86/instructions/MOVZX}
\input{appendix/x86/instructions/MOV}
\input{appendix/x86/instructions/MUL}
\input{appendix/x86/instructions/NEG}
\input{appendix/x86/instructions/NOP}
\input{appendix/x86/instructions/NOT}
\input{appendix/x86/instructions/OR}
\input{appendix/x86/instructions/POP}
\input{appendix/x86/instructions/PUSH}
\input{appendix/x86/instructions/RET}
\input{appendix/x86/instructions/SAHF}
\input{appendix/x86/instructions/SBB}
\input{appendix/x86/instructions/SCASB_W_D_Q}
\input{appendix/x86/instructions/SHx}
\input{appendix/x86/instructions/SHRD}
\input{appendix/x86/instructions/STOSB_W_D_Q}
\input{appendix/x86/instructions/SUB}
\input{appendix/x86/instructions/TEST}
\input{appendix/x86/instructions/XOR}
\end{description}

\subsubsection{\RU{Реже используемые инструкции}\EN{Less frequently used instructions}}

\begin{description}
\input{appendix/x86/instructions/BSF}
\input{appendix/x86/instructions/BSR}
\input{appendix/x86/instructions/BSWAP}
\input{appendix/x86/instructions/BTC}
\input{appendix/x86/instructions/BTR}
\input{appendix/x86/instructions/BTS}
\input{appendix/x86/instructions/BT}
\input{appendix/x86/instructions/CBW_CWDE_CDQ}
\input{appendix/x86/instructions/CLD}
\input{appendix/x86/instructions/CLI}
\input{appendix/x86/instructions/CMC}
\input{appendix/x86/instructions/CMOVcc}
\input{appendix/x86/instructions/CMPSB_W_D_Q}
\input{appendix/x86/instructions/CPUID}
\input{appendix/x86/instructions/DIV}
\input{appendix/x86/instructions/IDIV}
\input{appendix/x86/instructions/INT}
\input{appendix/x86/instructions/IN}
\input{appendix/x86/instructions/IRET}
\input{appendix/x86/instructions/LOOP}
\input{appendix/x86/instructions/OUT}
\input{appendix/x86/instructions/POPA}
\input{appendix/x86/instructions/POPCNT}
\input{appendix/x86/instructions/POPF}
\input{appendix/x86/instructions/PUSHA}
\input{appendix/x86/instructions/PUSHF}
\input{appendix/x86/instructions/RCx}
\input{appendix/x86/instructions/ROx}
\input{appendix/x86/instructions/SAL}
\input{appendix/x86/instructions/SAR}
\input{appendix/x86/instructions/SETcc}
\input{appendix/x86/instructions/STC}
\input{appendix/x86/instructions/STD}
\input{appendix/x86/instructions/STI}
\input{appendix/x86/instructions/SYSCALL}
\input{appendix/x86/instructions/SYSENTER}
\input{appendix/x86/instructions/UD2}
\input{appendix/x86/instructions/XCHG}
\end{description}

\subsubsection{\RU{Инструкции FPU}\EN{FPU instructions}}

\RU{Суффикс \TT{-R} в названии инструкции обычно означает, что операнды поменяны местами, суффикс \TT{-P} означает
что один элемент выталкивается из стека после исполнения инструкции, суффикс \TT{-PP} означает, что
выталкиваются два элемента}%
\EN{\TT{-R} suffix in the mnemonic usually implies that the operands are reversed,
\TT{-P} suffix implies that one element is popped
from the stack after the instruction's execution, \TT{-PP} suffix implies that two elements are popped}.

\TT{-P} \RU{инструкции часто бывают полезны, когда нам уже больше не нужно хранить значение в 
FPU-стеке после операции.}%
\EN{instructions are often useful when we do not need the value in the FPU stack to be 
present anymore after the operation.}

\begin{description}
\input{appendix/x86/instructions/FABS}
\input{appendix/x86/instructions/FADD} % + FADDP
\input{appendix/x86/instructions/FCHS}
\input{appendix/x86/instructions/FCOM} % + FCOMP + FCOMPP
\input{appendix/x86/instructions/FDIVR} % + FDIVRP
\input{appendix/x86/instructions/FDIV} % + FDIVP
\input{appendix/x86/instructions/FILD}
\input{appendix/x86/instructions/FIST} % + FISTP
\input{appendix/x86/instructions/FLD1}
\input{appendix/x86/instructions/FLDCW}
\input{appendix/x86/instructions/FLDZ}
\input{appendix/x86/instructions/FLD}
\input{appendix/x86/instructions/FMUL} % + FMULP
\input{appendix/x86/instructions/FSINCOS}
\input{appendix/x86/instructions/FSQRT}
\input{appendix/x86/instructions/FSTCW} % + FNSTCW
\input{appendix/x86/instructions/FSTSW} % + FNSTSW
\input{appendix/x86/instructions/FST}
\input{appendix/x86/instructions/FSUBR} % + FSUBRP
\input{appendix/x86/instructions/FSUB} % + FSUBP
\input{appendix/x86/instructions/FUCOM} % + FUCOMP + FUCOMPP
\input{appendix/x86/instructions/FXCH}
\end{description}

%\subsubsection{\RU{SIMD-инструкции}\EN{SIMD instructions}}

% TODO

%\begin{description}
%\input{appendix/x86/instructions/DIVSD}
%\input{appendix/x86/instructions/MOVDQA}
%\input{appendix/x86/instructions/MOVDQU}
%\input{appendix/x86/instructions/PADDD}
%\input{appendix/x86/instructions/PCMPEQB}
%\input{appendix/x86/instructions/PLMULHW}
%\input{appendix/x86/instructions/PLMULLD}
%\input{appendix/x86/instructions/PMOVMSKB}
%\input{appendix/x86/instructions/PXOR}
%\end{description}

% SHLD !
% SHRD !
% BSWAP !
% CMPXCHG
% XADD !
% CMPXCHG8B
% RDTSC !
% PAUSE!

% xsave
% fnclex, fnsave
% movsxd, movaps, wait, sfence, lfence, pushfq
% prefetchw
% REP RETN
% REP BSF
% movnti, movntdq, rdmsr, wrmsr
% ldmxcsr, stmxcsr, invlpg
% swapgs
% movq, movd
% mulsd
% POR
% IRETQ
% pslldq
% psrldq
% cqo, fxrstor, comisd, xrstor, wbinvd, movntq
% fprem
% addsb, subsd, frndint

% rare:
%\item[ENTER]
%\item[LES]
% LDS
% XLAT

\subsubsection{\RU{Инструкции с печатаемым ASCII-опкодом}\EN{Instructions having printable ASCII opcode}}

(\RU{В 32-битном режиме}\EN{In 32-bit mode}).

\label{printable_x86_opcodes}
\myindex{Shellcode}
\RU{Это может пригодиться для создания шеллкодов}\EN{These can be suitable for shellcode construction}.
\RU{См. также}\EN{See also}: \myref{subsec:EICAR}.

% FIXME: break table
% FIXME: start at 0x20...
\begin{center}
\begin{longtable}{ | l | l | l | }
\hline
\HeaderColor ASCII\RU{-символ}\EN{ character} & 
\HeaderColor \RU{шестнадцатеричный код}\EN{hexadecimal code} & 
\HeaderColor x86\RU{-инструкция}\EN{ instruction} \\
\hline
0	 &30	 &XOR \\
1	 &31	 &XOR \\
2	 &32	 &XOR \\
3	 &33	 &XOR \\
4	 &34	 &XOR \\
5	 &35	 &XOR \\
7	 &37	 &AAA \\
8	 &38	 &CMP \\
9	 &39	 &CMP \\
:	 &3a	 &CMP \\
;	 &3b	 &CMP \\
<	 &3c	 &CMP \\
=	 &3d	 &CMP \\
?	 &3f	 &AAS \\
@	 &40	 &INC \\
A	 &41	 &INC \\
B	 &42	 &INC \\
C	 &43	 &INC \\
D	 &44	 &INC \\
E	 &45	 &INC \\
F	 &46	 &INC \\
G	 &47	 &INC \\
H	 &48	 &DEC \\
I	 &49	 &DEC \\
J	 &4a	 &DEC \\
K	 &4b	 &DEC \\
L	 &4c	 &DEC \\
M	 &4d	 &DEC \\
N	 &4e	 &DEC \\
O	 &4f	 &DEC \\
P	 &50	 &PUSH \\
Q	 &51	 &PUSH \\
R	 &52	 &PUSH \\
S	 &53	 &PUSH \\
T	 &54	 &PUSH \\
U	 &55	 &PUSH \\
V	 &56	 &PUSH \\
W	 &57	 &PUSH \\
X	 &58	 &POP \\
Y	 &59	 &POP \\
Z	 &5a	 &POP \\
\lbrack{}	 &5b	 &POP \\
\textbackslash{}	 &5c	 &POP \\
\rbrack{}	 &5d	 &POP \\
\verb|^|	 &5e	 &POP \\
\_	 &5f	 &POP \\
\verb|`|	 &60	 &PUSHA \\
a	 &61	 &POPA \\
f	 &66	 &\RU{(в 32-битном режиме) переключиться на}\EN{(in 32-bit mode) switch to}\\
   & & \RU{16-битный размер операнда}\EN{16-bit operand size} \\
g	 &67	 &\RU{(в 32-битном режиме) переключиться на}\EN{in 32-bit mode) switch to}\\
   & & \RU{16-битный размер адреса}\EN{16-bit address size} \\
h	 &68	 &PUSH\\
i	 &69	 &IMUL\\
j	 &6a	 &PUSH\\
k	 &6b	 &IMUL\\
p	 &70	 &JO\\
q	 &71	 &JNO\\
r	 &72	 &JB\\
s	 &73	 &JAE\\
t	 &74	 &JE\\
u	 &75	 &JNE\\
v	 &76	 &JBE\\
w	 &77	 &JA\\
x	 &78	 &JS\\
y	 &79	 &JNS\\
z	 &7a	 &JP\\
\hline
\end{longtable}
\end{center}

\myindex{x86!\Instructions!AAA}
\myindex{x86!\Instructions!AAS}
\myindex{x86!\Instructions!CMP}
\myindex{x86!\Instructions!DEC}
\myindex{x86!\Instructions!IMUL}
\myindex{x86!\Instructions!INC}
\myindex{x86!\Instructions!JA}
\myindex{x86!\Instructions!JAE}
\myindex{x86!\Instructions!JB}
\myindex{x86!\Instructions!JBE}
\myindex{x86!\Instructions!JE}
\myindex{x86!\Instructions!JNE}
\myindex{x86!\Instructions!JNO}
\myindex{x86!\Instructions!JNS}
\myindex{x86!\Instructions!JO}
\myindex{x86!\Instructions!JP}
\myindex{x86!\Instructions!JS}
\myindex{x86!\Instructions!POP}
\myindex{x86!\Instructions!POPA}
\myindex{x86!\Instructions!PUSH}
\myindex{x86!\Instructions!PUSHA}
\myindex{x86!\Instructions!XOR}

\RU{В итоге}\EN{In summary}:
AAA, AAS, CMP, DEC, IMUL, INC, JA, JAE, JB, JBE, JE, JNE, JNO, JNS, JO, JP, JS, POP, POPA, PUSH, PUSHA, 
XOR.

 % subsection
\subsection{npad}
\label{sec:npad}

\RU{Это макрос в ассемблере, для выравнивания некоторой метки по некоторой границе.}
\EN{It is an assembly language macro for aligning labels on a specific boundary.}
\DE{Dies ist ein Assembler-Makro um Labels an bestimmten Grenzen auszurichten.}
\FR{C'est une macro du langage d'assemblage pour aligner les labels sur une limite
spécifique.}
\JPN{It is an assembly language macro for aligning labels on a specific boundary.}

\RU{Это нужно для тех \IT{нагруженных} меток, куда чаще всего передается управление, например, 
начало тела цикла. 
Для того чтобы процессор мог эффективнее вытягивать данные или код из памяти, через шину с памятью, 
кэширование, итд.}
\EN{That's often needed for the busy labels to where the control flow is often passed, e.g., loop body starts.
So the CPU can load the data or code from the memory effectively, through the memory bus, cache lines, etc.}
\DE{Dies ist oft nützlich Labels, die oft Ziel einer Kotrollstruktur sind, wie Schleifenköpfe.
Somit kann die CPU Daten oder Code sehr effizient vom Speicher durch den Bus, den Cache, usw. laden.}
\FR{C'est souvent necessaire pour des labels très utilisés, comme par exemple le
début d'un corps de boucle. Ainsi, le CPU peut charger les données ou le code depuis
la mémoire efficacement, à travers le bus mémoire, les caches, etc.}

\RU{Взято из}\EN{Taken from}\DE{Entnommen von}\FR{Pris de} \TT{listing.inc} (MSVC):

\myindex{x86!\Instructions!NOP}
\RU{Это, кстати, любопытный пример различных вариантов \NOP{}-ов. 
Все эти инструкции не дают никакого эффекта, но отличаются разной длиной.}
\EN{By the way, it is a curious example of the different \NOP variations.
All these instructions have no effects whatsoever, but have a different size.}
\DE{Dies ist übrigens ein Beispiel für die unterschiedlichen \NOP-Variationen.
Keine dieser Anweisungen hat eine Auswirkung, aber alle haben eine unterschiedliche Größe.}
\FR{À propos, c'est un exemple curieux des différentes variations de \NOP. Toutes
ces instructions n'ont pas d'effet, mais ont une taille différente.}

\RU{Цель в том, чтобы была только одна инструкция, а не набор NOP-ов, 
считается что так лучше для производительности CPU.}
\EN{Having a single idle instruction instead of couple of NOP-s,
is accepted to be better for CPU performance.}
\DE{Eine einzelne Idle-Anweisung anstatt mehrerer NOPs hat positive Auswirkungen
auf die CPU-Performance.}
\FR{Avoir une seule instruction sans effet au lieu de plusieurs est accepté comme
étant meilleur pour la performance du CPU.}

\begin{lstlisting}[style=customasmx86]
;; LISTING.INC
;;
;; This file contains assembler macros and is included by the files created
;; with the -FA compiler switch to be assembled by MASM (Microsoft Macro
;; Assembler).
;;
;; Copyright (c) 1993-2003, Microsoft Corporation. All rights reserved.

;; non destructive nops
npad macro size
if size eq 1
  nop
else
 if size eq 2
   mov edi, edi
 else
  if size eq 3
    ; lea ecx, [ecx+00]
    DB 8DH, 49H, 00H
  else
   if size eq 4
     ; lea esp, [esp+00]
     DB 8DH, 64H, 24H, 00H
   else
    if size eq 5
      add eax, DWORD PTR 0
    else
     if size eq 6
       ; lea ebx, [ebx+00000000]
       DB 8DH, 9BH, 00H, 00H, 00H, 00H
     else
      if size eq 7
	; lea esp, [esp+00000000]
	DB 8DH, 0A4H, 24H, 00H, 00H, 00H, 00H 
      else
       if size eq 8
        ; jmp .+8; .npad 6
	DB 0EBH, 06H, 8DH, 9BH, 00H, 00H, 00H, 00H
       else
        if size eq 9
         ; jmp .+9; .npad 7
         DB 0EBH, 07H, 8DH, 0A4H, 24H, 00H, 00H, 00H, 00H
        else
         if size eq 10
          ; jmp .+A; .npad 7; .npad 1
          DB 0EBH, 08H, 8DH, 0A4H, 24H, 00H, 00H, 00H, 00H, 90H
         else
          if size eq 11
           ; jmp .+B; .npad 7; .npad 2
           DB 0EBH, 09H, 8DH, 0A4H, 24H, 00H, 00H, 00H, 00H, 8BH, 0FFH
          else
           if size eq 12
            ; jmp .+C; .npad 7; .npad 3
            DB 0EBH, 0AH, 8DH, 0A4H, 24H, 00H, 00H, 00H, 00H, 8DH, 49H, 00H
           else
            if size eq 13
             ; jmp .+D; .npad 7; .npad 4
             DB 0EBH, 0BH, 8DH, 0A4H, 24H, 00H, 00H, 00H, 00H, 8DH, 64H, 24H, 00H
            else
             if size eq 14
              ; jmp .+E; .npad 7; .npad 5
              DB 0EBH, 0CH, 8DH, 0A4H, 24H, 00H, 00H, 00H, 00H, 05H, 00H, 00H, 00H, 00H
             else
              if size eq 15
               ; jmp .+F; .npad 7; .npad 6
               DB 0EBH, 0DH, 8DH, 0A4H, 24H, 00H, 00H, 00H, 00H, 8DH, 9BH, 00H, 00H, 00H, 00H
              else
	       %out error: unsupported npad size
               .err
              endif
             endif
            endif
           endif
          endif
         endif
        endif
       endif
      endif
     endif
    endif
   endif
  endif
 endif
endif
endm
\end{lstlisting}
 % subsection

}
\subsection{UNIX: struct tm}

% subsections here:
\EN{\input{patterns/15_structs/3_tm_linux/linux_EN}}
\RU{\input{patterns/15_structs/3_tm_linux/linux_RU}}
\DE{\input{patterns/15_structs/3_tm_linux/linux_DE}}
\FR{\input{patterns/15_structs/3_tm_linux/linux_FR}}
\JPN{\subsubsection{Linux}

Let's take the \TT{tm} structure from \TT{time.h} in Linux for example:

\lstinputlisting[style=customc]{patterns/15_structs/3_tm_linux/GCC_tm.c}

Let's compile it in GCC 4.4.1:

\lstinputlisting[caption=GCC 4.4.1,style=customasmx86]{patterns/15_structs/3_tm_linux/GCC_tm_EN.asm}

Somehow, \IDA did not write the local variables' names in the local stack.
But since we already are experienced reverse engineers :-) we may do it without this information in 
this simple example.

\myindex{x86!\Instructions!LEA}

Please also pay attention to the \TT{lea edx, [eax+76Ch]}~---this instruction just adds \TT{0x76C} (1900) to value in \EAX,
but doesn't modify any flags. See also the relevant section about \LEA{}~(\myref{sec:LEA}).

\myparagraph{GDB}

Let's try to load the example into GDB
\footnote{The \IT{date} result is slightly corrected for demonstration purposes.
Of course, it's not possible to run GDB that quickly, in the same second.}:

\lstinputlisting[caption=GDB]{patterns/15_structs/3_tm_linux/GCC_tm_GDB.txt}

We can easily find our structure in the stack.
First, let's see how it's defined in \IT{time.h}:

\begin{lstlisting}[caption=time.h, label=struct_tm,style=customc]
struct tm
{
  int	tm_sec;
  int	tm_min;
  int	tm_hour;
  int	tm_mday;
  int	tm_mon;
  int	tm_year;
  int	tm_wday;
  int	tm_yday;
  int	tm_isdst;
};
\end{lstlisting}

Pay attention that
32-bit \Tint is used here instead of WORD in SYSTEMTIME.
So, each field occupies 32-bit.

Here are the fields of our structure in the stack:

\begin{lstlisting}
0xbffff0dc:	0x080484c3	0x080485c0	0x000007de	0x00000000
0xbffff0ec:	0x08048301	0x538c93ed	0x00000025 sec	0x0000000a min
0xbffff0fc:	0x00000012 hour	0x00000002 mday	0x00000005 mon 	0x00000072 year
0xbffff10c:	0x00000001 wday	0x00000098 yday	0x00000001 isdst0x00002a30
0xbffff11c:	0x0804b090	0x08048530	0x00000000	0x00000000
\end{lstlisting}

Or as a table:

\begin{center}
\begin{tabular}{ | l | l | l | }
\hline
\headercolor{} Hexadecimal number & 
\headercolor{} decimal number & 
\headercolor{} field name \\
\hline
0x00000025 & 37 	& tm\_sec \\
\hline
0x0000000a & 10 	& tm\_min \\
\hline
0x00000012 & 18 	& tm\_hour \\	
\hline
0x00000002 & 2 		& tm\_mday \\	
\hline
0x00000005 & 5 		& tm\_mon \\	
\hline
0x00000072 & 114 	& tm\_year \\
\hline
0x00000001 & 1 		& tm\_wday \\	
\hline
0x00000098 & 152 	& tm\_yday \\	
\hline
0x00000001 & 1 		& tm\_isdst \\
\hline
\end{tabular}
\end{center}

Just like SYSTEMTIME (\myref{sec:SYSTEMTIME}), 

there are also other fields available that are not used, like tm\_wday, tm\_yday, tm\_isdst.
}

\EN{\input{patterns/15_structs/3_tm_linux/ARM/main_EN}}
\RU{\input{patterns/15_structs/3_tm_linux/ARM/main_RU}}
\DE{\input{patterns/15_structs/3_tm_linux/ARM/main_DE}}
\FR{\input{patterns/15_structs/3_tm_linux/ARM/main_FR}}
\JPN{\subsubsection{ARM}

\myparagraph{\OptimizingKeilVI (\ThumbMode)}

Same example:

\lstinputlisting[caption=\OptimizingKeilVI (\ThumbMode),style=customasmARM]{patterns/15_structs/3_tm_linux/ARM/tm_ARM_keil_thumb.asm}

\myparagraph{\OptimizingXcodeIV (\ThumbTwoMode)}

\IDA \q{knows} the \TT{tm} structure 
(because \IDA \q{knows} the types of the arguments of library functions like \TT{localtime\_r()}), 

so it shows here structure elements accesses and their names.

\lstinputlisting[caption=\OptimizingXcodeIV (\ThumbTwoMode),style=customasmARM]{patterns/15_structs/3_tm_linux/ARM/tm_ARM_xcode_thumb.asm}
}

\EN{\input{patterns/15_structs/3_tm_linux/MIPS/main_EN}}
\RU{\input{patterns/15_structs/3_tm_linux/MIPS/main_RU}}
\DE{\input{patterns/15_structs/3_tm_linux/MIPS/main_DE}}
\FR{\input{patterns/15_structs/3_tm_linux/MIPS/main_FR}}
\JPN{\subsubsection{MIPS}

\lstinputlisting[caption=\Optimizing GCC 4.4.5 (IDA),numbers=left,style=customasmMIPS]{patterns/15_structs/3_tm_linux/MIPS/MIPS_O3_IDA_EN.lst}

This is an example where the branch delay slots can confuse us.

For example, there is the instruction \INS{addiu \$a1, 1900} at line 35 which adds 1900 to the year number.

It's executed before the corresponding \INS{JALR} at line 34, do not forget about it.

}

% subsection:
\EN{\input{patterns/15_structs/3_tm_linux/as_array/main_EN}}
\RU{\input{patterns/15_structs/3_tm_linux/as_array/main_RU}}
\DE{\input{patterns/15_structs/3_tm_linux/as_array/main_DE}}
\FR{\input{patterns/15_structs/3_tm_linux/as_array/main_FR}}
\JPN{\subsubsection{Structure as a set of values}

In order to illustrate that the structure is just variables laying side-by-side in one place, 
let's rework our example while looking at the \IT{tm} structure definition again: \lstref{struct_tm}.

\lstinputlisting[style=customc]{patterns/15_structs/3_tm_linux/as_array/GCC_tm2.c}

\myindex{\CStandardLibrary!localtime\_r()}
N.B. 
The pointer to the \TT{tm\_sec} field is passed into \TT{localtime\_r}, i.e., 
to the first element of the \q{structure}.

The compiler warns us:

\begin{lstlisting}[caption=GCC 4.7.3]
GCC_tm2.c: In function 'main':
GCC_tm2.c:11:5: warning: passing argument 2 of 'localtime_r' from incompatible pointer type [enabled by default]
In file included from GCC_tm2.c:2:0:
/usr/include/time.h:59:12: note: expected 'struct tm *' but argument is of type 'int *'
\end{lstlisting}

But nevertheless, it generates this:

\lstinputlisting[caption=GCC 4.7.3,style=customasmx86]{patterns/15_structs/3_tm_linux/as_array/GCC_tm2.asm}

This code is identical to what we saw previously and it is
not possible to say, was it a structure in original source code or just a pack of variables.

And this works. 
However, it is not recommended to do this in practice. 

Usually, non-optimizing compilers allocates variables in the local stack in the 
same order as they were declared in the function.

Nevertheless, there is no guarantee.

By the way, some other compiler may warn about the \TT{tm\_year}, \TT{tm\_mon}, \TT{tm\_mday},
\TT{tm\_hour}, \TT{tm\_min} variables, but not \TT{tm\_sec}
 are used without being initialized.

Indeed, the compiler is not aware that these are to be filled by\\
\TT{localtime\_r()} function.

We chose this example, since all structure fields are of type \Tint.%

This would not work if structure fields are 16-bit (\TT{WORD}), 
like in the case of the \TT{SYSTEMTIME} structure---\TT{GetSystemTime()} 
will fill them incorrectly 
(because the local variables are aligned on a 32-bit boundary).
Read more about it in next section: 
\q{\StructurePackingSectionName} (\myref{structure_packing}).

So, a structure is just a pack of variables laying in one place, side-by-side.
We could say that the structure is the instruction to the compiler, directing it to hold variables in one place.
By the way, in some very early C versions (before 1972), there were no structures at all \RitchieDevC.

There is no debugger example here: it is just the same as you already saw.

\subsubsection{Structure as an array of 32-bit words}

\lstinputlisting[style=customc]{patterns/15_structs/3_tm_linux/as_array/GCC_tm3.c}

We just \IT{cast} a pointer to structure to an array of \Tint{}'s.
And that works!
We run the example at 23:51:45 26-July-2014.

\begin{lstlisting}[label=GCC_tm3_output]
0x0000002D (45)
0x00000033 (51)
0x00000017 (23)
0x0000001A (26)
0x00000006 (6)
0x00000072 (114)
0x00000006 (6)
0x000000CE (206)
0x00000001 (1)
\end{lstlisting}

The variables here 
are in the same order as they are enumerated in the definition of the structure: \myref{struct_tm}.

Here is how it gets compiled:

\lstinputlisting[caption=\Optimizing GCC 4.8.1,style=customasmx86]{patterns/15_structs/3_tm_linux/as_array/GCC_tm3_JPN.lst}

Indeed: the space in the local stack is first treated as a structure, and then it's treated as an array.

It's even possible to modify the fields of the structure through this pointer.

And again, it's dubiously hackish way to do things, not recommended for use in production code.

\mysubparagraph{\Exercise}

As an exercise, try to modify (increase by 1) the current month number, treating the structure as 
an array.

\subsubsection{Structure as an array of bytes}

We can go even further. Let's \IT{cast} the pointer to an array of bytes and dump it:

\lstinputlisting[style=customc]{patterns/15_structs/3_tm_linux/as_array/GCC_tm4.c}

\begin{lstlisting}
0x2D 0x00 0x00 0x00 
0x33 0x00 0x00 0x00 
0x17 0x00 0x00 0x00 
0x1A 0x00 0x00 0x00 
0x06 0x00 0x00 0x00 
0x72 0x00 0x00 0x00 
0x06 0x00 0x00 0x00 
0xCE 0x00 0x00 0x00 
0x01 0x00 0x00 0x00 
\end{lstlisting}

We also run this example at 23:51:45 26-July-2014
\footnote{The time and date are the same for demonstration purposes. Byte values are fixed up.}.
The values are just the same as in the previous dump 
(\myref{GCC_tm3_output}), and of course, the lowest byte goes first, because this is a little-endian architecture 
(\myref{sec:endianness}).

\lstinputlisting[caption=\Optimizing GCC 4.8.1,style=customasmx86]{patterns/15_structs/3_tm_linux/as_array/GCC_tm4_JPN.lst}
}


\subsection{UNIX: struct tm}

% subsections here:
\EN{\input{patterns/15_structs/3_tm_linux/linux_EN}}
\RU{\input{patterns/15_structs/3_tm_linux/linux_RU}}
\DE{\input{patterns/15_structs/3_tm_linux/linux_DE}}
\FR{\input{patterns/15_structs/3_tm_linux/linux_FR}}
\JPN{\subsubsection{Linux}

Let's take the \TT{tm} structure from \TT{time.h} in Linux for example:

\lstinputlisting[style=customc]{patterns/15_structs/3_tm_linux/GCC_tm.c}

Let's compile it in GCC 4.4.1:

\lstinputlisting[caption=GCC 4.4.1,style=customasmx86]{patterns/15_structs/3_tm_linux/GCC_tm_EN.asm}

Somehow, \IDA did not write the local variables' names in the local stack.
But since we already are experienced reverse engineers :-) we may do it without this information in 
this simple example.

\myindex{x86!\Instructions!LEA}

Please also pay attention to the \TT{lea edx, [eax+76Ch]}~---this instruction just adds \TT{0x76C} (1900) to value in \EAX,
but doesn't modify any flags. See also the relevant section about \LEA{}~(\myref{sec:LEA}).

\myparagraph{GDB}

Let's try to load the example into GDB
\footnote{The \IT{date} result is slightly corrected for demonstration purposes.
Of course, it's not possible to run GDB that quickly, in the same second.}:

\lstinputlisting[caption=GDB]{patterns/15_structs/3_tm_linux/GCC_tm_GDB.txt}

We can easily find our structure in the stack.
First, let's see how it's defined in \IT{time.h}:

\begin{lstlisting}[caption=time.h, label=struct_tm,style=customc]
struct tm
{
  int	tm_sec;
  int	tm_min;
  int	tm_hour;
  int	tm_mday;
  int	tm_mon;
  int	tm_year;
  int	tm_wday;
  int	tm_yday;
  int	tm_isdst;
};
\end{lstlisting}

Pay attention that
32-bit \Tint is used here instead of WORD in SYSTEMTIME.
So, each field occupies 32-bit.

Here are the fields of our structure in the stack:

\begin{lstlisting}
0xbffff0dc:	0x080484c3	0x080485c0	0x000007de	0x00000000
0xbffff0ec:	0x08048301	0x538c93ed	0x00000025 sec	0x0000000a min
0xbffff0fc:	0x00000012 hour	0x00000002 mday	0x00000005 mon 	0x00000072 year
0xbffff10c:	0x00000001 wday	0x00000098 yday	0x00000001 isdst0x00002a30
0xbffff11c:	0x0804b090	0x08048530	0x00000000	0x00000000
\end{lstlisting}

Or as a table:

\begin{center}
\begin{tabular}{ | l | l | l | }
\hline
\headercolor{} Hexadecimal number & 
\headercolor{} decimal number & 
\headercolor{} field name \\
\hline
0x00000025 & 37 	& tm\_sec \\
\hline
0x0000000a & 10 	& tm\_min \\
\hline
0x00000012 & 18 	& tm\_hour \\	
\hline
0x00000002 & 2 		& tm\_mday \\	
\hline
0x00000005 & 5 		& tm\_mon \\	
\hline
0x00000072 & 114 	& tm\_year \\
\hline
0x00000001 & 1 		& tm\_wday \\	
\hline
0x00000098 & 152 	& tm\_yday \\	
\hline
0x00000001 & 1 		& tm\_isdst \\
\hline
\end{tabular}
\end{center}

Just like SYSTEMTIME (\myref{sec:SYSTEMTIME}), 

there are also other fields available that are not used, like tm\_wday, tm\_yday, tm\_isdst.
}

\EN{\input{patterns/15_structs/3_tm_linux/ARM/main_EN}}
\RU{\input{patterns/15_structs/3_tm_linux/ARM/main_RU}}
\DE{\input{patterns/15_structs/3_tm_linux/ARM/main_DE}}
\FR{\input{patterns/15_structs/3_tm_linux/ARM/main_FR}}
\JPN{\subsubsection{ARM}

\myparagraph{\OptimizingKeilVI (\ThumbMode)}

Same example:

\lstinputlisting[caption=\OptimizingKeilVI (\ThumbMode),style=customasmARM]{patterns/15_structs/3_tm_linux/ARM/tm_ARM_keil_thumb.asm}

\myparagraph{\OptimizingXcodeIV (\ThumbTwoMode)}

\IDA \q{knows} the \TT{tm} structure 
(because \IDA \q{knows} the types of the arguments of library functions like \TT{localtime\_r()}), 

so it shows here structure elements accesses and their names.

\lstinputlisting[caption=\OptimizingXcodeIV (\ThumbTwoMode),style=customasmARM]{patterns/15_structs/3_tm_linux/ARM/tm_ARM_xcode_thumb.asm}
}

\EN{\input{patterns/15_structs/3_tm_linux/MIPS/main_EN}}
\RU{\input{patterns/15_structs/3_tm_linux/MIPS/main_RU}}
\DE{\input{patterns/15_structs/3_tm_linux/MIPS/main_DE}}
\FR{\input{patterns/15_structs/3_tm_linux/MIPS/main_FR}}
\JPN{\subsubsection{MIPS}

\lstinputlisting[caption=\Optimizing GCC 4.4.5 (IDA),numbers=left,style=customasmMIPS]{patterns/15_structs/3_tm_linux/MIPS/MIPS_O3_IDA_EN.lst}

This is an example where the branch delay slots can confuse us.

For example, there is the instruction \INS{addiu \$a1, 1900} at line 35 which adds 1900 to the year number.

It's executed before the corresponding \INS{JALR} at line 34, do not forget about it.

}

% subsection:
\EN{\input{patterns/15_structs/3_tm_linux/as_array/main_EN}}
\RU{\input{patterns/15_structs/3_tm_linux/as_array/main_RU}}
\DE{\input{patterns/15_structs/3_tm_linux/as_array/main_DE}}
\FR{\input{patterns/15_structs/3_tm_linux/as_array/main_FR}}
\JPN{\subsubsection{Structure as a set of values}

In order to illustrate that the structure is just variables laying side-by-side in one place, 
let's rework our example while looking at the \IT{tm} structure definition again: \lstref{struct_tm}.

\lstinputlisting[style=customc]{patterns/15_structs/3_tm_linux/as_array/GCC_tm2.c}

\myindex{\CStandardLibrary!localtime\_r()}
N.B. 
The pointer to the \TT{tm\_sec} field is passed into \TT{localtime\_r}, i.e., 
to the first element of the \q{structure}.

The compiler warns us:

\begin{lstlisting}[caption=GCC 4.7.3]
GCC_tm2.c: In function 'main':
GCC_tm2.c:11:5: warning: passing argument 2 of 'localtime_r' from incompatible pointer type [enabled by default]
In file included from GCC_tm2.c:2:0:
/usr/include/time.h:59:12: note: expected 'struct tm *' but argument is of type 'int *'
\end{lstlisting}

But nevertheless, it generates this:

\lstinputlisting[caption=GCC 4.7.3,style=customasmx86]{patterns/15_structs/3_tm_linux/as_array/GCC_tm2.asm}

This code is identical to what we saw previously and it is
not possible to say, was it a structure in original source code or just a pack of variables.

And this works. 
However, it is not recommended to do this in practice. 

Usually, non-optimizing compilers allocates variables in the local stack in the 
same order as they were declared in the function.

Nevertheless, there is no guarantee.

By the way, some other compiler may warn about the \TT{tm\_year}, \TT{tm\_mon}, \TT{tm\_mday},
\TT{tm\_hour}, \TT{tm\_min} variables, but not \TT{tm\_sec}
 are used without being initialized.

Indeed, the compiler is not aware that these are to be filled by\\
\TT{localtime\_r()} function.

We chose this example, since all structure fields are of type \Tint.%

This would not work if structure fields are 16-bit (\TT{WORD}), 
like in the case of the \TT{SYSTEMTIME} structure---\TT{GetSystemTime()} 
will fill them incorrectly 
(because the local variables are aligned on a 32-bit boundary).
Read more about it in next section: 
\q{\StructurePackingSectionName} (\myref{structure_packing}).

So, a structure is just a pack of variables laying in one place, side-by-side.
We could say that the structure is the instruction to the compiler, directing it to hold variables in one place.
By the way, in some very early C versions (before 1972), there were no structures at all \RitchieDevC.

There is no debugger example here: it is just the same as you already saw.

\subsubsection{Structure as an array of 32-bit words}

\lstinputlisting[style=customc]{patterns/15_structs/3_tm_linux/as_array/GCC_tm3.c}

We just \IT{cast} a pointer to structure to an array of \Tint{}'s.
And that works!
We run the example at 23:51:45 26-July-2014.

\begin{lstlisting}[label=GCC_tm3_output]
0x0000002D (45)
0x00000033 (51)
0x00000017 (23)
0x0000001A (26)
0x00000006 (6)
0x00000072 (114)
0x00000006 (6)
0x000000CE (206)
0x00000001 (1)
\end{lstlisting}

The variables here 
are in the same order as they are enumerated in the definition of the structure: \myref{struct_tm}.

Here is how it gets compiled:

\lstinputlisting[caption=\Optimizing GCC 4.8.1,style=customasmx86]{patterns/15_structs/3_tm_linux/as_array/GCC_tm3_JPN.lst}

Indeed: the space in the local stack is first treated as a structure, and then it's treated as an array.

It's even possible to modify the fields of the structure through this pointer.

And again, it's dubiously hackish way to do things, not recommended for use in production code.

\mysubparagraph{\Exercise}

As an exercise, try to modify (increase by 1) the current month number, treating the structure as 
an array.

\subsubsection{Structure as an array of bytes}

We can go even further. Let's \IT{cast} the pointer to an array of bytes and dump it:

\lstinputlisting[style=customc]{patterns/15_structs/3_tm_linux/as_array/GCC_tm4.c}

\begin{lstlisting}
0x2D 0x00 0x00 0x00 
0x33 0x00 0x00 0x00 
0x17 0x00 0x00 0x00 
0x1A 0x00 0x00 0x00 
0x06 0x00 0x00 0x00 
0x72 0x00 0x00 0x00 
0x06 0x00 0x00 0x00 
0xCE 0x00 0x00 0x00 
0x01 0x00 0x00 0x00 
\end{lstlisting}

We also run this example at 23:51:45 26-July-2014
\footnote{The time and date are the same for demonstration purposes. Byte values are fixed up.}.
The values are just the same as in the previous dump 
(\myref{GCC_tm3_output}), and of course, the lowest byte goes first, because this is a little-endian architecture 
(\myref{sec:endianness}).

\lstinputlisting[caption=\Optimizing GCC 4.8.1,style=customasmx86]{patterns/15_structs/3_tm_linux/as_array/GCC_tm4_JPN.lst}
}


\input{appendix/GCC_library}
\input{appendix/MSVC_library}
\input{appendix/cheatsheets}
}
\include{acronyms}

\bookmarksetup{startatroot}

\clearpage
\phantomsection
\addcontentsline{toc}{chapter}{%
    \RU{Глоссарий}%
    \EN{Glossary}%
    \ES{Glosario}%
    \PTBRph{}%
    \DE{Glossar}%
    \PLph{}%
    \ITAph{}%
    \THAph{}\NLph{}%
    \FR{Glossaire}%
    \JPN{用語}
    \TR{Bolum}
}
\printglossaries

\clearpage
\phantomsection
\printindex

\end{document}
