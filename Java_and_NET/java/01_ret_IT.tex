% TODO proof-reading
\subsection{Ritornare un valore}

Le funzioni che semplicemente ritornano indietro un qualche valore sono probabilmente tra le più semplici in Java.

Naturalmente, quando parliamo di funzioni in Java facciamo riferimento implicito ai metodi di una classe, 
non essendo possibile definire funzioni \q{libere}, nel senso comune del termine.

Infatti, in Java ogni metodo (statico o dinamico che sia) è sempre definito in relazione ad una classe, 
poiché non è possibile fare diversamente.

Ad ogni modo, per semplicità, faremo uso del termine \q{funzione}.

\begin{lstlisting}[style=customjava]
public class ret
{
	public static int main(String[] args)
	{
		return 0;
	}
}
\end{lstlisting}

Per compilare il file si esegue:

\begin{lstlisting}
javac ret.java
\end{lstlisting}

\dots e possiamo decompilarne il risultato utilizzando una \q{utility} standard di Java:

\begin{lstlisting}
javap -c -verbose ret.class
\end{lstlisting}

Ecco ciò che si ottiene dalla decompilazione:

\begin{lstlisting}[caption=JDK 1.7 (excerpt)]
  public static int main(java.lang.String[]);
    flags: ACC_PUBLIC, ACC_STATIC
    Code:
      stack=1, locals=1, args_size=1
         0: iconst_0
         1: ireturn
\end{lstlisting}

L'utilizzo della costante 0 è particolarmente frequente nella programmazione, 
per questo esiste un'istruzione apposita \GTT{iconst\_0}, di un byte, che carica la costante nello \q{stack}.

\footnote{Come in in MIPS, dove esiste un registro separato per la costante zero: \myref{MIPS_zero_register}.}.

Similarmente, esistono altre istruzioni apposite per caricare costanti nello stack: \GTT{iconst\_1} (che imposta 1), 
\GTT{iconst\_2}, etc., fino a \GTT{iconst\_5}.

E c'è un'istruzione \GTT{iconst\_m1} apposita per caricare la costante -1.

Nella JVM, lo \q{stack} viene impiegato per il passaggio dei dati alle funzioni al momento della loro invocazione, 
e per consentire a queste ultime di ritornare a loro volta dei dati indietro. 
Quindi, osservando il codice dell'esempio precedente, la funzione \GTT{iconst\_0} carica il valore 0 nello \q{stack}.
\GTT{ireturn} ritorna indietro un intero (\emph{i} come prefisso nel nome della istruzione significa \emph{integer}), prelevandolo 
dalla testa dello stack \ac{TOS}.

Riscriviamo l'esempio precedente in modo da far ritornare indietro il valore 1234 alla funzione:

\begin{lstlisting}[style=customjava]
public class ret
{
	public static int main(String[] args)
	{
		return 1234;
	}
}
\end{lstlisting}

\dots quello che otteniamo ora dalla decompilazione è:

\begin{lstlisting}[caption=JDK 1.7 (excerpt)]
  public static int main(java.lang.String[]);
    flags: ACC_PUBLIC, ACC_STATIC
    Code:
      stack=1, locals=1, args_size=1
         0: sipush        1234
         3: ireturn
\end{lstlisting}

\GTT{sipush} (\emph{short integer}) imposta il numero 1234 nello \q{stack}.
\emph{short} come prefisso nel nome della istruzione indica il fatto di lavorare con dati a 16-bit e infatti,
il numero 1234 può essere rappresentato con 16-bit.

Cosa accade per valori più grandi?

\begin{lstlisting}[style=customjava]
public class ret
{
	public static int main(String[] args)
	{
		return 12345678;
	}
}
\end{lstlisting}

\begin{lstlisting}[caption=Constant pool]
...
   #2 = Integer            12345678
...
\end{lstlisting}

\begin{lstlisting}
  public static int main(java.lang.String[]);
    flags: ACC_PUBLIC, ACC_STATIC
    Code:
      stack=1, locals=1, args_size=1
         0: ldc           #2                  // int 12345678
         2: ireturn
\end{lstlisting}

Nella JVM non è possibile codificare numeri a 32-bit in un'istruzione \q{opcode}, 
i progettisti non hanno lasciato questa possibilità.

Di conseguenza, il numero 12345678 che deve essere rappresentato con 32-bit viene registrato in uno spazio specifico detto
\q{constant pool}, diciamo che questo spazio sia una sorta di libreria delle costanti più frequenti 
(inclusi i tipi di dati stringa, oggetti, etc.).

Questa modalità di gestire le costanti non si riscontra solo nella JVM.

Anche MIPS, ARM ed altre CPUs RISC non permettono di codificare numeri di 32-bit
in una istruzione \q{opcode} di 32-bit, di conseguenza il codice della CPU RISC (incluso MIPS e ARM)
si trova a dover codificare il valore di una costante in più passi, a meno di poter utilizzare un
\q{segmento dati}: \myref{ARM_big_constants}, \myref{MIPS_big_constants}.

Tradizionalmente il codice MIPS dispone anche di un \q{constant pool} chiamato \q{literal pool},
dove i segmenti sono chiamati \q{.lit4} (per i valori costanti a virgola mobile indirizzati con 32-bit in precisione singola) e \q{.lit8}
(per i valori costanti a virgola mobile indirizzati con 64-bit in precisione doppia).

Proviamo ora con altri tipi di dati!

Booleani:

\begin{lstlisting}[style=customjava]
public class ret
{
	public static boolean main(String[] args)
	{
		return true;
	}
}
\end{lstlisting}

\begin{lstlisting}
  public static boolean main(java.lang.String[]);
    flags: ACC_PUBLIC, ACC_STATIC
    Code:
      stack=1, locals=1, args_size=1
         0: iconst_1
         1: ireturn
\end{lstlisting}

Il bytecode prodotto non è diverso da quello che si otterrebbe per una funzione che semplicemente ritorna indietro il valore numerico 1.

Gli spazi da 32-bit per i dati nello stack sono utilizzati anche per i valori booleani, come nel \CCpp.

Tuttavia il valore booleano che la funzione ritorna non può essere utilizzato come valore intero o viceversa --- 
le informazioni sui tipi di dati dichiarati sono registrati nel file della classe e verificati a runtime.

La stessa storia vale per indirizzare valori numerici con 16-bit \emph{short}:

\begin{lstlisting}[style=customjava]
public class ret
{
	public static short main(String[] args)
	{
		return 1234;
	}
}
\end{lstlisting}

\begin{lstlisting}
  public static short main(java.lang.String[]);
    flags: ACC_PUBLIC, ACC_STATIC
    Code:
      stack=1, locals=1, args_size=1
         0: sipush        1234
         3: ireturn
\end{lstlisting}

\dots e \emph{char}!

\begin{lstlisting}[style=customjava]
public class ret
{
	public static char main(String[] args)
	{
		return 'A';
	}
}
\end{lstlisting}

\begin{lstlisting}
  public static char main(java.lang.String[]);
    flags: ACC_PUBLIC, ACC_STATIC
    Code:
      stack=1, locals=1, args_size=1
         0: bipush        65
         2: ireturn
\end{lstlisting}

\INS{bipush} significa \q{caricare un byte nello stack}.
In Java un \emph{char} viene indirizzato con la codifica UTF-16 a 16-bit,
ed è equivalente a \emph{short}. Il codice ASCII del carattere \q{A} è 65, per cui 
l'istruzione può caricarlo nello stack come singolo byte.

Proviamo ora con un \emph{byte}:

\begin{lstlisting}[style=customjava]
public class retc
{
	public static byte main(String[] args)
	{
		return 123;
	}
}
\end{lstlisting}

\begin{lstlisting}
  public static byte main(java.lang.String[]);
    flags: ACC_PUBLIC, ACC_STATIC
    Code:
      stack=1, locals=1, args_size=1
         0: bipush        123
         2: ireturn
\end{lstlisting}

Ci si può chiedere: perché scegliere il tipo dati a 16-bit \emph{short}, quando poi internamente
viene gestito con numeri interi a 32-bit?

Perché usare il tipo di dato specifico \emph{char}, quando quest'ultimo è equivalente al tipo \emph{short}?

La ragione è nella necessità di controllo sul tipo di dati dichiarati e per una questione di leggibilità del codice.

Se da una parte il tipo di dato \emph{char} è essenzialmente lo stesso di \emph{short}, dall'altra ci permette di intuire facilmente che si tratta
di uno spazio che ospita un carattere UTF-16 e non un qualsiasi valore intero.

Quando usiamo \emph{short} per una variabile, rendiamo esplicito a chiunque come i possibili valori siano limitati da 16 bit.

Così è una buona prassi usare il tipo di dato \emph{boolean} quando necessario,
piuttosto che \emph{int} come in stile C.

Esiste in Java anche il tipo di dato intero a 64-bit:

\begin{lstlisting}[style=customjava]
public class ret3
{
	public static long main(String[] args)
	{
		return 1234567890123456789L;
	}
}
\end{lstlisting}

\begin{lstlisting}[caption=Constant pool]
...
   #2 = Long               1234567890123456789l
...
\end{lstlisting}

\begin{lstlisting}
  public static long main(java.lang.String[]);
    flags: ACC_PUBLIC, ACC_STATIC
    Code:
      stack=2, locals=1, args_size=1
         0: ldc2_w        #2                  // long 1234567890123456789l
         3: lreturn
\end{lstlisting}

Anche il tipo intero a 64-bit viene registrato nel (\q{constant pool}): \INS{ldc2\_w} è l'istruzione che 
carica il valore nello stack e \INS{lreturn} (\emph{long return}) lo preleva dallo stack per ritornarlo indietro.

L'istruzione \INS{ldc2\_w} è anche impiegata per caricare nello stack numeri in virgola mobile con precisione doppia
(appunto indirizzati con 64 bit) dal \q{constant pool}:

\begin{lstlisting}[style=customjava]
public class ret
{
	public static double main(String[] args)
	{
		return 123.456d;
	}
}
\end{lstlisting}

\begin{lstlisting}[caption=Constant pool]
...
   #2 = Double             123.456d
...
\end{lstlisting}

\begin{lstlisting}
  public static double main(java.lang.String[]);
    flags: ACC_PUBLIC, ACC_STATIC
    Code:
      stack=2, locals=1, args_size=1
         0: ldc2_w        #2                  // double 123.456d
         3: dreturn
\end{lstlisting}

\INS{dreturn} significa \q{return double}.

Infine abbiamo i numeri con virgola mobile e precisione singola:

\begin{lstlisting}[style=customjava]
public class ret
{
	public static float main(String[] args)
	{
		return 123.456f;
	}
}
\end{lstlisting}

\begin{lstlisting}[caption=Constant pool]
...
   #2 = Float              123.456f
...
\end{lstlisting}

\begin{lstlisting}
  public static float main(java.lang.String[]);
    flags: ACC_PUBLIC, ACC_STATIC
    Code:
      stack=1, locals=1, args_size=1
         0: ldc           #2                  // float 123.456f
         2: freturn
\end{lstlisting}

Notare come l'istruzione \INS{ldc} è la stessa usata per caricare nello stack numeri interi a 32-bit dal \q{constant pool}.

\INS{freturn} significa \q{return float}.

Cosa accadrebbe nel caso di una funzione che non ritorna indietro alcun dato?

\begin{lstlisting}[style=customjava]
public class ret
{
	public static void main(String[] args)
	{
		return;
	}
}
\end{lstlisting}

\begin{lstlisting}
  public static void main(java.lang.String[]);
    flags: ACC_PUBLIC, ACC_STATIC
    Code:
      stack=0, locals=1, args_size=1
         0: return
\end{lstlisting}

Notare come l'istruzione \INS{return} sia impiegata per restituire il controllo al termine dell'esecuzione di una
funzione che non ritorna indietro alcun dato.

Quanto detto fino ad ora rende facile dedurre il tipo di dato che una funzione (o metodo) ritorna,
semplicemente osservando l'ultima istruzione.

