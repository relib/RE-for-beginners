\subsection{Exemple cdecl: DLL}
\label{cdecl_DLL}

Revenons au fait que la façon de déclarer la fonction \verb|main()| n'est pas très importante: \myref{main_arguments}.

Ceci est une histoire vraie: il était une fois où je voulais remplacer un fichier
original de DLL par le mien.
Premièrement, j'ai énuméré les noms de tous les exports de la DLL et j'ai écrit
une fonction dans ma propre DLL de remplacement pour chaque fonction de la DLL
originale, comme:

\begin{lstlisting}[style=customc]
void function1 ()
{
	write_to_log ("function1() called\n");
};
\end{lstlisting}

Je voulais voir quelles fonctions étaient appelées lors de l'exécution, et quand.
Toutefois, j'étais pressé et n'avais pas le temps de déduire le nombre d'arguments
pour chaque fonction, encore moins pour les types des données.
Donc chaque fonction dans ma DLL de remplacement n'avait aucun argument que ce soit.
Mais tout fonctionnait, car toutes les fonctions utilisaient la convention d'appel
\emph{cdecl}.
(Ça ne fonctionnerait pas si les fonctions avaient la convention d'appel \emph{stdcall}.)
Ça a aussi fonctionné pour la version x64.

Et puis j'ai fait une étape de plus: j'ai déduit les types des arguments pour
certaines fonctions.
Mais j'ai fait quelques erreurs, par exemple, la fonction originale prend 3 arguments,
mais je n'en ai découvert que 2, etc.

Cependant, ça fonctionnait.
Au début, ma DLL de remplacement ignorait simplement tous les arguments.
Puis, elle ignorait le 3ème argument.
