\mysection{Patterns de code suspect}

\subsection{instructions XOR}
\myindex{x86!\Instructions!XOR}

Des instructions comme \TT{XOR op, op} (par exemple, \TT{XOR EAX, EAX}) sont utilisées
en général pour mettre la valeur d'un registre à zéro, mais si les opérandes sont
différentes, l'opération \q{ou exclusif} est exécutée.

Cette opération est rare en programmation courante, mais répandu en cryptographie,
y compris amateur.
C'est particulièrement suspect si le second opérande est un grand nombre.

Ceci peut indiquer du chiffrement/déchiffrement, du calcul de somme de contrôle, etc.\\
\\

Une exception à cette observation, qu'il est utile de noter, est le \q{canari} (\myref{subsec:BO_protection}).
Sa génération et sa vérification sont souvent effectuées en utilisant des instructions
\XOR. \\
\\
\myindex{AWK}

Ce script awk peut être utilisé pour traité les fichiers listing (.lst) d'\IDA:

\lstinputlisting{digging_into_code/awk.sh}

Il est aussi utile de noter que ce type de script peut aussi rapporter du code mal
désassemblé
(\myref{sec:incorrectly_disasmed_code}).

\subsection{Code assembleur écrit à la main}

\myindex{Function prologue}
\myindex{Function epilogue}
\myindex{x86!\Instructions!LOOP}
\myindex{x86!\Instructions!RCL}

Les compilateurs modernes ne génèrent pas les instructions \TT{LOOP} et \TT{RCL}.
D'un autre côté, ces instructions sont très connues des codeurs qui aiment écrire
directement en langage d'assemblage.
Si vous les rencontrez, on peut dire qu'il est très probable que ce morceau de code
ait été écrit à la main.
De telles instructions sont marquées avec un (M) dans la liste des instructions de
cet appendice: \myref{sec:x86_instructions}.

\par
De même, les prologue/épilogue de fonction sont rares dans de l'assembleur écrit
à la main.

\par
Il n'y a généralement pas de système fixé pour le passage des arguments aux fonctions
dans du code écrit à la main.

\par
Exemple du noyau de Windows 2003
(ntoskrnl.exe file):

\lstinputlisting[style=customasmx86]{digging_into_code/ntoskrnl.lst}

En effet, si nous regardons dans le code source de \ac{WRK} v1.2, ce code peut être
trouvé facilement dans le fichier \\
\emph{WRK-v1.2\textbackslash{}base\textbackslash{}ntos\textbackslash{}ke\textbackslash{}i386\textbackslash{}cpu.asm}.
% TBT
%\par 
%As of \INS{RCL}, I could find it in ntoskrnl.exe file from Windows 2003 x86 (MS Visual C compiler).
%It is occurred only once, in \TT{RtlExtendedLargeIntegerDivide()} function, and this might be inline assembler code case.
