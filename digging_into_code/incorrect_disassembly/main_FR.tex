\subsection{Code mal désassemblé}
\label{sec:incorrectly_disasmed_code}

Un rétro ingénieur pratiquant a souvent à faire avec du code mal désassemblé.

\subsubsection{Désassemblage depuis une adresse de début incorrecte (x86)}

Contrairement à ARM et MIPS (où toute instruction a une longueur de 2 ou 4 octets),
les instructions x86 ont une taille variable, donc tout désassembleur démarrant à
une mauvaise adresse qui se trouve au milieu d'une instruction x86 pourra produire
un résultat incorrect.

À titre d'exemple:

\lstinputlisting[style=customasmx86]{\CURPATH/x86_wrong_start_FR.asm}

Il y a des instructions incorrectement désassemblées au début, mais finalement le
désassembleur revient sur la bonne voie.

\subsubsection{À quoi ressemble du bruit aléatoire désassemblé?}

Des propriétés répandues qui peuvent être repérées facilement sont:

\begin{itemize}
\item Dispersion d'instructions inhabituellement grande.
\myindex{x86!\Instructions!PUSH}
\myindex{x86!\Instructions!MOV}
\myindex{x86!\Instructions!CALL}
\myindex{x86!\Instructions!IN}
\myindex{x86!\Instructions!OUT}
Les instructions x86 les plus fréquentes sont \PUSH{}, \MOV{}, \CALL{}, mais ici nous
voyons des instructions de tous les groupes d'instructions: \ac{FPU}, \INS{IN}/\INS{OUT},
instructions systèmes et rares.

\item Valeurs grandes et aléatoires, d'offsets et immédiates.

\item Sauts ayant des offsets incorrects, sautant au milieu d'autres instructions
\end{itemize}

\lstinputlisting[caption=\randomNoise{} (x86),style=customasmx86]{\CURPATH/x86.asm}

\myindex{x86-64}
\lstinputlisting[caption=\randomNoise{} (x86-64),style=customasmx86]{\CURPATH/x64.asm}

\myindex{ARM}
\lstinputlisting[caption=\randomNoise{} (ARM (\ARMMode)),style=customasmARM]{\CURPATH/ARM.asm}

\lstinputlisting[caption=\randomNoise{} (ARM (\ThumbMode)),style=customasmARM]{\CURPATH/ARM_thumb.asm}

\myindex{MIPS}
\lstinputlisting[caption=\randomNoise{} (MIPS little endian),style=customasmMIPS]{\CURPATH/MIPS.asm}

Il est important de garder à l'esprit que du code de dépaquetage et de déchiffrement
construit intelligemment (y compris auto-modifiant) peut avoir l'air aléatoire, mais
s'exécute toujours correctement.
% TODO таких примеров тоже бы добавить

