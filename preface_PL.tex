\section*{Przedmowa}

\subsection*{Skąd dwa tytuły?}
\label{TwoTitles}

W latach 2014-2018 książka nosiła tytuł "Inżynieria wsteczna dla początkujących" ale zawsze podejrzewałem, że zawęża to grono czytelników.

Ludzie zajmujący się bezpieczeństwem informacji (infosec) wiedzą o inżynierii wstecznej, jednak rzadko kiedy słyszałem, by używali słowa asembler.

Podobnie, termin ``inżynieria wsteczna'' jest nieco tajemniczy dla ogólnego grona programistów, jednak wiedzą oni o istnieniu asemblera.

W lipcu 2018 roku, w ramach eksperymentu, zmieniłem tytuł na ``Język maszynowy dla początkujących''
i umieściłem link na portalu Hacker News\footnote{\url{https://news.ycombinator.com/item?id=17549050}}, i książka została ogólnie dobrze przyjęta.

Niech więc tak będzie, książka ma teraz dwa tytuły.

Zmieniłem jednak drugi tytuł na "Rozumienie kodu maszynowego",
ponieważ ktoś już napisał książkę o tytule "Język maszynowy dla początkujących".
Ludzie twierdzą, że "dla początkujących" brzmi odrobinę ironicznie jak na \textasciitilde{}1000 stronicową książkę.

Dwie książki różnią się jedynie tytułem, nazwą pliku (UAL-XX.pdf versus RE4B-XX.pdf),
URLem i kilkoma pierwszymi stronami.

\subsection*{O inżynierii wstecznej}

Termin \q{\gls{reverse engineering}} ma kilka popularnych definicji:

1) inżynieria wsteczna oprogramowania; analiza skompilowanych programów; 

2) skanowanie modelu w 3D, żeby następnie go skopiować;

3) odzyskiwanie struktury \ac{DBMS}.

Nasza książka będzie powiązana z tą pierwszą definicją.

\subsection*{Pożądana wiedza}

Podstawowa znajomość C \ac{PL} .
Polecane materiały: \myref{CCppBooks}.

\subsection*{Ćwiczenia i zadania}

\dots 
wszystkie są na osobnej stronie: \url{http://challenges.re}.

\iffalse
\subsection*{O autorze}
\begin{tabularx}{\textwidth}{ l X }

\raisebox{-\totalheight}{
\includegraphics[scale=0.60]{Dennis_Yurichev.jpg}
}

&
Dennis Yurichev~--- reverse engineer i programista z dużym doświadczeniem.
Można się z nim skontaktować mailowo: \textbf{\EMAILS{}}.

% FIXME: no link. \tablefootnote doesn't work
\end{tabularx}
\fi

% subsections:
\subsection*{\PLph{}}

\url{https://beginners.re/\#praise}.


\subsection*{Uczelnie}

Ta książka jest polecana przynajmniej na poniższych uczelniach:
\url{https://beginners.re/\#uni}.


\ifdefined\RUSSIAN
\newcommand{\PeopleMistakesInaccuraciesRusEng}{Александр Лысенко, Федерико Рамондино, Марк Уилсон, Разихова Мейрамгуль Кайратовна, Анатолий Прокофьев, Костя Бегунец, Валентин ``netch'' Нечаев, Александр Плахов, Артем Метла, Александр Ястребов, Влад Головкин\footnote{goto-vlad@github}, Евгений Прошин, Александр Мясников, Алексей Третьяков}
\else
\newcommand{\PeopleMistakesInaccuraciesRusEng}{Alexander Lysenko, Federico Ramondino, Mark Wilson, Razikhova Meiramgul Kayratovna, Anatoly Prokofiev, Kostya Begunets, Valentin ``netch'' Nechayev, Aleksandr Plakhov, Artem Metla, Alexander Yastrebov, Vlad Golovkin\footnote{goto-vlad@github}, Evgeny Proshin, Alexander Myasnikov, Alexey Tretiakov}
\fi

\newcommand{\PeopleMistakesInaccuracies}{\PeopleMistakesInaccuraciesRusEng{}, Zhu Ruijin, Changmin Heo, Vitor Vidal, Stijn Crevits, Jean-Gregoire Foulon\footnote{\url{https://github.com/pixjuan}}, Ben L., Etienne Khan, Norbert Szetei\footnote{\url{https://github.com/73696e65}}, Marc Remy, Michael Hansen, Derk Barten, The Renaissance\footnote{\url{https://github.com/TheRenaissance}}, Hugo Chan, Emil Mursalimov, Tanner Hoke, Tan90909090@GitHub, Ole Petter Orhagen, Sourav Punoriyar, Vitor Oliveira, Alexis Ehret, Maxim Shlochiski,
Greg Paton, Pierrick Lebourgeois.}

\newcommand{\PeopleItalianTranslators}{Federico Ramondino\footnote{\url{https://github.com/pinkrab}},
Paolo Stivanin\footnote{\url{https://github.com/paolostivanin}}, twyK, Fabrizio Bertone, Matteo Sticco, Marco Negro\footnote{\url{https://github.com/Internaut401}}, bluepulsar}

\newcommand{\PeopleFrenchTranslators}{Florent Besnard\footnote{\url{https://github.com/besnardf}}, Marc Remy\footnote{\url{https://github.com/mremy}}, Baudouin Landais, Téo Dacquet\footnote{\url{https://github.com/T30rix}}, BlueSkeye@GitHub\footnote{\url{https://github.com/BlueSkeye}}}

\newcommand{\PeopleGermanTranslators}{Dennis Siekmeier\footnote{\url{https://github.com/DSiekmeier}},
Julius Angres\footnote{\url{https://github.com/JAngres}}, Dirk Loser\footnote{\url{https://github.com/PolymathMonkey}}, Clemens Tamme, Philipp Schweinzer}

\newcommand{\PeopleSpanishTranslators}{Diego Boy, Luis Alberto Espinosa Calvo, Fernando Guida, Diogo Mussi, Patricio Galdames,
Emiliano Estevarena}

\newcommand{\PeoplePTBRTranslators}{Thales Stevan de A. Gois, Diogo Mussi, Luiz Filipe, Primo David Santini}

\newcommand{\PeoplePolishTranslators}{Kateryna Rozanova, Aleksander Mistewicz, Wiktoria Lewicka, Marcin Sokołowski}

\newcommand{\PeopleJapaneseTranslators}{%
shmz@github\footnote{\url{https://github.com/shmz}},%
4ryuJP@github\footnote{\url{https://github.com/4ryuJP}}}

\EN{\input{thanks_EN}}
\ES{\input{thanks_ES}}
\NL{\input{thanks_NL}}
\RU{\input{thanks_RU}}
\IT{\input{thanks_IT}}
\FR{\input{thanks_FR}}
\DE{\input{thanks_DE}}
%\CN{\input{thanks_CN}}
\JA{\input{thanks_JA}}
\PL{\subsection*{Podziękowania}

Za cierpliwe odpowiadanie na wszystkie moje pytania: SkullC0DEr.

Za wskazanie błędów i nieścisłości: \PeopleMistakesInaccuracies{}

Za inną pomoc:
Andrew Zubinski,
Arnaud Patard (rtp on \#debian-arm IRC),
noshadow on \#gcc IRC,
Aliaksandr Autayeu,
Mohsen Mostafa Jokar,
Peter Sovietov,
Misha ``tiphareth'' Verbitsky.

Za przetłumaczenie tej książki na język chiński uproszczony:
Antiy Labs (\href{http://antiy.cn}{antiy.cn}), Archer.

Za tłumaczenie na język koreański: Byungho Min.

Za tłumaczenie na język holenderski: Cedric Sambre (AKA Midas).

Za tłumaczenie na język hiszpański: \PeopleSpanishTranslators{}.

Za tłumaczenie na język portugalski: \PeoplePTBRTranslators{}.

Za tłumaczenie na język włoski: \PeopleItalianTranslators{}.

Za tłumaczenie na język francuski: \PeopleFrenchTranslators{}.

Za tłumaczenie na język niemiecki: \PeopleGermanTranslators{}.

Za tłumaczenie na język polski: \PeoplePolishTranslators{}.

Za tłumaczenie na język japoński: \PeopleJapaneseTranslators{}.

Za korektę:
Vladimir Botov,
Andrei Brazhuk,
Mark ``Logxen'' Cooper, Yuan Jochen Kang, Mal Malakov, Lewis Porter, Jarle Thorsen, Hong Xie.

Vasil Kolev\footnote{\url{https://vasil.ludost.net/}} wprowadził wiele poprawek i wskazał sporo błędów.

Dziękuję również wszystkim użytkownikom z github.com za ich komentarze i poprawki.

Użyłem wielu pakietów \LaTeX. Chciałbym podziękować również ich autorom.

\subsubsection*{Darczyńcy}

Tym wszystkim, którzy mnie wspierali w czasie pisania tej książki:

\input{donors}

bardzo dziękuję.
}
\CN{\input{thanks_CN}}


\subsection*{mini-FAQ}

\par Q: Czy ta książka jest prostsza niż inne?
\par A: Nie, poziom trudności jest mniej więcej taki sam jak innych książek na ten temat.

\par Q: Obawiam się zacząć czytać tę książkę, ma ponad 1000 stron.
"... dla początkujących" w nazwie brzmi nieco ironicznie.
\par A: Wszelkiego rodzaju kody źródłowe stanowią większość tej książki.
Ta książka naprawdę jest dla początkujących, wiele w niej (jeszcze) brakuje.

\par Q: Co trzeba wiedzieć zanim się przystąpi do czytania książki?
\par A: Umiejętności С/С++ są pożądane, ale nie są niezbędne.

\par Q: Czy powinienem uczyć się jednocześnie x86/x64/ARM i MIPS? Czy to nie za dużo?
\par A: Myślę, że na początek wystarczy czytać tylko o x86/x64, części o ARM i MIPS można pominąć.

\par Q: Czy można zakupić książki w wersji papierowej w języku rosyjskim lub angielskim?
\par A: Niestety nie, żaden wydawca jeszcze się nie zainteresował wydaniem rosyjskiej lub angielskiej wersji. Natomiast można ją wydrukować i zbindować w każdym ksero.
\url{https://yurichev.com/news/20200222_printed_RE4B/}.

\par Q: Czy istnieje wersja epub/mobi?
\par A: Nie. W wielu miejscach książka korzysta z hacków specyficznych dla TeXa/LaTeXa, dlatego przerobienie jej na HTML
(epub/mobi to jest HTML) nie jest łatwe.

\par Q: Po co uczyć się asemblera w dzisiejszych czasach?
\par A: Jeśli nie jest się programistą \ac{OS}, to prawdopodobnie nie trzeba nic pisać w asemblerze: współczesne kompilatory optymalizują kod lepiej niż człowiek \footnote{Bardzo ciekawy artykuł na ten temat: \InSqBrackets{\AgnerFog}}.

Do tego współczesne \ac{CPU} są bardzo skomplikowanymi urządzeniami i znajomość asemblera nie pomoże poznać ich mechanizmów wewnętrznych.

Jednak zostają dwa obszary, w których dobra znajomość asemblera może być pomocna:
1) badanie malware (złośliwego oprogramowania) w celu jego analizy ; 2) lepsze zrozumienie skompilowanego kodu w trakcie debuggowania.

Wobec tego ta książka jest napisana dla tych ludzi, którzy raczej chcą rozumieć assembler, a nie w nim pisać. Stąd jest w niej bardzo dużo przykładów - wyjść kompilatora.

\par Q: Kliknąłem w odnośnik wewnątrz pliku PDF, jak teraz wrócić?
\par A: W Adobe Acrobat Reader trzeba wcisnąć Alt+LeftArrow. W Evince wcisnąć "<".

\par Q: Czy mogę wydrukować tę książkę? Korzystać z niej do nauczania?
\par A: Oczywiście, właśnie dlatego ta książka ma licencję Creative Commons (CC BY-SA 4.0).

\par Q: Dlaczego ta książka jest darmowa? Wykonałeś świetną robotę. To podejrzane, podobnie jak z innymi rzeczami za darmo.
\par A: Moim zdaniem autorzy literatury technicznej robią to dla autoreklamy. Taka praca nie przynosi za dużo pieniędzy.

\par Q: Jak znaleźć pracę w zawodzie reverse engineeraа?
\par A: Na reddit (RE\FNURLREDDIT), od czasu od czasu pojawiają się wątki poszukiwania pracowników.
Możesz spróbować tam poszukać.


\par Q: Wersje kompilatorów z tej książki są już przestarzałe...
\par A: Nie ma potrzeby by dokładnie wykonywać wszystkie kroki opisane w książce.
Użyj kompilatorów, które masz już zainstalowane w swoim \ac{OS}.
Możesz również skorzystać z: \href{https://godbolt.org/}{Compiler Explorer}.

\par Q: Mam pytanie...
\par A: Napisz do mnie maila (\EMAILS).



\subsection*{O tłumaczeniu na język koreański}

W styczniu 2015, wydawnictwo Acorn (\href{http://www.acornpub.co.kr}{www.acornpub.co.kr}) z Korei Południowej wykonało ciężką pracę, żeby przetłumaczyć i wydać moją książkę  (stanem na sierpień 2014) w języku koreańskim.
Jest ona teraz dostępna na \href{http://go.yurichev.com/17343}{ich stronie}.

\iffalse
\begin{figure}[H]
\centering
\includegraphics[scale=0.3]{acorn_cover.jpg}
\end{figure}
\fi

Tłumaczył Byungho Min (\href{http://go.yurichev.com/17344}{twitter/tais9}).
Okładka namalował mój dobry przyjaciel, artysta, Andy Nechaevsky
\href{http://go.yurichev.com/17023}{facebook/andydinka}.

Acorn również ma prawa autorskie do tłumaczenia koreańskiego.
Jakbyście chcieli mieć \emph{prawdziwą} książkę w języku koreańskim i chcielibyście wesprzeć moją pracę, możecie ją kupić.

\subsection*{O tłumaczeniu na język perski (farsi)}

W roku 2016 książkę przetłumaczył Mohsen Mostafa Jokar (znany w irańskiej społeczności z tłumaczenia instrukcji do Radare\footnote{\url{http://rada.re/get/radare2book-persian.pdf}})
Książka jest dostępna na stronie wydawnictwa \footnote{\url{http://goo.gl/2Tzx0H}} (Pendare Pars).

Pierwsze 40 stron: \url{https://beginners.re/farsi.pdf}.

Pozycja książki w Narodowej Bibliotece Iranu: \url{http://opac.nlai.ir/opac-prod/bibliographic/4473995}.

\subsection*{O tłumaczeniu na język chiński}

W kwietniu 2017, wydawnictwo PTPress skończyło tłumaczenie mojej książki na język chiński. Mają również prawo autorskie do tłumaczenia chińskiego.

Chińskie tłumaczenie można zamówić tutaj: \url{http://www.epubit.com.cn/book/details/4174}. Recenzje i historię tłumaczenia można znaleźć tutaj: \url{http://www.cptoday.cn/news/detail/3155}.

Głównym tłumaczem był Archer, u którego mam teraz dług wdzięczności.
Był bardzo dociekliwy i znalazł w książce sporo bugów i błędów, co jest szczególnie ważne w literaturze, której dotyczy ta książka.

Będę polecał go również innym autorom!

Chłopaki z \href{http://www.antiy.net/}{Antiy Labs} również pomogli z tłumaczeniem. \href{http://www.epubit.com.cn/book/onlinechapter/51413}{Tutaj słowo wstępne} napisane przez nich.


