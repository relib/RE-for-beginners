\newglossaryentry{caller}
{
  name=chiamante,
  description={Una funzione chiamante}
}

\newglossaryentry{callee}
{
  name=chiamata,
  description={Una funzione chiamata}
}

\newglossaryentry{stack pointer}
{
  name={stack pointer},
  description={Un registro che punta nello stack}
}

\newglossaryentry{prodotto}
{
  name={prodotto},
  description={Risultato di una moltiplicazione}
}

\newglossaryentry{GiB}
{
  name={GiB},
  description={Gibibyte: $2^{30}$ o 1024 megabyte o 1073741824 byte}
}

\newglossaryentry{increment}
{
  name={incrementa},
  description={Incrementa di 1}
}

\newglossaryentry{decrement}
{
  name={decrementa},
  description={Decrementa di 1}
}

\newglossaryentry{stdout}
{
  name={stdout},
  description={Standard output}
}

\newglossaryentry{endianness}
{
  name={endianness},
  description={L'ordine dei byte}
}

\newglossaryentry{thunk function}
{
  name={Funzione thunk},
  description={Piccola funzione con un solo scopo: chiamare un' altra funzione}
}

\newglossaryentry{leaf function}
{
  name={funzione foglia},
  description={Una funzione che non chiama nessun' altra funzione}
}

\newglossaryentry{heap}
{
  name={heap},
  description={di solito, una grossa locazione di memoria fornito da \ ac {OS} in modo che le applicazioni possano dividerla da sole come desiderano. malloc () / free () lavorano con l'heap}
}

\newglossaryentry{link register}
{
  name={registro link},
  description={(RISC) {Un registro in cui generalmente l'indirizzo di ritorno viene salvato.
Ciò rende possibile chiamare funzioni foglia più velocemente, ovvero senza l'utilizzo dello stack}}
}

\newglossaryentry{anti-pattern}
{
  name={anti-pattern},
  description={Generalmente considerata una cattiva pratica}
}

\newglossaryentry{stack frame}
{
  name={stack frame},
  description={\ITph{}}
}

\newglossaryentry{jump offset}
{
  name={offset di salto},
  description={\ITph{}}
}

\newglossaryentry{loop unwinding}
{
  name={srotolamente del ciclo},
  description={E' quando un compilatore, anzichè generare il codice di un ciclo per $n$ iterazioni, genera $n$ copie del corpo del ciclo, al fine di sbarazzarsi delle istruzioni per la manutenzione del ciclo}
}

\newglossaryentry{tracer}
{
  name={\ITph},
  description={\ITph{}}
}

\newglossaryentry{register allocator}
{
  name={registro allocatore},
  description={La parte del compilatore che assegna i registri della CPU alle variabili locali}
}

\newglossaryentry{quotient}
{
  name={quoziente},
  description={Risultato di una divisione}
}

\newglossaryentry{real number}
{
  name={numero reale},
  description={\ITph{}}
}

\newglossaryentry{NaN}
{
  name={Nan},
  description={non un numero:
	un caso speciale per i numeri a virgola mobile, generalmente segnalano errori}
}

\newglossaryentry{Windows NT}
{
  name={Windows NT},
  description={Windows NT, 2000, XP, Vista, 7, 8, 10}
}

\newglossaryentry{word}
{
  name={\ITph},
  description={\ITph{}}
}

\newglossaryentry{PDB}
{
  name={\ITph},
  description={\ITph{}}
}

\newglossaryentry{name mangling}
{
  name={\ITph},
  description={\ITph{}}
}

\newglossaryentry{padding}
{
  name={padding},
  description={in inglese significa riempire un cuscino con qualcosa per dargli una forma desiderata (più grande). In informatica, significa aggiungere più byte a un blocco in modo che abbia la dimensione desiderata, come $ 2 ^ n $ byte.}
}

\newglossaryentry{NOP}
{
  name={NOP},
  description={nessuna operazione (no operation)}
}

\newglossaryentry{POKE}
{
  name={POKE},
  description={Istruzione del linguaggio BASIC per scrivere un byte ad uno specifico indirizzo}
}

\newglossaryentry{xoring}
{
  name={\ITph},
  description={\ITph{}}
}

\newglossaryentry{atomic operation}
{
  name={operazione atomica},
  description={
  \q{$\alpha{}\tau{}o\mu{}o\varsigma{}$}
  %\q{atomic} 
  sta per \q{indivisibile} in Greco, quindi è garantito che un' operazione atomica non venga interrotta da altri thread}
}

\newglossaryentry{basic block}
{
  name={\ITph},
  description={Un gruppo di
	       istruzioni che non hanno salti / diramazioni e nemmeno salti da dentro il blocco verso fuori.
In \IDA è una lista di istruzioni senza linee vuote}
}

\newglossaryentry{reverse engineering}
{
  name={ingegneria inversa},
  description={L' atto di comprendere come una cosa funziona, a volte per clonarla}
}

\newglossaryentry{compiler intrinsic}
{
  name={\ITph},
  description={Una 	}
}
