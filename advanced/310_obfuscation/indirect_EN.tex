\subsubsection{Using indirect pointers}

\begin{lstlisting}[style=customasmx86]
dummy_data1	db	100h dup (0)
message1	db	'hello world',0

dummy_data2	db	200h dup (0)
message2	db	'another message',0

func		proc
		...
		mov	eax, offset dummy_data1 ; PE or ELF reloc here
		add	eax, 100h
		push	eax
		call	dump_string
		...
		mov	eax, offset dummy_data2 ; PE or ELF reloc here
		add	eax, 200h
		push	eax
		call	dump_string
		...
func		endp
\end{lstlisting}

\IDA{} will show references only to \TT{dummy\_data1} and \TT{dummy\_data2}, 
but not to the text strings.

Global variables and even functions may be accessed like that.

Now something slightly more advanced.

Honestly, I don't know its exact name, but I would call it \textit{shifted pointer}.
This technique is quite common, at least in copy protection schemes.

In short: while writing a value into global memory you use an address,
but by reading you use a sum of (other) addresses, or maybe a difference.
The goal is to hide a real address from a reverse engineer who debugs the code or exploring it in IDA
(or another disassembler).

This can be a nuisance.

\lstinputlisting[style=customc]{\CURPATH/indirect.c}

If compiled by non-optimizing MSVC 2015:

\lstinputlisting[style=customasmx86]{\CURPATH/indirect.asm}

You see, IDA can only get two addresses: \verb|secret_array[]| (start of the array) and \\
\verb|point_passed_to_get_byte_at_0x6000|.

How to deal with it: you can use hardware breakpoints on memory access operations
\tracer has \TT{BPMx} options) or symbolic execution engine or maybe you can write a plugin for IDA...

Surely, one array can be used for many values, not limited to boolean ones...

N.B.: Optimizing MSVC 2015 is smart enough to optimize the \verb|get_byte_at_0x6000()| function out.

