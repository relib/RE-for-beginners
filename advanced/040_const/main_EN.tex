\mysection{const correctness}
\myindex{\CLanguageElements!const}
\label{const_in_rdata}

This is undeservedly underused feature of many programming languages.
Read here about its importance:
\href{https://isocpp.org/wiki/faq/const-correctness}{1},
\href{https://stackoverflow.com/questions/136880/sell-me-on-const-correctness}{2}.

Ideally, everything you don't modify should have \emph{const} modifier.

Interestingly, how \emph{const correctness} is implemented at low level.
There are no runtime checks of local \emph{const} variables and function arguments (only compile-time checks).
But global variables of such a type are to be allocated in read-only data segments.

This is example is to be crashed, because if compiled by MSVC for win32,
the $a$ global variable is allocated in \verb|.rdata| read-only segment:

\lstinputlisting[style=customc]{\CURPATH/ex1.c}

\emph{Anonymous} (not linked to a variable name) C strings also have \verb|const char*| type.
You can't modify them:

\lstinputlisting[style=customc]{\CURPATH/ex2.c}

This code will crash on Linux (``segmentation fault'') and on Windows if compiled by MinGW.

GCC for Linux places all text strings info \TT{.rodata} data segment, which is explicitly read-only
(``read only data''):

\lstinputlisting{\CURPATH/ex2.txt}

When the \verb|alter_string()| function tries to write there, exception occurred.

Things are different in the code generated by MSVC, strings are located in \TT{.data} segment, which has no \TT{READONLY} flag.
MSVC's developers misstep?

\lstinputlisting{\CURPATH/ex22.txt}

However, MinGW hasn't this fault and allocates text strings in \verb|.rdata| segment.

\input{\CURPATH/two_strings_EN}

