\mysection{Использование const (const correctness)}
\myindex{\CLanguageElements!const}
\label{const_in_rdata}

Это незаслуженно малоиспользуемая возможность многих ЯП.
Тут можно почитать о её важности:
\href{https://isocpp.org/wiki/faq/const-correctness}{1},
\href{https://stackoverflow.com/questions/136880/sell-me-on-const-correctness}{2}.

Идеально, всё что вы не модифицируете, должно иметь модификатор \emph{const}.

Интересно, как \emph{const correctness} обеспечивается на низком уровне.
Локальные \emph{const}-переменные и аргументы ф-ций не проверяются во время исполнения (только во время компиляции).
Но глобальные переменные этого типа располагаются в сегментах данных только для чтения.

Вот пример который упадет, потому что, если будет скомпилирован MSVC для win32, глобальная переменная $a$ располагается
в сегменте \verb|.rdata|, который только для чтения:

\lstinputlisting[style=customc]{\CURPATH/ex1.c}

\emph{Анонимные} (не привязанные к имени переменной) строки в Си имеют тип \verb|const char*|.
Вы не можете их модифицировать:

\lstinputlisting[style=customc]{\CURPATH/ex2.c}

Это код упадет в Linux (``segmentation fault'') и в Windows, если скомпилирован MinGW.

GCC для Linux располагает все текстовые строки в сегменте данных \TT{.rodata}, который недвусмысленно защищен от записи
(``read only data''):

\lstinputlisting{\CURPATH/ex2.txt}

Когда ф-ция \verb|alter_string()| пытается там писать, срабатывает исключение.

Всё немного иначе в коде сгенерированном MSVC, строки располагаются в сегменте \TT{.data}, у которого нет флага \TT{READONLY}.
Оплошность разработчиков MSVC?

\lstinputlisting{\CURPATH/ex22.txt}

А в MinGW этой ошибки нет, и строки располагаются в сегменте \verb|.rdata|.

\input{\CURPATH/two_strings_RU}

