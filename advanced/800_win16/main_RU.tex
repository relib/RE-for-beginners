\mysection{Windows 16-bit}
\myindex{Windows!Windows 3.x}

16-битные программы под Windows в наше время редки, хотя иногда можно поработать с ними, в смысле ретрокомпьютинга,
либо которые защищенные донглами (\myref{dongles}).

16-битные версии Windows были вплоть до 3.11.
95/98/ME также поддерживает 16-битный код, как и все 32-битные OS линейки \gls{Windows NT}.
64-битные версии \gls{Windows NT} не поддерживают 16-битный код вообще.

Код напоминает тот что под MS-DOS.

Исполняемые файлы имеют NE-тип (так называемый \q{new executable}).

Все рассмотренные здесь примеры скомпилированы компилятором OpenWatcom 1.9 используя эти опции:

\begin{lstlisting}
wcl.exe -i=C:/WATCOM/h/win/ -s -os -bt=windows -bcl=windows example.c
\end{lstlisting}

\input{\CURPATH/ex1.tex}
\input{\CURPATH/ex2.tex}
\input{\CURPATH/ex3.tex}
\input{\CURPATH/ex4.tex}
\input{\CURPATH/ex5.tex}
\input{\CURPATH/ex6_RU.tex}

