\section*{Предисловие}

\subsection*{Почему два названия?}
\label{TwoTitles}

В 2014-2018 книга называлась ``Reverse Engineering для начинающих'', но я всегда подозревал что это слишком сужает аудиторию.

Люди от инфобезопасности знают о ``reverse engineering'', но я от них редко слышу слово ``ассемблер''.

Точно также, термин ``reverse engineering'' слишком незнакомый для общей аудитории программистов, но они знают про ``ассемблер''.

В июле 2018, для эксперимента, я заменил название на ``Assembly Language for Beginners''
и запостил ссылку на сайт Hacker News\footnote{\url{https://news.ycombinator.com/item?id=17549050}}, и книгу приняли, в общем, хорошо.

Так что, пусть так и будет, у книги будет два названия.

Хотя, я поменял второе название на ``Understanding Assembly Language'' (``Понимание языка ассемблера''), потому что кто-то уже написал книгу ``Assembly Language for Beginners''.
Также, люди говорят что ``для начинающих'' уже звучит немного саркастично для книги объемом в \textasciitilde{}1000 страниц.

Книги отличаются только названием, именем файла (UAL-XX.pdf и RE4B-XX.pdf), URL-ом и парой первых страниц.

\subsection*{О reverse engineering}

У термина \q{\gls{reverse engineering}} несколько популярных значений:
1) исследование скомпилированных
программ;
2) сканирование трехмерной модели для последующего копирования;
3) восстановление структуры СУБД.

Настоящая книга связана с первым значением.

\subsection*{Желательные знания перед началом чтения}

Очень желательно базовое знание \ac{PL} Си.
Рекомендуемые материалы: \myref{CCppBooks}.

\subsection*{Упражнения и задачи}

\dots 
все перемещены на отдельный сайт: \url{http://challenges.re}.

\iffalse
\subsection*{Об авторе}
\begin{tabularx}{\textwidth}{ l X }

\raisebox{-\totalheight}{
\includegraphics[scale=0.60]{Dennis_Yurichev.jpg}
}

&
Денис Юричев~--- опытный reverse engineer и программист.
С ним можно контактировать по емейлу: \textbf{\EMAIL{}}.

% FIXME: no link. \tablefootnote doesn't work
\end{tabularx}
\fi

% subsections:
\input{praise}
\subsection*{Университеты}

Эта книга рекомендуется по крайне мере в этих университетах:
\url{https://beginners.re/\#uni}.


\ifdefined\RUSSIAN
\newcommand{\PeopleMistakesInaccuraciesRusEng}{Александр Лысенко, Федерико Рамондино, Марк Уилсон, Разихова Мейрамгуль Кайратовна, Анатолий Прокофьев, Костя Бегунец, Валентин ``netch'' Нечаев, Александр Плахов, Артем Метла, Александр Ястребов, Влад Головкин\footnote{goto-vlad@github}, Евгений Прошин, Александр Мясников, Алексей Третьяков}
\else
\newcommand{\PeopleMistakesInaccuraciesRusEng}{Alexander Lysenko, Federico Ramondino, Mark Wilson, Razikhova Meiramgul Kayratovna, Anatoly Prokofiev, Kostya Begunets, Valentin ``netch'' Nechayev, Aleksandr Plakhov, Artem Metla, Alexander Yastrebov, Vlad Golovkin\footnote{goto-vlad@github}, Evgeny Proshin, Alexander Myasnikov, Alexey Tretiakov}
\fi

\newcommand{\PeopleMistakesInaccuracies}{\PeopleMistakesInaccuraciesRusEng{}, Zhu Ruijin, Changmin Heo, Vitor Vidal, Stijn Crevits, Jean-Gregoire Foulon\footnote{\url{https://github.com/pixjuan}}, Ben L., Etienne Khan, Norbert Szetei\footnote{\url{https://github.com/73696e65}}, Marc Remy, Michael Hansen, Derk Barten, The Renaissance\footnote{\url{https://github.com/TheRenaissance}}, Hugo Chan, Emil Mursalimov, Tanner Hoke, Tan90909090@GitHub, Ole Petter Orhagen, Sourav Punoriyar, Vitor Oliveira, Alexis Ehret, Maxim Shlochiski,
Greg Paton, Pierrick Lebourgeois.}

\newcommand{\PeopleItalianTranslators}{Federico Ramondino\footnote{\url{https://github.com/pinkrab}},
Paolo Stivanin\footnote{\url{https://github.com/paolostivanin}}, twyK, Fabrizio Bertone, Matteo Sticco, Marco Negro\footnote{\url{https://github.com/Internaut401}}, bluepulsar}

\newcommand{\PeopleFrenchTranslators}{Florent Besnard\footnote{\url{https://github.com/besnardf}}, Marc Remy\footnote{\url{https://github.com/mremy}}, Baudouin Landais, Téo Dacquet\footnote{\url{https://github.com/T30rix}}, BlueSkeye@GitHub\footnote{\url{https://github.com/BlueSkeye}}}

\newcommand{\PeopleGermanTranslators}{Dennis Siekmeier\footnote{\url{https://github.com/DSiekmeier}},
Julius Angres\footnote{\url{https://github.com/JAngres}}, Dirk Loser\footnote{\url{https://github.com/PolymathMonkey}}, Clemens Tamme, Philipp Schweinzer}

\newcommand{\PeopleSpanishTranslators}{Diego Boy, Luis Alberto Espinosa Calvo, Fernando Guida, Diogo Mussi, Patricio Galdames,
Emiliano Estevarena}

\newcommand{\PeoplePTBRTranslators}{Thales Stevan de A. Gois, Diogo Mussi, Luiz Filipe, Primo David Santini}

\newcommand{\PeoplePolishTranslators}{Kateryna Rozanova, Aleksander Mistewicz, Wiktoria Lewicka, Marcin Sokołowski}

\newcommand{\PeopleJapaneseTranslators}{%
shmz@github\footnote{\url{https://github.com/shmz}},%
4ryuJP@github\footnote{\url{https://github.com/4ryuJP}}}

\EN{\input{thanks_EN}}
\ES{\input{thanks_ES}}
\NL{\input{thanks_NL}}
\RU{\input{thanks_RU}}
\IT{\input{thanks_IT}}
\FR{\input{thanks_FR}}
\DE{\input{thanks_DE}}
%\CN{\input{thanks_CN}}
\JA{\input{thanks_JA}}
\PL{\subsection*{Podziękowania}

Za cierpliwe odpowiadanie na wszystkie moje pytania: SkullC0DEr.

Za wskazanie błędów i nieścisłości: \PeopleMistakesInaccuracies{}

Za inną pomoc:
Andrew Zubinski,
Arnaud Patard (rtp on \#debian-arm IRC),
noshadow on \#gcc IRC,
Aliaksandr Autayeu,
Mohsen Mostafa Jokar,
Peter Sovietov,
Misha ``tiphareth'' Verbitsky.

Za przetłumaczenie tej książki na język chiński uproszczony:
Antiy Labs (\href{http://antiy.cn}{antiy.cn}), Archer.

Za tłumaczenie na język koreański: Byungho Min.

Za tłumaczenie na język holenderski: Cedric Sambre (AKA Midas).

Za tłumaczenie na język hiszpański: \PeopleSpanishTranslators{}.

Za tłumaczenie na język portugalski: \PeoplePTBRTranslators{}.

Za tłumaczenie na język włoski: \PeopleItalianTranslators{}.

Za tłumaczenie na język francuski: \PeopleFrenchTranslators{}.

Za tłumaczenie na język niemiecki: \PeopleGermanTranslators{}.

Za tłumaczenie na język polski: \PeoplePolishTranslators{}.

Za tłumaczenie na język japoński: \PeopleJapaneseTranslators{}.

Za korektę:
Vladimir Botov,
Andrei Brazhuk,
Mark ``Logxen'' Cooper, Yuan Jochen Kang, Mal Malakov, Lewis Porter, Jarle Thorsen, Hong Xie.

Vasil Kolev\footnote{\url{https://vasil.ludost.net/}} wprowadził wiele poprawek i wskazał sporo błędów.

Dziękuję również wszystkim użytkownikom z github.com za ich komentarze i poprawki.

Użyłem wielu pakietów \LaTeX. Chciałbym podziękować również ich autorom.

\subsubsection*{Darczyńcy}

Tym wszystkim, którzy mnie wspierali w czasie pisania tej książki:

\input{donors}

bardzo dziękuję.
}
\CN{\input{thanks_CN}}


\input{FAQ_RU}

\subsection*{О переводе на корейский язык}

В январе 2015, издательство Acorn в Южной Корее сделало много работы в переводе 
и издании моей книги (по состоянию на август 2014) на корейский язык.
Она теперь доступна на \href{http://go.yurichev.com/17343}{их сайте}.

\iffalse
\begin{figure}[H]
\centering
\includegraphics[scale=0.3]{acorn_cover.jpg}
\end{figure}
\fi

Переводил Byungho Min (\href{http://go.yurichev.com/17344}{twitter/tais9}).
Обложку нарисовал мой хороший знакомый художник Андрей Нечаевский
\href{http://go.yurichev.com/17023}{facebook/andydinka}.
Они также имеют права на издание книги на корейском языке.
Так что если вы хотите иметь \emph{настоящую} книгу на полке на корейском языке и
хотите поддержать мою работу, вы можете купить её.

\subsection*{О переводе на персидский язык (фарси)}

В 2016 году книга была переведена Mohsen Mostafa Jokar (который также известен иранскому сообществу по переводу руководства Radare\footnote{\url{http://rada.re/get/radare2book-persian.pdf}}).
Книга доступна на сайте издательства\footnote{\url{http://goo.gl/2Tzx0H}} (Pendare Pars).

Первые 40 страниц: \url{https://beginners.re/farsi.pdf}.

Регистрация книги в Национальной Библиотеке Ирана: \url{http://opac.nlai.ir/opac-prod/bibliographic/4473995}.

\subsection*{О переводе на китайский язык}

В апреле 2017, перевод на китайский был закончен китайским издательством PTPress. Они также имеют права на издание книги на китайском языке.

Она доступна для заказа здесь: \url{http://www.epubit.com.cn/book/details/4174}. Что-то вроде рецензии и история о переводе: \url{http://www.cptoday.cn/news/detail/3155}.

Основным переводчиком был Archer, перед которым я теперь в долгу.
Он был крайне дотошным (в хорошем смысле) и сообщил о большинстве известных ошибок и баг, что крайне важно для литературы вроде этой книги.
Я буду рекомендовать его услуги всем остальным авторам!

Ребята из \href{http://www.antiy.net/}{Antiy Labs} также помогли с переводом. \href{http://www.epubit.com.cn/book/onlinechapter/51413}{Здесь предисловие} написанное ими.

