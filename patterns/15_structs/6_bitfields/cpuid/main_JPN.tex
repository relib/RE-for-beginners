\subsubsection{CPUID example}

The \CCpp language allows to define the exact number of bits for each structure field.
It is very useful if one needs to save memory space. 
For example, one bit is enough for a \Tbool variable.
But of course, it is not rational if speed is important.
% FIXME!
% another use of this is to parse binary protocols/packets, for example
% the definition of struct iphdr in include/linux/ip.h

\newcommand{\FNCPUID}{\footnote{\href{http://go.yurichev.com/17069}{wikipedia}}}

\myindex{x86!\Instructions!CPUID}
\label{cpuid}

Let's consider the \CPUID\FNCPUID instruction example.
This instruction returns information about the current CPU and its features.

If the \EAX is set to 1 before the instruction's execution, 
\CPUID returning this information packed into the \EAX register:

\begin{center}
\begin{tabular}{ | l | l | }
\hline
3:0 (4 bits)& Stepping \\
7:4 (4 bits) & Model \\
11:8 (4 bits) & Family \\
13:12 (2 bits) & Processor Type \\
19:16 (4 bits) & Extended Model \\
27:20 (8 bits) & Extended Family \\
\hline
\end{tabular}
\end{center}

\newcommand{\FNGCCAS}{\footnote{\href{http://go.yurichev.com/17070}
{More about internal GCC assembler}}}

MSVC 2010 has \CPUID macro, but GCC 4.4.1 does not.
So let's make this function by ourselves for GCC with the help of its built-in assembler\FNGCCAS.

\lstinputlisting[style=customc]{patterns/15_structs/6_bitfields/cpuid/CPUID.c}

After \CPUID fills \EAX/\EBX/\ECX/\EDX, these registers are to be written in the \TT{b[]} array.
Then, we have a pointer to the \TT{CPUID\_1\_EAX} structure and we point it to the value in \EAX from the \TT{b[]} array.

In other words, we treat a 32-bit \Tint value as a structure.
Then we read specific bits from the structure.

\myparagraph{MSVC}

Let's compile it in MSVC 2008 with \Ox option:

\lstinputlisting[caption=\Optimizing MSVC 2008,style=customasmx86]{patterns/15_structs/6_bitfields/cpuid/CPUID_msvc_Ox.asm}

\myindex{x86!\Instructions!SHR}

The \TT{SHR} instruction shifting the value in \EAX by the number of bits that must be
\IT{skipped}, e.g., we ignore some bits \IT{at the right side}.

\myindex{x86!\Instructions!AND}

The \AND instruction clears the unneeded bits \IT{on the left}, or, in other words, 
leaves only those bits in the \EAX register we need.

\clearpage
\myparagraph{MSVC + \olly}
\myindex{\olly}

Let's load our example into \olly 
and see, what values are set in EAX/EBX/ECX/EDX after the execution of CPUID: 

\begin{figure}[H]
\centering
\myincludegraphics{patterns/15_structs/6_bitfields/cpuid/olly.png}
\caption{\olly: After CPUID execution}
\label{fig:cpuid_olly_1}
\end{figure}

EAX has \TT{0x000206A7} (my \ac{CPU} is Intel Xeon E3-1220).\\
This is $0b0000 0000 0000 0010 0000 0110 1010 0111$ in binary form.

Here is how the bits are distributed by fields:

\begin{center}
\begin{tabular}{ | l | l | l | }
\hline
\headercolor{} field &
\headercolor{} in binary form &
\headercolor{} in decimal form \\
\hline
reserved2		& 0000 & 0 \\
\hline
extended\_family\_id	& 00000000 & 0 \\
\hline
extended\_model\_id	& 0010 & 2 \\
\hline
reserved1		& 00 & 0 \\
\hline
processor\_id		& 00 & 0 \\
\hline
family\_id		& 0110 & 6 \\
\hline
model			& 1010 & 10 \\
\hline
stepping		& 0111 & 7 \\
\hline
\end{tabular}
\end{center}

\lstinputlisting[caption=Console output]{patterns/15_structs/6_bitfields/cpuid/console.txt}


\myparagraph{GCC}

Let's try GCC 4.4.1 with \Othree option.

\lstinputlisting[caption=\Optimizing GCC 4.4.1,style=customasmx86]{patterns/15_structs/6_bitfields/cpuid/CPUID_gcc_O3.asm}

Almost the same.
The only thing worth noting is that GCC somehow combines the calculation of\\
\TT{extended\_model\_id} and \TT{extended\_family\_id} into one block,
instead of calculating them separately before each \printf call.
