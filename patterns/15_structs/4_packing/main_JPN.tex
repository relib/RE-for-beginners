\subsection{\StructurePackingSectionName}
\label{structure_packing}

One important thing is fields packing in structures\footnote{See also: \URLWPDA}.

Let's take a simple example:

\lstinputlisting[style=customc]{patterns/15_structs/4_packing/packing.c}

As we see, we have two \Tchar fields (each is exactly one byte) and two more~---\Tint (each --- 4 bytes).

% subsections:
\input{patterns/15_structs/4_packing/x86_EN}
\input{patterns/15_structs/4_packing/ARM_EN}
\input{patterns/15_structs/4_packing/MIPS_EN}

\subsubsection{One more word}

Passing a structure as a function argument (instead of a passing pointer to structure) is the same
as passing all structure fields one by one.

If the structure fields are packed by default, the f() function can be rewritten as:

\begin{lstlisting}[style=customc]
void f(char a, int b, char c, int d)
{
    printf ("a=%d; b=%d; c=%d; d=%d\n", a, b, c, d);
};
\end{lstlisting}

And that leads to the same code.
