\subsubsection{x86}

This compiles to:

\lstinputlisting[caption=MSVC 2012 /GS- /Ob0,label=src:struct_packing_4,numbers=left,style=customasmx86]{patterns/15_structs/4_packing/packing_JPN.asm}

We pass the structure as a whole, but in fact, as we can see, the structure
is being copied to a temporary one (a place in stack is allocated in line 10 for it,
and then all 4 fields, one by one, are copied in lines 12 \ldots\ 19), 
and then its pointer (address) is to be passed.

The structure is copied because it's not known whether the \ttf{} 
function going to modify the structure or not.
If it gets changed, then the structure in \main has to remain as it has been.

We could use \CCpp pointers, and the resulting code will be almost the same, but without
the copying.

As we can see, each field's address is aligned on a 4-byte boundary.
That's why each \Tchar occupies 4 bytes here (like \Tint). Why?
Because it is easier for the CPU to access memory at aligned addresses and to cache data from it.

However, it is not very economical.

Let's try to compile it with option (\TT{/Zp1}) 
(\IT{/Zp[n] pack structures on n-byte boundary}).

\lstinputlisting[caption=MSVC 2012 /GS- /Zp1,label=src:struct_packing_1,numbers=left,style=customasmx86]{patterns/15_structs/4_packing/packing_msvc_Zp1_JPN.asm}

Now the structure takes only 10 bytes and each \Tchar value takes 1 byte. What does it give to us?
Size economy. And as drawback~---the CPU accessing these fields slower than it could.

\label{short_struct_copying_using_MOV}

The structure is also copied in \main. Not field-by-field, but directly 10 bytes, using three pairs of \MOV.
Why not 4?

The compiler decided that it's better to copy 10 bytes using 3 \MOV pairs than to copy two 32-bit words
and two bytes using 4 \MOV pairs.

By the way, such copy implementation using \MOV instead of calling the \TT{memcpy()} function is widely
used, because it's faster than a call to \TT{memcpy()}---for short blocks, of course:
\myref{copying_short_blocks}.

As it can be easily guessed, if the structure is used in many source and object files,
all these must be compiled with the same convention about structures packing.

\newcommand{\FNURLMSDNZP}{\footnote{\href{http://go.yurichev.com/17067}
{MSDN: Working with Packing Structures}}}
\newcommand{\FNURLGCCPC}{\footnote{\href{http://go.yurichev.com/17068}
{Structure-Packing Pragmas}}}

Aside from MSVC \TT{/Zp} option which sets how to align each structure field, there is also
the \TT{\#pragma pack} compiler option, which can be defined right in the source code.
It is available in both MSVC\FNURLMSDNZP and GCC\FNURLGCCPC{}.

Let's get back to the \TT{SYSTEMTIME} structure that consists of 16-bit fields.
How does our compiler know to pack them on 1-byte alignment boundary?

\TT{WinNT.h} file has this:

\begin{lstlisting}[caption=WinNT.h,style=customc]
#include "pshpack1.h"
\end{lstlisting}

And this:

\begin{lstlisting}[caption=WinNT.h,style=customc]
#include "pshpack4.h"                   // 4 byte packing is the default
\end{lstlisting}

The file PshPack1.h looks like:

\begin{lstlisting}[caption=PshPack1.h,style=customc]
#if ! (defined(lint) || defined(RC_INVOKED))
#if ( _MSC_VER >= 800 && !defined(_M_I86)) || defined(_PUSHPOP_SUPPORTED)
#pragma warning(disable:4103)
#if !(defined( MIDL_PASS )) || defined( __midl )
#pragma pack(push,1)
#else
#pragma pack(1)
#endif
#else
#pragma pack(1)
#endif
#endif /* ! (defined(lint) || defined(RC_INVOKED)) */
\end{lstlisting}

This tell the compiler how to pack the structures defined after \TT{\#pragma pack}.

\clearpage
\myparagraph{\olly + fields are packed by default}
\myindex{\olly}

Let's try our example (where the fields are aligned by default (4 bytes)) in \olly:

\begin{figure}[H]
\centering
\myincludegraphics{patterns/15_structs/4_packing/olly_packing_4.png}
\caption{\olly: Before \printf execution}
\label{fig:packing_olly_4}
\end{figure}

We see our 4 fields in the data window.

But where do the random bytes (0x30, 0x37, 0x01) come from, that are next to the first (a) and third (c) fields?

By looking at our listing \myref{src:struct_packing_4}, we can see that the first and third fields
are \Tchar, therefore only one byte is written, 1 and 3 respectively (lines 6 and 8).

The remaining 3 bytes of the 32-bit words are not being modified in memory!
Hence, random garbage is left there.
\myindex{x86!\Instructions!MOVSX}

This garbage doesn't influence the \printf output in any way, because the values for it are prepared
using the \MOVSX instruction, which takes bytes, not words: 
\lstref{src:struct_packing_4} (lines 34 and 38).

By the way, the \MOVSX (sign-extending) instruction is used here, because 
\Tchar is signed by default in MSVC and GCC.
If the \TT{unsigned char} data type or \TT{uint8\_t} was used here, 
\MOVZX instruction would have been used instead.

\clearpage
\myparagraph{\olly + fields aligning on 1 byte boundary}
\myindex{\olly}

Things are much clearer here: 4 fields occupy 10 bytes and the values are stored side-by-side

\begin{figure}[H]
\centering
\myincludegraphics{patterns/15_structs/4_packing/olly_packing_1.png}
\caption{\olly: Before \printf execution}
\label{fig:packing_olly_1}
\end{figure}

