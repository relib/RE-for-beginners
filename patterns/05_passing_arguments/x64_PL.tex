\subsection{x64}

\myindex{x86-64}

Rzecz wygląda nieco inaczej na x86-64. Argumenty funkcji (pierwsze 4 lub 6) są przekazywane przez rejestry, a więc \glslink{callee}{funkcja wywoływana} odczytuje je z rejestrów zamiast ze stosu.

\subsubsection{MSVC}

\Optimizing MSVC:

\lstinputlisting[caption=\Optimizing MSVC 2012 x64,style=customasmx86]{patterns/05_passing_arguments/x64_MSVC_Ox_PL.asm}

Jak widać, funkcja \ttf odczytuje wartości wszystkich argumentów z rejestrów.

Instrukcja \LEA została użyta do zrealizowania dodawania,
najwyraźniej kompilator uznał, że będzie szybsza niż \TT{ADD}.
\myindex{x86!\Instructions!LEA}

\LEA jest również używana w funkcji \main do przygotowania pierwszego i trzeciego argumentu funkcji \ttf. Kompilator
zdecydował, że będzie to szybsze niż klasyczne załadowanie wartości do rejestru za pośrednictwem instrukcji \MOV.

Rzućmy okiem na wynik nieoptymalizującego MSVC:

\lstinputlisting[caption=MSVC 2012 x64,style=customasmx86]{patterns/05_passing_arguments/x64_MSVC_IDA_PL.asm}

Wynik kompilacji wygląda dość dziwnie, ponieważ wszystkie 3 argumenty z rejestrów zostały z jakiegoś powodu odłożone na stos.

\myindex{Shadow space}
\label{shadow_space}
Nazywamy to \q{shadow space}
\footnote{\href{http://msdn.microsoft.com/en-us/library/zthk2dkh(v=vs.80).aspx}{MSDN}}: 

Każda funkcja w Win64 może (ale nie musi) zapisać tam wartości 4 argumentów, przekazywanych przez rejestry.
Dzieje się to z dwóch powodów:
1) przeznaczanie całego rejestru (lub nawet 4 rejestrów) na argumenty jest rozrzutne, więc dostęp do nich będzie zachodził przez stos
2) ułatwia to debuggowanie, gdyż debugger zawsze wie, gdzie znaleźć argumenty funkcji
\footnote{\href{http://msdn.microsoft.com/en-us/library/ew5tede7(v=VS.90).aspx}{MSDN}}.

Czasami duże funkcje mogą zapisywać swoje argumenty do \q{shadow space}, jeśli będą one wykorzystywane podczas wykonania,
ale małe funkcje nie muszą tego robić.

Odpowiedzialnością \glslink{caller}{funkcji wywołującej} jest zaalokowanie na stosie miejsca na \q{shadow space}.

\subsubsection{GCC}

\Optimizing GCC generuje dość zrozumiały kod:

\lstinputlisting[caption=\Optimizing GCC 4.4.6 x64,style=customasmx86]{patterns/05_passing_arguments/x64_GCC_O3_PL.s}

\NonOptimizing GCC:

\lstinputlisting[caption=GCC 4.4.6 x64,style=customasmx86]{patterns/05_passing_arguments/x64_GCC_PL.s}

\myindex{Shadow space}

W System V *NIX (\SysVABI) nie ma wymagania o \q{shadow space}, ale \glslink{callee}{funkcje wywoływane} mogą zapisywać swoje argumenty przy niedoborze rejestrów.

\subsubsection{GCC: uint64\_t zamiast int}

Nasz przykład wykorzystuje 32-bitowy typ \Tint, dlatego używane są 32-bitowe części rejestrów (z prefiksem \TT{E-}).

Zmodyfikujmy nieco przykład, by użyć wartości 64-bitowych:

\lstinputlisting[style=customc]{patterns/05_passing_arguments/ex64.c}

\lstinputlisting[caption=\Optimizing GCC 4.4.6 x64,style=customasmx86]{patterns/05_passing_arguments/ex64_GCC_O3_IDA_PL.asm}

Kod jest taki sam, ale tym razem użyto \emph{całych} rejestrów (z prefiksem \TT{R-}).

