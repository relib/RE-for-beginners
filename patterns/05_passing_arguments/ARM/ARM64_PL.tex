\subsubsection{ARM64}

\myparagraph{\Optimizing GCC (Linaro) 4.9}

\myindex{Fused multiply–add}
\myindex{ARM!\Instructions!MADD}
Wynik kompilacji jest bardzo prosty.
Instrukcja \TT{MADD} przeprowadza jednoczesne mnożenie i dodawanie (podobnie jak \TT{MLA}, którą widzieliśmy wcześniej).
Wszystkie 3 argumenty są w 32-bitowych częściach rejestrów X, ponieważ są one 32-bitowymi liczbami typu \emph{int}.
Wynik jest zwracana przez \TT{W0}.

\lstinputlisting[caption=\Optimizing GCC (Linaro) 4.9,style=customasmARM]{patterns/05_passing_arguments/ARM/ARM64_O3_PL.s}

Rozszerzmy typ danych do 64-bitowych liczb typu \TT{uint64\_t}:

\lstinputlisting[style=customc]{patterns/05_passing_arguments/ex64.c}

\begin{lstlisting}[style=customasmARM]
f:
	madd	x0, x0, x1, x2
	ret
main:
	mov	x1, 13396
	adrp	x0, .LC8
	stp	x29, x30, [sp, -16]!
	movk	x1, 0x27d0, lsl 16
	add	x0, x0, :lo12:.LC8
	movk	x1, 0x122, lsl 32
	add	x29, sp, 0
	movk	x1, 0x58be, lsl 48
	bl	printf
	mov	w0, 0
	ldp	x29, x30, [sp], 16
	ret

.LC8:
	.string	"%lld\n"
\end{lstlisting}

Funkcja \ttf{} jest niemal taka sama, używa jednak pełnych 64-bitowych rejestrów X.
Duże 64-bit wartości są zapisywane do rejestrów w cześciach, patrz: \myref{ARM_big_constants_loading}.

\myparagraph{\NonOptimizing GCC (Linaro) 4.9}

Kompilator z wyłączoną optymalizacją generuje dłuższy kod:

\begin{lstlisting}[style=customasmARM]
f:
	sub	sp, sp, #16
	str	w0, [sp,12]
	str	w1, [sp,8]
	str	w2, [sp,4]
	ldr	w1, [sp,12]
	ldr	w0, [sp,8]
	mul	w1, w1, w0
	ldr	w0, [sp,4]
	add	w0, w1, w0
	add	sp, sp, 16
	ret
\end{lstlisting}

Kod zapisuje swoje argumenty wejściowe na stosie lokalnym,
na wypadek gdyby coś w funkcji musiało użyć rejestrów \TT{W0...W2}.
Dzięki temu unikniemy nadpisywania oryginalnych wartości
argumentów funkcji, które mogą być potrzebne w przyszłości.

Jest to nazywane \emph{Register Save Area.} \ARMPCS.
Funkcja wywoływana nie ma obowiązku tego robić.
Przypomina to \q{Shadow Space}: \myref{shadow_space}.

Dlaczego optymalizujący GCC 4.9 pominął kod zapisujący argumenty?
Ponieważ przeprowadził analizę i wywnioskował,
że argumenty funkcji nie będą potrzebne w przyszłości i rejestry
\TT{W0...W2} nie będą używane.

\myindex{ARM!\Instructions!MUL}
\myindex{ARM!\Instructions!ADD}

Widać również parę instrukcji \TT{MUL}/\TT{ADD} zamiast pojedynczej \TT{MADD}.
