\mysection{Пранк: невозможность выйти из Windows 7}

\myindex{Windows 98}
\myindex{user32.dll}
Я уже не помню, как я нашел ф-цию \verb|ExitWindowsEx()| в Windows 98 (это был конец 1990-х), в файле user32.dll.
Вероятно я просто заметил само себя описывающее имя.
И затем я попробовал \emph{заблокировать} ей изменив первый байт на 0xC3 (\INS{RETN}).

В итоге стало смешно: из Windows 98 нельзя было выйти.
Пришлось нажимать на кнопку reset.

И вот теперь, на днях, я попробовал сделать то же самое в Windows 7, созданную почти на 10 лет позже, на базе принципиально другой
Windows NT.
И все еще, ф-ция \verb|ExitWindowsEx()| присутствует в файле user32.dll и служит тем же целям.

В начале, я отключил \emph{Windows File Protection} добавив это в реестр
(а иначе Windows будет молча восстанавливать модифицированные системные файлы):

\myindex{Windows File Protection}
\begin{lstlisting}
Windows Registry Editor Version 5.00

[HKEY_LOCAL_MACHINE\SOFTWARE\Microsoft\Windows NT\CurrentVersion\Winlogon]
"SFCDisable"=dword:ffffff9d
\end{lstlisting}

\myindex{Hiew}
\myindex{IDA}
Затем я переименовал \verb|c:\windows\system32\user32.dll| в \verb|user32.dll.bak|.
Я нашел точку входа ф-ции \verb|ExitWindowsEx()| в списке экспортируемых адресов в Hiew (то же можно было бы сделать и в IDA) и записал туда байт 0xC3.
Я перезагрузил Windows 7 и теперь её нельзя \emph{зашатдаунить}.
Кнопки "Restart" и "Logoff" больше не работают.

Не знаю, смешно ли это в наше время или нет, но тогда, в конце 90-х, мой товарищ отнес пропатченный файл user32.dll
на дискете в свой университет и скопировал его на все компьютеры
(бывшие в его доступе, работавшие под Windows 98 (почти все)).
После этого, выйти корректно из Windows было нельзя и его преподаватель информатики был адово злой.
(Надеюсь, он мог бы простить нас, если он это сейчас читает.)

Если вы делаете это, сохраните оригиналы всех файлов.
Лучше всего, запускать Windows в виртуальной машине.

