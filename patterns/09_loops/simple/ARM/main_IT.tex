\subsubsection{ARM}

\myparagraph{\NonOptimizingKeilVI (\ARMMode)}

\lstinputlisting[label=Keil_number_sign,style=customasmARM]{patterns/09_loops/simple/ARM/Keil_ARM_O0.asm}

Il contatore di iterazioni $i$ viene salvato nel registro \Reg{4}.
L' istruzione \INS{MOV R4, \#2} inizializza solamente $i$.
Le istruzioni \INS{MOV R0, R4} e \INS{BL printing\_function}
compongono il corpo del ciclo, la prima istruzione prepara l'argomento per la funzione 
\ttf e la seconda la chiama.
\myindex{ARM!\Instructions!ADD}
L'istruzione \INS{ADD R4, R4, \#1} aggiunge solamente 1 alla variabie $i$ ad ogni iterazione.
\myindex{ARM!\Instructions!CMP}
\myindex{ARM!\Instructions!BLT}
\INS{CMP R4, \#0xA} compara $i$ con \TT{0xA} (10). 
L' istruzione successiva \INS{BLT} (\emph{Branch Less Than}) salta se $i$ è minore di 10.
Altrimenti, 0 deve essere scritto in \Reg{0} (poiché la nostra funzione restituisce 0)
e termina l'esecuzione della funzione.

\myparagraph{\OptimizingKeilVI (\ThumbMode)}

\lstinputlisting[style=customasmARM]{patterns/09_loops/simple/ARM/Keil_thumb_O3.asm}

Sostanzialmente è uguale.

\myparagraph{\OptimizingXcodeIV (\ThumbTwoMode)}
\label{ARM_unrolled_loops}

\lstinputlisting[style=customasmARM]{patterns/09_loops/simple/ARM/xcode_thumb_O3.asm}

Infatti, questo era nella mia funzione \ttf:

\begin{lstlisting}[style=customc]
void printing_function(int i)
{
    printf ("%d\n", i);
};
\end{lstlisting}

\myindex{Unrolled loop}
\myindex{Inline code}
Quindi, LLVM non ha solamente \emph{srotolato} il ciclo, 
ma ha anche "inserito tra le linee" la mia
semplice funzione \ttf,
inserendo il suo codice 8 volte anzichè chiamarla. 

Ciò è possibile nel caso in cui la funzione sia molto semplice (come la mia) e non venga chiamata molto spesso (come qui).

\myparagraph{ARM64: \Optimizing GCC 4.9.1}

\lstinputlisting[caption=\Optimizing GCC 4.9.1,style=customasmARM]{patterns/09_loops/simple/ARM/ARM64_GCC491_O3_IT.s}

\myparagraph{ARM64: \NonOptimizing GCC 4.9.1}

\lstinputlisting[caption=\NonOptimizing GCC 4.9.1 -fno-inline,style=customasmARM]{patterns/09_loops/simple/ARM/ARM64_GCC491_O0_IT.s}
