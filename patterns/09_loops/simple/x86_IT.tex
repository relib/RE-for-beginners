\subsubsection{x86}

\myindex{x86!\Instructions!LOOP}

Nell' instruction set x86, c'è una speciale istruzione di \LOOP per controllare il valore nel registro \ECX e
se non è 0, \gls{decrement} \ECX
e passa il controllo del flusso alla label nell' operando di \LOOP. 
Probabilmente questa istruzione non è molto conveniente, e non ci sono moderni compilatori che la inseriscono automaticamente.
Di conseguenza, se la vedete da qualche parte, probabilmente quella parte di codice assembly è stata scritta a mano.

\par

In \CCpp i cicli sono solitamente costruiti usando le istruzioni \TT{for()}, \TT{while()} o \TT{do/while()}.

Iniziamo con \TT{for()}.
\myindex{\CLanguageElements!for}

Questa istruzione definisce l'inizializzazione del ciclo (imposta un contatore di cicli ad un valore iniziale), 
la condizione di ciclo (il contatore è maggiore di un valore limite?), cosa viene eseguito ad ogni iterazione (\gls{increment}/\gls{decrement} il contatore)
e ovviamente il corpo del ciclo.

\lstinputlisting[style=customc]{patterns/09_loops/simple/loops_1_IT.c}

Anche il codice generato è composto da quattro parti.

Iniziamo con un semplice esempio:

\lstinputlisting[label=loops_src,style=customc]{patterns/09_loops/simple/loops_2.c}

Il risultato (MSVC 2010):

\lstinputlisting[caption=MSVC 2010,style=customasmx86]{patterns/09_loops/simple/1_MSVC_IT.asm}

Come possiamo vedere, nulla di speciale.

GCC 4.4.1 emette quasi lo stesso codice, con una sottile differenza:

\lstinputlisting[caption=GCC 4.4.1,style=customasmx86]{patterns/09_loops/simple/1_GCC_IT.asm}

Ora vediamo cosa ottieniamo con l'ottimizzazione impostata su (\TT{\Ox}):

\lstinputlisting[caption=\Optimizing MSVC,style=customasmx86]{patterns/09_loops/simple/1_MSVC_Ox.asm}

Quello che abbiamo è che lo spazio per la veriabile $i$ non è più allocato nello stack,
ma viene utilizzato un registro, \ESI.
Questo è possibile nelle piccole funzioni dove non ci somo molte variabili locali.

Una cosa importante è che la funzione \ttf non deve cambiare il valore in \ESI.
Il nostro compilatore c'è lo assicura. 
E se il compilatore decide di usare il registro \ESI anche nella funzione \ttf, il suo valore viene salvato durante il prologo della funzione e ripristinato durante l'epilogo della funzione,
similmente al nostro esempio: notare \TT{PUSH ESI/POP ESI}
all'inizio e fine della funzione.

Proviamo GCC 4.4.1 con la massima ottimizzazione impostata (opzione \Othree):

\lstinputlisting[caption=\Optimizing GCC 4.4.1,style=customasmx86]{patterns/09_loops/simple/1_GCC_O3.asm}

\myindex{Loop unwinding}

Huh, GCC ha appena "srotolato" il nostro ciclo.

\Gls{loop unwinding} è vantaggioso nel caso in cui non ci siano molte iterazioni, perchè possiamo ridurre il tempo di esecuzione rimuovendo tutte le istruzioni di supporto ai cicli. 
Dall' altro lato, il codice risultante è ovviamente maggiore.

Srotolare grossi cicli non è raccomandato al giorno d'oggi, perchè grosse funzioni possono richiedere un ingombro della cache maggiore%
%
\footnote{Un ottimo articolo a riguardo: \DrepperMemory.
Qui ci sono altre raccomandazioni da Intel riguardo lo srotolamento dei cicli: 
\InSqBrackets{\IntelOptimization 3.4.1.7}.}.

OK, aumentiamo a 100 il massimo valore della variabile $i$ e proviamo nuovamente. GCC fa:

\lstinputlisting[caption=GCC,style=customasmx86]{patterns/09_loops/simple/2_GCC_IT.asm}

E' abbastanza simile a quello che produce MSVC 2010 con ottimizzazione (\Ox), 
con l'eccezione che il registro \EBX è allocato per la variabile $i$.

GCC è sicuro che questo registro non verrà modificato nella funzione \ttf 
e nel caso, verrà salvato durante il prologo della funzione e verrà ripristinato durante l'epilogo, 
proprio come in questo caso nella funzione \main.

\clearpage
\subsubsection{x86: \olly}
\myindex{\olly}

Compiliamo il nostro esempio con MSVC 2010 con le opzioni \Ox e \Obzero,
carichiamolo poi in \olly.

Sembrerebbe che \olly sia in grado di rilevare dei semplici cicli e ce li mostra tra parentesi quadre, per convenienza:

\begin{figure}[H]
\centering
\myincludegraphics{patterns/09_loops/simple/olly1.png}
\caption{\olly: inizio \main}
\label{fig:loops_olly_1}
\end{figure}

Tracciando (F8~--- \stepover) vediamo \ESI 
\glslink{increment}{incrementare}.
Qui, per esempio, $ESI=i=6$:

\begin{figure}[H]
\centering
\myincludegraphics{patterns/09_loops/simple/olly2.png}
\caption{\olly: corpo del ciclo appena eseguito con $i=6$}
\label{fig:loops_olly_2}
\end{figure}

9 è l'ultimo valore del ciclo.
Motivo per il quale, \JL non si attiva dopo \gls{increment} e la funzione concluderà:

\begin{figure}[H]
\centering
\myincludegraphics{patterns/09_loops/simple/olly3.png}
\caption{\olly: $ESI=10$, fine ciclo}
\label{fig:loops_olly_3}
\end{figure}

\subsubsection{x86: tracer}
\myindex{tracer}

Come possimo vedere, non è molto comodo tracciare manualmente nel debugger.
Questa è la ragione per cui proveremo ad usare \tracer.

Apriamo l'esempio compilato in \IDA, cerchiamo l'indirizzo dell' istruzione \INS{PUSH ESI}
(che passa l'unico argomento a \ttf), che è \TT{0x401026} in questo caso ed eseguiamo il \tracer:

\begin{lstlisting}
tracer.exe -l:loops_2.exe bpx=loops_2.exe!0x00401026
\end{lstlisting}

\TT{BPX} imposta solamente un breakpoint all' indirizzo e \tracer stamperà poi lo stato dei registri.

Questo è ciò che vediamo in  \TT{tracer.log}:

\lstinputlisting{patterns/09_loops/simple/tracer.log}

Vediamo il valore del registro \ESI, cambiare da 2 a 9.

Oltre a ciò, il \tracer può collezionare i valori del registro a tutti gli indirizzi all' interno della funzione.
Questo si chiama \emph{trace}.
Ogni istruzione viene tracciata, tutti i valori interessanti del registro vengono collezionati.

Dopodichè, viene generato un \IDA .idc-script, che aggiunge i commenti.
Quindi, abbiamo appreso in \IDA che l' indirizzo della funzione \main è \TT{0x00401020}, quindi eseguiamo:

\begin{lstlisting}
tracer.exe -l:loops_2.exe bpf=loops_2.exe!0x00401020,trace:cc
\end{lstlisting}

\TT{BPF} sta per "imposta breakpoint alla funzione".

Come risultato, otteniamo gli script \TT{loops\_2.exe.idc} e \TT{loops\_2.exe\_clear.idc}.

\clearpage
Carichiamo \TT{loops\_2.exe.idc} in \IDA e osserviamo:

\begin{figure}[H]
\centering
\myincludegraphics{patterns/09_loops/simple/IDA_tracer_cc.png}
\caption{\IDA con .idc-script caricato}
\label{fig:loops_IDA_tracer}
\end{figure}

Notiamo che \ESI può assumere valori da 2 a 9 all' inizio del corpo del ciclo,
ma da 3 a 0xA (10) dopo l'incremento.
Notiamo inoltre che il \main termina con 0 in \EAX.

\tracer genera inoltre \TT{loops\_2.exe.txt}, 
che contiene informazioni riguardo a quante volte ogni istruzione è stata eseguita e i valori del registro:

\lstinputlisting[caption=loops\_2.exe.txt]{patterns/09_loops/simple/loops_2.exe.txt}
\myindex{\GrepUsage}
Possiamo usare grep qui.

