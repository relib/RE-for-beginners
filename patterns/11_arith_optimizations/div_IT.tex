\subsection{Divisioni}

\subsubsection{Dividere usando gli shift}
\label{division_by_shifting}

Esempio di divisione per 4:

\begin{lstlisting}[style=customc]
unsigned int f(unsigned int a)
{
	return a/4;
};
\end{lstlisting}

Otteniamo (MSVC 2010):

\begin{lstlisting}[caption=MSVC 2010,style=customasmx86]
_a$ = 8		; size = 4
_f	PROC
	mov	eax, DWORD PTR _a$[esp-4]
	shr	eax, 2
	ret	0
_f	ENDP
\end{lstlisting}

\label{SHR}
\myindex{x86!\Instructions!SHR}

L' istruzione \SHR (\emph{SHift Right}) in questo esempio fa scorrere un numero di 2 bit a destra.
I due bit liberati a sinistra (e.g., i due bit più significativi) sono impostati a zero.
I due bit meno significativi sono scartati.
Infatti, questi due bit scartati sono il resto della divisione.

\myindex{x86!\Instructions!SHR}

L' istruzione \SHR funziona proprio come \SHL, ma nella direzione opposta.

\input{shift_right}

E' semplice da capire se immaginiamo il numero 23 nel sistema numerico decimale.
23 può essere facilmente diviso per 10, semplicemente scartando l' ultima cifra (3---resto divisione). 
2 rimane dopo l' operazione come \gls{quotient}.

Quindi il resto viene scartato, ma questo è OK, lavoriamo comunque su valori interi, 
che non sono \glslink{real number}{numeri reali}!

Divisione per 4 in ARM:

\begin{lstlisting}[caption=\NonOptimizingKeilVI (\ARMMode),style=customasmARM]
f PROC
        LSR      r0,r0,#2
        BX       lr
        ENDP
\end{lstlisting}

Divisione per 4 in MIPS:

\begin{lstlisting}[caption=\Optimizing GCC 4.4.5 (IDA),style=customasmMIPS]
        jr      $ra
        srl     $v0, $a0, 2 ; branch delay slot
\end{lstlisting}

\myindex{MIPS!\Instructions!SRL}
L' istruzione SRL corrisponde a \q{Shift Right Logical}.
