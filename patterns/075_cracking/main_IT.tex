\mysection{Software cracking}
\myindex{\SoftwareCracking}

La maggior parte dei software può essere craccata in questo modo --- cercando la reale posizione dove la protezione viene controllata, un dongle
(\myref{dongles}), license key, serial number, etc.

Solitamente è del tipo:

\begin{lstlisting}[style=customasmx86]
...
call check_protection
jz   all_OK
call message_box_protection_missing
call exit
all_OK:
; procedi
...
\end{lstlisting}

Quindi se vedete una patch (o ``crack''), che cracca un software, e questa patch rimpiazza i byte 0x74/0x75 (JZ/JNZ) con 0xEB (JMP),
è tutto qui.

Il processo di cracking del software si riduce alla ricerca di quel \INS{JMP}.

\myhrule{}

Ci sono anche casi, dove un software controlla la protezione ogni tanto, questo può essere un dongle, o un server di licenza può richiederlo attraverso internet.
In questo caso bisogna cercare la funzione che controlla la protezione.
Quindi per applicare una patch, inserire \TT{xor eax, eax / retn}, o \TT{mov eax, 1 / retn}.

E' importante capire che dopo aver applicato una patch all'inizio di una funzione, di solito, un garbage esegue queste due istruzioni.
Il garbage è composto da una parte di un' istruzione e diverse altre succesive.

Questo è un caso reale.
L'inizio della funzione che vogliamo \emph{rimpiazzare} con \TT{return 1;}

\begin{lstlisting}[style=customasmx86,caption=Prima]
8BFF                           mov         edi,edi
55                             push        ebp
8BEC                           mov         ebp,esp
81EC68080000                   sub         esp,000000868
A110C00001                     mov         eax,[00100C010]
33C5                           xor         eax,ebp
8945FC                         mov         [ebp][-4],eax
53                             push        ebx
8B5D08                         mov         ebx,[ebp][8]
...
\end{lstlisting}

\begin{lstlisting}[style=customasmx86,caption=Dopo]
B801000000                     mov         eax,1
C3                             retn
EC                             in          al,dx
68080000A1                     push        0A1000008
10C0                           adc         al,al
0001                           add         [ecx],al
33C5                           xor         eax,ebp
8945FC                         mov         [ebp][-4],eax
53                             push        ebx
8B5D08                         mov         ebx,[ebp][8]
...
\end{lstlisting}

Diverse istruzioni sbagliate appaiono --- \INS{IN}, \INS{PUSH}, \INS{ADC}, \INS{ADD},
dopo che, il disassemblatore Hiew (che ho appena usato) ha sincronizzato e continuato a disassemblare tutto ilresto.

Ciò non è importante --- tutte queste istruzioni successivve a \INS{RETN} non saranno mai eseguited,
a meno che da qualche parte avvenga un salto diretto, e questo non dovrebbe essere possibile in generale.

\myhrule{}

Inoltre, potrebbe essere presente una variabile booleana globale, con una flag, se il software è registrato oppure no.

\begin{lstlisting}[style=customasmx86]
init_etc proc
...
call check_protection_or_license_file
mov  is_demo, eax
...
retn
init_etc endp

...

save_file proc
...
mov  eax, is_demo
cmp  eax, 1
jz   all_OK1

call message_box_it_is_a_demo_no_saving_allowed
retn

:all_OK1
; continua a salvare il file

...

save_proc endp

somewhere_else proc

mov  eax, is_demo
cmp  eax, 1
jz   all_OK

; controlla se siamo in esecuzione da 15 minuti
; esci se è così
; o mostra sullo schermo qualcosa di fastidioso 

:all_OK2
; continua

somewhere_else endp
\end{lstlisting}

All'inizio di una funzione \verb|check_protection_or_license_file()| si potrebbe applicare una patch, cosicchè ritorni sempre 1 o se conviene, per diverse ragioni, applicare una patch a tutte le funzioni \INS{JZ}/\INS{JNZ}.

Altro sulle patch: \ref{patching}.

