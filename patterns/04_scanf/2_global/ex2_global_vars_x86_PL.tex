\subsubsection{MSVC: x86}

\lstinputlisting[style=customasmx86]{patterns/04_scanf/2_global/ex2_MSVC.asm}

W tym przypadku zmienna \TT{x} jest zdefiniowana w segmencie \TT{\_DATA} i nie następuje alokacja pamięci na stosie lokalnym. Dostęp do zmiennej jest bezpośredni, z pominięciem stosu.
Niezainicjalizowane zmienne globalne nie zajmują miejsca w pliku wykonywalnym (po co alokować wyzerowaną pamięć?), dopiero w momencie odwołania pod adres zmiennej, \ac{OS} alokuje blok pamięci, wypełniony zerami.

Przypiszmy teraz wprost wartość do zmiennej:

\lstinputlisting[style=customc]{patterns/04_scanf/2_global/default_value_PL.c}

Otrzymamy:

\begin{lstlisting}[style=customasmx86]
_DATA	SEGMENT
_x	DD	0aH

...
\end{lstlisting}

Widać przypisaną wartość \TT{0xA} typu DWORD  (DD oznacza DWORD, czyli 32 bity).

Jeśli otworzysz skompilowany plik .exe w programie \IDA, zobaczysz zmienną \emph{x} na początku segmentu \TT{\_DATA}, a zaraz za nią ciąg znaków.

Gdybyś otworzył w programie \IDA skompilowany plik .exe z poprzedniego przykładu (gdy \emph{x} była niezainicjalizowana), zobaczyłbyś podobny wynik:

\lstinputlisting[caption=\IDA,style=customasmx86]{patterns/04_scanf/2_global/IDA.lst}

\label{BSSClearedByCStd}
Zmienna \TT{\_x} razem z innymi zmiennymi, które nie muszą być zainicjalizowane, jest oznaczona za pomocą \TT{?}.
To powoduje, że po załadowaniu pliku wykonalnego do pamięci, miejsce na te zmienne jest zaalokowane i wypełnione zerami \InSqBrackets{\CNineNineStd 6.7.8p10}.
W samym pliku wykonywalnym niezainicjalizowane zmienne nie zajmują żadnego miejsca. Ma to swoje zalety, np. przy dużych tablicach.

\input{patterns/04_scanf/2_global/olly.tex}

\subsubsection{GCC: x86}

\myindex{ELF}
Wynik kompilacji na Linuksie jest prawi taki sam, różnicą jest to, że niezainicjalizowana zmienna jest umieszczona w segmencie \TT{\_bss}.
W plikach \ac{ELF} ten segment ma następujące atrybuty:

\begin{lstlisting}
; Segment type: Uninitialized
; Segment permissions: Read/Write
\end{lstlisting}

Jeśli jednak nadasz zmiennej jakość wartość, np. 10,
zostanie umieszczona w segmencie \TT{\_data}, który z kolei ma atrybuty:

\begin{lstlisting}
; Segment type: Pure data
; Segment permissions: Read/Write
\end{lstlisting}

\subsubsection{MSVC: x64}

\lstinputlisting[caption=MSVC 2012 x64,style=customasmx86]{patterns/04_scanf/2_global/ex2_MSVC_x64_PL.asm}

Kod jest niemal taki sam jak na x86.
Zauważ, że adres zmiennej $x$ jest przekazany do funkcji \TT{scanf()} za pomocą instrukcji \LEA,
ale wartość zmiennej jest przekazywana do drugiego wywołania \printf za pomocą instrukcji \MOV.
\TT{DWORD PTR}---to część kodu w asemblerze (bez związku z kodem maszynowym),
pokazująca, że zmienna jest 32-bitowa i instrukcja \MOV musi być odpowiednio zakodowana (opcode stosowny do rozmiaru).

