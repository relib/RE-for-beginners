\subsubsection{x86}

\myparagraph{MSVC}

Poniższy kod otrzymamy po kompilacji za pomocą MSVC 2010:

\lstinputlisting[style=customasmx86]{patterns/04_scanf/1_simple/ex1_MSVC_PL.asm}

\TT{x} jest zmienną lokalną.

Według standardu \CCpp zmienna lokalna może być widoczna tylko w konkretnej funkcji. Tradycyjnie zmienne lokalne są przechowywane na stosie. Prawdopodobnie są inne moliwości przechowywania tych zmiennych, ale tak akurat jest w x86.

\myindex{x86!\Instructions!PUSH}
Zadaniem instrukcji rozpoczynającej funkcję, \TT{PUSH ECX}, nie jest zapisanie stanu \ECX (można zauważyć brak odpowiadającej instrukcji \TT{POP ECX} na końcu funkcji).

Tak naprawdę instrukcja ta alokuje 4 bajty na stosie do przechowania zmiennej x.

\label{stack_frame}
\myindex{\Stack!Stack frame}
\myindex{x86!\Registers!EBP}
Dostęp do \TT{x} odbywa się za pomocą makra \TT{\_x\$} (-4) i rejestru \EBP, który wskazuje na bieżącą ramkę.

W trakcie wykonywania funkcji \EBP wskazuje na bieżącą \glslink{stack frame}{ramkę stosu}, umożliwiając dostęp do zmiennych lokalnych i argumentów funkcji poprzez \TT{EBP+offset}.

\myindex{x86!\Registers!ESP}
Można by użyć w tym celu rejestru \ESP, ale nie byłoby to zbyt wygodne, ponieważ wartość tego rejestru często się zmienia.
Wartość \EBP może być postrzegana jako \emph{zachowana} wartość \ESP z początku wykonania funkcji.

% FIXME1 это уже было в 02_stack?
Poniżej pokazano typowy układ ramki stosu w środowisku 32-bitowym:

\begin{center}
\begin{tabular}{ | l | l | }
\hline
\dots & \dots \\
\hline
EBP-8 & zmienna lokalna \#2, \MarkedInIDAAs{} \TT{var\_8} \\
\hline
EBP-4 & zmienna lokalna \#1, \MarkedInIDAAs{} \TT{var\_4} \\
\hline
EBP & zapisana wartość \EBP \\
\hline
EBP+4 & adres powrotu \\
\hline
EBP+8 & \argument \#1, \MarkedInIDAAs{} \TT{arg\_0} \\
\hline
EBP+0xC & \argument \#2, \MarkedInIDAAs{} \TT{arg\_4} \\
\hline
EBP+0x10 & \argument \#3, \MarkedInIDAAs{} \TT{arg\_8} \\
\hline
\dots & \dots \\
\hline
\end{tabular}
\end{center}

Funkcja \scanf w naszym przykładzie ma dwa argumenty.

Pierwszy jest wskaźnikiem na łańcuch znaków \TT{\%d} a drugi jest adresem zmiennej \TT{x}.

\myindex{x86!\Instructions!LEA}
Na początku adres zmiennej \TT{x} jest ładowany do rejestru \EAX przy pomocy instrukcji \\
\TT{lea eax, DWORD PTR \_x\$[ebp]}.

\LEA oznacza \emph{load effective address} i jest często używana do formowania adresów ~(\myref{sec:LEA}).

Można powiedzieć, że w tym przypadku \LEA po prostu umieszcza sumę rejestru \EBP i makra \TT{\_x\$} w rejestrze \EAX.

W tym przypadku (\TT{\_x\$} = -4) jest to samo co \INS{lea eax, [ebp-4]}. Więc od rejestru \EBP jest odejmowane 4 i wynik zostaje umieszczony w rejestrze \EAX.
Następnie wartość rejestru \EAX jest odkładana na stos i funkcja \scanf zostaje wywołana.

Kolejno następuje przygotowanie do wywołania funkcji \printf. Pierwszym argumentem jest wskaźnik na łańcuch znaków:
\TT{You entered \%d...\textbackslash{}n}.

Drugi argument jest przygotowywany za pomocą: \TT{mov ecx, [ebp-4]}.
Instrukcja kopiuje zmienną \TT{x} (nie jej adres) do rejestru \ECX.

Następnie wartość z \ECX jest odkładana na stos, a na koniec zostaje wywołana funkcja  \printf.

\input{patterns/04_scanf/1_simple/olly.tex}

\myparagraph{GCC}

Tak wygląda skompilowany kod w GCC 4.4.1 w systemie Linux:

\lstinputlisting[style=customasmx86]{patterns/04_scanf/1_simple/ex1_GCC.asm}

\myindex{puts() instead of printf()}
GCC zamienia wywołanie funkcji \printf na wywołanie funkcji \puts. Powód tego został wyjaśniony w ~(\myref{puts}).

% TODO: rewrite
%\RU{Почему \scanf переименовали в \TT{\_\_\_isoc99\_scanf}, я честно говоря, пока не знаю.}
%\EN{Why \scanf is renamed to \TT{\_\_\_isoc99\_scanf}, I do not know yet.}
% 
% Apparently it has to do with the ISO c99 standard compliance. By default GCC allows specifying a standard to adhere to.
% For example if you compile with -std=c89 the outputted assmebly file will contain scanf and not __isoc99__scanf. I guess current GCC version adhares to c99 by default.
% According to my understanding the two implementations differ in the set of suported modifyers (See printf man page)

Jak w przykładzie z MSVC---argumenty funkcji są umieszczane na stosie przy użyciu instrukcji \MOV.

\myparagraph{Nawiasem mówiąc\dots}

Ten prosty przykład pokazuje, że kompilatory rzeczywiście tłumaczą listę instrukcji języka C na serię instrukcji kodu maszynowego.
W kodzie C wykonanie odbywa się instrukcja po instrukcji i podobnie jest w kodzie maszynowym - między instrukcjami nie ma nic więcej.
