\subsection{Popularny błąd}

Bardzo popularnym błędem jest podanie jako argumentu zmiennej \emph{x} zamiast wskaźnika na zmienną \emph{x}:

\lstinputlisting[style=customc]{patterns/04_scanf/error.c}

Co się wtedy stanie?
% TBT: was incorrect "not uninitialized"
% correct:
% \emph{x} is not initialized and contains some random noise from local stack.
Zmienna \emph{x} jest niezainicjalizowana i zawiera losowe śmieci z lokalnego stosu.
Kiedy funkcja \scanf jest wywoływana, pobiera ciąg znaków od użytkownika, parsuje jako liczbę i próbuje ją zapisać w \emph{x}, traktując wartość \emph{x} jak adres w pamięci.
Jednak skoro tam są losowe śmieci ze stosu, \scanf będzie próbować uzyskać dostęp do losowego adresu.
Najprawdopodobniej proces natychmiast zakończy się błędem.

\myindex{0xCCCCCCCC}
\myindex{0x0BADF00D}
Co ciekawe, niektóre biblioteki \ac{CRT} w wersji do debugowania, umieszczają w pamięci, która jest alokowana, widoczne wzorce takie jak 0xCCCCCCCC albo 0x0BADF00D itp.
W tym przypadku \emph{x} może zawierać 0xCCCCCCCC, więc \scanf będzie próbowała zapisać pod adres 0xCCCCCCCC.
Jeśli zauważysz, że coś w procesie próbuje zapisać pod 0xCCCCCCCC, to znaczy, że
została użyta niezainicjalizowana zmienna (lub wskaźnik).
Jest to lepsze rozwiązanie niż gdyby nowo alokowana pamięć była po prostu wyzerowana.
