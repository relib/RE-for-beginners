\subsection{scanf()}

Jak wspomniano wcześniej, używanie \scanf w dzisiejszych czasach jest nieco staroświeckie.
Jeśli jednak musisz to zrobić, należy się upewnić czy wykonanie \scanf zakończyło się poprawnie, bez żadnego błędu.

\lstinputlisting[style=customc]{patterns/04_scanf/3_checking_retval/ex3.c}

Zgodnie ze standardem \scanf\footnote{scanf, wscanf: \MSDNscanf{}} zwraca liczbę pól, które zostały z sukcesem wczytane i zapisane.

W naszym przypadku, jeśli wszystko pójdzie dobrze i użytkownik wprowadził liczbę, \scanf zwróci 1. W przypadku wystąpienia błędu (lub \ac{EOF}), zwróci 0.

Dodaliśmy więcej kodu by sprawdzić co zwraca \scanf i wypiszmy ewentualne komunikaty o błędach.

Poniżej pokazano program w działaniu:

\begin{lstlisting}
C:\...>ex3.exe
Enter X:
123
You entered 123...

C:\...>ex3.exe
Enter X:
ouch
What you entered? Huh?
\end{lstlisting}

% subsections
\input{patterns/04_scanf/3_checking_retval/x86}
\input{patterns/04_scanf/3_checking_retval/x64}
\input{patterns/04_scanf/3_checking_retval/ARM}
\input{patterns/04_scanf/3_checking_retval/MIPS}

\subsubsection{\Exercise}

\myindex{x86!\Instructions!Jcc}
\myindex{ARM!\Instructions!Bcc}
Jak widać, instrukcja \INS{JNE}/\INS{JNZ} może być łatwo zastąpiona przez \INS{JE}/\INS{JZ} i vice versa
(a \INS{BNE} przez \INS{BEQ} i vice versa).
Jednak należy pamiętać o zamianie miejscami bloków kodu do wykonania. Spróbuj to zrobić w ramach ćwiczeń.

