\subsubsection{MSVC: x86}

Poniżej wynik kompilacji pod MSVC 2010:

\lstinputlisting[style=customasmx86]{patterns/04_scanf/3_checking_retval/ex3_MSVC_x86.asm}

\myindex{x86!\Registers!EAX}
\glslink{caller}{Funkcja wywołująca} (\main) potrzebuje rezultatu zwróconego przez \glslink{callee}{funkcję wywoływaną} (\scanf),
więc \glslink{callee}{funkcja wywoływana} zwraca go za pomocą rejestru \EAX.

\myindex{x86!\Instructions!CMP}
Rezultat sprawdzamy za pomocą instrukcji \TT{CMP EAX, 1} (\emph{CoMPare}) ~--- porównujemy wartość w rejestrze \EAX z liczbą 1.

\myindex{x86!\Instructions!JNE}
Instrukcja \JNE to skok warunkowy, następujący po \CMP. \JNE oznacza \emph{Jump if Not Equal}.

Jeśli wartość w \EAX jest różna od 1, \ac{CPU} przekaże sterowanie pod adres z operandu instrukcji \JNE, w naszym przypadku jest to \TT{\$LN2@main}.
Przekazanie sterowania pod ten adres oznacza wykonanie funkcji \printf z argumentem \TT{What you entered? Huh?}.
Jeśli natomiast \scanf zakończyła się sukcesem i wartość w EAX jest równa 1, skok warunkowy nie zostanie wykonany i kolejno zostanie wywołana funkcja \printf, z dwoma argumentami:\\
\TT{'You entered \%d...'} i wartością \TT{x}.

\myindex{x86!\Instructions!XOR}
\myindex{\CLanguageElements!return}
W tym drugin przypadku - gdy \scanf zakończyła się poprawnie - nie ma potrzeby wykonywać drugiego wywołania funkcji \printf, stąd przed wywołaniem znajduje się instrukcja \JMP (skok bezwarunkowy).
Instrukcja przekazuje sterowanie w miejsce za drugim wywołaniem \printf, ale przed instrukcją \TT{XOR EAX, EAX}, która realizuje \TT{return 0}.

% FIXME internal \ref{} to x86 flags instead of wikipedia
\myindex{x86!\Registers!\Flags}
Można powiedzieć, że porówywanie dwóch wartości jest zwykle realizowane przez parę instrukcji \CMP/\Jcc, gdzie \emph{cc} oznacza \emph{condition code} (\emph{kod warunku}).
\CMP porównuje dwie wartości i ustawia flagę procesora \footnote{rejestr FLAGS, więcej o tym przeczytasz pod adresem: \href{http://en.wikipedia.org/wiki/FLAGS_register_(computing)}{wikipedia}.}.
\Jcc sprawdza te flagi i decyduje czy przekazać sterowanie pod podany adres.

\myindex{x86!\Instructions!CMP}
\myindex{x86!\Instructions!SUB}
\myindex{x86!\Instructions!JNE}
\myindex{x86!\Registers!ZF}
\label{CMPandSUB}
Zabrzmi to paradoksalnie, ale instrukcja \CMP to tak na prawdę \SUB (subtract - \emph{odejmij}).
Nie tylko \CMP, ale wszystkie instrukcje arytmetyczne modyfikują flagi procesora.
Jeśli porównamy 1 z 1, $1-1$ daje 0, więc flaga \ZF zostanie ustawiona.
W żadnym innym przypadku flaga \ZF nie zostanie ustawiona, poza tym gdy operandy są sobie równe.
\JNE sprawdza tylko flagę \ZF i wykonuje skok, jeśli nie jest ustawiona. \JNE jest synonimem \JNZ (\emph{Jump if Not Zero}).
Asembler tłumaczy zarówno \JNE jak i \JNZ na ten sam kod operacji (opcode).
Instrukcja \CMP może być zastąpiona przez \SUB i prawie wszystko powinno działać poprawnie, poza tym, że \SUB zmieni wartość pierwszego operandu na wynik operacji odejmowania.
\CMP to \emph{SUB bez zapisywania wyniku operacji, ale ze zmianą flag}.

\subsubsection{MSVC: x86: IDA}

\myindex{IDA}
Nadszedł czas na uruchomienie programu \IDA i pokazanie jego możliwości.
Przy okazji, początkującym pomoże ustawienie opcji \TT{/MD} w MSVC, co spowoduje, że wszystkie
funkcji biblioteki standardowej nie będą statycznie zlinkowane do pliku wykonywalnego,
ale zostaną zaimportowane z \TT{MSVCR*.DLL} podczas wykonania.
Dzięki temu łatwiej będzie zobaczyć, które funkcje z biblioteki standardowej zostały użyte i gdzie.

Podczas analizy kodu w programie \IDA warto dla siebie (i innych) robić notatki.
W tym przypadku widzimy, że skok \TT{JNZ} wykona się w przypadku błędu.
Można przesunąć kursor do etykiety, nacisnąć \q{n} i zmienić nazwę na \q{error}.
Zmienimy również nazwę kolejnej etykiety na \q{exit}.

Poniżej listing po zmianiach nazw:

\lstinputlisting[style=customasmx86]{patterns/04_scanf/3_checking_retval/ex3.lst}

Te drobne modyfikacje ułatwiły zrozumienie kodu, jednak nie warto przesadzać i komentować każdej instrukcji.

% FIXME draw button?
W \IDA nożesz również ukryć (zwinąć) kod wybranej funkcji.
Zaznacz blok kodu, wciśnij \q{CTRL- -} na klawiaturze numerycznej i wpisz tekst, który ma
zostać wyświetlony zamiast kodu.

Ukryjmy dwa bloki kodu i nadajmy im nazwy:

\lstinputlisting[style=customasmx86]{patterns/04_scanf/3_checking_retval/ex3_2.lst}

% FIXME draw button?
By rozwinąć poprzednio zwinięte fragmenty, użyj \q{CTRL-+} na klawiaturze numerycznej.

\clearpage
Po naciśnięciu \q{spacji} zobaczymy reprezentację funkcji w postaci grafu.

\begin{figure}[H]
\centering
\myincludegraphics{patterns/04_scanf/3_checking_retval/IDA.png}
\caption{Tryb grafu w IDA}
\label{fig:ex3_IDA_1}
\end{figure}

Z każdego skoku warunkowego wychodzą dwie strzałki: zielona i czerwona.
Zielona wskazuje blok, który się wykona w przypadku wykonania skoku, a czerwona - blok, który się wykona, gdy do skoku nie dojdzie.

\clearpage
W tym trybie można zwinąć węzły i nadać nazwę tak stworzonej \q{grupie węzłów}.
Zróbmy to dla 3 bloków:

\begin{figure}[H]
\centering
\myincludegraphics{patterns/04_scanf/3_checking_retval/IDA2.png}
\caption{Tryb grafu w IDA przy 3 zwiniętych węzłach}
\label{fig:ex3_IDA_2}
\end{figure}

Jest to dość użyteczne.
Można powiedzieć, że istotną częścią pracy osoby zajmującej się inżynierią wsteczną (a także każdego innego badacza) jest ograniczenie ilości informacji.

\clearpage
\subsubsection{MSVC: x86 + \olly}

Spróbujmy zhackować nasz program w \olly, zmuszając go, by uznał, że funkcja \scanf wykonała się bez błędów.
Kiedy adres zmiennej lokalnej jest przekazywany do \scanf,
zmienna początkowo zawiera przypadkową wartość, w tym wypadku \TT{0x6E494714}:

\begin{figure}[H]
\centering
\myincludegraphics{patterns/04_scanf/3_checking_retval/olly_1.png}
\caption{\olly: przekazywanie adresu zmiennej do \scanf}
\label{fig:scanf_ex3_olly_1}
\end{figure}

\clearpage
Kiedy wykonywana jest funkcja \scanf , w konsoli wpiszmy coś, co z pewnością nie jest liczbą, na przykład \q{asdasd}.
\scanf kończy działanie z 0 w \EAX, co wskazuje na wystąpienie błędu:

\begin{figure}[H]
\centering
\myincludegraphics{patterns/04_scanf/3_checking_retval/olly_2.png}
\caption{\olly: \scanf zwraca błąd}
\label{fig:scanf_ex3_olly_2}
\end{figure}

Możemy sprawdzić wartość zmiennej lokalnej na stosie i zauważyć, że się ona nie zmieniła.
W rzeczy samej, dlaczego funkcja \scanf miałaby cokolwiek tam zapisać?
Jej wykonanie nie spowodowało nic, poza zwróceniem zera.

Spróbujmy \q{zhackować} nasz program.
Kliknij prawym przyciskiem na \EAX,
wśród opcji znajduje się \q{Set to 1} (\emph{ustaw na 1}).
To jest to, czego szukamy.

Mamy teraz 1 w \EAX, a więc kolejne sprawdzenie powinno się wykonać zgodnie z oczekiwaniami i
\printf powinna wyświetlić wartość zmiennej ze stosu.

Po wznowieniu wykonania programu (F9) widzimy następujący efekt w oknie konsoli:

\lstinputlisting[caption=console window]{patterns/04_scanf/3_checking_retval/console.txt}

1850296084 to postać dziesiętna liczby, którą widzieliśmy na stosie (\TT{0x6E494714})!


\clearpage
\subsubsection{MSVC: x86 + Hiew}
\myindex{Hiew}

Na tym przykładzie pokażemy proste \emph{poprawianie} plików wykonalnych.
Tak zmodyfikujmy program, by zawsze wypisał wejście wprowadzony przez użytkownika, niezależnie od jego treści.

Zakładając, że plik wykonywalny jest linkowany dynamicznie z \TT{MSVCR*.DLL} (kompilacja z opcją \TT{/MD}),
zobaczymy funkcję \main na początku sekcji \TT{.text}.
Otwórzmy plik w Hiew i znajdźmy początek sekcji \TT{.text} (Enter, F8, F6, Enter, Enter).

Widzimy:

\begin{figure}[H]
\centering
\myincludegraphics{patterns/04_scanf/3_checking_retval/hiew_1.png}
\caption{Hiew: funkcja \main}
\label{fig:scanf_ex3_hiew_1}
\end{figure}

Hiew znajduje łańuchy znaków \ac{ASCIIZ} i je wyświetla, tak samo dzieje się również z nazwami zaimportowanych funkcji.

\clearpage
Przesuń kursor do adresu \TT{.00401027} (znajduje się tam instrukcja \TT{JNZ}, którą musimy ominąć), naciśnij F3 i wpisz \q{9090} (co oznacza dwie instrukcje \ac{NOP}):

\begin{figure}[H]
\centering
\myincludegraphics{patterns/04_scanf/3_checking_retval/hiew_2.png}
\caption{Hiew: zastąpienie \TT{JNZ} przez dwie instrukcje \ac{NOP}}
\label{fig:scanf_ex3_hiew_2}
\end{figure}

Następnie naciśnij F9 (update). Plik wykonywalny został zapisany na dysk i będzie się zachowywał zgodnie z naszymi oczekiwaniami.

Dwie instrukcje \ac{NOP} nie są najbardziej eleganckim rozwiązaniem.
Innym sposobem byłoby poprawienie instrukcji przez zapisanie 0 do drugiego bajtu kodu operacji (\glslink{jump offset}{przesunięcie skoku}),
by \TT{JNZ} zawsze skakała do kolejnej instrukcji.

Można też program zmodyfikować w drugą stronę: zastąpić pierwszy bajt przez \TT{EB}, nie zmieniając drugiego bajtu (\glslink{jump offset}{przesunięcie skoku}).
Otrzymamy wtedy skok bezwarunkowy, który zawsze będzie zachodził, przez co za każdym razem dostaniemy wiadomość o błędzie.
