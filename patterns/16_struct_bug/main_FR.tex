\mysection{Le bogue \emph{struct} classique}

Ceci est un bogue \emph{struct} classique.

Voici un exemple de définition:

\lstinputlisting[style=customc]{patterns/16_struct_bug/old.h}

Et puis les fichiers C:

\lstinputlisting[style=customc]{patterns/16_struct_bug/setter.c}

\lstinputlisting[style=customc]{patterns/16_struct_bug/printer.c}

Jusqu'ici, tout va bien.

Maintenant vous ajoutez un troisième champ dans la structure, entre les deux champs:

\lstinputlisting[style=customc]{patterns/16_struct_bug/new.h}

Et vous modifiez probablement la fonction \verb|setter()|, mais oubliez \verb|printer()|:

\lstinputlisting[style=customc]{patterns/16_struct_bug/setter_new.c}

Vous compilez votre projet, mais le fichier C où se trouve \verb|printer()| qui
est séparé, n'est pas recompilé car votre \ac{IDE} ou système de compilation n'a
pas d'idée que ce module dépend d'une définition de structure \emph{test}.
Peut-être car \verb|#include <new.h>| est oublié.
Ou peut-être que le fichier d'entête \verb|new.h| est inclus dans \verb|printer.c|
via un autre fichier d'entête.
Le fichier objet n'est pas modifié (l'\ac{IDE} pense qu'il n'a pas besoin d'être
recompilé), tandis que la fonction \verb|setter()| est déjà la nouvelle version.
Ces deux fichiers objets (ancien et nouveau) peuvent tôt ou tard être liés dans un
fichier exécutable.

Ensuite, vous le lancez, et le \verb|setter()| met les 3 champs aux offsets +0,
+4 et +8.
Toutefois. \verb|printer()| connait seulement 2 champs, et les prends aux offsets
+0 et +4 lors de l'affichage.

Ceci conduits à des bogues obscurs et méchants.
La raison est que l'\ac{IDE} ou le système de construction ou le Makefile ne savent
pas que les deux fichiers C (ou modules) dépendent de l'entête.
Un remède courant est de tout supprimer et de recompiler.

Ceci est également vrai pour les classes C++, puisqu'elles fonctionnent tout comme
des structures: \myref{CppClasses}.

Ceci est une maladie ce \CCpp, et une source de critique, oui.
De nombreux \ac{PL}s ont un meilleur support des modules et interfaces.
Mais gardez à l'esprit l'époque de création du ocmpilateur C: dans les années 70,
sur de vieux ordinateurs PDP.
Donc tout a été simplifié à ceci par les créateurs du C.
